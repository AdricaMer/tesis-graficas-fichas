\chapter{Cap\'i{}tulo 3}%
\label{cap:num cromatico}

%\section{Teoremas y demostraciones de clanes}%
%\label{sec:etiquetas}


\begin{teorema}
    \label{teo:sexto}
    Sea $G$ una gr\'afica y $F_k(G)$ su $k$-gr\'afica de fichas, entonces
    $\chi(F_k(G)) \leq \chi (G)$.
\end{teorema}

\begin{proof}
    Sea $c: V(G) \rightarrow \{0,1, \dots \chi(G)-1\}$ una coloraci\'on
    propia de $G$. Ahora, a cada v\'ertice $A$ de $F_k(G)$ le asignamos el
    color $ c'(A)= \Bigl(\Sigma_{x \in A}c(x)\Bigl) mod \chi(G)$. Basta ver
    que $c'$ es una coloraci\'on propia. Supongamos lo contrario, existen
    $A$ y $B$ v\'ertices adyacentes en $F_k(G)$ tales que $c'(A) = c'(B)$.
    Como $A$ y $B$ son adyacentes, entonces $A \triangle B = \{a.b\}$ para
    algunos v\'ertices $a$ y $b$ de $G$ tales que $ab \in E(G)$. Al tomar
    $c'(A) = c'(B)$ tenemos que $\Bigl(\Sigma_{x \in A}c(x)\Bigl) =
    \Bigl(\Sigma_{y \in B}c(y)\Bigl) (mod \chi(G))$. Sin p\'erdida de
    generalidad consideramos $a \in A$ y $b \in B$, entonces tenemos que
    $c(a) = c(b) (mod \chi(G))$. Entonces llegamos a la contradicci\'on de
    que $c$ no es una coloraci\'on propia pues $c(a) = c(b)$ y $ab \in
    E(G)$. Por lo que tenemos que $c'$ es una coloraci\'on propia de
    $F_k(G)$ con a lo m\'as $\chi (G)$ colores. Por lo tanto $\chi(F_k(G))
    \leq \chi (G)$.
\end{proof}

\begin{proposicion}
    \label{prop:primera}
    Sea $F_k(G)$ una gr\'afica bipartita para $k \geq 1$, entonces $F_l(G)$
    es una gr\'afica bipartita para todo $l \geq 1$.
\end{proposicion}

%demostrar en el cap de gr\'aficas que una gráfica es 1-cromática si y sólo
%si es vacı́a, y es 2-cromática si
%y sólo si es bipartita y no vacı́a
\begin{proof}
    Una gr\'afica es $2$-crom\'atica si y s\'olo si es bipartita y no
    vac\'ia. Por lo que basta demostrar que si $F_k(G)$ es una gr\'afica
    bipartita, entonces $G$ es una gr\'afica bipartita. Lo probaremos por
    contrapositiva, es decir si $G$ no es una gr\'afica bipartita, entonces
    $F_k(G)$ tampoco es una gr\'afica bipartita. Sea $G$ una gr\'afica que
    no es bipartita, entonces $G$ contiene un ciclo impar $C=(v_1, \dots,
    v_p)$. Si $p \geq k+1$, entonces tenemos un $p$-ciclo en $F_k(G)$ de la
    siguiente manera: $\{v_1, v_2, \dots, v_{k-2}, v_{k-1}, v_k\}$ $\{v_1,
    v_2, \dots, v_{k-2}, v_{k-1}, v_{k+1}\}$ $\{v_1, v_2, \dots, v_{k-2},
    v_{k-1}, v_{k+2}\}$ \dots $\{v_1, v_2, \dots, v_{k-2}, v_{k-1}, v_p\}$
    $\{v_1, v_2, \dots, v_{k-2}, v_k, v_p\}$ $\{v_1, v_2, \dots, v_{k-1},
    v_k, v_p\}$ \dots $\{v_1 v_3, \dots, v_{k-1}, v_k, v_p\}$ $\{v_2, v_3,
    \dots, v_{k-1}, v_k, v_p\}$. Entonces $F_k(G)$ no es bipartita. Ahora
    consideramos $p \leq k$. Como $n \geq k+1$, entonces existe un conjunto
    $A$ de $k-p+1$ v\'ertice en $V(G)\setminus C$. Dado que $C$ est\'a en $G
    -A$, usamos la construcci\'on anterior y obtenemos un ciclo impar en
    $F_j(G)$. Como $F_k(G,A) \simeq F_{p-1}(G-A)$ entonces existe un ciclo
    impar en $F_k(G,A)$ que es subgr\'afica de $F_k(G)$. Por lo tanto
    $F_k(G)$ no es bipartita.

    Tenemos que, si $F_k(G)$ es una gr\'afica bipartita, entonces $G$
    tamb\'en es una gr\'afica bipartita. Entonces $G$ es $2$-crom\'atica y
    por el \cref*{teo:sexto} tenemos que para cualquier $l\geq 1$, $\chi
    (F_l(G)) \geq \chi (G)=2$. Por lo tanto tenemos que $F_l(G)$ es
    $2$-crom\'atica, es decir es una gr\'afica bipartita.
\end{proof}

