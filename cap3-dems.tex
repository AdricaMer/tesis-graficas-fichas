\chapter{N\'umero Cr\'omatico}%
\label{cap:num cromatico}

\section{Teoremas y demostraciones}%
%\label{sec:etiquetas}


\begin{teorema}
\label{teo:num cromatico de G y F(G)}
    Sea $G$ una gr\'afica y $F_k(G)$ su $k$-gr\'afica de fichas, entonces
    $\chi(F_k(G)) \leq \chi (G)$.
\end{teorema}

\begin{proof}
    Sea $c: V(G) \rightarrow \{0,1, \dots \chi(G)-1\}$ una coloraci\'on
    propia de $G$. Ahora, a cada v\'ertice $A$ de $F_k(G)$ le asignamos el
    color $ c'(A)= \Bigl(\Sigma_{x \in A}c(x)\Bigl) \mod \chi(G)$. Basta ver
    que $c'$ es una coloraci\'on propia. Supongamos lo contrario, existen
    $A$ y $B$ v\'ertices adyacentes en $F_k(G)$ tales que $c'(A) = c'(B)$.
    Como $A$ y $B$ son adyacentes, entonces $A \triangle B = \{a.b\}$ para
    algunos v\'ertices $a$ y $b$ de $G$ tales que $ab \in E(G)$. Al tomar
    $c'(A) = c'(B)$ tenemos que $\Bigl(\Sigma_{x \in A}c(x)\Bigl) =
    \Bigl(\Sigma_{y \in B}c(y)\Bigl) (\mod \chi(G))$. Sin p\'erdida de
    generalidad consideramos $a \in A$ y $b \in B$, entonces tenemos que
    $c(a) = c(b) (\mod \chi(G))$. Entonces llegamos a la contradicci\'on de
    que $c$ no es una coloraci\'on propia pues $c(a) = c(b)$ y $ab \in
    E(G)$. Por lo que tenemos que $c'$ es una coloraci\'on propia de
    $F_k(G)$ con a lo m\'as $\chi (G)$ colores. Por lo tanto $\chi(F_k(G))
    \leq \chi (G)$.
\end{proof}

\begin{proposicion}
\label{prop:biparticion F(G)}
    Sea $F_k(G)$ una gr\'afica bipartita para $k \geq 1$, entonces $F_l(G)$
    es una gr\'afica bipartita para todo $l \geq 1$.
\end{proposicion}

%demostrar en el cap de gr\'aficas que una gráfica es 1-cromática si y sólo
%si es vacı́a, y es 2-cromática si
%y sólo si es bipartita y no vacı́a
\begin{proof}
    Una gr\'afica es $2$-crom\'atica si y s\'olo si es bipartita y no
    vac\'ia. Por lo que basta demostrar que si $F_k(G)$ es una gr\'afica
    bipartita, entonces $G$ es una gr\'afica bipartita. Lo probaremos por
    contrapositiva, es decir si $G$ no es una gr\'afica bipartita, entonces
    $F_k(G)$ tampoco es una gr\'afica bipartita. Sea $G$ una gr\'afica que
    no es bipartita, entonces $G$ contiene un ciclo impar $C=(v_1, \dots,
    v_p)$. Si $p \geq k+1$, entonces tenemos un $p$-ciclo en $F_k(G)$ de la
    siguiente manera: $\{v_1, v_2, \dots, v_{k-2}, v_{k-1}, v_k\}$ $\{v_1,
    v_2, \dots, v_{k-2}, v_{k-1}, v_{k+1}\}$ $\{v_1, v_2, \dots, v_{k-2},
    v_{k-1}, v_{k+2}\}$ \dots $\{v_1, v_2, \dots, v_{k-2}, v_{k-1}, v_p\}$
    $\{v_1, v_2, \dots, v_{k-2}, v_k, v_p\}$ $\{v_1, v_2, \dots, v_{k-1},
    v_k, v_p\}$ \dots $\{v_1 v_3, \dots, v_{k-1}, v_k, v_p\}$ $\{v_2, v_3,
    \dots, v_{k-1}, v_k, v_p\}$. Entonces $F_k(G)$ no es bipartita. Ahora
    consideramos $p \leq k$. Como $n \geq k+1$, entonces existe un conjunto
    $A$ de $k-p+1$ v\'ertice en $V(G)\setminus C$. Dado que $C$ est\'a en $G
    -A$, usamos la construcci\'on anterior y obtenemos un ciclo impar en
    $F_j(G)$. Como $F_k(G,A) \simeq F_{p-1}(G-A)$ entonces existe un ciclo
    impar en $F_k(G,A)$ que es subgr\'afica de $F_k(G)$. Por lo tanto
    $F_k(G)$ no es bipartita.

    Tenemos que, si $F_k(G)$ es una gr\'afica bipartita, entonces $G$
    tamb\'en es una gr\'afica bipartita. Entonces $G$ es $2$-crom\'atica y
    por el \cref{teo:num cromatico de G y F(G)} tenemos que para cualquier $l\geq 1$, $\chi
    (F_l(G)) \geq \chi (G)=2$. Por lo tanto tenemos que $F_l(G)$ es
    $2$-crom\'atica, es decir es una gr\'afica bipartita.
\end{proof}

\begin{teorema}
    \label{relacion num cromatico G y F(G) con k}
        Sea $G$ una gr\'afica y $F_k(G)$ su $k$-gr\'afica de fichas. Entonces
        $\chi(F_k(G)) \geq \frac{n-k+2}{n} \chi(G) -1$.
    \end{teorema}
    
    
    %definici\ón de grafica inducida
    %dem n\'umero crom\'atico de subgr\'afica menor a n\'umero cr\'omatioc de gr\'afica
    % \begin{proof}
    %     Primero consideramos $k=1$. Sabemos que $n \geq \chi(G)$, por lo que tenemos
    %     que $n\chi(G) \geq n\chi(G) + \chi(G) -n$. Entonces obtenemos $\chi (G) \geq
    %     \frac{(n+1)\chi(G)-n}{n} = \frac{n+1}{n}\chi (G) -1$. Por otra parte,
    %     $F_1(G) \simeq (G)$. Por lo tanto tenemos que $\chi(F_1(G)) \geq
    %     \frac{n-1+2}{n} \chi(G) -1$.
    
    %     Ahora sumongamos $k \geq 2$ y una coloraci\'on de $\chi(G)$ colores en $G$.
    %     Nombramos $V_1 V_2, \dots, V_{\chi(G)}$ a las clases de colores de $G$. Sin
    %     p\'erdida de generalidad, consideramos que $|V_1|\geq |V_2|\geq \cdots \geq
    %     |V_{\chi(G)}|$. 
    %     %Entonces tenemso que para todo $m \in \{1, \dots, \chi(G)\}$ se cumple que
    %     %$\Sigma_{i=1}^{m}|V_i| \geq \frac{mn}{\chi(G)}$.
    %     Denotamos $m'$ el menor \'indice tal que $\Sigma_{i=1}^{m'}|V_i| \geq k-1$.
    %     Notamos que $\Sigma_{i=1}^{m'-1}|V_i| \leq k-2$, pues de lo contrario
    %     tendr\'iamos que $k-2<\Sigma_{i=1}^{m'-1}|V_i| < k-1$, es decir
    %     $\Sigma_{i=1}^{m'-1}|V_i|<1$. Entonces tenemos que
    %     $\frac{(m-1)n}{\chi(G)}\leq \Sigma_{i=1}^{m}|V_i| \leq k-2$. Ahora, tomamos
    %     $X \subseteq \bigcup_{i=1}^{m'} V_i$ con cardinalidad $k-1$. Por como se
    %      defini\'o el conjunto, $G[X]$ es $m$-coloreable.% Entonces tenemos que
    %     % $\chi(G) \leq \chi(G[X])+\chi(G-X) \leq m + \chi(G-X)$.
    
    %     Ahora, sabemos que $F_{k-r}(G-H) \simeq F_k(G,H)$, donde $r = |X|$. En este
    %     caso $|X| = k-1$, por lo que tenemos que $F_1(G-X) \simeq F_k(G,X)$.
    %     Adem\'as, sabemos que $F_1(G) \simeq G$. Por lo que tenemos que $G-X \simeq
    %     F_k(G,X)$, que es una subr\'afica de $F_k(G)$. Por lo tanto tenemos que
    %     $\chi(G) \leq m + \chi(G-X) = m + \chi(F_k(G,X)) \leq m + \chi(F_k(G))$. 
        
    %     Por otro lado, tenemos $\frac{(m-1)n}{\chi(G)}\leq \Sigma_{i=1}^{m'-1}|V_i|
    %     \leq k-2$. Entonces $m-1 \leq \frac{(k-2)}{n}\chi(G)$, de donde se sigue que
    %     $\chi(G) \leq m + \chi(F_k(G)) = \frac{(k-2)}{n}\chi(G) +1 + \chi(F_k(G))$.
    %     Por lo tanto $\chi(F_k(G)) \geq \frac{n-k+2}{n} \chi(G) -1$
    % \end{proof}
    
    \begin{teorema}
    \label{relacion num cromatico indep k}
        Sea $G$ una gr\'afica y $F_k(G)$ su $k$-gr\'afica de fichas. Entonces $\chi
        (F_k(G)) \geq (\frac{1}{2}+ \frac{2}{n})\chi(G) -1 $ para toda $k \geq 1$.
    \end{teorema}
    
    \begin{proof}
        Sea $G$ una gr\'afica y $F_k(G)$ su gr\'afica de fichas, con $k \geq 1$. Por
        \cref{relacion num cromatico G y F(G) con k} tenemos que $\chi(F_k(G)) \geq
        \frac{n-k+2}{n} \chi(G) -1$. Basta demostrar que $\frac{n-k+2}{n} \geq
        \frac{1}{2}+\frac{2}{n}$. Sabemos que $F_K(G) \simeq F_{n-k}(G)$, por lo que
        podemos asumir, sin p\'erdida de generalidad, que $k\leq \frac{n}{2}$.
        Entonces tenemos que $\frac{2n-n}{2}\geq k$, de donde obtenemos que $n-k
        \geq \frac{n}{2}$. Entonces tenemos que $\frac{n-k}{n}\geq \frac{1}{2}$. Por
        lo tanto $\frac{n-k+2}{n} \geq \frac{1}{2}+\frac{2}{n}$.
    \end{proof}