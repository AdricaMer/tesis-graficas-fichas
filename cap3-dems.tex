\chapter{N\'umero Cr\'omatico}%
\label{cap:num cromatico}

\section{Teoremas y demostraciones}%
%\label{sec:etiquetas}


\begin{teorema}
\label{teo:num cromatico de G y F(G)}
    Sea $G$ una gr\'afica y $F_k(G)$ su $k$-gr\'afica de fichas, entonces
    $\chi(F_k(G)) \leq \chi (G)$.
\end{teorema}

\begin{proof}
    Sea $c: V(G) \to \{0,1, \dots, \chi(G)-1\}$ una coloraci\'on propia de $G$.
    Ahora, a cada v\'ertice $A$ de $F_k(G)$ le asignamos el color $ c'(A)=
    \Sigma_{x \in A}c(x) \mod \chi(G)$. Basta ver que $c'$ es una coloraci\'on
    propia. Supongamos lo contrario, existen $A$ y $B$ v\'ertices adyacentes en
    $F_k(G)$ tales que $c'(A) = c'(B)$. Como $A$ y $B$ son adyacentes, entonces
    $A \triangle B = \{a,b\}$ para algunos v\'ertices $a$ y $b$ de $G$ tales que
    $ab \in E(G)$. Al tomar $c'(A) = c'(B)$ tenemos que $\Sigma_{x \in A}c(x)
    \equiv \Sigma_{y \in B}c(y) \mod \chi(G)$. Sin p\'erdida de generalidad
    consideramos $a \in A$ y $b \in B$, entonces podemos ver la congruencia de
    la siguiente manera: $c(a) + \Sigma_{x \in A \setminus\{a\}}c(x) \equiv c(b)
    + \Sigma_{y \in B\setminus\{b\}}c(y) \mod \chi(G)$. Pero $A \triangle B =
    \{a,b\}$, por lo que $\Sigma_{x \in A\setminus\{a\}}c(x)=\Sigma_{y \in
    B\setminus\{b\}}c(y)$. Entonces tenemos que $c(a) \equiv c(b) \mod \chi(G)$
    y de este modo llegamos a la contradicci\'on de que $c$ no es una
    coloraci\'on propia pues $c(a) = c(b)$ y $ab \in E(G)$. Por lo que tenemos
    que $c'$ es una coloraci\'on propia de $F_k(G)$ con a lo m\'as $\chi (G)$
    colores. Por lo tanto $\chi(F_k(G)) \leq \chi (G)$.
\end{proof}

\begin{proposicion}
\label{prop:biparticion F(G)}
    Sea $F_k(G)$ una gr\'afica bipartita para $k \geq 1$, entonces $F_l(G)$
    es una gr\'afica bipartita para todo $l \geq 1$.
\end{proposicion}

%demostrar en el cap de gr\'aficas que una gráfica es 1-cromática si y sólo
%si es vacı́a, y es 2-cromática si
%y sólo si es bipartita y no vacı́a
\begin{proof}
    Para empezar, veamos que si $F_k(G)$ es una gr\'afica bipartita, entonces
    $G$ es una gr\'afica bipartita. Lo probaremos por contrapositiva, es decir,
    si $G$ no es una gr\'afica bipartita, entonces $F_k(G)$ tampoco es una
    gr\'afica bipartita. Sea $G$ una gr\'afica que no es bipartita, entonces $G$
    contiene un ciclo impar $C=(v_1, \dots, v_{p-1}, v_p, v_1)$. Primero veamos
    el caso en el que $p \geq k+1$. En este caso tenemos un $p$-ciclo en
    $F_k(G)$ de la siguiente manera: $\{v_1, v_2, \dots, v_{k-2}, v_{k-1},
    v_k\}$ $\{v_1, v_2, \dots, v_{k-2}, v_{k-1}, v_{k+1}\}$ $\{v_1, v_2, \dots,
    v_{k-2}, v_{k-1}, v_{k+2}\}, \dots, \{v_1, v_2, \dots, v_{k-2}, v_{k-1},
    v_p\}$ $\{v_1, v_2, \dots, v_{k-2}, v_k, v_p\}$ $\{v_1, v_2, \dots, v_{k-1},
    v_k, v_p\} \dots \{v_1, v_3, \dots, v_{k-1}, v_k, v_p\}$ $\{v_2, v_3,
    \dots, v_{k-1}, v_k, v_p\}$ $\{v_1, v_2, \dots, v_{k-2}, v_{k-1}, v_k\}$.
    Notamos que el hay $p-k+1$ v\'ertices al mover la ficha de $v_k$ a $p$.
    Adem\'as tenemos $k-1$ v\'ertices en el resto del ciclo. Por lo tanto
    $F_k(G)$ tiene un $p$-ciclo por lo que no es bipartita.
    
    Ahora consideramos el caso en el que $p \leq k$. Notamos que el conjunto
    $V(G)\setminus C$ tiene cardinalidad $n-p$, donde $C$ es el ciclo impar
    antes mencionado. Como $n \geq k+1$, entonces $n-p \geq k+1-p$, por lo que
    existe un conjunto $A$ de $k-p+1$ v\'ertices en $V(G)\setminus C$. Notemos
    que al colocar una ficha en cada v\'ertice de $A$, y mantenerlas fijas, o
    equivalentemente s\'olo mover las $p-1$ fichas restantes, se obtiene una
    gr\'afica isomorfa a la gr\'afica de fichas $F_{p-1}(G-A)$. Entonces tenemos
    del caso anterior, en el que $k \leq p$, que $F_{p-1}(G-A)$ no es una
    gr\'afica bipartita. Pero como ya notamos, $F_{p-1}(G-A) = F_{k-(k+1-p)}
    (G-A)$ es isomorfa a la subgr\'afica de $F_k(G)$ inducida por los v\'ertices
    de $F_k(G)$ que contienen a $A$, $F_k(G,A)$. Por lo tanto existe un ciclo
    impar en una gr\'afica inducida de $F_k(G)$, de manera que $F_k(G)$ no es
    bipartita. Por lo tanto tenemos que si $F_k(G)$ es una gr\'afica bipartita,
    entonces $G$ tambi\'en lo es.

    Ahora pasamos a demostrar la proposici\'on. Sea $F_k(G)$ una gr\'afica
    bipartita, por lo demostrado anteriormente, tenemos que $G$ tambi\'en es una
    gr\'afica bipartita. Sabemos que una gr\'afica es $2$-crom\'atica si y
    s\'olo si es bipartita y no vac\'ia, por lo que $\chi(G)=2$. Pero, por
    \cref{teo:num cromatico de G y F(G)}, tenemos para cualquier $l\geq 1$,
    $\chi (F_l(G)) \le \chi (G) = 2$. Por lo tanto tenemos que $F_l(G)$ es
    $2$-crom\'atica, es decir es una gr\'afica bipartita.
\end{proof}



\begin{figure}[ht!]
    \centering
       \begin{tikzpicture}
    
        \begin{scope}[xshift=-5.5cm]
            \draw (0,1.5) node (1) [vertex,fill=salmon ,label=90:{\scriptsize $1$}] {};
            \draw (-1,0) node (2) [vertex,fill=amarilloOscuro ,label=115:{\scriptsize $2$}] {};
            \draw (1,0) node (3) [vertex,fill=amarilloOscuro ,label=80:{\scriptsize $3$}] {};
            \draw (-1,-1.5) node (4) [vertex,fill=salmon ,label=240:{\scriptsize $4$}] {};
            \draw (1,-1.5) node (5) [vertex,fill=salmon ,label=330:{\scriptsize $5$}] {};

            \foreach \i/\j in{1/2,1/3,4/2,4/3,5/2,5/3} 
            \draw [edge,grisOscuro] (\i) to (\j);
            \end{scope}
        
        \begin{scope}[xshift=-1cm,yshift=0cm,scale=1]
            \draw (-2,1.4) node (1) [vertex,fill=verde ,label=180:{\scriptsize $12$}] {};
            \draw (2,1.4) node (2) [vertex,fill=verde ,label=0:{\scriptsize $13$}] {};
            \draw (0,0.7) node (3) [vertex,fill=salmon ,label=90:{\scriptsize $14$}] {}; 
            \draw (0,2) node (4) [vertex,fill=salmon ,label=90:{\scriptsize $15$}] {};
            \draw (0,-0.7) node (5) [vertex,fill=salmon ,label=270:{\scriptsize $23$}] {};
            \draw (-2,-1.4) node (6) [vertex,fill=verde ,label=180:{\scriptsize $24$}] {};
            \draw (-2,0) node (7) [vertex,fill=verde ,label=180:{\scriptsize $25$}] {};
            \draw (2,-1.4) node (8) [vertex,fill=verde ,label=0:{\scriptsize $34$}] {};
            \draw (2,0) node (9) [vertex,fill=verde ,label=0:{\scriptsize $35$}] {};
            \draw (0,-2) node (10) [vertex,fill=salmon ,label=270:{\scriptsize $45$}] {};
           
            \foreach \i/\j in{1/5,1/3,1/4,6/5,6/3,6/10,7/5,7/4,7/10,5/2,5/8,5/9,
            2/3,2/4,8/3,8/10,9/4,9/10} 
            \draw [edge,grisOscuro] (\i) to (\j);
       \end{scope}
    
       \begin{scope}[xshift=5cm,yshift=0cm,scale=1]
            \draw (0,-0.7) node (1) [vertex,fill=salmon ,label=270:{\scriptsize $123$}] {};
            \draw (-2,-1.4) node (2) [vertex,fill=verde ,label=180:{\scriptsize $124$}] {};
            \draw (-2,1.4) node (3) [vertex,fill=verde ,label=180:{\scriptsize $125$}] {};
            \draw (2,-1.4) node (4) [vertex,fill=verde ,label=0:{\scriptsize $134$}] {};
            \draw (2,1.4) node (5) [vertex,fill=verde ,label=0:{\scriptsize $135$}] {};
            \draw (0,0.7) node (6) [vertex,fill=salmon ,label=90:{\scriptsize $145$}] {}; 
            \draw (0,-2) node (7) [vertex,fill=salmon ,label=270:{\scriptsize $234$}] {};
            \draw (0,2) node (8) [vertex,fill=salmon ,label=90:{\scriptsize $235$}] {};
            \draw (-2,0) node (9) [vertex,fill=verde ,label=180:{\scriptsize $245$}] {};
            \draw (2,0) node (10) [vertex,fill=verde ,label=0:{\scriptsize $345$}] {};
            
        \foreach \i/\j in{5/1,5/6,5/8,2/1,2/6,2/7,10/1,10/8,10/7,1/3,1/4,1/9,
            3/6,3/8,4/6,4/7,9/8,9/7} 
            \draw [edge,grisOscuro] (\i) to (\j);
   \end{scope}
            
    
    \end{tikzpicture}
    \caption{Gr\'afica bipartita, su 2-gr\'afica de fichas y su 3-gr\'afica de fichas.}
    \label{fig:ex-bip}
\end{figure}

Notamos que, dada la bipartici\'on
$({\color{amarilloOscuro}X},{\color{salmon}Y})$, toda $F_l(G)$, con $l\geq 2$,
tiene la bipartici\'on $({\color{verde}X},{\color{salmon}Y})$, donde
${\color{verde}X=\{V\in X\colon\ (\{2\}\subseteq V  \lor \{3\} \subseteq V)
\land \{2,3\}\not\subseteq V\}}$ y 

${\color{salmon}Y=\{V \in Y\colon\ \{2,3\}\subseteq V \lor \{2,3\}\not\subseteq
V\}}$. En otras palabras, para toda gr\'afica de fichas de $G$,el conjunto
${\color{verde}X}$ es el conjunto que contiene a
${\color{amarilloOscuro}X}\in G$ o no lo contiene. Adem\'as, el conjunto
${\color{salmon}Y}$ es el conjunto que tiene exactamente un elemente de
${\color{amarilloOscuro}X}\in G$    

    \begin{teorema}
        \label{relacion num cromatico G y F(G) con k}
            Sea $G$ una gr\'afica y $F_k(G)$ su $k$-gr\'afica de fichas.
            Entonces $\chi(F_k(G)) \geq \frac{n-k+2}{n} \chi(G) -1$.
        \end{teorema}
        
        %definici\ón de grafica inducida dem n\'umero crom\'atico de
        %subgr\'afica menor a n\'umero cr\'omatioc de gr\'afica 
        \begin{proof}
        Primero consideramos $k=1$. Sabemos que $n \geq \chi(G)$, por lo que
        tenemos que $n\chi(G) \geq n\chi(G) + \chi(G) -n$. Entonces obtenemos
        $\chi (G) \geq \frac{(n+1)\chi(G)-n}{n} = \frac{n+1}{n}\chi (G) -1$.
        Por otra parte, $F_1(G) \cong (G)$. Por lo tanto tenemos que
        $\chi(F_1(G)) \geq \frac{n-1+2}{n} \chi(G) -1$.
        
            Ahora supongamos $k \geq 2$ y una coloraci\'on de $\chi(G)$ colores
            en $G$. Nombramos $V_1 V_2, \dots, V_{\chi(G)}$ a las clases
            crom\'aticas de $G$. Sin p\'erdida de generalidad, consideramos que
            $|V_1|\geq |V_2|\geq \cdots \geq |V_{\chi(G)}|$. Al tomar este
            orden, todo \'indice $m$ cumple que $\Sigma_{i=1}^{m}|V_i| \geq
            \frac{mn}{\chi(G)}$, con la igualdad en el caso en el que $|V_i| =
            |V_j| = \frac{n}{\chi(G)}$, para todo $i,j \in \{1, \dots,
            \chi(G)\}$. Denotamos con $m'$ el menor \'indice tal que
            $\Sigma_{i=1}^{m'}|V_i| \geq k-1$. Notamos que
            $\Sigma_{i=1}^{m'-1}|V_i| \leq k-2$, pues de lo contrario
            tendr\'iamos que $k-2<\Sigma_{i=1}^{m'-1}|V_i| < k-1$, es decir
            $\Sigma_{i=1}^{m'-1}|V_i| \notin \mathbb{Z}$. Entonces tenemos que
            $\frac{(m'-1)n}{\chi(G)}\leq \Sigma_{i=1}^{m'-1}|V_i| \leq k-2$.
            Ahora, tomamos $X \subseteq \bigcup_{i=1}^{m'} V_i$ con cardinalidad
            $k-1$. Por como se defini\'o el conjunto, $G[X]$ es $m'$-coloreable.
            %Notamos que para un conjunto $S \subset V(G)$ se cumple que $\chi(G)
            %\leq \chi(G[S])+\¢hi(G-S)$. 
            Entonces tenemos que  $\chi(G) \leq \chi(G[X])+\chi(G-X) \leq m' +
            \chi(G-X)$.
            %TODO convertir el comentario en lema en la introduccion
        
            Ahora, sabemos que $F_{k-r}(G-H) \cong F_k(G,H)$, donde $r = |X|$.
            En este caso $|X| = k-1$, por lo que tenemos que $F_1(G-X) \cong
            F_k(G,X)$. Adem\'as, sabemos que $F_1(G) \cong G$. Por lo que
            tenemos que $G-X \cong F_k(G,X)$, que es una subr\'afica de
            $F_k(G)$. Por lo tanto tenemos que $\chi(G) \leq m + \chi(G-X) = m'
            + \chi(F_k(G,X)) \leq m' + \chi(F_k(G))$. X
            
            Por otro lado, tenemos $\frac{(m'-1)n}{\chi(G)}\leq
            \Sigma_{i=1}^{m'-1}|V_i| \leq k-2$. Entonces $m'-1 \leq
            \frac{(k-2)}{n}\chi(G)$, de donde se sigue que $\chi(G) \leq m' +
            \chi(F_k(G)) = \frac{(k-2)}{n}\chi(G) +1 + \chi(F_k(G))$. Por lo
            tanto $\chi(F_k(G)) \geq \frac{n-k+2}{n} \chi(G) -1$
        \end{proof}
    \begin{teorema}
    \label{relacion num cromatico indep k}
        Si $G$ es una gr\'afica y $F_k(G)$ su $k$-gr\'afica de fichas, entonces
        $\chi (F_k(G)) \geq (\frac{1}{2}+ \frac{2}{n})\chi(G) -1 $ para toda $k
        \geq 1$.
    \end{teorema}
    
    \begin{proof}
        Sea $G$ una gr\'afica y $F_k(G)$ su gr\'afica de fichas, con $k \geq 1$. Por
        \cref{relacion num cromatico G y F(G) con k} tenemos que $\chi(F_k(G)) \geq
        \frac{n-k+2}{n} \chi(G) -1$. Basta demostrar que $\frac{n-k+2}{n} \geq
        \frac{1}{2}+\frac{2}{n}$. Sabemos que $F_K(G) \cong F_{n-k}(G)$, por lo que
        podemos asumir, sin p\'erdida de generalidad, que $k\leq \frac{n}{2}$.
        Entonces tenemos que $\frac{2n-n}{2}\geq k$, de donde obtenemos que $n-k
        \geq \frac{n}{2}$. Entonces tenemos que $\frac{n-k}{n}\geq \frac{1}{2}$. Por
        lo tanto $\frac{n-k+2}{n} \geq \frac{1}{2}+\frac{2}{n}$.
    \end{proof}