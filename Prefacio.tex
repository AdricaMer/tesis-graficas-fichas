\chapter{Prefacio}
\label{sec:prefacio}

El trabajo a continuaci\'on presentado se estudia desde la Teor\'ia de
Gr\'aficas. La intenci\'on de este trabajo es generar literatura que pueda
servir como referencia a alumnos de la carrera de Matem\'aticas, as\'i como
cualquier persona interesada, por lo que busca busca ser un trabajo
autocontenido. Por ello, todas las proposiciones, lemas y teoremas del \'area de
Teor\'ia de Gr\'aficas son demostrados en el trabajo. Sin embargo, es deseable
que el lector tenga conocimientos b\'asicos en el \'area de Matem\'aticas para
entender con m\'as facilidad el contenido. 

El objetivo es estudiar varias propiedades de un tipo de gr\'aficas llamadas
gr\'aficas de fichas, traducido de su nombre en ingl\'es ``token graphs''.
% Tomando una gr\'afica $G$ y un entero positivo $k$, vamos a definir la
% gr\'afica de $k$-fichas de $G$ como la gr\'afica cuyo conjunto de v\'ertices
% es $\binom{V(G)}{k}$ y donde dos v\'ertices $A$ y $B$ son adyacentes si y
% s\'olo si $|A \triangle B| = \{a,b\}$, con $a \in A$, $b \in B$ y $ab \in
% E(G)$. 
La mayor\'ia de las propiedades a considerar salen del art\'iculo en el cu\'al
se bas\'o este trabajo \cite{fabilaToken}, donde se trabaja la conexidad, los
clanes y el n\'umero de crom\'atico de las gr\'aficas de fichas, El \'ultimo
capitulo se enfoca en la hamiltonicidad de estas gr\'aficas y para esto se toma
como base \cite{adameHamilt}. 

El primer cap\'itulo es una introducci\'on a la Teor\'ia de Gr\'aficas. Se
definen conceptos como el de una gr\'afica, subgr\'afica.....

El segundo cap\'itulo es una introducci\'on a las gr\'aficas de fichas. Se da
una breve explicaci\'on historica de el estudio de este tipo de gr\'aficas.
Luego, se pasa a explicar m\'as a detalle como son estas gr\'aficas. Finalmente
se dan algunos isomorfisos de gr\'aficas.

El cuarto cap\'itulo se centra en la conexidad de las gr\'aficas de fichas. Se
trabaja la relaci\'on entre la conexidad de una gr\'afica y la cantidad de
trayectorias internamente ajenas entre v\'ertices de la gr\'afica de fichas de
dicha gr\'afica. \'Esto para ayudar a probar la relaci\'on entre la conexidad de
una gr\'afica y la conexidad de su gr\'afica de fichas. En este cap\'itulo
tambien se ve un teorema sobre el di\'ametro de una gr\'afica y un teorema sobre
producto carteciano.

El cuarto cap\'itulo se  divide en dos secciones. La primera secci\'on se dan
cotas el n\'umero crom\'atico de una gr\'afica de fichas. En la segunda
secci\'on se da una caracterizaci\'on de los clanes de una gr\'afica de fichas.
Tamb\'ien se da un aproximado para el n\'umero de caln.

En el \'ultimo capitulo se estudia la hamiltonicidad de una familia de
gr\'aficas espec\'ifica, los abanicos. \'Esto con la intenci\'on de encontrar
una familia de gr\'aficas no hamiltonianas cuyas gr\'afcias de fichas si sean
hamiltonianas.