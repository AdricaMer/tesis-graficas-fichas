\chapter{Prefacio}
\label{sec:prefacio}

Dentro de la historia de las Matem\'aticas, la Teor\'ia de Gr\'aficas es un
\'area relativamente nueva que ha ido cambiando durante su crecimiento. 

Durante el proceso de este trabajo, al momento de explicar el tema de mi
proyecto, sobretodo a mis conocidos no matem\'aticos, es el pensar a una
gr\'afica como un tablero sobre el cu\'al se colocan fichas. Pensamos a los
v\'ertices de la gr\'afica como los lugares donde pueden estar las fichas y a
las aristas como los lugares por los cuales se pueden mover las fichas de un
v\'ertice a otro. De esta manera, la gr\'afica de fichas es la gr\'afica que
optienes de ver todas las posibilidades que tienes en tu tablero con un n\'umero
fijo de fichas. Los v\'ertices de esta nueva gr\'afica son todas las
configuraciones en las que puedes poner las fichas en tu gr\'afica original, una
``fotografia'' de la gr\'afica original con las fichas en los v\'ertices. Las
aristas de la gr\'afica de fichas nos dicen si se puede pasar de una
configuraci\'on a otra, de acuerdo a las reglas dadas. Otra manera de verlo es
el pensar a las aristas como un posible movimiento en el tablero, es decir, la gr\'afica original.

