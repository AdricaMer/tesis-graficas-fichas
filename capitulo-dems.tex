\chapter{Algunos Teoremas}%
\label{cap:ejemplos}

\section{Teoremas y demostraciones}%
\label{sec:etiquetas}

Todos los ambientes que se desee referir por n\'umero m\'as adelante deben de
tener una etiqueta.  Consideremos por ejemplo el siguiente lema.

%TODO: Cambiar delta por $d$.
\begin{teorema}%
\label{teo:primero}
Sea $G$ una gr\'afica conexa con di\'ametro $\delta$. Entonces, $F_{k}(G)$ es 
conexa con di\'ametro al menos $k(\delta -k+1)$ y a lo m\'as $\delta k$.
\end{teorema}

\begin{proof}
Sean $A$ y $B$ v\'ertices de $F_{k}(G)$. Primero nos enfocamos en la cota
superior. Por definici\'on tenemos que $|A \triangle B| \leq |A \cup B|$, con
igualdad cuando $A \cap B = \varnothing$. Observamos que, al ser $A$ y $B$
v\'ertices de $F_{k}(G)$, tenemos que $|A|=k$ y $|B|=k$ por lo que $|A \cup B|
\le 2k$. Entonces, tenemos que $|A \triangle B| \leq 2k$, por lo que
$\frac{1}{2} |A \triangle B| \leq k$.

Buscamos demostrar que el di\'ametro de $F_{k}(G)$ es a lo m\'as $\delta k$, por
lo que basta demostrar, por inducci\'on, que para cualesquiera dos v\'ertices
$A$ y $B$ de $F_{k}(G)$ hay una $AB$-trayectoria de a lo m\'as
$\frac{\delta}{2}|A\triangle B|$. Observamos que esto tambi\'en implica que
$F_{k}(G)$ es conexa.

Si $A\triangle B=\varnothing$, entonces $A=B$ por lo que no hay nada que probar.
Ahora consideramos $A$ y $B$ tales que $A\triangle B \neq \varnothing$. Tomamos
como hip\'otesis que para cualesquiera dos v\'ertices de $F_{k}(G)$, $C$ y $D$,
tales que $|C\triangle D|<|A \triangle B|$, existe una $CD$-trayectoria con
longitud a lo m\'as $\frac{\delta}{2}|C\triangle D|$. Al tomar $A\triangle B
\neq \varnothing$ tenemos un v\'ertice de $G$ en $A\setminus B$ y un v\'ertice
en $B\setminus A$, que denotamos $a$ y $b$ respectivamente. Dado que el
di\'ametro de $G$ es $\delta$, entonces hay una $ab$-trayectoria de longitud a
lo m\'as $\delta$, digamos $P$.

Definimos $A'=(A\setminus \{a\})\cup \{b\}$ y la trayectoria $A\xrightarrow[P]{}
A'$ en $F_{k}(G)$. Observamos que, por un lado $b\in B\cap A'$ y $b\notin B\cap
A$, pero $b\in A\cup B$. Por otro lado tenemos que $a\notin A'$ por lo que
$a\notin A'\cup B$ y $a\notin A\cap B$, pero $a\in A\cup B$. Entonces, tenemos
que $a,b \in A\triangle B$ y $a,b \notin A'\triangle B$. Ahora Tomamos $v\in A$
tal que $v \neq a$. Entonces, tenemos que $v \in A\triangle B$ si y s\'olo si
$v\in A'\triangle B$. Por lo tanto tenemos que $|A'\triangle B|=|A \triangle B|-
2$. Por hip\'otesis inductiva, sabemos que hay una $A'B$-trayectoria en
$F_{k}(G)$ de longitud a lo m\'as $\frac{\delta}{2}|A'\triangle B|$, que como se
observ\'o anteriormente, coincide con $\frac{\delta}{2}|A\triangle B| - \delta$.

Sabemos que $A\xrightarrow[P]{} A'$ tiene la misma longitud que $P$, que es a lo
m\'as $\delta$. Entonces, tenemos una $AB$-trayectoria de la forma $A\rightarrow
A'\rightarrow B$ que tiene longitud a lo m\'as $\frac{\delta}{2}|A\triangle
B|-\delta +\delta =\frac{\delta}{2}|A\triangle B|$. Por lo tanto tenemos que
$F_{k}(G)$ es conexa y tiene di\'ametro a lo m\'as $\delta k$.

Ahora demostraremos la cota inferior. Sabemos que $G$ es una gr\'afica conexa
con di\'ametro $\delta$, por lo que existen vertices que est\'an a distancia
$\delta$, digamos $x$ y $y$. Ahora construimos una partici\'on de $V$ usando  la
distancia que tiene cada v\'ertice a $x$. Es decir, para cada $i\in [0,\delta]$,
sea $V_{i}$ el conjunto de v\'ertices de $G$ a distancia $i$ de $x$. Entonces,
tenemos que $V_{0}=\{x\}$ y $y\in V_{\delta}$. Denotamos $d_x(v)$ a la distancia
entre $x$ y el v\'ertice $v$.

Sea $a$ el m\'\i{}nimo \'\i{}ndice para el cu\'al se tiene $k \leq |V_{0}\cup
V_{1}\cup \dots \cup V_{a}|$ y sea $b$ el m\'aximo \'\i{}ndice para el cu\'al se
tiene $k\leq |V_{b}\cup V_{b+1}\cup \dots \cup V_{\delta}|$. Tomamos $A$ un
$k$-\textit{subconjunto} de $V_{0}\cup \dots \cup V_{a}$  tal que $A\subseteq
V_{0}$ o $V_{0}\cup \dots V_{a-1}\subseteq A$. Tomamos $B$ un
$k$-\textit{subconjunto} de $V_{b}\cup \dots \cup V_{\delta}$ tale que
$B\subseteq V_{\delta}$ o $V_{b+1}\cup \dots \cup V_{\delta}$. 

Consideramos cualquier trayectoria entre $A$ y $B$ en $F_{k}(G)$. Cualquier
ficha inicialmente en $A$, digamos en el v\'ertice $v$ de $G$, se mueve a
alg\'un v\'ertice en $B$, digamos el v\'ertice $v'$ de $G$. Observamos que todas
las aristas de $G$ est\'an dentro de alg\'un $V_{i}$ o a lo m\'as entre alg\'un
$V_{i}$ y $V_{i+1}$, con $i\in[0,\delta]$. Entonces, para la ficha en $v$ se
necesitan al menos $d_x(v')-d_x(v)$ movimientos para llegar a $v'$, ocupando
s\'olo las aristas entre $V_{i}$ y $V_{i+1}$, $i\in [0,\delta]$. Por lo tanto,
el di\'ametro de $F_{k}(G)$ es al menos $\sum_{v\in A}(d_x(v')-d_x(v))=
\sum_{w\in B}d_x(w)-\sum_{v\in A}d_x(v)$. Observamos que, al ser $G$ conexa,
toda $V_{i}$ tiene al menos un elemento y por construcci\'on $V_{i} \cap
V_{i+1}=\varnothing$, para toda $i\in [0,\delta]$. Tomamos el caso en el que
$|V_{i}|=1$ para toda $i\in [0,\delta]$. Entonces, tenemos que $k\leq
|V_{b}\cup\dots\cup V_{\delta}|=|V_{b}|+|V_{b+1}|+\cdots +|V_\delta|$
$=\sum_{b}^{\delta}1 = \delta -b+1$. An\'alogamente tenemos que $k\leq
|V_{0}\cup V_{1}\cup \dots V_{a}|=|V_{0}|+|V_{1}|+\cdots + |V_{a}|$
$=\sum_{0}^{a} 1 = a+1$ En ambos casos la cota m\'\i{}nima se alcanza en la
igualdad, por lo que tomamos $a=k-1$ y $b=\delta-k+1$. Por lo tanto tenemos que
el di\'ametro de $F_{k}(G)$ es al menos $\sum_{j=\delta -k+1}^{\delta}j -
\sum_{i=0}^{k-1}i = k(\delta-k+1)$.
\end{proof}


\begin{lema}%
\label{lem:primero}
Sea $A$ un $k$-\textit{conjunto} en la gr\'afica $G$ y $a, b \in V(G)$ tales que
$a \in A$ y $b \notin B$. Sean $P, Q$ $ab$-\textit{trayectorias} internamente
ajenas en $G$. Entonces $A \xrightarrow[P]{} A'$ y $A \xrightarrow[Q]{} A'$ son
trayectorias internamente ajenas en $F_{k}(G)$.
\end{lema}

\begin{proof}
    Dado $A$ un $k$-\textit{conjunto} en la gr\'afica $G$ y $P$ y $Q$
    $ab$-\textit{trayectoria} internamente ajenas en $G$ tal que $a \in A$ y $b
    \notin A$, con $a, b \in V(G)$. Primero, supongamos que $|P \cap A| \geq 2$,
    con $P \cap A = \{v_{1}, v_{2}, \dots , v_{p}\}$, con $v_{1} = a$. Ahora
    consideremos $R$ un v\'ertice interno de $A \xrightarrow[P]{} A'$. Entonces
    tenemos que  $|R \cap P| = p$, las fichas de $A$ que se mueven por $P$
    conforme avanzamos en la trayectoria. 

    Observamos que si $k>p$, entonces $k -p$ fichas no est\'an ein la
    intersecci\'on, es decir, no est\'an sobre $P$ por lo que est\'an
    est\'aticas en todos los v\'ertices de $A \xrightarrow[P]{} A'$. 

    Por construcci\'on de $A \xrightarrow[P]{} A'$, $R$ tiene una ficha en la
    subtrayectoria %$\(v_{p},b\]$. 
    Entonces $R$ no contiene al conjunto $\{v_{2}, \dots, v_{p}\}$. Por otro
    lado, como $\{v_{2}, \dots, v_{p}\}$ est\'a en $A \cap P$ y $Q$ y $P$ son
    internamente ajenas, entonces $\{v_{2}, \dots, v_{p}\}$ est\'an est\'aticos
    en cada v\'ertice de $A \xrightarrow[Q]{} A'$. Por lo tanto tenemos que $A
    \xrightarrow[P]{} A'$ y $A \xrightarrow[Q]{}A'$ son trayectorias
    internamente ajenas.

    Al suponer que $|A \cap Q| \geq 2$ tenemos el caso an\'alogo al caso
    anterior.

    Ahora suponemos que $|A \cap P| = 1$ y $|A \cap Q| = 1$, es decir $A \cap P
    = \{a\} = A \cap Q$. Sin p\'erdida de generalidad suponemos que $P$ no es la
    arista $ab$. Entonces tenemos que $P \setminus \{a,b\} \neq \varnothing$.
    Entonces tenemos que todo v\'ertice interno de $A \xrightarrow[P]{} A'$
    tiene alg\'un v\'ertice de $P \setminus \{a, b\}$. Por otro lado, ning\'un
    v\'ertice interno de $A \xrightarrow[Q]{} A'$ tiene un v\'ertice de $P
    \setminus \{a, b\}$ pues $P$ y $Q$ son internamente ajenos y s\'olo
    comparten el v\'ertice $a$ con $A$. Entonces, cada v\'ertice interno de
    ambas trayectorias tiene intersecci\'on vac\'\i{}a con $A$. Por lo tanto
    tenemos que $A \xrightarrow[P]{} A'$ y $A \xrightarrow[Q]{} A'$ son
    internamente ajenos.
\end{proof}

\begin{lema}%
\label{lem:segundo}
    Sea $H$ una gr\'afica bipartita completa con clases de color $Y$ y $Z$,
    donde $|Y|<|Z|$. Supongamos que las aristas de $H$ est\'an coloreadas de
    azul y rojo de manera que cada v\'ertice de $Y$ es incidente en a lo m\'as
    una arista roja. Entonces $H$ tiene un conjunto $M$ de aristas azules tal
    que cada v\'ertice en $Y$ incide en exactamente una arista de $M$ y la
    uni\'on de aristas rojas y aristas de $M$ es ac\'\i{}clica.
\end{lema}

\begin{proof}
    Sea $H$ una gr\'afica bipartita completa con clases de color $Y$ y $Z$ como
    se especificaron. Demostramos el Lema por inducci\'on sobre $|Y|$. 

    Primero supongamos que $Y=\{y\}$. Por construcci\'on, existe una arista $e
    \in M$ tal que $y$ es incidente en $e$. Por otro lado, a lo m\'as existe una
    arista roja incidente en $y$, la nombramos $e'$. Entonces $\{e, e'\}$ no es
    un ciclo.

    Ahora supongamos que $|Y|>1$. Como tenemos que $|Y|<|Z|$, entonces existe
    alg\'un $x \in Z$ tal que no tiene aristas rojas incidentes. Sea $v \in Y$ y
    sea $e$ la arista roja incidente en $v$, en caso de que exista. Tomamos
    $H'= (H-v)-x$ con $R'$ el conjunto de aristas rojas de $H'$. Observamos que
    $Y' = Y- v$ y $Z'= Z- x$ son las clases de color que $H'$. Por inducci\'on
    existe un conjunto $M'$ de aristas azules incidentes en $H'$ tal que cada
    v\'ertice de $Y'$incide exactamente en una arista de $M'$ y adem\'as $R'\cup
    M'$ es ac\'\i{}clico. Ahora nos fijamos en $H$. Definimo $M= M'\cup \{xv\}$.
    Al tomar $x$ sin aristas rojas tenemos que $M$ cumple tener una arista azul
    por v\'ertice en $Y$. Tambi\'en definimos $R= R'\cup e$, si es que existe
    $e$.  Dado que  $M'\cup R'$ es ac\'\i{}clica y las aristas $\{vx\}$ y $e$, en caso de que exista, no
    est\'an en $M'\cup R'$, tenemos que $M \cup R$ es ac\'\i{}clica.
\end{proof}