\chapter{Algunos Teoremas}%
\label{cap:ejemplos}

\section{Teoremas y demostraciones}%
\label{sec:etiquetas}

Todos los ambientes que se desee referir por n\'umero m\'as adelante deben de
tener una etiqueta.  Consideremos por ejemplo el siguiente lema.

\begin{teorema}%
\label{teo:primero}
Sea $G$ una gr\'afica conexa con di\'ametro $d$. Entonces, $F_{k}(G)$ es 
conexa con di\'ametro al menos $k(d -k+1)$ y a lo m\'as $d k$.
\end{teorema}

\begin{proof}
Sean $A$ y $B$ v\'ertices de $F_{k}(G)$. Primero nos enfocamos en la cota
superior. Por definici\'on tenemos que $|A \triangle B| \leq |A \cup B|$, con
igualdad cuando $A \cap B = \varnothing$. Observamos que, al ser $A$ y $B$
v\'ertices de $F_{k}(G)$, tenemos que $|A|=k$ y $|B|=k$ por lo que $|A \cup B|
\le 2k$. Entonces, tenemos que $|A \triangle B| \leq 2k$, por lo que
$\frac{1}{2} |A \triangle B| \leq k$.

Buscamos demostrar que el di\'ametro de $F_{k}(G)$ es a lo m\'as $d k$, por
lo que basta demostrar, por inducci\'on, que para cualesquiera dos v\'ertices
$A$ y $B$ de $F_{k}(G)$ hay una $AB$-trayectoria de a lo m\'as
$\frac{d}{2}|A\triangle B|$. Observamos que esto tambi\'en implica que
$F_{k}(G)$ es conexa.

Si $A\triangle B=\varnothing$, entonces $A=B$ por lo que no hay nada que probar.
Ahora consideramos $A$ y $B$ tales que $A\triangle B \neq \varnothing$. Tomamos
como hip\'otesis que para cualesquiera dos v\'ertices de $F_{k}(G)$, $C$ y $D$,
tales que $|C\triangle D|<|A \triangle B|$, existe una $CD$-trayectoria con
longitud a lo m\'as $\frac{d}{2}|C\triangle D|$. Al tomar $A\triangle B
\neq \varnothing$ tenemos un v\'ertice de $G$ en $A\setminus B$ y un v\'ertice
en $B\setminus A$, que denotamos $a$ y $b$ respectivamente. Dado que el
di\'ametro de $G$ es $d$, entonces hay una $ab$-trayectoria de longitud a
lo m\'as $d$, digamos $P$.

Definimos $A'=(A\setminus \{a\})\cup \{b\}$ y la trayectoria $A\xrightarrow[P]{}
A'$ en $F_{k}(G)$. Observamos que, por un lado $b\in B\cap A'$ y $b\notin B\cap
A$, pero $b\in A\cup B$. Por otro lado tenemos que $a\notin A'$ por lo que
$a\notin A'\cup B$ y $a\notin A\cap B$, pero $a\in A\cup B$. Entonces, tenemos
que $a,b \in A\triangle B$ y $a,b \notin A'\triangle B$. Ahora tomamos $v\in A$
tal que $v \neq a$. Entonces, tenemos que $v \in A\triangle B$ si y s\'olo si
$v\in A'\triangle B$. Por lo tanto tenemos que $|A'\triangle B|=|A \triangle B|-
2$. Por hip\'otesis inductiva, sabemos que hay una $A'B$-trayectoria en
$F_{k}(G)$ de longitud a lo m\'as $\frac{d}{2}|A'\triangle B|$, que como se
observ\'o anteriormente, coincide con $\frac{d}{2}|A\triangle B| - d$.

Sabemos que $A\xrightarrow[P]{} A'$ tiene la misma longitud que $P$, que es a lo
m\'as $d$. Entonces, tenemos una $AB$-trayectoria de la forma $A\rightarrow
A'\rightarrow B$ que tiene longitud a lo m\'as $\frac{d}{2}|A\triangle
B|-d +d =\frac{d}{2}|A\triangle B|$. Por lo tanto tenemos que
$F_{k}(G)$ es conexa y tiene di\'ametro a lo m\'as $d k$.

Ahora demostraremos la cota inferior. Sabemos que $G$ es una gr\'afica conexa
con di\'ametro $d$, por lo que existen vertices que est\'an a distancia
$d$, digamos $x$ y $y$. Ahora construimos una partici\'on de $V$ usando  la
distancia que tiene cada v\'ertice a $x$. Es decir, para cada $i\in [0,d]$,
sea $V_{i}$ el conjunto de v\'ertices de $G$ a distancia $i$ de $x$. Entonces,
tenemos que $V_{0}=\{x\}$ y $y\in V_{d}$. Denotamos $d_x(v)$ a la distancia
entre $x$ y el v\'ertice $v$.

Sea $a$ el m\'\i{}nimo \'\i{}ndice para el cu\'al se tiene $k \leq |V_{0}\cup
V_{1}\cup \dots \cup V_{a}|$ y sea $b$ el m\'aximo \'\i{}ndice para el cu\'al se
tiene $k\leq |V_{b}\cup V_{b+1}\cup \dots \cup V_{d}|$. Tomamos $A$ un
$k$-\textit{subconjunto} de $V_{0}\cup \dots \cup V_{a}$  tal que $A\subseteq
V_{0}$ o $V_{0}\cup \dots V_{a-1}\subseteq A$. Tomamos $B$ un
$k$-\textit{subconjunto} de $V_{b}\cup \dots \cup V_{d}$ tale que
$B\subseteq V_{d}$ o $V_{b+1}\cup \dots \cup V_{d}$. 

Consideramos cualquier trayectoria entre $A$ y $B$ en $F_{k}(G)$. Cualquier
ficha inicialmente en $A$, digamos en el v\'ertice $v$ de $G$, se mueve a
alg\'un v\'ertice en $B$, digamos el v\'ertice $v'$ de $G$. Observamos que todas
las aristas de $G$ est\'an dentro de alg\'un $V_{i}$ o a lo m\'as entre alg\'un
$V_{i}$ y $V_{i+1}$, con $i\in[0,d]$. Entonces, para la ficha en $v$ se
necesitan al menos $d_x(v')-d_x(v)$ movimientos para llegar a $v'$, ocupando
s\'olo las aristas entre $V_{i}$ y $V_{i+1}$, $i\in [0,d]$. Por lo tanto,
el di\'ametro de $F_{k}(G)$ es al menos $\sum_{v\in A}(d_x(v')-d_x(v))=
\sum_{w\in B}d_x(w)-\sum_{v\in A}d_x(v)$. Observamos que, al ser $G$ conexa,
toda $V_{i}$ tiene al menos un elemento y por construcci\'on $V_{i} \cap
V_{i+1}=\varnothing$, para toda $i\in [0,d]$. Tomamos el caso en el que
$|V_{i}|=1$ para toda $i\in [0,d]$. Entonces, tenemos que $k\leq
|V_{b}\cup\dots\cup V_{d}|=|V_{b}|+|V_{b+1}|+\cdots +|V_d|$
$=\sum_{b}^{d}1 = d -b+1$. An\'alogamente tenemos que $k\leq
|V_{0}\cup V_{1}\cup \dots V_{a}|=|V_{0}|+|V_{1}|+\cdots + |V_{a}|$
$=\sum_{0}^{a} 1 = a+1$ En ambos casos la cota m\'\i{}nima se alcanza en la
igualdad, por lo que tomamos $a=k-1$ y $b=d-k+1$. Por lo tanto tenemos que
el di\'ametro de $F_{k}(G)$ es al menos $\sum_{j=d -k+1}^{d}j -
\sum_{i=0}^{k-1}i = k(d-k+1)$.
\end{proof}


\begin{lema}%
\label{lem:primero}
Sea $A$ un $k$-conjunto en la gr\'afica $G$ y $a, b \in V(G)$ tales que $a \in
A$ y $b \notin A$. Sea $A' = (A \setminus \{ a \}) \cup \{ b \}$. Si $P$ y $Q$
son $ab$-trayectorias internamente ajenas en $G$, entonces $A \xrightarrow[P]{}
A'$ y $A \xrightarrow[Q]{} A'$ son trayectorias internamente ajenas en
$F_{k}(G)$.
\end{lema}

\begin{proof}
    Primero, supongamos que $|V(P) \cap A| \geq 2$, con $V(P) \cap A = \{v_{1},
    v_{2}, \dots , v_{p}\}$, con $v_{1} = a$. Notemos que, si $k > p$, entonces
    $k-p$ fichas no est\'an sobre $P$, por lo que est\'an est\'aticas en todos
    los v\'ertices de $A \xrightarrow[P]{} A'$. Ahora, consideremos $R$ un
    v\'ertice interno de $A \xrightarrow[P]{} A'$. Por la observaci\'on anterior
    tenemos que  $|R \cap V(P)| = p$. Por construcci\'on de $A \xrightarrow[P]{}
    A'$, $R$ tiene una ficha en la trayectoria $(v_{p},b ]$. Entonces, $R$ no
    contiene al conjunto $\{v_{2}, \dots, v_{p}\}$. Por otro lado, como
    $\{v_{2}, \dots, v_{p}\}$ est\'a en $A \cap V(P)$, y $Q$ y $P$ son
    internamente ajenas, entonces $\{v_{2}, \dots, v_{p}\}$ est\'an est\'aticos
    en cada v\'ertice de $A \xrightarrow[Q]{} A'$. Por lo tanto tenemos que $A
    \xrightarrow[P]{} A'$ y $A \xrightarrow[Q]{}A'$ son trayectorias
    internamente ajenas. Al suponer que $|A \cap Q| \geq 2$, tenemos un caso
    an\'alogo al anterior.

    Ahora, supongamos que $|A \cap V(P)| = 1$ y $|A \cap V(Q)| = 1$, es decir,
    $A \cap V(P) = \{a\} = A \cap V(Q)$. Sin p\'erdida de generalidad suponemos
    que $P$ no es la arista $ab$. Entonces, tenemos que $P \setminus \{a,b\}
    \neq \varnothing$. Por lo tanto, tenemos que todo v\'ertice interno de $A
    \xrightarrow[P]{} A'$ tiene alg\'un v\'ertice de $P \setminus \{a, b\}$. Por
    otro lado, ya que $P$ y $Q$ son internamente ajenos y s\'olo comparten el
    v\'ertice $a$ con $A$, ning\'un v\'ertice interno de $A \xrightarrow[Q]{}
    A'$ tiene un v\'ertice de $P \setminus \{a, b\}$. Luego, cada v\'ertice
    interno de ambas trayectorias tiene intersecci\'on vac\'\i{}a con $A$.
    Concluimos que $A \xrightarrow[P]{} A'$ y $A \xrightarrow[Q]{} A'$ son
    internamente ajenas.
\end{proof}

\begin{lema}%
\label{lem:segundo}
    Sea $H$ una gr\'afica bipartita completa con clases de color $Y$ y $Z$,
    donde $|Y|<|Z|$. Si las aristas de $H$ est\'an coloreadas de azul y rojo de
    manera que cada v\'ertice de $Y$ es incidente en a lo m\'as una arista roja,
    entonces $H$ tiene un conjunto $M$ de aristas azules tal que cada v\'ertice
    en $Y$ incide en exactamente una arista de $M$. Adem\'as, la uni\'on de
    aristas rojas y aristas de $M$ es ac\'\i{}clica.
\end{lema}

\begin{proof}
    Sea $H$ una gr\'afica bipartita completa con clases de color $Y$ y $Z$ como
    se especificaron. Demostramos el resultado por inducci\'on sobre $|Y|$. 

    Primero, supongamos que $Y=\{y\}$. Como $H$ es bipartita completa, existe
    una arista azul $e$ tal que $y$ es incidente en $e$.   Sea $M = \{ e \}$.
    Por otro lado, a lo m\'as existe una arista roja incidente en $y$, digamos
    $e'$. Entonces la uni\'on de aristas rojas y aristas en $M$ es $\{e, e'\}$,
    que no es un ciclo.

    Ahora supongamos que $|Y|>1$. Como tenemos que $|Y|<|Z|$, y hay a lo m\'as
    una arista roja incidente en cada v\'ertice de $Y$, entonces existe alg\'un
    $x \in Z$ tal que no tiene aristas rojas incidentes. Sea $v \in Y$ y sea
    $e$, en caso de existir, la arista roja incidente en $v$. Tomamos $H'=
    (H-v)-x$ con $R'$ el conjunto de aristas rojas de $H'$. Observamos que $Y' =
    Y- v$ y $Z'= Z- x$ son las clases de color de $H'$. Por hip\'otesis de
    inducci\'on, existe un conjunto $M'$ de aristas azules incidentes en $H'$
    tal que cada v\'ertice de $Y'$ incide exactamente en una arista de $M'$.
    Adem\'as $R'\cup M'$ es ac\'\i{}clico.
    
    En $H$, definimos $M = M'\cup \{xv\}$. Al tomar $x$ sin aristas rojas
    tenemos que $M$ cumple tener una arista azul por v\'ertice en $Y$. Tambi\'en
    definimos $R= R'\cup \{ e \}$, si es que existe $e$.  Dado que  $M'\cup R'$
    es ac\'\i{}clica y las aristas $\{vx\}$ y $e$, en caso de existir, no
    est\'an en $M'\cup R'$, entonces tenemos que $M \cup R$ es ac\'\i{}clica.
\end{proof}

%TODO: Terminar demostraci\'on
\begin{lema}%
\label{lem:tercero}
    Sea $G$ una gr\'afica $t$-conexa. Sean $A$ y $B$ v\'ertices de $F_{k}(G)$
    tales que $|A \triangle B| = 2$. Entonces hay $t$ $AB$-trayectorias
    internamente ajenas en $F_{k}(G)$. Adem\'as, si $t \geq k$, entonces hay
    $k(t- k + 1)$ $AB$-trayectorias internamente ajenas en $F_{k}(G)$.
\end{lema}

\begin{proof}
    Sea $G$ una gr\'afica $t$-conexa y sean $A$ y $B$ v\'ertices de $F_{k}(G)$
    tales que $|A \triangle B| = 2$. Primero buscamos probar el n\'umero de
    $AB$-trayectorias internamente ajenas en $F_{k}(G)$ es al menos $t$. 
    
    Sean $a \in A \setminus B$ y $b \in B \setminus A$ los v\'ertices en la
    diferencia sim\'etrica. Al ser $G$ una gr\'afica $t$-conexa, entonces por el
    Teorema General de Menger, existen $P_{1}, \dots, P_{t}$ $ab$-trayectorias
    internamente ajenas en $G$. Por lo tanto, por el Lema 1.1.2, existen $A
    \xrightarrow[P_1]{}  B, \dots, A \xrightarrow[P_t]{}  B$ $AB$-trayectorias
    internamente ajenas en $F_{k}(G)$. 

    Ahora buscamos demostrar que, si $t \geq k$, entonces el n\'umero de
    $AB$-trayectorias internamente ajenas en $F_{k}(G)$ es al menos $k(t- k
    +1)$. Asumimos que $t \geq k + 1$. Sea $\mathcal{P}$ un conjunto m\'aximo de
    $ab$-trayectorias internamente ajenas en $G$. Por el Teorema General de
    Menger sabemos que $|\mathcal{P}| \ge t$. Elegimos un conjunto $\mathcal{P}$
    para el que ninguna trayectoria tenga cuerdas. Definimos a las trayectorias
    $P_{1}, \dots, P_{l}$ como aquellas en $\mathcal{P}$ que no intersectan a $A
    \cap B$ y $Q_{1}, \dots, Q_{s}$ las trayectorias en $\mathcal{P}$ que
    intersectan a $A \cap B$. Entonces $l + s = |\mathcal{P}| \ge t$.

    Definimos a $C$ como el conjunto de v\'ertices en $A \cap B$ que intersectan
    alg\'un $Q_i$, con $i \in \{1, \dots, s\}$. Observamos que, al ser $Q_1,
    \dots, Q_s$ internamente ajenos, entonces cada v\'ertice de $C$ esta en
    exactamente un $Q_i$, con $i \in \{1, \dots, s\}$. 

    Definimos $D$ como el conjunto de v\'ertices en $A \cap B$ que no intersecta
    a ning\'un $Q_i$, con $i \in \{1, \dots, l\}$. Observamos que $C$ y $D$
    dividen a $A \cap B$. Entonces, tenemos que $|A\cap B| = |C| + |D| = k-1$.
    Adem\'as podemos ver que $s \leq |C| \leq k-1$ y como $ t - s \leq l$,
    entonces $l \geq t -|C| = t- (k-1-|D|)$.

    Podemos separar las $AB$-trayectorias construidas en $F_{k}(G)$ en tres
    tipos. Los primeros dos tipos son las trayectorias obtenidas del Lema 1.1.2,
    es decir, las trayectorias $A \xrightarrow[P_1]{}  B, \dots, A
    \xrightarrow[P_s]{}  B$ y $A \xrightarrow[Q_1]{}  B, \dots, A
    \xrightarrow[Q_l]{}  B$. Nombramos a estos tipos de $AB$-trayectorias
    trayectorias de tipo $P$ y de tipo $Q$ respectivamente. Notamos que, por
    como se defini\'o, toda $AB$-trayectoria de tipo $P$ no pasa por $A\cap B$,
    por lo que cada trayectoria de este tipo corresponde a la sequencia de
    fichas obtenidas al mover la ficha de $a$ a trav\'es de $P_i$ hacia $b$, con
    $i \in \{1, \dots, s\}$.

\end{proof}

\begin{teorema}%
\label{teo:segundo}
    Sea $G$ una gr\'afica $t$-conexa. Entonces $F_{k}(G)$ es $t$-conexa para
    todo $k>1$
\end{teorema}
        
\begin{proof}
Primero, observamos que $A$ y $B$ son v\'ertices adyacentes en $F_k(G)$ si y
s\'olo si $V(G) \setminus A$ y $V(G)\setminus B$ son adyacentes en $F_{n-k}(G)$.
Entonces tenemos que $F_k(G) \simeq F_{n-k}(G)$. Por lo tanto asumimos sin
p\'erdida de generalidad que $k \leq \frac{n}{2}$.

Sea $\mathcal{C}$ el m\'i{}nimo corte por v\'ertices de $F_k(G)$. Basta
demostrar que $|\mathcal{C}| \geq t$. Definimos $A$ y $B$ como los v\'ertices en
distintas componentes de $F_k(G)- \mathcal{C}$ tales que $|A \triangle B|$ es
m\'i{}nimo.

Si $|A \triangle B| = 2$, entonces por el Lema 1.1.4 tenemos que hay $t$
$AB$-trayectorias internamente ajenas en $F_k(G)$. Por lo que tenemos que
$|\mathcal{C}| \geq t$.

Ahora consideramos $|A \triangle B| = 2r \geq 4$. Definimos $A \setminus B
=\{a_1, \dots, a_r\}$ y $B \setminus A =\{b_1, \dots, b_r\}$. Tambi\'en
definimos $A_{i,x} = A\setminus \{a_i\} \cup \{x\}$ y $B_{j,x} = B\setminus
\{b_j\} \cup \{x\}$ para cada $i \in \{1, \dots, r\}$ y $x \in V(G)\setminus
(A\cup B)$. Supongamos que para algunos $i, j$ y $x$ tenemos que $A_{i,x} \notin
\mathcal{C}$ y $B_{j,x} \notin \mathcal{C}$. Como $A \triangle A_{i,x} = \{a_i,
x\}$, entonces tenemos que $|A \triangle A_{i,x}|< |A \triangle B|$ por lo que
$A$ y $A_{i,x}$ est\'an en la misma componente de $F_k(G)- \mathcal{C}$.
An\'alogamente tenemos que  $B$ y $B_{j,x}$ est\'an en la misma componente de
$F_k(G)-\mathcal{C}$

Luego nos fijamos en que $A_{i,x} \triangle B_{j,x} = (A \triangle B) \setminus
\{a_i, b_j\}$. Por lo que $|A_{i,x} \triangle B_{j,x}| = 2(r-1)$. As\'i{}
tenemos que $A_{i,x}$ y $B_{j,x}$ est\'an en la misma componente de $F_k(G)-
\mathcal{C}$. Pero tenemos que $A$ y$A_{i,x}$ est\'an en la misma componente, de
$F_K(G) - \mathcal{C}$, de igual manera que $B$ y $B_{j,x}$. Por lo tanto $A$ y
$B$ est\'an en la misma componente de $F_k(G)- \mathcal{C}$. Esto es una
cotradicci\'on pues tomamos a $A$ y $B$ en distintas componentes, lo que implica
que $A_{i,x}$ o $B_{j,x}$ est\'a en $\mathcal{C}$, para todas las $i,j$ y $x$.

Por lo anterior tenemos que, para cada $x \in V(G)\setminus (A \cup B)$,
$\mathcal{C}$ tiene todos los $\{A_i,x | i \in \{1, \dots, r\}\}$ o todos los
$\{B_j,x | j \in \{1, \dots, r\}\}$. Dado que $|A\cup B|=k +r$, sabemos que
$\mathcal{C}$ tiene al menos $r(n-k-r)$ v\'ertices.

Ahora definimos $A_{i,j} = (A\setminus \{a_i\}) \cup \{b_j\}$ y $B_{i,j} =
(B\setminus \{b_i\}) \cup \{a_j\}$, para toda $ i, j \in \{1, \dots, r\}$.
Notamos que hay $2r^2$ de estos conjuntos. Ahora supongamos que $A_{i,j} \notin
\mathcal{C}$, para algunos $ i, j \in \{1, \dots, r\}$. Entonces tenemos que $A
\triangle A_{i,j} = \{a_i, b_j\}$. Entonces $A$ y $A_{i,j}$ est\'an en la misma
componente de $F_k(G)- \mathcal{C}$. Por otro lado, $|B \triangle A_{i,j}| = 2
(r-1)$, por lo que $B$ y $A_{i,j}$ est\'an en la misma componente de $F_k(G) -
\mathcal{C}$. Por lo tanto tenemos que $A$ y $B$ est\'an en la misma componente
de $F_k(G)-\mathcal{C}$, lo cu\'al nos lleva a la misma contradicci\'on que el
caso anterior. Por lo tanto tenemos que $A_{i,j} \in \mathcal{C}$ para toda $i,
j \in \{1, \dots, r\}$. De manera an\'aloga $B_{i,j} \in \mathcal{C}$. Por lo
tanto tenemos que $\mathcal{C}$ tiene $2r^2$ v\'ertices, adem\'as de los del
caso anterior.

As\'i, tenemos que $|\mathcal{C}|\geq r(n-k-r)+2r^2 = r(n-k) + r^2$, pero $r \geq
2$ y $k \leq \frac{n}{2}$, por lo que tenemos que $|\mathcal{C}| \geq r(n-k)+r^2
> r(n-k) \geq 2(n-k) \geq n >t$. Por lo tanto $|\mathcal{C}|>t$.
\end{proof} 
