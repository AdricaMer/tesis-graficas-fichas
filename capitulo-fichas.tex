\chapter{Gr\'aficas de fichas}%
\label{cap:fichass}

\section{Introducci\'on a las gr\'aficas de fichas}%
\label{sec:intro-fichas}


Durante los a\~{n}os 90, se empiezan a estudiar las gr\'aficas de $2$-fichas, en
ese entonces referidas como ``Dobule Vertex Graphs''. Se estudian varias de sus
propiedades, como conectividad, planaridad, regularidad y hamiltonicidad
\cite{alaviPlanarity, alaviDVGraphs, alaviHamilt, zhuConnect}. Al mismo tiempo,
en 1992, se comienzan a estudiar las gr\'aficas de $k$-fichas, para una $k \in
\mathbb{N^{+}}$ m\'as general, refiri\'endose a ellas como ``$k$-tuplex Vertex
Graphs'' \cite{zhuNTuples}. Posteriormente, en 2002, Rudolph estudia las
gr\'aficas de $2$-fichas. En el art\'iculo, presenta un ejemplo de dos
gr\'aficas coespectrales, cuyas gr\'aficas de $2$-fichas no lo son
\cite{rudolphGInv}. Recordemos que la Teor\'ia de Gr\'aficas Espectrales estudia
las gr\'aficas en relaci\'on a su polinomio caracter\'istico, valores y vectores
propios de las matrices asociadas a las gr\'aficas. Dos gr\'aficas son
coespectrales si sus matrices de adyacencia tienen multiconjuntos de valores
propios iguales. En el 2007, se estudian nuevamente las gr\'aficas de fichas en
\cite{audeanetSymPower}, donde se refieren a ellas como las ``potencias
sim\'etricas'' de una gr\'afica. En el art\'iculo, se demuestra que las
gr\'aficas de $2$-fichas de las gr\'aficas fuertemente regulares, con los mismos
par\'ametros, son coespectrales. M\'as adelante, de manera independiente, Barghi
y Ponomarenko \cite{barghi-ponomarenko} y Alzaga, Iglesias y Pignol
\cite{alzagaSymPower} prueban que, para alg\'un entero positivo dado, existen
una cantidad infinita de pares de gr\'aficas no isomorfas con gr\'aficas de
$k$-fichas coespectrales. En el 2012 se reintroduce el concepto de gr\'aficas de
fichas \cite{fabilaToken}, asign\'andole este nombre al pensarlo como
configuraciones de $k$ fichas indistinguibles sobre los v\'ertices de una
gr\'afica $G$, a lo m\'as una ficha por v\'ertice. Una ficha se puede ``mover''
de un v\'ertice de $G$ a otro siempre y cuando exista una arista entre ellos y
no haya una ficha en el segundo v\'ertice. Cada configuraci\'on de $k$ fichas es
un v\'ertice en nuestra nueva gr\'afica, donde dos v\'ertices son adyacentes
siempre que se pueda llegar de una configuraci\'on a otra al mover una ficha a
trav\'es de una arista. A esta nueva gr\'afica la nombramos la gr\'afica de
$k$-fichas de $G$ y la denotamos $F_k(G)$. M\'as adelante, tomando como base
este \'ultimo art\'iculo, se han seguido estudiando propiedades de las
gr\'aficas de fichas como conexidad, planaridad, hamiltonicidad, entre otras
\cite{carballosaRegPlan, leaConnect, riveraHamilt, adameHamilt, leaEConnect}.

Formalmente hablando, dada una gr\'afica $G$ y un entero positivo $k$, la
\textbf{gr\'afica de} \indiceSub{gr\'afica}{$k$-fichas} \textbf{de
$G$}\index{fichas!gr\'afica} es la gr\'afica cuyo conjunto de v\'ertices es
$\binom{V(G)}{k}$ y donde dos v\'ertices $A$ y $B$ son adyacentes si y s\'olo si
$|A \triangle B| = \{a,b\}$, con $a \in A$, $b \in B$ y $ab \in E(G)$.
Observamos que $F_1(G)$ es isomorfa a $G$, pues cada v\'ertice de $F_1(G)$ es el
v\'ertice de $G$ donde se encuentra la ficha. Por lo mismo, $G$ y $F_1(G)$
tambi\'en tienen las mismas adyacencias y no adyacencias. Por lo tanto, en
general, nos enfocamos en $F_k(G)$, para $k > 1$. En \cref{fig:ex-tok-graph} se
muestra un ejemplo de una gr\'afica $G$ y su gr\'afica de $2$-fichas. Dentro de
cada v\'ertice de la gr\'afica de $2$-fichas se muestra una copia a escala de
$G$, resaltando en ${\color{rosa}\bf rosa}$ los v\'ertices de $G$ donde se
encuentran las fichas. \'Esto para facilitar el entendimiento de la relacion
entre $G$ y sus gr\'aficas de fichas.

\begin{figure}[ht!]
    \centering
       \begin{tikzpicture}
    
        \begin{scope}[xshift=-7.2cm,yshift=-2.6cm,scale=0.8]
            \foreach \i in {1,...,5}
                \draw ({(360/5)*\i}:2) node(\i)[vertex, label=(360/5)*\i:{${\i}$}]{};
            \foreach \i/\j in {1/2,1/3,1/4,1/5,2/3,3/4,4/5}
                \draw [edge,grisOscuro] (\i) to (\j);
        \end{scope}
        
        \begin{scope}[xshift=0cm,yshift=0cm,scale=2]
            \draw (0,-2.6) node (1) [Bvertex, label=270:{\large $34$}] {};
            \draw (0,0.8) node (2) [Bvertex, label=90:{\large $25$}] {};
            \draw (1,-1.6) node (3) [Bvertex, label=290:{\large $35$}] {};
            \draw (-1,-1.6) node (4) [Bvertex, label=240:{\large $24$}] {};
            \draw (2.3,0) node (5) [Bvertex, label=0:{\large $45$}] {}; 
            \draw (-2.3,0) node (6) [Bvertex, label=180:{\large $23$}] {};
            \draw (0.9,-0.1) node (7) [Bvertex, label=270:{\large $14$}] {};
            \draw (-0.9,-0.1) node (8) [Bvertex, label=270:{\large $13$}] {};
            \draw (1.6,1) node (9) [Bvertex, label=45:{\large $15$}] {};
            \draw (-1.6,1) node (10) [Bvertex, label=135:{\large $12$}] {};
           
            \foreach \i/\j in{1/4,1/3,1/7,1/8,2/3,2/4,2/9,2/10,3/5,3/8,3/9,4/6,
            4/7,4/10,5/7,5/9,6/8,6/10,7/8,7/9,8/10} 
            \draw[edge,grisOscuro] (\i) to (\j);
       \end{scope} 
       
       %node (1), $34$
       \begin{scope}[xshift=0cm,yshift=-5.2cm,scale=0.3]
        \foreach \i in {1,2,5}
            \draw ({(360/5)*\i}:2) node(\i)[svertex]{};
        \foreach \i in {3,4}
            \draw ({(360/5)*\i}:2) node(\i)[svertex,fill=rosa]{};
        \foreach \i/\j in {1/2,1/3,1/4,1/5,2/3,3/4,4/5}
            \draw [edge,grisOscuro] (\i) to (\j);
        \end{scope}

        %node (4), $24$
        \begin{scope}[xshift=-2cm,yshift=-3.2cm,scale=0.3]
            \foreach \i in {1,3,5}
                \draw ({(360/5)*\i}:2) node(\i)[svertex]{};
            \foreach \i in {2,4}
                \draw ({(360/5)*\i}:2) node(\i)[svertex,fill=rosa]{};
            \foreach \i/\j in {1/2,1/3,1/4,1/5,2/3,3/4,4/5}
                \draw [edge,grisOscuro] (\i) to (\j);
        \end{scope}

        %node (3), $35$
        \begin{scope}[xshift=2cm,yshift=-3.2cm,scale=0.3]
            \foreach \i in {1,2,4}
                \draw ({(360/5)*\i}:2) node(\i)[svertex]{};
            \foreach \i in {3,5}
               \draw ({(360/5)*\i}:2) node(\i)[svertex,fill=rosa]{};
            \foreach \i/\j in {1/2,1/3,1/4,1/5,2/3,3/4,4/5}
                \draw [edge,grisOscuro] (\i) to (\j);
        \end{scope}

        %node (6), $23$
        \begin{scope}[xshift=-4.6cm,yshift=0cm,scale=0.3]
            \foreach \i in {1,4,5}
                \draw ({(360/5)*\i}:2) node(\i)[svertex]{};
            \foreach \i in {2,3}
                \draw ({(360/5)*\i}:2) node(\i)[svertex,fill=rosa]{};
            \foreach \i/\j in {1/2,1/3,1/4,1/5,2/3,3/4,4/5}
                \draw [edge,grisOscuro] (\i) to (\j);
        \end{scope}
        
        %node (5), $45$
        \begin{scope}[xshift=4.6cm,yshift=0cm,scale=0.3]
            \foreach \i in {1,2,3}
                \draw ({(360/5)*\i}:2) node(\i)[svertex]{};
            \foreach \i in {4,5}
                \draw ({(360/5)*\i}:2) node(\i)[svertex,fill=rosa]{};
            \foreach \i/\j in {1/2,1/3,1/4,1/5,2/3,3/4,4/5}
                \draw [edge,grisOscuro] (\i) to (\j);
        \end{scope}

        %node (8), $13$
        \begin{scope}[xshift=-1.8cm,yshift=-0.2cm,scale=0.3]
            \foreach \i in {2,4,5}
                \draw ({(360/5)*\i}:2) node(\i)[svertex]{};
            \foreach \i in {1,3}
                \draw ({(360/5)*\i}:2) node(\i)[svertex,fill=rosa]{};
            \foreach \i/\j in {1/2,1/3,1/4,1/5,2/3,3/4,4/5}
                \draw [edge,grisOscuro] (\i) to (\j);
        \end{scope}

        %node (7), $14$
        \begin{scope}[xshift=1.8cm,yshift=-0.2cm,scale=0.3]
            \foreach \i in {2,3,5}
                \draw ({(360/5)*\i}:2) node(\i)[svertex]{};
            \foreach \i in {1,4}
                \draw ({(360/5)*\i}:2) node(\i)[svertex,fill=rosa]{};
            \foreach \i/\j in {1/2,1/3,1/4,1/5,2/3,3/4,4/5}
                \draw [edge,grisOscuro] (\i) to (\j);
        \end{scope}

        %node (2), $25$
        \begin{scope}[xshift=0cm,yshift=1.6cm,scale=0.3]
            \foreach \i in {1,3,4}
                \draw ({(360/5)*\i}:2) node(\i)[svertex]{};
            \foreach \i in {2,5}
                \draw ({(360/5)*\i}:2) node(\i)[svertex,fill=rosa]{};
            \foreach \i/\j in {1/2,1/3,1/4,1/5,2/3,3/4,4/5}
                \draw [edge,grisOscuro] (\i) to (\j);
        \end{scope}

        %node (10), $12$
        \begin{scope}[xshift=-3.2cm,yshift=2cm,scale=0.3]
            \foreach \i in {3,4,5}
                \draw ({(360/5)*\i}:2) node(\i)[svertex]{};
            \foreach \i in {1,2}
                \draw ({(360/5)*\i}:2) node(\i)[svertex,fill=rosa]{};
            \foreach \i/\j in {1/2,1/3,1/4,1/5,2/3,3/4,4/5}
                \draw [edge,grisOscuro] (\i) to (\j);
        \end{scope}

        %node (9), $15$
        \begin{scope}[xshift=3.2cm,yshift=2cm,scale=0.3]
            \foreach \i in {2,3,4}
                \draw ({(360/5)*\i}:2) node(\i)[svertex]{};
            \foreach \i in {1,5}
                \draw ({(360/5)*\i}:2) node(\i)[svertex,fill=rosa]{};
            \foreach \i/\j in {1/2,1/3,1/4,1/5,2/3,3/4,4/5}
                \draw [edge,grisOscuro] (\i) to (\j);
        \end{scope}
    
    \end{tikzpicture}
    \caption{Una gr\'afica $G$ (izquierda) y $F_2(G)$ (derecha) donde, dentro de
    cada v\'ertice, se muestra como es la configuraci\'on de las fichas en $G$
    para dicho v\'ertice.}
    \label{fig:ex-tok-graph}
\end{figure}

Como muchas veces en matem\'aticas, hay ciertas gr\'aficas de fichas que han
sido estudiadas con otro enfoque y bajo otros nombres. Hay un caso que vale la
pena resaltar, el de las gr\'aficas de Johnson. Para un entero $1 \leq k \leq
n$, una \textbf{gr\'afica de} \indiceSub{gr\'afica}{Johnson}, denotada $J(n,k)$,
es la gr\'afica que tiene como conjunto de v\'ertices a todos los
$k$-subconjuntos de $\{1, \dots, n\}$ y donde dos v\'ertices $A$ y $B$ son
adyacentes si y s\'olo si $|A \cap B| = k-1$. Nos fijamos que, para toda $1 \leq
k \leq n$, la gr\'afica de Johnson $J(n,k)$ es isomorfa a la gr\'afica de
$k$-fichas de $K_n$. Las gr\'aficas de Johnson son estudiadas tambi\'en desde la
Teor\'ia de C\'odigos, por lo que hay m\'as ejemplos de este tipo de gr\'aficas
de fichas. De esta manera, las gr\'aficas de Johnson pueden servir como ejemplos
a trav\'es de este trabajo. M\'as a\'un, algunas propiedades conocidas de las
gr\'aficas de Johnson son casos particulares de algunos resultados de este
trabajo.

\section{Algunas estructuras en gr\'aficas de fichas}%
\label{sec:}

Las fichas en una gr\'afica se pueden mover \'unicamente por aristas existentes,
por lo que es natural pensar que una trayectoria en la gr\'afica original tiene
alguna trayectoria en la gr\'afica de fichas con la misma longitud. Teniendo eso
en mente, definimos el siguiente concepto. Sean $P$ una $ab$-trayectoria en la
gr\'afica $G$ y $A$ un $k$-conjunto en $V(G)$, en otras palabras $A \in
V(F_k(G))$. Si al conjunto $A$ le pedimos que $a\in A$ y $b \notin A$, entonces
a la pareja $(A,P)$ le podemos asignar una trayectoria en la gr\'afica de
$k$-fichas, tal que el v\'ertice final est\'a dado por $A'=(A \setminus \{a\})
\cup \{b\}$. Para definir esta nueva trayectoria tomamos $A\cap P =\{v_1, \dots,
v_q\}$, con $v_1 = a$, y ``movemos'' las fichas de la siguiente manera. Primero,
movemos la ficha de $v_q \in V(G)$ hacia el v\'ertice $b \in V(G)$ por $P$.
Luego, para los v\'ertices $v_i$ en $G$, con $i \in \{q-1, q-2, \dots 1\}$,
movemos la ficha de $v_i$ a $v_{i+1}$. As\'i, sucesivamente, vamos moviendo
fichas a trav\'es de $P$ por v\'ertices que est\'an libres de fichas. Observamos
que, al recorrer las fichas a trav\'es de $P$, obtenemos una trayectoria de la
misma longitud que $P$. Esta trayectoria en $F_k(G)$ la denotamos
\indiceSub{fichas}{$A \xrightarrow[P]{} A'$}. A continuaci\'on, se encuentra un
ejemplo de este tipo de trayectoria en una gr\'afica de $3$-fichas, mostrada en
\cref{fig:ex-tok-Path}. Tomamos a \cref{fig:ex-tok-aux} como figura auxiliar
para entender mejor el movimiento de las fichas a trav\'es de la trayectoria. La
gr\'afica $G$ se muestra del lado izquierdo en ambas figuras, adem\'as, en
\cref{fig:ex-tok-Path} se muestra $F_3(G)$ del lado derecho. Definamos
$A=\{2,3,4\}$ en $F_3(G)$ y nuestra trayectoria en $G$ como ${\color{vino}\bf P=
(3,1,4,5)}$, resaltada en ambas figuras. De ah\'i se sigue que $A'=\{2,4,5\}$.
Adem\'as, del lado derecho de \cref{fig:ex-tok-aux} se muestra el movimiento de
las fichas que empiezan en $A \cap P =\{3,4\}$, considerando de arriba a abajo
como se van moviendo las fichas.

\begin{figure}[ht!]
    \centering
       \begin{tikzpicture}
    
        \begin{scope}[xshift=-8.5cm,scale=0.8]
            \foreach \i in {1,...,5}
                \draw ({(360/5)*\i}:2) node(\i)[vertex, label=(360/5)*\i:{${\i}$}]{};
            
            \foreach \i/\j in {1/2,1/5,2/3,3/4}
                \draw [edge,grisOscuro!70] (\i) to (\j);
            \foreach \i/\j in {1/3,1/4,4/5}
                \draw [wedge,vino] (\i) to (\j);
        \end{scope}

        \begin{scope}[yshift=54]
            \draw (-1,0) node (1) [vertex, label=90:{$1$}] {};
            \draw (-3,0) node (3) [vertex, fill=naranja, label=90:{$3$}] {};
            \draw (1,0) node (4) [vertex, fill=azulMetal, label=90:{$4$}] {};
            \draw (3,0) node (5) [vertex, label=90:{$5$}] {};

            \foreach \i/\j in {1/3,1/4,4/5}
                \draw [edge,vino] (\i) to (\j);
        \end{scope}

        \begin{scope}[yshift=18]
            \draw (-1,0) node (1) [vertex, label=90:{$1$}] {};
            \draw (-3,0) node (3) [vertex, fill=naranja, label=90:{$3$}] {};
            \draw (1,0) node (4) [vertex, label=90:{$4$}] {};
            \draw (3,0) node (5) [vertex, fill=azulMetal, label=90:{$5$}] {};

            \foreach \i/\j in {1/3,1/4,4/5}
                \draw [edge,vino] (\i) to (\j);
        \end{scope}

        \begin{scope}[yshift=-18]
            \draw (-1,0) node (1) [vertex, fill=naranja, label=90:{$1$}] {};
            \draw (-3,0) node (3) [vertex, label=90:{$3$}] {};
            \draw (1,0) node (4) [vertex, label=90:{$4$}] {};
            \draw (3,0) node (5) [vertex, fill= azulMetal, label=90:{$5$}] {};

            \foreach \i/\j in {1/3,1/4,4/5}
                \draw [edge,vino] (\i) to (\j);
        \end{scope}

        \begin{scope}[yshift=-54]
            \draw (-1,0) node (1) [vertex, label=90:{$1$}] {};
            \draw (-3,0) node (3) [vertex, label=90:{$3$}] {};
            \draw (1,0) node (4) [vertex, fill=naranja, label=90:{$4$}] {};
            \draw (3,0) node (5) [vertex, fill=azulMetal, label=90:{$5$}] {};

            \foreach \i/\j in {1/3,1/4,4/5}
                \draw [edge,vino] (\i) to (\j);
        \end{scope}
        
    \end{tikzpicture}
    \caption{Una gr\'afica $G$ con una trayectoria $P$ resaltada (izquierda) y
     el movimiento de dos fichas a trav\'es de $P$ (derecho).}
    \label{fig:ex-tok-aux}
\end{figure}
    
M\'as adelante, en \cref{fig:ex-tok-Path}, se muestra la trayectoria
${\color{vino}\bf \{2,3,4\}\xrightarrow[P]{}\{2,4,5\}}$ en $F_3(G)$. En cada
v\'ertice de esta trayectoria se resaltan los n\'umeros del color de la ficha
que se usa en \cref{fig:ex-tok-aux}, para resaltar la relaci\'on entre la
trayectoria de $G$ y la de $F_3(G)$. Observamos que $2$ no est\'a en $A \cap P
=\{3,4\}$, por lo que no se ve en \cref{fig:ex-tok-aux}. Esto significa que la
ficha en el v\'ertice $2$ se queda fija a trav\'es de la trayectoria en la
gr\'afica de fichas. Al estar fija, todos los v\'ertices de la nueva trayectoria
en $F_3(G)$ contienen a $2$. Ese hecho se puede ver en \cref{fig:ex-tok-Path} al
fijarnos que el $2$ est\' en todos los v\'ertices de la trayectoria, resaltado
en gris.

\begin{figure}[ht!]
    \centering
       \begin{tikzpicture}
    
        \begin{scope}[xshift=-8.5cm,scale=0.8]
            \foreach \i in {1,...,5}
                \draw ({(360/5)*\i}:2) node(\i)[vertex, label=(360/5)*\i:{\normalsize ${\i}$}]{};
            
            \foreach \i/\j in {1/2,1/5,2/3,3/4}
                \draw [edge,grisOscuro!75] (\i) to (\j);
            \foreach \i/\j in {1/3,1/4,4/5}
                \draw [wedge,vino] (\i) to (\j);
            \end{scope}
        
        \begin{scope}[xshift=-2cm]
            \draw (0,1.5) node (1) [vertex, label=90:{\footnotesize $134$}] {};
            \draw (-2,2) node (2) [vertex, label=90:{${\bf {\color{grisOscuro} 2}{\color{naranja} 3}{\color{azulMetal} 4}}$}] {};
            \draw (2,2) node (3) [vertex, label=90:{\footnotesize $345$}] {};
            \draw (-3,1) node (4) [vertex, label=180:{\footnotesize $123$}] {};
            \draw (3,1) node (5) [vertex, label=0:{\footnotesize $145$}] {}; 
            \draw (-1.1,0.7) node (6) [vertex, label=87:{${\bf {\color{grisOscuro} 2}{\color{naranja} 3}{\color{azulMetal} 5}}$}] {};
            \draw (1.1,0.7) node (7) [vertex, label=93:{${\bf {\color{grisOscuro} 2}{\color{naranja} 4}{\color{azulMetal} 5}}$}] {};
            \draw (-1.1,-1) node (8) [vertex, label=240:{\footnotesize $124$}] {};
            \draw (1.1,-1) node (9) [vertex, label=330:{\footnotesize $135$}] {};
            \draw (0,-2.3) node (10) [vertex, label=270:{${\bf {\color{naranja} 1}{\color{grisOscuro} 2}{\color{azulMetal} 5}}$}] {};
           
            \foreach \i/\j in{1/2,1/3,1/8,1/9,2/4,2/8,3/5,3/7,3/9,4/6,4/8,5/7,
            5/9,6/7,6/9,7/8,8/10,9/10} 
                \draw [edge,grisOscuro!75] (\i)to (\j); 
            \foreach \i/\j in { 2/6,6/10,7/10} 
                \draw [wedge,vino] (\i) to (\j);
       \end{scope}       
    
    \end{tikzpicture}
    \caption{Una gr\'afica $G$, resaltando una trayectoria $P$(derecha) y 
    su gr\'afica de $3$-fichas (izquierda), resaltando la trayectoria 
    generada por $P$.}
    \label{fig:ex-tok-Path}
\end{figure}


Del ejemplo anterior, podemos concluir que tener una ficha fija resulta
interesante al considerar una trayectoria en la gr\'afica de fichas. Vale la
pena analizar c\'omo se comportan otras subgr\'aficas al dejar fija una ficha.
Generalizando un poco m\'as, nos podemos preguntar, qu\'e gr\'afica de
$3$-fichas se obtendr\'ia si una ficha se queda fija, digamos en $2$. Es f\'acil
notar que esta gr\'afica es una subgr\'afica de la gr\'afica de $3$-fichas de
$G$, pues el conjunto de v\'ertices de esta gr\'afica es el conjunto de
v\'ertices de $F_3(G)$ que contengan a $2$. Este ejemplo se representa en
\cref{fig:ex-tok-subgraph}. La idea de fijar una ficha se puede extender a
$r\leq k$ fichas, aunque el caso interesante ser\'a para $r<k$, pues, si fijamos
todas las fichas, obtenemos una gr\'afica trivial. Dado un conjunto $X \subseteq
V(G)$ con $|X|=r<k$, definimos a $F_k(G,X)$ como la \textbf{subgr\'afica de
$F_k(G)$ inducida}\index{subgr\'afica!inducida de
fichas}\index{fichas!subgr\'afica inducida} por los v\'ertices de $F_k(G)$ que
contienen al subconjunto $X$. 

\begin{figure}[ht!]
    \centering
       \begin{tikzpicture}
    
        \begin{scope}[xshift=-8.5cm,scale=0.7]
            \foreach \i in {1,3,4,5}
                \draw ({(360/5)*\i}:2) node(\i)[vertex, label=(360/5)*\i:{\small ${\i}$}]{};
            \draw ({(360/5)*2}:2) node(2)[vertex, fill=rosa, label=(360/5)*2:{\small ${2}$}]{};
            
            \foreach \i/\j in {1/2,1/3,1/4,1/5,2/3,3/4,4/5}
                \draw [edge,grisOscuro] (\i) to (\j);
            \end{scope}

        \begin{scope}[xshift=-2cm, scale=0.8]
            \draw (0,1.5) node (1) [vertex, label=90:{{ {\color{grisOscuro}\footnotesize $134$}}}] {};
            \draw (-2,2) node (2) [vertex, label=90:\small $234$] {};
            \draw (2,2) node (3) [vertex, label=90:{{ {\color{grisOscuro}\footnotesize $345$}}}] {};
            \draw (-3,1) node (4) [vertex, label=180:\small $123$] {};
            \draw (3,1) node (5) [vertex, label=0:{{ {\color{grisOscuro}\footnotesize $145$}}}] {}; 
            \draw (-1.1,0.7) node (6) [vertex, label=87:\small $235$] {};
            \draw (1.1,0.7) node (7) [vertex, label=93:\small $245$] {};
            \draw (-1.1,-1) node (8) [vertex, label=240:\small $124$] {};
            \draw (1.1,-1) node (9) [vertex, label=330:{{ {\color{grisOscuro}\footnotesize $135$}}}] {};
            \draw (0,-2.3) node (10) [vertex, label=270:\small $125$] {};
           
            \foreach \i/\j in{1/2,1/3,1/8,1/9,3/5,3/7,3/9,5/7,
            5/9,9/10} 
                \draw [edge,grisOscuro!75] (\i)to (\j); 
            \foreach \i/\j in {2/4,4/6,4/8,2/8,7/8,8/10,2/6,6/7,6/10,7/10} 
                \draw [wedge,rosa] (\i) to (\j);
       \end{scope}       

    \end{tikzpicture}
    \caption{Una gr\'afica $G$ (izquierda) y su gr\'afica de $3$-fichas 
    (derecha), donde se resalta $F_3(G,\{2\})$.}
    \label{fig:ex-tok-subgraph}
\end{figure}

Al momento de generar $F_3(G,2)$, es f\'acil notar que las fichas se mueven por
$V(G) \setminus \{2\}$, excepto la ficha que est\'a fija. De igual manera, las
aristas por las que se mueven las fichas son las aristas de $G$ que no tienen
extremos en $2$. Vi\'endolo as\'i, podemos interpretar $F_3(G,\{2\})$ como la
gr\'afica de $(3-1)$-fichas de la gr\'afica $G-2$, es decir, $F_2(G-2)$. Esta
gr\'afica est\'a exhibida en \cref{fig:ex-tok-subgraph-aux}. De manera general,
esto quiere decir que existe una relaci\'on entre la subgr\'afica de $F_k(G)$
inducida por $X$, donde $|r|=X \subset V(G)$, y la gr\'afica de $(k-r)$-fichas
de la gr\'afica $G-X$, es decir, $F_{k-r}(G-X)$. Afirmamos que, para $k,r \in
\mathbb{N}$ y $k>r = |X|$, con $X \subseteq V(G)$, la gr\'afica $F_k(G,X)$ es
isomorfa a la gr\'afica $F_{k-r}(G-X)$, con el isomorfismo que a cada $A \in
F_k(G,X)$ le asocia $A \setminus X$ en $F_{k-r}(G-X)$.   Observamos que, las
aristas en $G$ con alg\'un extremo en $X$ no se usan en $F_k(G,X)$, por lo que
la funci\'on s\'i preserva adyacencias y no adyacencias.

\begin{figure}[ht!]
    \centering
       \begin{tikzpicture}
    
       \begin{scope}[xshift=-8.5cm,scale=0.7]
        \foreach \i in {1,3,4,5}
            \draw ({(360/5)*\i}:2) node(\i)[vertex, label=(360/5)*\i:{\small ${\i}$}]{};
        \draw ({(360/5)*2}:2) node(2)[cvertex, label=(360/5)*2:{{ {\color{grisOscuro!75}\small $2$}}}]{};
        
        \foreach \i/\j in {1/3,1/4,1/5,3/4,4/5}
            \draw [edge,grisOscuro] (\i) to (\j);

        \foreach \i/\j in {1/2,2/3}
            \draw [edge,grisOscuro!50] (\i) to (\j);
        \end{scope}

    \begin{scope}[xshift=-2cm]
        \draw (0.5,0.87) node (1) [vertex,label=87:$235$] {};
        \draw ({(360/6)*2}:2) node(2)[vertex, label=(360/5)*2:{\small ${234}$}]{};
        \draw ({(360/6)*3}:2) node(3)[vertex, label=(360/5)*3:{\small ${123}$}]{};
        \draw (-0.5,-0.87) node (4) [vertex,label=240:\small $124$] {};
        \draw ({(360/6)*5}:2) node(5)[vertex, label=(360/5)*5:{\small ${125}$}]{};
        \draw ({(360/6)*6}:2) node(6)[vertex, label=(360/5)*6:{\small ${245}$}]{};
   
    \foreach \i/\j in {1/2,1/3,1/5,1/6,2/3,2/4,3/4,4/5,4/6,5/6}
        \draw [edge,grisOscuro] (\i) to (\j);
    \end{scope}

    \end{tikzpicture}
    \caption{Una gr\'afica $G$ (izquierda), resaltando $G-2$, y la 
    gr\'afica de fichas $F_2(G-2)$ (derecha).}
    \label{fig:ex-tok-subgraph-aux}
\end{figure}

\newpage

Otra cosa interesante del ejemplo de \cref{fig:ex-tok-Path} es que, si nos
fijamos bien, podemos ver que tiene el mismo diagrama que la gr\'afica de
$2$-fichas en \cref{fig:ex-tok-graph}, i.e., son gr\'aficas isomorfas. Al ser
ambas gr\'aficas de fichas de la misma gr\'afica, vale la pena preguntarse,
cu\'al es la relaci\'on entre ellas. Fij\'andonos a\'un m\'as, vemos que el
isomorfismo entre ambas env\'ia el v\'ertice $A \in V(F_2(G))$ al v\'ertice
$V(G) \setminus A \in V(F_3(G))$. Adem\'as, notamos que dos v\'ertices de $ A,B
\in V(F_2(G))$ son adyacentes si y s\'olo si $V(G) \setminus A$ y $V(G)
\setminus B$ son adyacentes en $F_3(G)$. As\'i, la diferencia sim\'etrica entre
$A$ y $B$ es igual a la diferencia sim\'etrica entre sus complementos. Podemos
verlo como recorrer la gr\'afica ``al rev\'es''. Generalizando este concepto,
afirmamos que, para toda gr\'afica $G$, $F_k(G)$ y $F_{n-k}(G)$ son gr\'aficas
isomorfas. Empezamos viendo que $|V(F_k(G))| =\binom{n}{k}= \binom{n}{n-k}=
|V(F_{n-k}(G))|$. Tomando la funci\'on antes mencionada, falta ver que preserva
adyacencias y no adyacencias. Volviendo a observar que, para todo $A$ y $B$ en
$V(F_k(G))$ tenemos que $A \triangle B = (V(G)\setminus A) \triangle
(V(G)\setminus B)$, entonces las adyacencias y no adyacencias dependen de $G$,
por lo que la funci\'on cumple ser un isomorfismo. 

El resultado anterior resulta de mucha utilidad al momento de estudiar las
gr\'aficas de fichas, pues nos dice que basta en enfocarnos en las gr\'aficas de
$k$-fichas, para $k \leq \frac{n}{2}$.