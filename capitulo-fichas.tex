\chapter{Gr\'aficas de fichas}%
\label{cap:fichass}

\section{Introducci\'on a las gr\'aficas de fichas}%
\label{sec:intro-fichas}


    Durante los a\~{n}os 90, se empiezan a estudiar las de gr\'aficas de
$2$-fichas, en ese entonces referidas como ``Dobule Vertex Graphs''. Se empiezan
a estudiar varias de sus proiedades como conectividad, planaridad, regularidad y
hamiltonicidad \cite{alaviPlanarity, alaviDVGraphs, alaviHamilt, zhuConnect}. Al
mismo tiempo, en 1992, se comienzan a estudiar las gr\'aficas de $k$-fichas,
para una $k \in \mathbb{N^{+}}$ m\'as general, refiriendose a ellas como
``$k$-tuplex Vertex Graphs'' \cite{zhuNTuples}. Posteriormente, en 2002, se
Rudolph estudia en \cite{rudolphGInv} las gr\'aficas de $2$-fichas (bajo otro
nombre), y presenta un ejemplo de dos gr\'aficas coespectrales donde sus
gr\'aficas de $2$-fichas no lo son. Recordemos que la Teor\'ia de Gr\'aficas
Espectrales estudia las gr\'aficas en relaci\'on a su polinomio
caracter\'istico, valores y vectores propios de las matrices asociadas a las
gr\'aficas. Dos gr\'aficas son coespectrales si las matrices de adyacencia de
las gr\'aficas tienen multiconjuntos iguales de valores propios. En el 2007, se
estudian nuevamente las gr\'aficas de fichas en \cite{audeanetSymPower}, donde
se refieren a ellas como ``potencias sim\'etricas'' de una gr\'afica. En este
art\'iculo, se demuestra que las gr\'aficas de $2$-fichas de las gr\'aficas
fuertemente regulares con los mismos par\'ametros son coespectrales. M\'as
adelante, de manera independiente, Barghi y Ponomarenko
\cite{barghi-ponomarenko} y Alzaga, Iglesias y Pignol \cite{alzagaSymPower}
prueban que para alg\'un entero positivo dado existen una cantidad infinita de
pares de gr\'aficas no isomorfas con gr\'aficas de $k$-fichas coespectrales. En
el 2012 se reintroduce el concepto de gr\'aficas de fichas \cite{fabilaToken},
asign\'andole este nombre al pensarlo como configuraciones de $k$ fichas
indistinguibles sobre los v\'ertices de una gr\'afica $G$, a lo m\'as una ficha
por v\'ertice. Una ficha se puede ``mover'' de un v\'ertice de $G$ a otro
siempre y cuando exista una arista entre ellos y no haya una ficha en el segundo
v\'ertice. Cada configuraci\'on de $k$ fichas ser\'a un v\'ertice en nuestra
nueva gr\'afica donde dos v\'ertices ser\'an adyacentes siempre que se pueda
llegar de una configuraci\'on a otra al mover una ficha a trav\'es de una
arista. A esta nueva gr\'afica la nombramos la gr\'afica de $k$-fichas de $G$ y
la denotamos $F_k(G)$. M\'as adelante, tomando como base este \'ultimo
art\'iculo, se han seguido estudiando propiedades de las gr\'aficas de fichas
como conexidad, planaridad, hamiltonicidad, entre otras cosas
\cite{carballosaRegPlan, leaConnect, riveraHamilt, adameHamilt, leaEConnect}.

Formalmente hablando, dada una gr\'afica $G$ y $k$ un entero positivo, la
gr\'afica cuyo conjunto de v\'ertices es $\binom{V(G)}{k}$ y donde dos
v\'ertices $A$ y $B$ en son adyacentes si y s\'olo si $|A \triangle B| =
\{a,b\}$, con $a \in A$, $b \in B$ y $ab \in E(G)$, es la \textbf{gr\'afica de}
\indiceSub{gr\'afica}{$k$-fichas} \textbf{de $G$}\index{fichas!gr\'afica}.
Observamos que la gr\'afica de $1$-fichas es isomorfa a la gr\'afica original,
pues cada v\'ertice de la gr\'afica de fichas es el v\'ertice de la gr\'afica
original donde se encuentra la ficha. Por lo mismo tambi\'en tiene las mismas
adyacencias y no adyacencias que la gr\'afica original. Por lo tanto, en general
para los ejemplos usaremos gr\'aficas de $k$-fichas, con $k > 1$. En
\cref{fig:ex-tok-graph} se muestra un ejemplo de una gr\'afica $G$ y su
gr\'afica de $2$-fichas. Para ayudar al entendimiento de las gr\'aficas de
fichas, en la gr\'afica del lado derecho de \cref{fig:ex-tok-graph} en cada
v\'ertice de $F_2(G)$ se muestra una copia a escala de $G$, resaltando en
${\color{rosa}\bf rosa}$ los v\'ertices de $G$ donde se encuentran las fichas.

\begin{figure}[ht!]
    \centering
       \begin{tikzpicture}
    
        \begin{scope}[xshift=-9cm,scale=0.9]
            \foreach \i in {1,...,5}
                \draw ({(360/5)*\i}:2) node(\i)[vertex, label=(360/5)*\i:{${\i}$}]{};
            \foreach \i/\j in {1/2,1/3,1/4,1/5,2/3,3/4,4/5}
                \draw [edge,grisOscuro] (\i) to (\j);
        \end{scope}
        
        \begin{scope}[xshift=0cm,yshift=0cm,scale=2]
            \draw (0,1.8) node (1) [Bvertex, label=90:{\large $34$}] {};
            \draw (0,-1.4) node (2) [Bvertex, label=270:{\large $25$}] {};
            \draw (1,1) node (3) [Bvertex, label=45:{\large $35$}] {};
            \draw (-1,1) node (4) [Bvertex, label=135:{\large $24$}] {};
            \draw (2.1,0) node (5) [Bvertex, label=0:{\large $45$}] {}; 
            \draw (-2.1,0) node (6) [Bvertex, label=180:{\large $23$}] {};
            \draw (0.85,-0.6) node (7) [Bvertex, label=270:{\large $14$}] {};
            \draw (-0.85,-0.6) node (8) [Bvertex, label=270:{\large $13$}] {};
            \draw (1.6,-1.8) node (9) [Bvertex, label=290:{\large $15$}] {};
            \draw (-1.6,-1.8) node (10) [Bvertex, label=240:{\large $12$}] {};
           
            \foreach \i/\j in{1/4,1/3,1/7,1/8,2/3,2/4,2/9,2/10,3/5,3/8,3/9,4/6,
            4/7,4/10,5/7,5/9,6/8,6/10,7/8,7/9,8/10} 
            \draw[edge,grisOscuro] (\i) to (\j);
       \end{scope} 
       
       \begin{scope}[xshift=0cm,yshift=3.6cm,scale=0.3]
        \foreach \i in {1,2,5}
            \draw ({(360/5)*\i}:2) node(\i)[svertex]{};
        \foreach \i in {3,4}
            \draw ({(360/5)*\i}:2) node(\i)[svertex,fill=rosa]{};
        \foreach \i/\j in {1/2,1/3,1/4,1/5,2/3,3/4,4/5}
            \draw [edge,grisOscuro] (\i) to (\j);
        \end{scope}

        \begin{scope}[xshift=-2cm,yshift=2cm,scale=0.3]
            \foreach \i in {1,3,5}
                \draw ({(360/5)*\i}:2) node(\i)[svertex]{};
            \foreach \i in {2,4}
                \draw ({(360/5)*\i}:2) node(\i)[svertex,fill=rosa]{};
            \foreach \i/\j in {1/2,1/3,1/4,1/5,2/3,3/4,4/5}
                \draw [edge,grisOscuro] (\i) to (\j);
        \end{scope}

        \begin{scope}[xshift=2cm,yshift=2cm,scale=0.3]
            \foreach \i in {1,2,4}
                \draw ({(360/5)*\i}:2) node(\i)[svertex]{};
            \foreach \i in {3,5}
               \draw ({(360/5)*\i}:2) node(\i)[svertex,fill=rosa]{};
            \foreach \i/\j in {1/2,1/3,1/4,1/5,2/3,3/4,4/5}
                \draw [edge,grisOscuro] (\i) to (\j);
        \end{scope}

        \begin{scope}[xshift=-4.2cm,yshift=0cm,scale=0.3]
            \foreach \i in {1,4,5}
                \draw ({(360/5)*\i}:2) node(\i)[svertex]{};
            \foreach \i in {2,3}
                \draw ({(360/5)*\i}:2) node(\i)[svertex,fill=rosa]{};
            \foreach \i/\j in {1/2,1/3,1/4,1/5,2/3,3/4,4/5}
                \draw [edge,grisOscuro] (\i) to (\j);
        \end{scope}
        
        \begin{scope}[xshift=4.2cm,yshift=0cm,scale=0.3]
            \foreach \i in {1,2,3}
                \draw ({(360/5)*\i}:2) node(\i)[svertex]{};
            \foreach \i in {4,5}
                \draw ({(360/5)*\i}:2) node(\i)[svertex,fill=rosa]{};
            \foreach \i/\j in {1/2,1/3,1/4,1/5,2/3,3/4,4/5}
                \draw [edge,grisOscuro] (\i) to (\j);
        \end{scope}

        \begin{scope}[xshift=-1.7cm,yshift=-1.2cm,scale=0.3]
            \foreach \i in {2,4,5}
                \draw ({(360/5)*\i}:2) node(\i)[svertex]{};
            \foreach \i in {1,3}
                \draw ({(360/5)*\i}:2) node(\i)[svertex,fill=rosa]{};
            \foreach \i/\j in {1/2,1/3,1/4,1/5,2/3,3/4,4/5}
                \draw [edge,grisOscuro] (\i) to (\j);
        \end{scope}

        \begin{scope}[xshift=1.7cm,yshift=-1.2cm,scale=0.3]
            \foreach \i in {2,3,5}
                \draw ({(360/5)*\i}:2) node(\i)[svertex]{};
            \foreach \i in {1,4}
                \draw ({(360/5)*\i}:2) node(\i)[svertex,fill=rosa]{};
            \foreach \i/\j in {1/2,1/3,1/4,1/5,2/3,3/4,4/5}
                \draw [edge,grisOscuro] (\i) to (\j);
        \end{scope}

        \begin{scope}[xshift=0cm,yshift=-2.8cm,scale=0.3]
            \foreach \i in {1,3,4}
                \draw ({(360/5)*\i}:2) node(\i)[svertex]{};
            \foreach \i in {2,5}
                \draw ({(360/5)*\i}:2) node(\i)[svertex,fill=rosa]{};
            \foreach \i/\j in {1/2,1/3,1/4,1/5,2/3,3/4,4/5}
                \draw [edge,grisOscuro] (\i) to (\j);
        \end{scope}

        \begin{scope}[xshift=-3.2cm,yshift=-3.6cm,scale=0.3]
            \foreach \i in {3,4,5}
                \draw ({(360/5)*\i}:2) node(\i)[svertex]{};
            \foreach \i in {1,2}
                \draw ({(360/5)*\i}:2) node(\i)[svertex,fill=rosa]{};
            \foreach \i/\j in {1/2,1/3,1/4,1/5,2/3,3/4,4/5}
                \draw [edge,grisOscuro] (\i) to (\j);
        \end{scope}

        \begin{scope}[xshift=3.2cm,yshift=-3.6cm,scale=0.3]
            \foreach \i in {2,3,4}
                \draw ({(360/5)*\i}:2) node(\i)[svertex]{};
            \foreach \i in {1,5}
                \draw ({(360/5)*\i}:2) node(\i)[svertex,fill=rosa]{};
            \foreach \i/\j in {1/2,1/3,1/4,1/5,2/3,3/4,4/5}
                \draw [edge,grisOscuro] (\i) to (\j);
        \end{scope}
    
    \end{tikzpicture}
    \caption{Una gr\'afica $G$ (izquierda) y su gr\'afica de
    $2$-fichas (derecha) donde, dentro de cada v\'ertice, se muestra como es la
    configuraci\'on de las fichas en $G$ para dicho v\'ertice.}
    \label{fig:ex-tok-graph}
\end{figure}

\section{Algunas estructuras en gr\'aficas de fichas}%
\label{sec:}

Las fichas en una gr\'afica se pueden mover \'unicamente por aristas existentes,
por lo que ser\'ia natural pensar que una trayectoria en la gr\'afica tiene
alguna trayectoria en la gr\'afica de fichas con la misma longitud. Teniendo eso
en mente, definimos el siguiente concepto. Sean $P$ una $ab$-trayectoria en la
gr\'afica $G$ y $A$ un $k$-conjunto en $G$, en otras palabras $A \in V(F_k(G))$.
Si al conjunto $A$ le pedimos que $a\in A$ y $b \notin A$, entonces a la pareja
$(A,P)$ le podemos asignar una trayectoria en la gr\'afica de $k$-fichas, tal
que el v\'ertice final est\'a dado por $A'=(A \setminus \{a\}) \cup \{b\}$. Para
definir esta nueva trayectoria tomamos $A\cap P =\{v_1, \dots, v_q\}$, con $v_1
= a$, y ``movemos las fichas'' de la siguiente manera. Primero movemos la ficha
de $v_q$, v\'ertice de $G$, hacia el v\'ertice $b$, v\'ertice de $G$, por $P$.
Luego, para los v\'ertices $v_i$ en $G$, con $i \in \{q-1, q-2, \dots 1\}$,
movemos la ficha de $v_i$ a $v_{i+1}$. As\'i, susesivamente, vamos moviendo
fichas a trav\'es de $P$ por v\'ertices que estan libres de fichas. Observamos
que, al recorrer las fichas a trav\'es de $P$ obtenemos una trayectoria de la
misma longitud que $P$. Esta trayectoria en $F_k(G)$ la denotamos
\indiceSub{fichas}{$A \xrightarrow[P]{} A'$}. A continuaci\'on daremos un
ejemplo de esta trayectoria en una gr\'afica de $3$-fichas, apoyandonos de
\cref{fig:ex-tok-aux} y \cref{fig:ex-tok-path}. Para este ejemplo usaremos la
misma gr\'afica $G$ que en \cref{fig:ex-tok-graph}. Tanto en
\cref{fig:ex-tok-aux} como \cref{fig:ex-tok-path} la gr\'afica $G$ se muestra
del lado izquierdo y en \cref{fig:ex-tok-path} se muestra $F_3(G)$ del lado
derecho. Primero definamos $A=\{2,3,4\}$ en $F_3(G)$ y nuestra trayectoria en
$G$ como ${\color{vino}\bf P= (3,1,4,5)}$, resaltada en \cref{fig:ex-tok-aux}.
Por lo tanto, tenemos que $A'=\{2,4,5\}$. Del lado derecho de la misma figura se
muestra el movimiento de las fichas que empiezan en $A \cap P =\{3,4\}$,
considerando de arriba a abajo como se van moviendo las fichas.

\begin{figure}[ht!]
    \centering
       \begin{tikzpicture}
    
        \begin{scope}[xshift=-8.5cm,scale=0.8]
            \foreach \i in {1,...,5}
                \draw ({(360/5)*\i}:2) node(\i)[vertex, label=(360/5)*\i:{${\i}$}]{};
            
            \foreach \i/\j in {1/2,1/5,2/3,3/4}
                \draw [edge,grisOscuro!70] (\i) to (\j);
            \foreach \i/\j in {1/3,1/4,4/5}
                \draw [wedge,vino] (\i) to (\j);
        \end{scope}

        \begin{scope}[yshift=54]
            \draw (-1,0) node (1) [vertex, label=90:{$1$}] {};
            \draw (-3,0) node (3) [vertex, fill=naranja, label=90:{$3$}] {};
            \draw (1,0) node (4) [vertex, fill=azulMetal, label=90:{$4$}] {};
            \draw (3,0) node (5) [vertex, label=90:{$5$}] {};

            \foreach \i/\j in {1/3,1/4,4/5}
                \draw [edge,vino] (\i) to (\j);
        \end{scope}

        \begin{scope}[yshift=18]
            \draw (-1,0) node (1) [vertex, label=90:{$1$}] {};
            \draw (-3,0) node (3) [vertex, fill=naranja, label=90:{$3$}] {};
            \draw (1,0) node (4) [vertex, label=90:{$4$}] {};
            \draw (3,0) node (5) [vertex, fill=azulMetal, label=90:{$5$}] {};

            \foreach \i/\j in {1/3,1/4,4/5}
                \draw [edge,vino] (\i) to (\j);
        \end{scope}

        \begin{scope}[yshift=-18]
            \draw (-1,0) node (1) [vertex, fill=naranja, label=90:{$1$}] {};
            \draw (-3,0) node (3) [vertex, label=90:{$3$}] {};
            \draw (1,0) node (4) [vertex, label=90:{$4$}] {};
            \draw (3,0) node (5) [vertex, fill= azulMetal, label=90:{$5$}] {};

            \foreach \i/\j in {1/3,1/4,4/5}
                \draw [edge,vino] (\i) to (\j);
        \end{scope}

        \begin{scope}[yshift=-54]
            \draw (-1,0) node (1) [vertex, label=90:{$1$}] {};
            \draw (-3,0) node (3) [vertex, label=90:{$3$}] {};
            \draw (1,0) node (4) [vertex, fill=naranja, label=90:{$4$}] {};
            \draw (3,0) node (5) [vertex, fill=azulMetal, label=90:{$5$}] {};

            \foreach \i/\j in {1/3,1/4,4/5}
                \draw [edge,vino] (\i) to (\j);
        \end{scope}
        
    \end{tikzpicture}
    \caption{Una gr\'afica $G$ con una trayectoria $P$ resaltada (izquierda) y
     el movimiento de dos fichas a trav\'es de $P$ (derecho).}
    \label{fig:ex-tok-aux}
\end{figure}
    
M\'as adelante, en \cref{fig:ex-tok-path}, se muestra la trayectoria
${\color{vino}\bf \{2,3,4\}\xrightarrow[P]{}\{2,4,5\}}$ en $F_3(G)$. En cada
v\'ertice de esta trayecotira se resaltan los n\'umeros del color de la ficha
que se us\'o en \cref{fig:ex-tok-aux}, \'esto para facilitar el entendimiento de
la relaci\'on entre la trayectoria de $G$ y la de $F_3(G)$. Observamos que $2$
no est\'a en $A \cap P =\{3,4\}$ por lo que no se ve en \cref{fig:ex-tok-aux}.
Esto significa que la ficha en el v\'ertice $2$ se queda fija a trav\'es de la
trayectoria en la gr\'afica de fichas. Esto se puede ver en
\cref{fig:ex-tok-path} al fijarnos que el $2$ en todos los v\'ertices de la
trayectoria, resaltado en gris.

%A lo largo de este trabajo nos ser\'a de utilidad $A'$ por lo que .....


\begin{figure}[ht!]
    \centering
       \begin{tikzpicture}
    
        \begin{scope}[xshift=-8.5cm,scale=0.8]
            \foreach \i in {1,...,5}
                \draw ({(360/5)*\i}:2) node(\i)[vertex, label=(360/5)*\i:{\normalsize ${\i}$}]{};
            
            \foreach \i/\j in {1/2,1/5,2/3,3/4}
                \draw [edge,grisOscuro!75] (\i) to (\j);
            \foreach \i/\j in {1/3,1/4,4/5}
                \draw [wedge,vino] (\i) to (\j);
            \end{scope}
        
        %{\small $12{\bf 6}$}
        %${\color{azulCielo}\bf azul}$
        \begin{scope}[xshift=-2cm]
            \draw (0,1.5) node (1) [vertex, label=90:{\footnotesize $134$}] {};
            \draw (-2,2) node (2) [vertex, label=90:{${\bf {\color{grisOscuro} 2}{\color{naranja} 3}{\color{azulMetal} 4}}$}] {};
            \draw (2,2) node (3) [vertex, label=90:{\footnotesize $345$}] {};
            \draw (-3,1) node (4) [vertex, label=180:{\footnotesize $123$}] {};
            \draw (3,1) node (5) [vertex, label=0:{\footnotesize $145$}] {}; 
            \draw (-1.1,0.7) node (6) [vertex, label=87:{${\bf {\color{grisOscuro} 2}{\color{naranja} 3}{\color{azulMetal} 5}}$}] {};
            \draw (1.1,0.7) node (7) [vertex, label=93:{${\bf {\color{grisOscuro} 2}{\color{naranja} 4}{\color{azulMetal} 5}}$}] {};
            \draw (-1.1,-1) node (8) [vertex, label=240:{\footnotesize $124$}] {};
            \draw (1.1,-1) node (9) [vertex, label=330:{\footnotesize $135$}] {};
            \draw (0,-2.3) node (10) [vertex, label=270:{${\bf {\color{naranja} 1}{\color{grisOscuro} 2}{\color{azulMetal} 5}}$}] {};
           
            \foreach \i/\j in{1/2,1/3,1/8,1/9,2/4,2/8,3/5,3/7,3/9,4/6,4/8,5/7,
            5/9,6/7,6/9,7/8,8/10,9/10} 
                \draw [edge,grisOscuro!75] (\i)to (\j); 
            \foreach \i/\j in { 2/6,6/10,7/10} 
                \draw [wedge,vino] (\i) to (\j);
       \end{scope}       
    
    \end{tikzpicture}
    \caption{Una gr\'afica $G$ resaltando una trayectoria $P$ del lado derecho y 
    su gr\'afica de $3$-fichas del lado izquierdo, resaltando la trayectoria 
    generada por $P$.}
    \label{fig:ex-tok-path}
\end{figure}


Algo que tambi\'en resulta interesante en el ejemplo anterior es la ficha que se
queda fija en el v\'ertice $2$ de la gr\'afica $G$. Al estar fija, todos los
v\'ertices de la nueva trayectoria en $F_3(G)$ contienen a $2$. Generalizando un
poco m\'as, nos preguntamos que gr\'afica de $3$-fichas se obtendr\'ia si una
ficha se queda fija, digamos en $2$. Es f\'acil notar que esta gr\'afica es una
sugr\'afica de la gr\'afica de $3$-fichas de $G$, pues el conjunto de v\'ertices
de esta gr\'afica es el conjunto de v\'ertices de $F_3(G)$ que contengan a $2$.
(A continuaci\'on, este ejemplo se representa en \cref{fig:ex-tok-subgraph}.)
La idea de fijar una ficha se puede extender a $r\leq k$ fichas, aunque el caso
interesante ser\'a para $r<k$ pues fijamos todas las fichas obtenemos una
gr\'afica trivial. Dado un conjunto $X \subseteq V(G)$ con $|X|=r<k$, definimos
a $F_k(G,X)$ como la \textbf{subgr\'afica de $F_k(G)$
inducida}\index{subgr\'afica!inducida de fichas}\index{fichas!subgr\'afica
inducida} por los v\'ertices de $F_k(G)$ que contienen al subconjunto $X$. 

\begin{figure}[ht!]
    \centering
       \begin{tikzpicture}
    
        \begin{scope}[xshift=-8.5cm,scale=0.7]
            \foreach \i in {1,3,4,5}
                \draw ({(360/5)*\i}:2) node(\i)[vertex, label=(360/5)*\i:{\small ${\i}$}]{};
            \draw ({(360/5)*2}:2) node(2)[vertex, fill=rosa, label=(360/5)*2:{\small ${2}$}]{};
            
            \foreach \i/\j in {1/2,1/3,1/4,1/5,2/3,3/4,4/5}
                \draw [edge,grisOscuro] (\i) to (\j);
            \end{scope}
        
        %{\small $12{\bf 6}$}
        %${\color{azulCielo}\bf azul}$
        \begin{scope}[xshift=-2cm, scale=0.8]
            \draw (0,1.5) node (1) [vertex, label=90:{{ {\color{grisOscuro}\footnotesize $134$}}}] {};
            \draw (-2,2) node (2) [vertex, label=90:\small $234$] {};
            \draw (2,2) node (3) [vertex, label=90:{{ {\color{grisOscuro}\footnotesize $345$}}}] {};
            \draw (-3,1) node (4) [vertex, label=180:\small $123$] {};
            \draw (3,1) node (5) [vertex, label=0:{{ {\color{grisOscuro}\footnotesize $145$}}}] {}; 
            \draw (-1.1,0.7) node (6) [vertex, label=87:\small $235$] {};
            \draw (1.1,0.7) node (7) [vertex, label=93:\small $245$] {};
            \draw (-1.1,-1) node (8) [vertex, label=240:\small $124$] {};
            \draw (1.1,-1) node (9) [vertex, label=330:{{ {\color{grisOscuro}\footnotesize $135$}}}] {};
            \draw (0,-2.3) node (10) [vertex, label=270:\small $125$] {};
           
            \foreach \i/\j in{1/2,1/3,1/8,1/9,3/5,3/7,3/9,5/7,
            5/9,9/10} 
                \draw [edge,grisOscuro!75] (\i)to (\j); 
            \foreach \i/\j in {2/4,4/6,4/8,2/8,7/8,8/10,2/6,6/7,6/10,7/10} 
                \draw [wedge,rosa] (\i) to (\j);
       \end{scope}       

    \end{tikzpicture}
    \caption{Una gr\'afica $G$ (izquierda) y su gr\'afica de $3$-fichas 
    (derecha), donde se resalta $F_3(G,\{2\})$}
    \label{fig:ex-tok-subgraph}
\end{figure}

Al momento de generar $F_3(G,2)$ es f\'acil notar que las fichas se mueven
por $V(G) \setminus \{2\}$, excepto la ficha que est\'a fija. De igual manera,
las aristas por las que se mueven las fichas son las aristas de $G$ que no
tienen extremos en $2$. Entonces, podr\'iamos interpretar $F_3(G,\{2\})$ como la
gr\'afica de $(3-1)$-fichas de la gr\'afica $G-2$, es decir $F_2(G-2)$. Esta
gr\'afica est\'a exhibida en \cref{fig:ex-tok-subgraph-aux}. De manera general,
esto quiere decir que existe una relaci\'on entre la subgr\'afica de $F_k(G)$
inducida por $X$, donde $|r|=X \subset V(G)$, y la gr\'afica de $(k-r)$-fichas
de la gr\'afica $G-X$, es decir $F_{k-r}(G-X)$. Afirmamos que, para $k,r \in
\mathbb{N}$ y $k>r = |X|$, con $X \subseteq V(G)$, la gr\'afica $F_k(G,X)$ es
isomorfa a la gr\'afica $F_{k-r}(G-X)$, con el isomorfismo que a cada $A \in
F_k(G,X)$ le asocia $A \setminus X$ en $F_{k-r}(G-X)$.   Observamos que las
aristas en $G$ con alg\'un extremos en $X$ no se usan en $F_k(G,X)$ por lo que
el la funci\'on si preserva adyacencias y no adyacencias..... 

\begin{figure}[ht!]
    \centering
       \begin{tikzpicture}
    
       \begin{scope}[xshift=-8.5cm,scale=0.7]
        \foreach \i in {1,3,4,5}
            \draw ({(360/5)*\i}:2) node(\i)[vertex, label=(360/5)*\i:{\small ${\i}$}]{};
        \draw ({(360/5)*2}:2) node(2)[cvertex, label=(360/5)*2:{{ {\color{grisOscuro!75}\small $2$}}}]{};
        
        \foreach \i/\j in {1/3,1/4,1/5,3/4,4/5}
            \draw [edge,grisOscuro] (\i) to (\j);

        \foreach \i/\j in {1/2,2/3}
            \draw [edge,grisOscuro!50] (\i) to (\j);
        \end{scope}
    
    %{\small $12{\bf 6}$}
    %${\color{azulCielo}\bf azul}$
%     \begin{scope}[xshift=-1.3cm]
%         \draw (-2,2) node (2) [vertex, label=90:$234$] {};
%         \draw (-3,1) node (4) [vertex, label=180:{$123$}] {};
%         \draw (-1.1,0.7) node (6) [vertex, label=87:$235$] {};
%         \draw (1.1,0.7) node (7) [vertex, label=93:$245$] {};
%         \draw (-1.1,-1) node (8) [vertex, label=240:{$124$}] {};
%         \draw (0,-2.3) node (10) [vertex, label=270:$125$] {};
       
%        \foreach \i/\j in {2/4,4/6,4/8,2/8,7/8,8/10,2/6,6/7,6/10,7/10} 
%             \draw [edge,grisOscuro] (\i) to (\j);
%    \end{scope}       

    \begin{scope}[xshift=-2cm]
        \draw (0.5,0.87) node (1) [vertex,label=87:$235$] {};
        \draw ({(360/6)*2}:2) node(2)[vertex, label=(360/5)*2:{\small ${234}$}]{};
        \draw ({(360/6)*3}:2) node(3)[vertex, label=(360/5)*3:{\small ${123}$}]{};
        \draw (-0.5,-0.87) node (4) [vertex,label=240:\small $124$] {};
        \draw ({(360/6)*5}:2) node(5)[vertex, label=(360/5)*5:{\small ${125}$}]{};
        \draw ({(360/6)*6}:2) node(6)[vertex, label=(360/5)*6:{\small ${245}$}]{};
    %\draw ({(360/5)*2}:2) node(2)[vertex, fill=rosa, label=(360/5)*2:{\small ${2}$}]{};
    
    \foreach \i/\j in {1/2,1/3,1/5,1/6,2/3,2/4,3/4,4/5,4/6,5/6}
        \draw [edge,grisOscuro] (\i) to (\j);
    \end{scope}

    \end{tikzpicture}
    \caption{Una gr\'afica $G$ (izquierda), donde se resalta $G-2$ y la 
    gr\'afica de fichas $F_2(G-2)$ (derecha).}
    \label{fig:ex-tok-subgraph-aux}
\end{figure}
