\chapter{Gr\'aficas de fichas}%
\label{cap:fichass}

\section{Introducci\'on a las gr\'aficas de fichas}%
\label{sec:intro-fichas}


Durante los a\~{n}os 90, se empiezan a estudiar las gr\'aficas de $2$-fichas, en
ese entonces referidas como \textit{Double Vertex Graphs}. Se estudian varias de
sus propiedades, como conectividad, planaridad, regularidad y hamiltonicidad
\cite{alaviPlanarity, alaviDVGraphs, alaviHamilt, zhuConnect}. Al mismo tiempo,
en 1992, se comienzan a estudiar las gr\'aficas de $k$-fichas, para una $k \in
\mathbb{N^{+}}$ m\'as general, refiri\'endose a ellas como \textit{$k$-tuplex
Vertex Graphs} \cite{zhuNTuples}. Posteriormente, en 2002, Rudolph estudia
nuevamente las gr\'aficas de $2$-fichas, presentando un ejemplo de dos
gr\'aficas coespectrales\footnote{Recordemos que la Teor\'ia de Gr\'aficas
Espectrales estudia las gr\'aficas en relaci\'on a su polinomio
caracter\'istico, valores y vectores propios de las matrices asociadas a las
gr\'aficas. Dos gr\'aficas son coespectrales si sus matrices de adyacencia
tienen multiconjuntos de valores propios iguales.}, cuyas gr\'aficas de
$2$-fichas no lo son \cite{rudolphGInv}. En el 2007, se estudian nuevamente las
gr\'aficas de fichas en \cite{audeanetSymPower}, donde se refieren a ellas como
las ``potencias sim\'etricas'' de una gr\'afica. En el art\'iculo, se demuestra
que las gr\'aficas de $2$-fichas de las gr\'aficas fuertemente regulares, con
los mismos par\'ametros, son coespectrales. M\'as adelante, de manera
independiente, Barghi y Ponomarenko \cite{barghi-ponomarenko} y Alzaga, Iglesias
y Pignol \cite{alzagaSymPower} prueban que, para alg\'un entero positivo dado,
existen una cantidad infinita de pares de gr\'aficas no isomorfas con gr\'aficas
de $k$-fichas coespectrales. En el 2012 se reintroduce el concepto de gr\'aficas
de fichas \cite{fabilaToken}, asign\'andole este nombre al pensarlo como
configuraciones de $k$ fichas indistinguibles sobre los v\'ertices de una
gr\'afica $G$, a lo m\'as una ficha por v\'ertice. Una ficha se puede ``mover''
de un v\'ertice de $G$ a otro siempre y cuando exista una arista entre ellos y
no haya una ficha en el segundo v\'ertice. Cada configuraci\'on de $k$ fichas es
un v\'ertice en nuestra nueva gr\'afica, donde dos v\'ertices son adyacentes
siempre que se pueda llegar de una configuraci\'on a otra al mover una ficha a
trav\'es de una arista. A esta nueva gr\'afica la nombramos la gr\'afica de
$k$-fichas de $G$ y la denotamos $F_k(G)$. M\'as adelante, tomando como base
este \'ultimo art\'iculo, se han seguido estudiando propiedades de las
gr\'aficas de fichas como conexidad, planaridad, hamiltonicidad, entre otras
\cite{carballosaRegPlan, leaConnect, riveraHamilt, adameHamilt, leaEConnect}.

Formalmente hablando, dada una gr\'afica $G$ y un entero positivo $k$, la
\textbf{gr\'afica de} $k$-\textbf{fichas} \index{gr\'afica!k @$k$ fichas}
\textbf{de $G$}\index{fichas!gr\'afica} es la gr\'afica cuyo conjunto de
v\'ertices es $\binom{V(G)}{k}$ y donde dos v\'ertices $A$ y $B$ son adyacentes
si y s\'olo si $|A \triangle B| = \{a,b\}$, con $a \in A$, $b \in B$ y $ab \in
E(G)$. Observamos que $F_1(G)$ es isomorfa a $G$, pues cada v\'ertice de
$F_1(G)$ es el v\'ertice de $G$ donde se encuentra la ficha. Por lo mismo, $G$ y
$F_1(G)$ tambi\'en tienen las mismas adyacencias y no adyacencias. Por lo tanto,
en general, nos enfocamos en $F_k(G)$, para $k > 1$. En \cref{fig:ex-tok-graph}
se muestra un ejemplo de una gr\'afica $G$ y su gr\'afica de $2$-fichas. Dentro
de cada v\'ertice de la gr\'afica de $2$-fichas se muestra una copia a escala de
$G$, resaltando en ${\color{rosa}\bf rosa}$ los v\'ertices de $G$ donde se
encuentran las fichas, esto para facilitar el entendimiento de la relacion entre
$G$ y sus gr\'aficas de fichas.

\begin{figure}[ht!]
    \centering
       \begin{tikzpicture}
    
        \begin{scope}[xshift=-7.2cm,yshift=-2.6cm,scale=0.8]
            \foreach \i in {1,...,5}
                \draw ({(360/5)*\i}:2) node(\i)[vertex, label=(360/5)*\i:{${\i}$}]{};
            \foreach \i/\j in {1/2,1/3,1/4,1/5,2/3,3/4,4/5}
                \draw [edge,grisOscuro] (\i) to (\j);
        \end{scope}
        
        \begin{scope}[xshift=0cm,yshift=0cm,scale=2]
            \draw (0,-2.6) node (1) [Bvertex, label=270:{\large $34$}] {};
            \draw (0,0.8) node (2) [Bvertex, label=90:{\large $25$}] {};
            \draw (1,-1.6) node (3) [Bvertex, label=290:{\large $35$}] {};
            \draw (-1,-1.6) node (4) [Bvertex, label=240:{\large $24$}] {};
            \draw (2.3,0) node (5) [Bvertex, label=0:{\large $45$}] {}; 
            \draw (-2.3,0) node (6) [Bvertex, label=180:{\large $23$}] {};
            \draw (0.9,-0.1) node (7) [Bvertex, label=270:{\large $14$}] {};
            \draw (-0.9,-0.1) node (8) [Bvertex, label=270:{\large $13$}] {};
            \draw (1.6,1) node (9) [Bvertex, label=45:{\large $15$}] {};
            \draw (-1.6,1) node (10) [Bvertex, label=135:{\large $12$}] {};
           
            \foreach \i/\j in{1/4,1/3,1/7,1/8,2/3,2/4,2/9,2/10,3/5,3/8,3/9,4/6,
            4/7,4/10,5/7,5/9,6/8,6/10,7/8,7/9,8/10} 
            \draw[edge,grisOscuro] (\i) to (\j);
       \end{scope} 
       
       %node (1), $34$
       \begin{scope}[xshift=0cm,yshift=-5.2cm,scale=0.3]
        \foreach \i in {1,2,5}
            \draw ({(360/5)*\i}:2) node(\i)[svertex]{};
        \foreach \i in {3,4}
            \draw ({(360/5)*\i}:2) node(\i)[svertex,fill=rosa]{};
        \foreach \i/\j in {1/2,1/3,1/4,1/5,2/3,3/4,4/5}
            \draw [edge,grisOscuro] (\i) to (\j);
        \end{scope}

        %node (4), $24$
        \begin{scope}[xshift=-2cm,yshift=-3.2cm,scale=0.3]
            \foreach \i in {1,3,5}
                \draw ({(360/5)*\i}:2) node(\i)[svertex]{};
            \foreach \i in {2,4}
                \draw ({(360/5)*\i}:2) node(\i)[svertex,fill=rosa]{};
            \foreach \i/\j in {1/2,1/3,1/4,1/5,2/3,3/4,4/5}
                \draw [edge,grisOscuro] (\i) to (\j);
        \end{scope}

        %node (3), $35$
        \begin{scope}[xshift=2cm,yshift=-3.2cm,scale=0.3]
            \foreach \i in {1,2,4}
                \draw ({(360/5)*\i}:2) node(\i)[svertex]{};
            \foreach \i in {3,5}
               \draw ({(360/5)*\i}:2) node(\i)[svertex,fill=rosa]{};
            \foreach \i/\j in {1/2,1/3,1/4,1/5,2/3,3/4,4/5}
                \draw [edge,grisOscuro] (\i) to (\j);
        \end{scope}

        %node (6), $23$
        \begin{scope}[xshift=-4.6cm,yshift=0cm,scale=0.3]
            \foreach \i in {1,4,5}
                \draw ({(360/5)*\i}:2) node(\i)[svertex]{};
            \foreach \i in {2,3}
                \draw ({(360/5)*\i}:2) node(\i)[svertex,fill=rosa]{};
            \foreach \i/\j in {1/2,1/3,1/4,1/5,2/3,3/4,4/5}
                \draw [edge,grisOscuro] (\i) to (\j);
        \end{scope}
        
        %node (5), $45$
        \begin{scope}[xshift=4.6cm,yshift=0cm,scale=0.3]
            \foreach \i in {1,2,3}
                \draw ({(360/5)*\i}:2) node(\i)[svertex]{};
            \foreach \i in {4,5}
                \draw ({(360/5)*\i}:2) node(\i)[svertex,fill=rosa]{};
            \foreach \i/\j in {1/2,1/3,1/4,1/5,2/3,3/4,4/5}
                \draw [edge,grisOscuro] (\i) to (\j);
        \end{scope}

        %node (8), $13$
        \begin{scope}[xshift=-1.8cm,yshift=-0.2cm,scale=0.3]
            \foreach \i in {2,4,5}
                \draw ({(360/5)*\i}:2) node(\i)[svertex]{};
            \foreach \i in {1,3}
                \draw ({(360/5)*\i}:2) node(\i)[svertex,fill=rosa]{};
            \foreach \i/\j in {1/2,1/3,1/4,1/5,2/3,3/4,4/5}
                \draw [edge,grisOscuro] (\i) to (\j);
        \end{scope}

        %node (7), $14$
        \begin{scope}[xshift=1.8cm,yshift=-0.2cm,scale=0.3]
            \foreach \i in {2,3,5}
                \draw ({(360/5)*\i}:2) node(\i)[svertex]{};
            \foreach \i in {1,4}
                \draw ({(360/5)*\i}:2) node(\i)[svertex,fill=rosa]{};
            \foreach \i/\j in {1/2,1/3,1/4,1/5,2/3,3/4,4/5}
                \draw [edge,grisOscuro] (\i) to (\j);
        \end{scope}

        %node (2), $25$
        \begin{scope}[xshift=0cm,yshift=1.6cm,scale=0.3]
            \foreach \i in {1,3,4}
                \draw ({(360/5)*\i}:2) node(\i)[svertex]{};
            \foreach \i in {2,5}
                \draw ({(360/5)*\i}:2) node(\i)[svertex,fill=rosa]{};
            \foreach \i/\j in {1/2,1/3,1/4,1/5,2/3,3/4,4/5}
                \draw [edge,grisOscuro] (\i) to (\j);
        \end{scope}

        %node (10), $12$
        \begin{scope}[xshift=-3.2cm,yshift=2cm,scale=0.3]
            \foreach \i in {3,4,5}
                \draw ({(360/5)*\i}:2) node(\i)[svertex]{};
            \foreach \i in {1,2}
                \draw ({(360/5)*\i}:2) node(\i)[svertex,fill=rosa]{};
            \foreach \i/\j in {1/2,1/3,1/4,1/5,2/3,3/4,4/5}
                \draw [edge,grisOscuro] (\i) to (\j);
        \end{scope}

        %node (9), $15$
        \begin{scope}[xshift=3.2cm,yshift=2cm,scale=0.3]
            \foreach \i in {2,3,4}
                \draw ({(360/5)*\i}:2) node(\i)[svertex]{};
            \foreach \i in {1,5}
                \draw ({(360/5)*\i}:2) node(\i)[svertex,fill=rosa]{};
            \foreach \i/\j in {1/2,1/3,1/4,1/5,2/3,3/4,4/5}
                \draw [edge,grisOscuro] (\i) to (\j);
        \end{scope}
    
    \end{tikzpicture}
    \caption{Una gr\'afica $G$ (izquierda) y $F_2(G)$ (derecha) donde, dentro de
    cada v\'ertice, se muestra c\'omo es la configuraci\'on de las fichas en $G$
    para dicho v\'ertice.}
    \label{fig:ex-tok-graph}
\end{figure}

Como muchas veces en matem\'aticas, hay ciertas gr\'aficas de fichas que han
sido estudiadas con otro enfoque y bajo otros nombres. Hay un caso que vale la
pena resaltar, el de las gr\'aficas de Johnson. Para un entero $1 \leq k \leq
n$, una \textbf{gr\'afica} \indiceSub{gr\'afica}{de Johnson}, denotada $J(n,k)$,
es la gr\'afica que tiene como conjunto de v\'ertices a todos los
$k$-subconjuntos de $\{1, \dots, n\}$ y donde dos v\'ertices $A$ y $B$ son
adyacentes si y s\'olo si $|A \cap B| = k-1$. Nos fijamos que, para toda $1 \leq
k \leq n$, la gr\'afica de Johnson $J(n,k)$ es isomorfa a la gr\'afica de
$k$-fichas de $K_n$. Las gr\'aficas de Johnson son estudiadas tambi\'en desde la
Teor\'ia de C\'odigos, por lo que hay m\'as ejemplos de este tipo de gr\'aficas
de fichas. De esta manera, las gr\'aficas de Johnson pueden servir como ejemplos
a trav\'es de este trabajo. M\'as a\'un, algunas propiedades conocidas de las
gr\'aficas de Johnson son casos particulares de algunos resultados de este
trabajo.

\section{Algunas estructuras en gr\'aficas de fichas}%
\label{sec:}

Las fichas en una gr\'afica se pueden mover \'unicamente por aristas existentes,
por lo que es natural pensar que una trayectoria en la gr\'afica original tiene
alguna trayectoria en la gr\'afica de fichas con la misma longitud. Teniendo eso
en mente, definimos el siguiente concepto. Sean $P$ una $ab$-trayectoria en la
gr\'afica $G$ y $A$ un $k$-conjunto en $V(G)$, en otras palabras $A \in
V(F_k(G))$. Si al conjunto $A$ le pedimos que $a\in A$ y $b \notin A$, entonces
a la pareja $(A,P)$ le podemos asignar una trayectoria en la gr\'afica de
$k$-fichas, tal que el v\'ertice final est\'a dado por $A'=(A \setminus \{a\})
\cup \{b\}$. Para definir esta nueva trayectoria tomamos $A\cap P =\{v_1, \dots,
v_q\}$, con $v_1 = a$, y ``movemos'' las fichas de la siguiente manera. Primero,
movemos la ficha de $v_q \in V(G)$ hacia el v\'ertice $b \in V(G)$ por $P$.
Luego, para los v\'ertices $v_i$ en $G$, con $i \in \{q-1, q-2, \dots 1\}$,
movemos la ficha de $v_i$ a $v_{i+1}$. As\'i, sucesivamente, vamos moviendo
fichas a trav\'es de $P$ por v\'ertices que est\'an libres de fichas. Observamos
que, al recorrer las fichas a trav\'es de $P$, obtenemos una trayectoria de la
misma longitud que $P$. Esta trayectoria en $F_k(G)$ la denotamos
\indiceSub{fichas}{$A \xrightarrow[P]{} A'$}. A continuaci\'on, se encuentra un
ejemplo de este tipo de trayectoria en una gr\'afica de $3$-fichas, mostrada en
\cref{fig:ex-tok-Path}. Tomamos a \cref{fig:ex-tok-aux} como figura auxiliar
para entender mejor el movimiento de las fichas a trav\'es de la trayectoria. La
gr\'afica $G$ se muestra del lado izquierdo en ambas figuras, adem\'as, en
\cref{fig:ex-tok-Path} se muestra $F_3(G)$ del lado derecho. Definamos
$A=\{2,3,4\}$ en $F_3(G)$ y nuestra trayectoria en $G$ como
${\color{vino}\boldsymbol {P= (3,1,4,5)}}$, resaltada en ambas figuras. De ah\'i
se sigue que $A'=\{2,4,5\}$. Adem\'as, del lado derecho de \cref{fig:ex-tok-aux}
se muestra el movimiento de las fichas que empiezan en $A \cap P =\{3,4\}$,
considerando de arriba a abajo como se van moviendo las fichas.

\begin{figure}[ht!]
    \centering
       \begin{tikzpicture}
    
        \begin{scope}[xshift=-8.5cm,scale=0.8]
            \foreach \i in {1,...,5}
                \draw ({(360/5)*\i}:2) node(\i)[vertex, label=(360/5)*\i:{${\i}$}]{};
            
            \foreach \i/\j in {1/2,1/5,2/3,3/4}
                \draw [edge,grisOscuro!70] (\i) to (\j);
            \foreach \i/\j in {1/3,1/4,4/5}
                \draw [wedge,vino] (\i) to (\j);
        \end{scope}

        \begin{scope}[yshift=54]
            \draw (-1,0) node (1) [vertex, label=90:{$1$}] {};
            \draw (-3,0) node (3) [vertex, fill=naranja, label=90:{$3$}] {};
            \draw (1,0) node (4) [vertex, fill=azulMetal, label=90:{$4$}] {};
            \draw (3,0) node (5) [vertex, label=90:{$5$}] {};

            \foreach \i/\j in {1/3,1/4,4/5}
                \draw [edge,vino] (\i) to (\j);
        \end{scope}

        \begin{scope}[yshift=18]
            \draw (-1,0) node (1) [vertex, label=90:{$1$}] {};
            \draw (-3,0) node (3) [vertex, fill=naranja, label=90:{$3$}] {};
            \draw (1,0) node (4) [vertex, label=90:{$4$}] {};
            \draw (3,0) node (5) [vertex, fill=azulMetal, label=90:{$5$}] {};

            \foreach \i/\j in {1/3,1/4,4/5}
                \draw [edge,vino] (\i) to (\j);
        \end{scope}

        \begin{scope}[yshift=-18]
            \draw (-1,0) node (1) [vertex, fill=naranja, label=90:{$1$}] {};
            \draw (-3,0) node (3) [vertex, label=90:{$3$}] {};
            \draw (1,0) node (4) [vertex, label=90:{$4$}] {};
            \draw (3,0) node (5) [vertex, fill= azulMetal, label=90:{$5$}] {};

            \foreach \i/\j in {1/3,1/4,4/5}
                \draw [edge,vino] (\i) to (\j);
        \end{scope}

        \begin{scope}[yshift=-54]
            \draw (-1,0) node (1) [vertex, label=90:{$1$}] {};
            \draw (-3,0) node (3) [vertex, label=90:{$3$}] {};
            \draw (1,0) node (4) [vertex, fill=naranja, label=90:{$4$}] {};
            \draw (3,0) node (5) [vertex, fill=azulMetal, label=90:{$5$}] {};

            \foreach \i/\j in {1/3,1/4,4/5}
                \draw [edge,vino] (\i) to (\j);
        \end{scope}
        
    \end{tikzpicture}
    \caption{Una gr\'afica $G$ con una trayectoria $P$ resaltada (izquierda) y
     el movimiento de dos fichas a trav\'es de $P$ (derecho).}
    \label{fig:ex-tok-aux}
\end{figure}
    
M\'as adelante, en \cref{fig:ex-tok-Path}, se muestra la trayectoria
${\color{vino}\boldsymbol {\{2,3,4\}\xrightarrow[P]{}\{2,4,5\}}}$ en $F_3(G)$.
En cada v\'ertice de esta trayectoria se resaltan los n\'umeros del color de la
ficha que se usa en \cref{fig:ex-tok-aux}, para resaltar la relaci\'on entre la
trayectoria de $G$ y la de $F_3(G)$. Observamos que $2$ no est\'a en $A \cap P
=\{3,4\}$, por lo que no se ve en \cref{fig:ex-tok-aux}. Esto significa que la
ficha en el v\'ertice $2$ se queda fija a trav\'es de la trayectoria en la
gr\'afica de fichas. Al estar fija, todos los v\'ertices de la nueva trayectoria
en $F_3(G)$ contienen a $2$. Ese hecho se puede ver en \cref{fig:ex-tok-Path} al
fijarnos que el $2$ est\'a en todos los v\'ertices de la trayectoria, resaltado
en gris.

\begin{figure}[ht!]
    \centering
       \begin{tikzpicture}
    
        \begin{scope}[xshift=-8.5cm,scale=0.8]
            \foreach \i in {1,...,5}
                \draw ({(360/5)*\i}:2) node(\i)[vertex, label=(360/5)*\i:{\normalsize ${\i}$}]{};
            
            \foreach \i/\j in {1/2,1/5,2/3,3/4}
                \draw [edge,grisOscuro!75] (\i) to (\j);
            \foreach \i/\j in {1/3,1/4,4/5}
                \draw [wedge,vino] (\i) to (\j);
            \end{scope}
        
        \begin{scope}[xshift=-2cm]
            \draw (0,1.5) node (1) [vertex, label=90:{\footnotesize $134$}] {};
            \draw (-2,2) node (2) [vertex, label=90:{${\bf {\color{grisOscuro} 2}{\color{naranja} 3}{\color{azulMetal} 4}}$}] {};
            \draw (2,2) node (3) [vertex, label=90:{\footnotesize $345$}] {};
            \draw (-3,1) node (4) [vertex, label=180:{\footnotesize $123$}] {};
            \draw (3,1) node (5) [vertex, label=0:{\footnotesize $145$}] {}; 
            \draw (-1.1,0.7) node (6) [vertex, label=87:{${\bf {\color{grisOscuro} 2}{\color{naranja} 3}{\color{azulMetal} 5}}$}] {};
            \draw (1.1,0.7) node (7) [vertex, label=93:{${\bf {\color{grisOscuro} 2}{\color{naranja} 4}{\color{azulMetal} 5}}$}] {};
            \draw (-1.1,-1) node (8) [vertex, label=240:{\footnotesize $124$}] {};
            \draw (1.1,-1) node (9) [vertex, label=330:{\footnotesize $135$}] {};
            \draw (0,-2.3) node (10) [vertex, label=270:{${\bf {\color{naranja} 1}{\color{grisOscuro} 2}{\color{azulMetal} 5}}$}] {};
           
            \foreach \i/\j in{1/2,1/3,1/8,1/9,2/4,2/8,3/5,3/7,3/9,4/6,4/8,5/7,
            5/9,6/7,6/9,7/8,8/10,9/10} 
                \draw [edge,grisOscuro!75] (\i)to (\j); 
            \foreach \i/\j in { 2/6,6/10,7/10} 
                \draw [wedge,vino] (\i) to (\j);
       \end{scope}       
    
    \end{tikzpicture}
    \caption{Una gr\'afica $G$, resaltando una trayectoria $P$(derecha) y 
    su gr\'afica de $3$-fichas (izquierda), resaltando la trayectoria 
    generada por $P$.}
    \label{fig:ex-tok-Path}
\end{figure}


Del ejemplo anterior, podemos concluir que tener una ficha fija resulta
interesante al considerar una trayectoria en la gr\'afica de fichas. Vale la
pena analizar c\'omo se comportan otras subgr\'aficas al dejar fija una ficha.
Generalizando un poco m\'as, nos podemos preguntar, qu\'e gr\'afica de
$3$-fichas se obtendr\'ia si una ficha se queda fija, digamos en $2$. Es f\'acil
notar que esta gr\'afica es una subgr\'afica de la gr\'afica de $3$-fichas de
$G$, pues el conjunto de v\'ertices de esta gr\'afica es el conjunto de
v\'ertices de $F_3(G)$ que contengan a $2$. Este ejemplo se representa en
\cref{fig:ex-tok-subgraph}. La idea de fijar una ficha se puede extender a
$r\leq k$ fichas, aunque el caso interesante ser\'a para $r<k$, pues, si fijamos
todas las fichas, obtenemos una gr\'afica trivial. Dado un conjunto $X \subseteq
V(G)$ con $|X|=r<k$, definimos a $F_k(G,X)$ como la \textbf{subgr\'afica de
$F_k(G)$ inducida}\index{subgr\'afica!inducida de
fichas}\index{fichas!subgr\'afica inducida} por los v\'ertices de $F_k(G)$ que
contienen al subconjunto $X$. 

\begin{figure}[ht!]
    \centering
       \begin{tikzpicture}
    
        \begin{scope}[xshift=-8.5cm,scale=0.7]
            \foreach \i in {1,3,4,5}
                \draw ({(360/5)*\i}:2) node(\i)[vertex, label=(360/5)*\i:{\small ${\i}$}]{};
            \draw ({(360/5)*2}:2) node(2)[vertex, fill=rosa, label=(360/5)*2:{\small ${2}$}]{};
            
            \foreach \i/\j in {1/2,1/3,1/4,1/5,2/3,3/4,4/5}
                \draw [edge,grisOscuro] (\i) to (\j);
            \end{scope}

        \begin{scope}[xshift=-2cm, scale=0.8]
            \draw (0,1.5) node (1) [vertex, label=90:{{ {\color{grisOscuro}\footnotesize $134$}}}] {};
            \draw (-2,2) node (2) [vertex, label=90:\small $234$] {};
            \draw (2,2) node (3) [vertex, label=90:{{ {\color{grisOscuro}\footnotesize $345$}}}] {};
            \draw (-3,1) node (4) [vertex, label=180:\small $123$] {};
            \draw (3,1) node (5) [vertex, label=0:{{ {\color{grisOscuro}\footnotesize $145$}}}] {}; 
            \draw (-1.1,0.7) node (6) [vertex, label=87:\small $235$] {};
            \draw (1.1,0.7) node (7) [vertex, label=93:\small $245$] {};
            \draw (-1.1,-1) node (8) [vertex, label=240:\small $124$] {};
            \draw (1.1,-1) node (9) [vertex, label=330:{{ {\color{grisOscuro}\footnotesize $135$}}}] {};
            \draw (0,-2.3) node (10) [vertex, label=270:\small $125$] {};
           
            \foreach \i/\j in{1/2,1/3,1/8,1/9,3/5,3/7,3/9,5/7,
            5/9,9/10} 
                \draw [edge,grisOscuro!75] (\i)to (\j); 
            \foreach \i/\j in {2/4,4/6,4/8,2/8,7/8,8/10,2/6,6/7,6/10,7/10} 
                \draw [wedge,rosa] (\i) to (\j);
       \end{scope}       

    \end{tikzpicture}
    \caption{Una gr\'afica $G$ (izquierda) y su gr\'afica de $3$-fichas 
    (derecha), donde se resalta $F_3(G,\{2\})$.}
    \label{fig:ex-tok-subgraph}
\end{figure}

Al momento de generar $F_3(G,{2})$, es f\'acil notar que las fichas se mueven por
$V(G) \setminus \{2\}$, excepto la ficha que est\'a fija. De igual manera, las
aristas por las que se mueven las fichas son las aristas de $G$ que no tienen
extremos en $2$. Vi\'endolo as\'i, podemos interpretar $F_3(G,\{2\})$ como la
gr\'afica de $(3-1)$-fichas de la gr\'afica $G-2$, es decir, $F_2(G-2)$. Esta
gr\'afica est\'a exhibida en \cref{fig:ex-tok-subgraph-aux}. De manera general,
esto quiere decir que existe una relaci\'on entre la subgr\'afica de $F_k(G)$
inducida por $X$, donde $|r|=X \subset V(G)$, y la gr\'afica de $(k-r)$-fichas
de la gr\'afica $G-X$, es decir, $F_{k-r}(G-X)$. Afirmamos que, para $k,r \in
\mathbb{N}$ y $k>r = |X|$, con $X \subseteq V(G)$, la gr\'afica $F_k(G,X)$ es
isomorfa a la gr\'afica $F_{k-r}(G-X)$, con el isomorfismo que a cada $A \in
F_k(G,X)$ le asocia $A \setminus X$ en $F_{k-r}(G-X)$.   Observamos que, las
aristas en $G$ con alg\'un extremo en $X$ no se usan en $F_k(G,X)$, por lo que
la funci\'on s\'i preserva adyacencias y no adyacencias.

\begin{figure}[ht!]
    \centering
       \begin{tikzpicture}
    
       \begin{scope}[xshift=-8.5cm,scale=0.7]
        \foreach \i in {1,3,4,5}
            \draw ({(360/5)*\i}:2) node(\i)[vertex, label=(360/5)*\i:{\small ${\i}$}]{};
        \draw ({(360/5)*2}:2) node(2)[cvertex, label=(360/5)*2:{{ {\color{grisOscuro!75}\small $2$}}}]{};
        
        \foreach \i/\j in {1/3,1/4,1/5,3/4,4/5}
            \draw [edge,grisOscuro] (\i) to (\j);

        \foreach \i/\j in {1/2,2/3}
            \draw [edge,grisOscuro!50] (\i) to (\j);
        \end{scope}

    \begin{scope}[xshift=-2cm]
        \draw (0.5,0.87) node (1) [vertex,label=87:$235$] {};
        \draw ({(360/6)*2}:2) node(2)[vertex, label=(360/5)*2:{\small ${234}$}]{};
        \draw ({(360/6)*3}:2) node(3)[vertex, label=(360/5)*3:{\small ${123}$}]{};
        \draw (-0.5,-0.87) node (4) [vertex,label=240:\small $124$] {};
        \draw ({(360/6)*5}:2) node(5)[vertex, label=(360/5)*5:{\small ${125}$}]{};
        \draw ({(360/6)*6}:2) node(6)[vertex, label=(360/5)*6:{\small ${245}$}]{};
   
    \foreach \i/\j in {1/2,1/3,1/5,1/6,2/3,2/4,3/4,4/5,4/6,5/6}
        \draw [edge,grisOscuro] (\i) to (\j);
    \end{scope}

    \end{tikzpicture}
    \caption{Una gr\'afica $G$ (izquierda), resaltando $G-2$, y la 
    gr\'afica de fichas $F_2(G-2)$ (derecha).}
    \label{fig:ex-tok-subgraph-aux}
\end{figure}

\newpage

Otra cosa interesante del ejemplo de \cref{fig:ex-tok-Path} es que, si nos
fijamos bien, podemos ver que tiene el mismo diagrama que la gr\'afica de
$2$-fichas en \cref{fig:ex-tok-graph}, i.e., son gr\'aficas isomorfas. Al ser
ambas gr\'aficas de fichas de la misma gr\'afica, vale la pena preguntarse,
cu\'al es la relaci\'on entre ellas. Fij\'andonos a\'un m\'as, vemos que el
isomorfismo entre ambas env\'ia el v\'ertice $A \in V(F_2(G))$ al v\'ertice
$V(G) \setminus A \in V(F_3(G))$. Adem\'as, notamos que dos v\'ertices de $ A,B
\in V(F_2(G))$ son adyacentes si y s\'olo si $V(G) \setminus A$ y $V(G)
\setminus B$ son adyacentes en $F_3(G)$. As\'i, la diferencia sim\'etrica entre
$A$ y $B$ es igual a la diferencia sim\'etrica entre sus complementos. Podemos
verlo como recorrer la gr\'afica ``al rev\'es''. Generalizando este concepto,
afirmamos que, para toda gr\'afica $G$, $F_k(G)$ y $F_{n-k}(G)$ son gr\'aficas
isomorfas. Empezamos viendo que $|V(F_k(G))| =\binom{n}{k}= \binom{n}{n-k}=
|V(F_{n-k}(G))|$. Tomando la funci\'on antes mencionada, falta ver que preserva
adyacencias y no adyacencias. Volviendo a observar que, para todo $A$ y $B$ en
$V(F_k(G))$ tenemos que $A \triangle B = (V(G)\setminus A) \triangle
(V(G)\setminus B)$, entonces las adyacencias y no adyacencias dependen de $G$,
por lo que la funci\'on cumple ser un isomorfismo. 

El resultado anterior resulta de mucha utilidad al momento de estudiar las
gr\'aficas de fichas, pues nos dice que basta en enfocarnos en las gr\'aficas de
$k$-fichas, para $k \leq \frac{n}{2}$.

Teniendo estos isomorfismos en mente, pasamos a desarrollar las propiedades de
las gr\'aficas de fichas mencionadas en los art\'iculos \textit{Token Graphs}
\cite{fabilaToken} y \textit{Hamiltonicity of Token Graphs of Some Join Graphs}
\cite{adameHamilt}.


\section{Di\'ametro y producto cartesiano}%
\label{sec:etiquetas}

En esta secci\'on estudiamos dos teoremas del art\'iculo \textit{Token Graphs}
\cite{fabilaToken}, el primero sobre el d\'iametro de una gr\'afica de fichas y
el segundo sobre el producto carteciano. Primero, establecemos una cota para el
di\'ametro de una gr\'afica de fichas, la demostraci\'on del teorema esta basada
en la demostraci\'on del art\'iculo antes mencionado.

\begin{teorema}%
\label{teo:diamFG}
    Si $G$ una gr\'afica conexa con di\'ametro $d$, entonces, $F_{k}(G)$ es
    conexa con di\'ametro, al menos, $k(d -k+1)$ y, a lo m\'as, $d k$.
\end{teorema}

\begin{proof}
    Sean $A$ y $B$ v\'ertices de $F_{k}(G)$. Primero, nos enfocamos en la cota
    superior. Por definici\'on, tenemos que $|A \triangle B| \leq |A \cup B|$,
    con igualdad cuando $A \cap B = \varnothing$. Observamos que, al ser $A$ y
    $B$ v\'ertices de $F_{k}(G)$, tenemos que $|A|=k$ y $|B|=k$, por lo que $|A
    \cup B| \le 2k$. Luego, tenemos que $|A \triangle B| \leq 2k$, por lo que
    $\frac{1}{2} |A \triangle B| \leq k$.

    Buscamos demostrar que el di\'ametro de $F_{k}(G)$ es, a lo m\'as, $d k$,
    por lo que basta demostrar por inducci\'on que, para cualesquiera dos
    v\'ertices $A$ y $B$ de $F_{k}(G)$, hay una $AB$-trayectoria de, a lo m\'as,
    $\frac{d}{2}|A\triangle B|$. Observamos que esto tambi\'en implica que
    $F_{k}(G)$ es conexa.

    Si $A\triangle B=\varnothing$, entonces $A=B$, por lo que no hay nada que
    probar. Ahora consideramos $A$ y $B$, tales que $A\triangle B \neq
    \varnothing$. Tomamos como hip\'otesis que, para cualesquiera dos v\'ertices
    de $F_{k}(G)$, $C$ y $D$, tales que $|C\triangle D|<|A \triangle B|$, existe
    una $CD$-trayectoria con longitud, a lo m\'as, $\frac{d}{2}|C\triangle D|$.
    Al tomar $A\triangle B \neq \varnothing$, tenemos un v\'ertice de $G$ en
    $A\setminus B$ y un v\'ertice en $B\setminus A$, que denotamos $a$ y $b$,
    respectivamente. Dado que el di\'ametro de $G$ es $d$, tenemos que hay una
    $ab$-trayectoria de longitud a lo m\'as $d$, digamos $P$.

    Definimos $A'=(A\setminus \{a\})\cup \{b\}$ y la trayectoria
    $A\xrightarrow[P]{} A'$ en $F_{k}(G)$. Por un lado, observamos que $b\in
    B\cap A'$ y $b\notin B\cap A$. Por otro lado, tenemos que $a \notin A'$, por
    lo que $a\notin A'\cup B$ y $a\notin A\cap B$. Sin embargo, tanto $a$ como
    $b$ est\'an en $A\cup B$. Por eso, tenemos que $a,b \in A\triangle B$ y $a,b
    \notin A'\triangle B$. Ahora, tomamos $v\in A$ tal que $v \neq a$, se sigue
    que $v \in A\triangle B$ si y s\'olo si $v\in A'\triangle B$. Por lo tanto,
    tenemos que $|A'\triangle B|=|A \triangle B|- 2$. Por hip\'otesis inductiva,
    sabemos que hay una $A'B$-trayectoria en $F_{k}(G)$ de longitud, a lo m\'as,
    $\frac{d}{2}|A'\triangle B|$ que, como se observ\'o anteriormente, coincide
    con $\frac{d}{2}|A\triangle B| - d$.

    Sabemos que $A\xrightarrow[P]{} A'$ tiene la misma longitud que $P$, que es,
    a lo m\'as, $d$. Por ello, tenemos una $AB$-trayectoria de la forma
    $A\rightarrow A'\rightarrow B$ que tiene longitud, a lo m\'as,
    $\frac{d}{2}|A\triangle B|-d +d =\frac{d}{2}|A\triangle B|$. Por lo tanto,
    tenemos que $F_{k}(G)$ es conexa y tiene di\'ametro, a lo m\'as, $d k$.

    Ahora, demostramos la cota inferior. Sabemos que $G$ es una gr\'afica conexa
    con di\'ametro $d$, por lo que existen vertices que est\'an a distancia $d$,
    digamos $x$ y $y$. Ahora, construimos una partici\'on de $V$ usando la
    distancia que tiene cada v\'ertice a $x$. Es decir, para cada $i\in [0,d]$,
    sea $V_{i}$ el conjunto de v\'ertices de $G$ a distancia $i$ de $x$. Por
    tanto, tenemos que $V_{0}=\{x\}$ y $y\in V_{d}$. Denotamos $d_x(v)$ a la
    distancia entre $x$ y el v\'ertice $v$.

    Sea $a$ el m\'inimo \'indice para el cu\'al se tiene $k \leq |V_{0}\cup
    V_{1}\cup \cdots \cup V_{a}|$ y sea $b$ el m\'aximo \'indice para el cu\'al
    se tiene $k\leq |V_{b}\cup V_{b+1}\cup \cdots \cup V_{d}|$. Tomamos $A$ un
    $k$-subconjunto de $V_{0}\cup \cdots \cup V_{a}$  tal que $A\subseteq V_{0}$
    o $V_{0}\cup \cdots \cup V_{a-1}\subseteq A$. Adem\'as, tomamos $B$ un
    $k$-subconjunto de $V_{b}\cup \cdots \cup V_{d}$ tal que $B\subseteq V_{d}$
    o $V_{b+1}\cup \cdots \cup V_{d} \subseteq B$. 

    Consideremos una trayectoria entre $A$ y $B$ en $F_{k}(G)$. Cualquier ficha
    inicialmente en $A$, digamos $v \in (G)$, se mueve a alg\'un v\'ertice en
    $B$, digamos $v'\in (G)$. Notamos que todas las aristas de $G$ est\'an
    dentro de alg\'un $V_{i}$ o, a lo m\'as, entre alg\'un $V_{i}$ y $V_{i+1}$,
    con $i\in[0,d]$. As\'i pues, para la ficha en $v$ se necesitan, al menos,
    $d_x(v')-d_x(v)$ movimientos para llegar a $v'$, ocupando s\'olo las aristas
    entre $V_{i}$ y $V_{i+1}$. Por lo tanto, el di\'ametro de $F_{k}(G)$ es, al
    menos, $\sum_{v\in A}(d_x(v')-d_x(v))= \sum_{w\in B}d_x(w)-\sum_{v\in
    A}d_x(v)$. Observamos que, al ser $G$ conexa, toda $V_{i}$ tiene al menos un
    elemento y, por construcci\'on, $V_{i} \cap V_{i+1}=\varnothing$, para toda
    $i\in [0,d]$. Tomamos el caso en el que $|V_{i}|=1$, para toda $i\in [0,d]$.
    Luego, tenemos que $k\leq |V_{b}\cup\cdots\cup
    V_{d}|=|V_{b}|+|V_{b+1}|+\cdots +|V_d|$ $=\sum_{b}^{d}1 = d -b+1$.
    An\'alogamente, tenemos que $k\leq |V_{0}\cup V_{1}\cup \cdots \cup
    V_{a}|=|V_{0}|+|V_{1}|+\cdots + |V_{a}|$ $=\sum_{0}^{a} 1 = a+1$. En ambos
    casos, la cota m\'inima se alcanza en la igualdad, por lo que tomamos
    $a=k-1$ y $b=d-k+1$. Por lo tanto, tenemos que el di\'ametro de $F_{k}(G)$
    es al menos $\sum_{j=d -k+1}^{d}j - \sum_{i=0}^{k-1}i = k(d-k+1)$.
\end{proof}

Es interesante notar que las cotas obtenidos en \cref{teo:diamFG} son
alcanzables. A continuaci\'on se muestran algunos ejemplos para dichas cotas. La
cota inferior se cumple para toda trayectoria de $d +1$ v\'ertices, con $k \leq
d +1$, es decir, una trayectoria de di\'ametro $d$. Es f\'acil observar que,
para toda $k$, obtenemos una gr\'afica de $k$-fichas de di\'ametro $k(d-k+1)$.
Para ejemplificar este tipo de gr\'aficas, consideremos a la trayectoria $P_6$.
Recordamos que $F_2(P_6) \cong F_4(P_6)$, $F_1(P_6) \cong F_5(P_6)$ y $F_6(P_6)
\cong K_1$, por lo que, en \cref{fig:ex-diamP} se muestran las gr\'aficas de $2$
y $3$-fichas de la $P_6$. Las gr\'aficas $F_2(P_6)$ y $F_3(P_6)$ tienen
di\'ametro $8$ y $9$, respectivamente, i.e., tiene di\'ametro $k(d-k+1)$

\begin{figure}[ht!]
    \centering
    \begin{tikzpicture}
    
        \begin{scope}[xshift=-6cm,yshift=-3cm,rotate=135]
            \draw (2,2) node (1) [vertex,fill=menta,label=270:{\footnotesize $56$}] {};
            \draw (2,1) node (2) [vertex,fill=menta,label=270:{\footnotesize $46$}] {};
            \draw (2,0) node (3) [vertex,label=270:{\footnotesize $36$}] {};
            \draw (2,-1) node (4) [vertex,label=270:{\footnotesize $26$}] {};
            \draw (2,-2) node (5) [vertex,label=270:{\footnotesize $16$}] {};
            \draw (-2,-2) node (6) [vertex,fill=menta,label=270:{\footnotesize $12$}] {};
            \draw (-1,-2) node (7) [vertex,fill=menta,label=270:{\footnotesize $13$}] {};
            \draw (0,-2) node (8) [vertex,label=270:{\footnotesize $14$}] {};
            \draw (1,-2) node (9) [vertex,label=270:{\footnotesize $15$}] {};
            \draw (-1,-1) node (10) [vertex,fill=menta,label=270:{\footnotesize $23$}] {};
            \draw (0,-1) node (11) [vertex,fill=menta,label=270:{\footnotesize $24$}] {};
            \draw (1,-1) node (12) [vertex,label=270:{\footnotesize $25$}] {};
            \draw (0,0) node (13) [vertex,fill=menta,label=270:{\footnotesize $34$}] {};
            \draw (1,0) node (14) [vertex,fill=menta,label=270:{\footnotesize $35$}] {};
            \draw (1,1) node (15) [vertex,fill=menta,label=270:{\footnotesize $45$}] {};

            \foreach \i/\j in{2/3,3/4,4/5,7/8,8/9,5/9,11/12,4/12,
            3/14,9/12,12/14,8/11} 
                \draw [edge,grisOscuro!75] (\i) to (\j);

            \foreach \i/\j in{1/2,2/15,14/15,13/14,11/13,10/11,7/10,6/7} 
                    \draw [wedge,menta] (\i) to (\j);
        \end{scope}
        
        \begin{scope}[rotate=-55]
           
            \draw (2,0) node (21) [vertex,label=180:{\footnotesize $236$}] {};
            \draw (2,-1) node (22) [vertex,fill=azulMetal,label=180:{\footnotesize $136$}] {};
            \draw (2,-2) node (23) [vertex,fill=azulMetal,label=270:{\footnotesize $126$}] {};
            \draw (-1,-2) node (24) [vertex,fill=azulMetal,label=250:{\footnotesize $123$}] {};
            \draw (0,-2) node (25) [vertex,fill=azulMetal,label=250:{\footnotesize $124$}] {};
            \draw (1,-2) node (26) [vertex,fill=azulMetal,label=250:{\footnotesize $125$}] {};
            \draw (0,-1) node (27) [vertex,label=120:{\footnotesize $134$}] {};
            \draw (1,-1) node (28) [vertex,label=180:{\footnotesize $135$}] {};
            \draw (0,0) node (29) [vertex,label=90:{\footnotesize $234$}] {};
            \draw (1,0) node (20) [vertex,label=80:{\footnotesize $235$}] {};
            
            
            \foreach \i/\j in{21/22,27/28,22/28,29/20,
            21/20,29/27,25/27,28/20,26/28} 
                \draw [edge,grisOscuro!75] (\i) to (\j);
        \end{scope}

        \begin{scope}[xshift=3.2cm,yshift=0cm,rotate=-55]
           
            \draw (2,0) node (31) [vertex,fill=azulMetal,label=0:{\footnotesize $356$}] {};
            \draw (2,-1) node (32) [vertex,fill=azulMetal,label=0:{\footnotesize $256$}] {};
            \draw (2,-2) node (33) [vertex,fill=azulMetal,label=270:{\footnotesize $156$}] {};
            \draw (2,1) node (34) [vertex,fill=azulMetal,label=0:{\footnotesize $456$}] {};
            \draw (0,-2) node (35) [vertex,label=0:{\footnotesize $145$}] {};
            \draw (1,-2) node (36) [vertex,fill=azulMetal,label=250:{\footnotesize $146$}] {};
            \draw (0,-1) node (37) [vertex,label=90:{\footnotesize $245$}] {};
            \draw (1,-1) node (38) [vertex,label=0:{\footnotesize $246$}] {};
            \draw (0,0) node (39) [vertex,label=90:{\footnotesize $345$}] {};
            \draw (1,0) node (30) [vertex,label=80:{\footnotesize $346$}] {};
            
            
            \foreach \i/\j in{35/36,37/38,32/38,39/30,
            31/30,39/37,35/37,38/30,36/38} 
                \draw [edge,grisOscuro!75] (\i) to (\j);

           
        \end{scope}
        
        \foreach \i/\j in{24/25,25/26,23/26,22/23,22/36,33/36,32/33,31/32,31/34} 
                    \draw [wedge,azulMetal] (\i) to (\j);
        
                    \foreach \i/\j in{28/35,21/38,20/37} 
            \draw [edge,grisOscuro!75] (\i) to (\j);
        
    \end{tikzpicture}

\caption{La gr\'afica de $2$-fichas (izquierda) y la gr\'afica de $3$-fichas
(derecha) de la trayectoria $P_6$, resaltando una trayectoria de longitud
m\'axima en ambas gr\'aficas.}
\label{fig:ex-diamP}       
\end{figure}

Para la cota superior d\cref{teo:diamFG} consideramos la gr\'afica $G$, obtenida
al a\~{n}adirle $k$ v\'ertices a cada extremo de la trayectoria $P_{d -1}$ y
hacerlos adyacentes a los extremos de esta trayectoria. Observamos que $G$ tiene
di\'ametro $d$. Adem\'as, para toda $k \leq n$, la gr\'afica de $k$-fichas de
$G$ tiene di\'ametro $dk$. Para ejemplificar esto, \cref{fig:ex-diamT} muestra
una gr\'afica $G$ construida al agregar cuatro v\'ertices en cada extremo de la
trayectoria $P_5$, i.e., $k=4$. Notamos que, los v\'ertices
$\{v_1,v_2,v_3,v_4\}$ y $\{w_1,w_2,w_3,w_4\}$ son extremos en $F_4(G)$, pues
ambos tienen \'unicamente un vecino. \Cref{fig:ex-diamT} muestra el movimiento
de las fichas en $F_4(G)$ de un extremo al otro. Observamos que, una ficha en
alg\'un $v_i$ tiene que moverse por $d +1$ v\'ertices a alg\'un $w_i$ libre,
para $i \in \{1,2,3,4\}$. Adem\'as, cada movimiento corresponde a un v\'ertice
en la trayectoria en $F_4(G)$, que va de un extremo a otro. Por lo tanto,
tenemos que la trayectoria que va de $\{v_1,v_2,v_3,v_4\}$ a
$\{w_1,w_2,w_3,w_4\}$ tiene longitud $4 \cdot 6 = 24$.

\begin{figure}[ht!]
    \centering
       \begin{tikzpicture}
    
        \begin{scope}[]
            \draw (-2,0) node (1) [vertex,label=270:{\footnotesize $1$}] {};
            \draw (-1,0) node (2) [vertex,label=270:{\footnotesize $2$}] {};
            \draw (0,0) node (3) [vertex,label=270:{\footnotesize $3$}] {};
            \draw (1,0) node (4) [vertex,label=270:{\footnotesize $4$}] {};
            \draw (2,0) node (5) [vertex,label=270:{\footnotesize $5$}] {};
            \draw (-3.3,-1) node (6) [vertex,fill=naranja,label=180:{\footnotesize $v_4$}] {};
            \draw (-3.3,-0.3) node (7) [vertex,fill=naranja,label=180:{\footnotesize $v_3$}] {};
            \draw (-3.3,0.3) node (8) [vertex,fill=naranja,label=180:{\footnotesize $v_2$}] {};
            \draw (-3.3,1) node (9) [vertex,fill=naranja,label=180:{\footnotesize $v_1$}] {};
            \draw (3.3,-1) node (10) [vertex,label=0:{\footnotesize $w_4$}] {};
            \draw (3.3,-0.3) node (11) [vertex,label=0:{\footnotesize $w_3$}] {};
            \draw (3.3,0.3) node (12) [vertex,label=0:{\footnotesize $w_2$}] {};
            \draw (3.3,1) node (13) [vertex,label=0:{\footnotesize $w_1$}] {};

            \foreach \i/\j in{1/2,1/6,1/7,1/8,1/9,2/3,3/4,4/5,5/10,5/11,5/12,5/13} 
                \draw [edge,grisOscuro!75] (\i) to (\j);
        \end{scope}

        \begin{scope}[yshift=-3cm]
            \draw (-2,0) node (1) [vertex,fill=naranja,label=270:{\footnotesize $1$}] {};
            \draw (-1,0) node (2) [vertex,label=270:{\footnotesize $2$}] {};
            \draw (0,0) node (3) [vertex,label=270:{\footnotesize $3$}] {};
            \draw (1,0) node (4) [vertex,label=270:{\footnotesize $4$}] {};
            \draw (2,0) node (5) [vertex,label=270:{\footnotesize $5$}] {};
            \draw (-3.3,-1) node (6) [vertex,label=180:{\footnotesize $v_4$}] {};
            \draw (-3.3,-0.3) node (7) [vertex,fill=naranja,label=180:{\footnotesize $v_3$}] {};
            \draw (-3.3,0.3) node (8) [vertex,fill=naranja,label=180:{\footnotesize $v_2$}] {};
            \draw (-3.3,1) node (9) [vertex,fill=naranja,label=180:{\footnotesize $v_1$}] {};
            \draw (3.3,-1) node (10) [vertex,label=0:{\footnotesize $w_4$}] {};
            \draw (3.3,-0.3) node (11) [vertex,label=0:{\footnotesize $w_3$}] {};
            \draw (3.3,0.3) node (12) [vertex,label=0:{\footnotesize $w_2$}] {};
            \draw (3.3,1) node (13) [vertex,label=0:{\footnotesize $w_1$}] {};

            \foreach \i/\j in{1/2,1/6,1/7,1/8,1/9,2/3,3/4,4/5,5/10,5/11,5/12,5/13} 
                \draw [edge,grisOscuro!75] (\i) to (\j);
        \end{scope}


        \begin{scope}[]
            \node at (0,-5) {\large \vdots};
        \end{scope}

        \begin{scope}[yshift=-6.7cm]
            \draw (-2,0) node (1) [vertex,label=270:{\footnotesize $1$}] {};
            \draw (-1,0) node (2) [vertex,label=270:{\footnotesize $2$}] {};
            \draw (0,0) node (3) [vertex,label=270:{\footnotesize $3$}] {};
            \draw (1,0) node (4) [vertex,label=270:{\footnotesize $4$}] {};
            \draw (2,0) node (5) [vertex,label=270:{\footnotesize $5$}] {};
            \draw (-3.3,-1) node (6) [vertex,label=180:{\footnotesize $v_4$}] {};
            \draw (-3.3,-0.3) node (7) [vertex,label=180:{\footnotesize $v_3$}] {};
            \draw (-3.3,0.3) node (8) [vertex,label=180:{\footnotesize $v_2$}] {};
            \draw (-3.3,1) node (9) [vertex,label=180:{\footnotesize $v_1$}] {};
            \draw (3.3,-1) node (10) [vertex,fill=naranja,label=0:{\footnotesize $w_4$}] {};
            \draw (3.3,-0.3) node (11) [vertex,fill=naranja,label=0:{\footnotesize $w_3$}] {};
            \draw (3.3,0.3) node (12) [vertex,fill=naranja,label=0:{\footnotesize $w_2$}] {};
            \draw (3.3,1) node (13) [vertex,fill=naranja,label=0:{\footnotesize $w_1$}] {};

            \foreach \i/\j in{1/2,1/6,1/7,1/8,1/9,2/3,3/4,4/5,5/10,5/11,5/12,5/13} 
                \draw [edge,grisOscuro!75] (\i) to (\j);
        \end{scope}

        
    \end{tikzpicture}
    \caption{La gr\'afica $G$, utilizada como ejemplo d\cref{teo:diamFG},
    resaltando el movimiento de las $4$ fichas al pasar de un extremo de la
    gr\'afica al otro.}
\label{fig:ex-diamT}       
\end{figure}

Nuestro siguiente resultado muestra que algunas subgr\'aficas inducidas de las
gr\'aficas de fichas son productos cartesianos de algunas subgr\'aficas
inducidas de $G$. El art\'iculo \textit{Token Graphs} \cite{fabilaToken}, de
donde sale el siguiente teorema, no proporciona una demostraci\'on del dicho
teorema, por lo que la siguiente demostraci\'on es independiente del art\'iculo.

\begin{teorema}
    \label{teo:PCartes}
    Sean $H_1, \dots, H_m$ subgr\'aficas inducidas de una gr\'afica $G$ tales
    que son ajenas dos a dos. Para todos los enteros positivos $s_1, \dots,
    s_m$, tales que $1 \leq s_i \leq |V(H_i)|$ y $\sum\limits_{i=1}^{m}  s_i =
    k$, se tiene que $F_{s_1}(H_1) \square \cdots \square F_{s_m}(H_m)$ es una
    subgr\'afica inducida de $F_k(G)$.
\end{teorema}

\begin{proof}
    Sean $H_1, \dots, H_m$ subgr\'aficas inducidas de  $G$ y sean $s_1, \dots,
    s_m \in \mathbb{Z}$ como en el enunciado. Al conjunto de los
    $k$-subconjuntos de $G$ que cumplen tener exactamente $s_i$ elementos de
    $H_i$, para $i \in \{1, \dots, m\}$, lo llamamos $L$, i.e., $L = \{A
    \subseteq G \colon\ |A\cap V(H_1)|=s_1 , |A \cap V(H_2)|=s_2, \dots, |A \cap
    V(H_m)|=m \}$. Observamos que, debido a  que $\sum\limits_{i=1}^{m} s_i =
    k$, todo elemento de $L$ tiene cardinalidad $k$. Buscamos ver que la
    subgr\'afica inducida de $F_k(G)$ por $L$, es decir, $F_k(G,L)$, es isomorfa
    a $F_{s_1}(H_1) \square \cdots \square F_{s_m}(H_m)$. Primero, analizamos el
    conjunto de v\'ertices de ambas gr\'aficas de fichas. Sea $A \in
    V(F_k(G,L))$, de tal manera que $A=\{a_1, \dots, a_{s_1}, b_1,\dots,
    b_{s_2}, \dots, m_1, \dots, m_{s_m}\}$. Por definici\'on, $|A \cap V(H_i)|=
    s_i$, para $i \in \{1, \dots, m\}$. Adem\'as, sabemos que cada v\'ertice de
    $F_{s_i}(H_i)$ tiene cardinalidad $s_i$, para $i \in \{1, \dots, m\}$, por
    lo que tenemos que $A \in V(H_1) \times \cdots \times V(H_m)$. De manera
    an\'aloga, si tomamos $A \in V(H_1) \times \cdots \times V(H_m)$, $A =
    \{A'_1, A'_2, \dots, A'_m\}$ con $A'_i \in F_{s_i}(H_i)$. De lo anterior,
    tenemos que $A \cap V(H_i) = A'_i$, por lo que $|A \cap V(H_i)|= s_i$, para
    toda $i \in \{1, \dots, m\}$. Adicionalmente, tenemos que $|A|= k$, puesto
    que $\sum\limits_{i=1}^{m} s_i = k$. Por lo tanto, $A \in V(F_k(G,L))$.

    Ahora, nos enfocamos en la relaci\'on entre las aristas de ambas gr\'aficas
    de fichas. Sean $B, C \in V(F_k(G,L))$, tales que $BC \in E(F_k(G,L))$, en
    otras palabras, existen $b \in B$ y $c \in C$, tales que $B \triangle C =
    \{b, c\}$ y $bc \in E(G,L)$. Por definici\'on, tenemos que $V(H_i) \cap
    V(H_j) = \varnothing$, con $i \neq j$, por lo que $b, c \in V(H_i)$. Esto
    significa que las fichas en $H_j$ est\'an est\'aticas, para todo $j \in \{1,
    \dots, m\} \setminus \{i\}$. Lo cu\'al pasa si y s\'olo s\'i $BC$ es una
    arista en la subgr\'afica de fichas inducida por $H_i$. Como $V(H_i) \cap
    V(H_j) = \varnothing$, para $i \neq j$, entonces la subgr\'afica de fichas
    inducida por $H_i$ es isomorfa a $F_{s_i}(H_i)$. Al tener el resto de las
    fichas est\'aticas, tenemos que $BC \in E(F_{s_1}(H_1) \square \cdots
    \square F_{s_m}(H_m))$.

    Por lo tanto, tenemos un isomorfismo entre $F_{s_1}(H_1) \square \cdots
    \square F_{s_m}(H_m)$ y $F_k(G,L)$, una subgr\'afica inducida de $F_k(G)$.
    Por ende, $F_{s_1}(H_1) \square \cdots \square F_{s_m}(H_m)$ es una
    subgr\'afica inducida de $F_k(G)$.
\end{proof}

\Cref{fig:ex-cart}, exhibida a continuaci\'on, es un ejemplo sencillo
d\cref{teo:PCartes}. Del lado izquierdo de la figura tenemos una gr\'afica $G$,
resaltando dos subgr\'aficas inducidas ${\color{baige}\boldsymbol {H_1}}$ y
${\color{azulMetal}\boldsymbol {H_2}}$. El conjunto de v\'ertices de
${\color{azulMetal}\boldsymbol {H_2}}$ es $\{1,2,4,6\}$, mientras que el de
${\color{baige}\boldsymbol {H_1}}$ es $\{3\}$. Por un lado tenemos que $1\leq
s_1 \leq |V(H_1)|$, por lo que $s_1 =1$. Por otro lado, $1\leq s_2 \leq |V(H_2)|
= 4$ y, adem\'as, $s_1+s_2 = k =3$, por lo que $s_2 =2$. Del lado derecho de
\cref{fig:ex-cart} se muestra la gr\'afica $F_3(G)$, resaltando de
$\color{fushia} \bf rosa$ el producto cartesiano $F_1(H_1) \square F_2(H_2)$.
Observamos que dicha gr\'afica es, en efecto, una subgr\'afica inducida de
$F_3(G)$.

\begin{figure}[ht!]
    \centering
       \begin{tikzpicture}
    
        \begin{scope}[xshift=-8.5cm]
            \foreach \i in {0,1,3,5}
                \draw ({(360/6)*\i}:2) node(\i)[wvertex,fill=azulMetal]{};
            \draw ({(360/6)*2}:2) node(2)[wvertex,fill=baige]{};
            \draw ({(360/6)*4}:2) node(4)[wvertex]{}; 
            \foreach \i in {0,...,5}
                \draw ({(360/6)*\i}:2.5) node(e0){\pgfmathparse{int(\i+1)}
               \pgfmathresult};
            \foreach \i/\j in {0/2,1/2,1/4,2/3,2/4,2/5,3/4,4/5}
                \draw [edge,grisOscuro!50] (\i) to (\j);
            
            \foreach \i/\j in {0/1,0/5,0/3,1/5,3/5}
                \draw [wedge,azulMetal] (\i) to (\j); 
            \end{scope}
        
        \begin{scope}[xshift=0cm,yshift=0cm,scale=0.9]
            \foreach \i in {0,1,5,8,9,13} 
                \draw ({(360/16)*\i}:4) node(\i)[wvertex,fill=fushia]{}; 
            \foreach \i in {2,3,4,6,7,10,11,12,14,15} 
                \draw({(360/16)*\i}:4) node(\i)[wvertex]{};
    
            \draw ({(360/16)*1}:4.5) node (e1) {{${\color{azulMetal}\bf 2}{\color{baige}\bf 3}{\color{azulMetal}\bf6}$}};
            \draw ({(360/16)*2}:4.5) node (e1) {{\footnotesize $256$}};
            \draw ({(360/16)*3}:4.5) node (e1) {{\footnotesize $246$}};
            \draw ({(360/16)*4}:4.5) node (e1) {{\footnotesize $126$}};
            \draw ({(360/16)*5}:4.5) node (e1) {{${\color{azulMetal}\bf12}{\color{baige}\bf 3}$}};
            \draw ({(360/16)*6}:4.5) node (e1) {{\footnotesize $125$}};
            \draw ({(360/16)*7}:4.5) node (e1) {{\footnotesize $245$}};
            \draw ({(360/16)*8}:4.5) node (e1) {{${\color{azulMetal}\bf2}{\color{baige}\bf 3}{\color{azulMetal}\bf4}$}};
            \draw ({(360/16)*9}:4.5) node (e1) {{${\color{azulMetal}\bf1}{\color{baige}\bf 3}{\color{azulMetal}\bf4}$}};
            \draw ({(360/16)*10}:4.5) node (e1) {{\footnotesize $145$}};
            \draw ({(360/16)*11}:4.5) node (e1) {{\footnotesize $135$}};
            \draw ({(360/16)*12}:4.5) node (e1) {{\footnotesize $345$}};
            \draw ({(360/16)*13}:4.5) node (e1) {{${\color{baige}\bf 3}{\color{azulMetal}\bf46}$}};
            \draw ({(360/16)*14}:4.5) node (e1) {{\footnotesize $456$}};
            \draw ({(360/16)*15}:4.5) node (e1) {{\footnotesize $156$}};
            \draw ({(360/16)*16}:4.5) node (e1) {{${\color{azulMetal}\bf1}{\color{baige}\bf 3}{\color{azulMetal}\bf6}$}};
    
            \draw (-1.1,0.2) node (17) [vertex,label=10:{\tiny $124$}] {};
            \draw (0.1,-0.3) node (18) [vertex,label=70:{\tiny $146$}] {};
            \draw (1.2,0.7) node (19) [vertex,label=180:{\tiny $356$}] {};
            \draw (0.3,1.4) node (20) [vertex,label=88:{\tiny $235$}] {};
            
            \foreach\i/\j in{0/4,0/11,0/15,0/18,1/2,1/3,1/4,1/19,1/20,2/3,2/6,
            2/7,2/15,2/19,2/20,3/4,3/7,3/8,3/13,3/14,3/17,3/18,4/5,4/6,4/15,4/17,
            5/6,5/11,5/17,6/7,6/11,6/15,6/17,6/20,7/8,7/10,7/12,7/14,7/20,8/12,
            8/17,8/20,9/10,9/11,9/17,9/18,10/11,10/12,10/14,10/15,10/17,10/18,
            11/12,11/15,11/19,11/20,12/13,12/14,12/19,13/14,13/18,13/19,14/15,
            14/19,15/18,15/19,17/18,19/20} 
               \draw [edge,grisOscuro!50] (\i) to (\j);

               \foreach \i/\j in {0/1,0/5,0/9,0/13,1/5,1/8,5/8,8/9,8/13,9/13}
               \draw [wedge,fushia] (\i) to (\j);
       \end{scope}

    \end{tikzpicture}
    \caption{Una gr\'afica $G$ (izquierda), resaltando dos subgr\'aficas
    inducidas, ${\color{baige}\boldsymbol {H_1}}$ y
    ${\color{azulMetal}\boldsymbol {H_2}}$. $F_3(G)$ (derecha), resaltando su
    subgr\'afica inducida ${\color{fushia}\boldsymbol {F_1(H_1) \square
    F_2(H_2)}}$.}
    \label{fig:ex-cart}
    \end{figure}