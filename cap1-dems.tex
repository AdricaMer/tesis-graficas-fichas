\chapter{Algunos Teoremas}%
\label{cap:ejemplos}

\section{Teoremas y demostraciones de conexidad}%
\label{sec:etiquetas}

Todos los ambientes que se desee referir por n\'umero m\'as adelante deben de
tener una etiqueta.  Consideremos por ejemplo el siguiente lema.

\begin{teorema}%
\label{teo:diametro F(G)}
Sea $G$ una gr\'afica conexa con di\'ametro $d$. Entonces, $F_{k}(G)$ es 
conexa con di\'ametro al menos $k(d -k+1)$ y a lo m\'as $d k$.
\end{teorema}

\begin{proof}
Sean $A$ y $B$ v\'ertices de $F_{k}(G)$. Primero nos enfocamos en la cota
superior. Por definici\'on tenemos que $|A \triangle B| \leq |A \cup B|$, con
igualdad cuando $A \cap B = \varnothing$. Observamos que, al ser $A$ y $B$
v\'ertices de $F_{k}(G)$, tenemos que $|A|=k$ y $|B|=k$ por lo que $|A \cup B|
\le 2k$. Entonces, tenemos que $|A \triangle B| \leq 2k$, por lo que
$\frac{1}{2} |A \triangle B| \leq k$.

Buscamos demostrar que el di\'ametro de $F_{k}(G)$ es a lo m\'as $d k$, por
lo que basta demostrar, por inducci\'on, que para cualesquiera dos v\'ertices
$A$ y $B$ de $F_{k}(G)$ hay una $AB$-trayectoria de a lo m\'as
$\frac{d}{2}|A\triangle B|$. Observamos que esto tambi\'en implica que
$F_{k}(G)$ es conexa.

Si $A\triangle B=\varnothing$, entonces $A=B$ por lo que no hay nada que probar.
Ahora consideramos $A$ y $B$ tales que $A\triangle B \neq \varnothing$. Tomamos
como hip\'otesis que para cualesquiera dos v\'ertices de $F_{k}(G)$, $C$ y $D$,
tales que $|C\triangle D|<|A \triangle B|$, existe una $CD$-trayectoria con
longitud a lo m\'as $\frac{d}{2}|C\triangle D|$. Al tomar $A\triangle B
\neq \varnothing$ tenemos un v\'ertice de $G$ en $A\setminus B$ y un v\'ertice
en $B\setminus A$, que denotamos $a$ y $b$ respectivamente. Dado que el
di\'ametro de $G$ es $d$, entonces hay una $ab$-trayectoria de longitud a
lo m\'as $d$, digamos $P$.

Definimos $A'=(A\setminus \{a\})\cup \{b\}$ y la trayectoria $A\xrightarrow[P]{}
A'$ en $F_{k}(G)$. Observamos que, por un lado $b\in B\cap A'$ y $b\notin B\cap
A$, pero $b\in A\cup B$. Por otro lado tenemos que $a\notin A'$ por lo que
$a\notin A'\cup B$ y $a\notin A\cap B$, pero $a\in A\cup B$. Entonces, tenemos
que $a,b \in A\triangle B$ y $a,b \notin A'\triangle B$. Ahora tomamos $v\in A$
tal que $v \neq a$. Entonces, tenemos que $v \in A\triangle B$ si y s\'olo si
$v\in A'\triangle B$. Por lo tanto tenemos que $|A'\triangle B|=|A \triangle B|-
2$. Por hip\'otesis inductiva, sabemos que hay una $A'B$-trayectoria en
$F_{k}(G)$ de longitud a lo m\'as $\frac{d}{2}|A'\triangle B|$, que como se
observ\'o anteriormente, coincide con $\frac{d}{2}|A\triangle B| - d$.

Sabemos que $A\xrightarrow[P]{} A'$ tiene la misma longitud que $P$, que es a lo
m\'as $d$. Entonces, tenemos una $AB$-trayectoria de la forma $A\rightarrow
A'\rightarrow B$ que tiene longitud a lo m\'as $\frac{d}{2}|A\triangle
B|-d +d =\frac{d}{2}|A\triangle B|$. Por lo tanto tenemos que
$F_{k}(G)$ es conexa y tiene di\'ametro a lo m\'as $d k$.

Ahora demostraremos la cota inferior. Sabemos que $G$ es una gr\'afica conexa
con di\'ametro $d$, por lo que existen vertices que est\'an a distancia
$d$, digamos $x$ y $y$. Ahora construimos una partici\'on de $V$ usando  la
distancia que tiene cada v\'ertice a $x$. Es decir, para cada $i\in [0,d]$,
sea $V_{i}$ el conjunto de v\'ertices de $G$ a distancia $i$ de $x$. Entonces,
tenemos que $V_{0}=\{x\}$ y $y\in V_{d}$. Denotamos $d_x(v)$ a la distancia
entre $x$ y el v\'ertice $v$.

Sea $a$ el m\'\i{}nimo \'\i{}ndice para el cu\'al se tiene $k \leq |V_{0}\cup
V_{1}\cup \dots \cup V_{a}|$ y sea $b$ el m\'aximo \'\i{}ndice para el cu\'al se
tiene $k\leq |V_{b}\cup V_{b+1}\cup \dots \cup V_{d}|$. Tomamos $A$ un
$k$-subconjunto de $V_{0}\cup \dots \cup V_{a}$  tal que $A\subseteq
V_{0}$ o $V_{0}\cup \dots V_{a-1}\subseteq A$. Tomamos $B$ un
$k$-subconjunto de $V_{b}\cup \dots \cup V_{d}$ tale que
$B\subseteq V_{d}$ o $V_{b+1}\cup \dots \cup V_{d}$. 

Consideramos cualquier trayectoria entre $A$ y $B$ en $F_{k}(G)$. Cualquier
ficha inicialmente en $A$, digamos en el v\'ertice $v$ de $G$, se mueve a
alg\'un v\'ertice en $B$, digamos el v\'ertice $v'$ de $G$. Observamos que todas
las aristas de $G$ est\'an dentro de alg\'un $V_{i}$ o a lo m\'as entre alg\'un
$V_{i}$ y $V_{i+1}$, con $i\in[0,d]$. Entonces, para la ficha en $v$ se
necesitan al menos $d_x(v')-d_x(v)$ movimientos para llegar a $v'$, ocupando
s\'olo las aristas entre $V_{i}$ y $V_{i+1}$, $i\in [0,d]$. Por lo tanto,
el di\'ametro de $F_{k}(G)$ es al menos $\sum_{v\in A}(d_x(v')-d_x(v))=
\sum_{w\in B}d_x(w)-\sum_{v\in A}d_x(v)$. Observamos que, al ser $G$ conexa,
toda $V_{i}$ tiene al menos un elemento y por construcci\'on $V_{i} \cap
V_{i+1}=\varnothing$, para toda $i\in [0,d]$. Tomamos el caso en el que
$|V_{i}|=1$ para toda $i\in [0,d]$. Entonces, tenemos que $k\leq
|V_{b}\cup\dots\cup V_{d}|=|V_{b}|+|V_{b+1}|+\cdots +|V_d|$
$=\sum_{b}^{d}1 = d -b+1$. An\'alogamente tenemos que $k\leq
|V_{0}\cup V_{1}\cup \dots V_{a}|=|V_{0}|+|V_{1}|+\cdots + |V_{a}|$
$=\sum_{0}^{a} 1 = a+1$ En ambos casos la cota m\'\i{}nima se alcanza en la
igualdad, por lo que tomamos $a=k-1$ y $b=d-k+1$. Por lo tanto tenemos que
el di\'ametro de $F_{k}(G)$ es al menos $\sum_{j=d -k+1}^{d}j -
\sum_{i=0}^{k-1}i = k(d-k+1)$.
\end{proof}


\begin{lema}%
\label{lem:relacion trayectorias int. ajenas de G y F(G)}
Sea $A$ un $k$-conjunto en la gr\'afica $G$ y $a, b \in V(G)$ tales que $a \in
A$ y $b \notin A$. Sea $A' = (A \setminus \{ a \}) \cup \{ b \}$. Si $P$ y $Q$
son $ab$-trayectorias internamente ajenas en $G$, entonces $A \xrightarrow[P]{}
A'$ y $A \xrightarrow[Q]{} A'$ son trayectorias internamente ajenas en
$F_{k}(G)$.
\end{lema}

\begin{proof}
    Primero, supongamos que $|V(P) \cap A| \geq 2$, con $V(P) \cap A = \{v_{1},
    v_{2}, \dots , v_{p}\}$, con $v_{1} = a$. Notemos que, si $k > p$, entonces
    $k-p$ fichas no est\'an sobre $P$, por lo que est\'an est\'aticas en todos
    los v\'ertices de $A \xrightarrow[P]{} A'$. Ahora, consideremos $R$ un
    v\'ertice interno de $A \xrightarrow[P]{} A'$. Por la observaci\'on anterior
    tenemos que  $|R \cap V(P)| = p$. Por construcci\'on de $A \xrightarrow[P]{}
    A'$, $R$ tiene una ficha en la trayectoria $(v_{p},b ]$. Entonces, $R$ no
    contiene al conjunto $\{v_{2}, \dots, v_{p}\}$. Por otro lado, como
    $\{v_{2}, \dots, v_{p}\}$ est\'a en $A \cap V(P)$, y $Q$ y $P$ son
    internamente ajenas, entonces $\{v_{2}, \dots, v_{p}\}$ est\'an est\'aticos
    en cada v\'ertice de $A \xrightarrow[Q]{} A'$. Por lo tanto tenemos que $A
    \xrightarrow[P]{} A'$ y $A \xrightarrow[Q]{}A'$ son trayectorias
    internamente ajenas. Al suponer que $|A \cap Q| \geq 2$, tenemos un caso
    an\'alogo al anterior.

    Ahora, supongamos que $|A \cap V(P)| = 1$ y $|A \cap V(Q)| = 1$, es decir,
    $A \cap V(P) = \{a\} = A \cap V(Q)$. Sin p\'erdida de generalidad suponemos
    que $P$ no es la arista $ab$. Entonces, tenemos que $P \setminus \{a,b\}
    \neq \varnothing$. Por lo tanto, tenemos que todo v\'ertice interno de $A
    \xrightarrow[P]{} A'$ tiene alg\'un v\'ertice de $P \setminus \{a, b\}$. Por
    otro lado, ya que $P$ y $Q$ son internamente ajenos y s\'olo comparten el
    v\'ertice $a$ con $A$, ning\'un v\'ertice interno de $A \xrightarrow[Q]{}
    A'$ tiene un v\'ertice de $P \setminus \{a, b\}$. Luego, cada v\'ertice
    interno de ambas trayectorias tiene intersecci\'on vac\'\i{}a con $A$.
    Concluimos que $A \xrightarrow[P]{} A'$ y $A \xrightarrow[Q]{} A'$ son
    internamente ajenas.
\end{proof}

\begin{lema}%
\label{lem:azul-rojo graf. bipartita}
    Sea $H$ una gr\'afica bipartita completa con clases de color $Y$ y $Z$,
    donde $|Y|<|Z|$. Si las aristas de $H$ est\'an coloreadas de azul y rojo de
    manera que cada v\'ertice de $Y$ es incidente en a lo m\'as una arista roja,
    entonces $H$ tiene un conjunto $M$ de aristas azules tal que cada v\'ertice
    en $Y$ incide en exactamente una arista de $M$. Adem\'as, la uni\'on de
    aristas rojas y aristas de $M$ es ac\'\i{}clica.
\end{lema}

\begin{proof}
    Sea $H$ una gr\'afica bipartita completa con clases de color $Y$ y $Z$ como
    se especificaron. Demostramos el resultado por inducci\'on sobre $|Y|$. 

    Primero, supongamos que $Y=\{y\}$. Como $H$ es bipartita completa, existe
    una arista azul $e$ tal que $y$ es incidente en $e$.   Sea $M = \{ e \}$.
    Por otro lado, a lo m\'as existe una arista roja incidente en $y$, digamos
    $e'$. Entonces la uni\'on de aristas rojas y aristas en $M$ es $\{e, e'\}$,
    que no es un ciclo.

    Ahora supongamos que $|Y|>1$. Como tenemos que $|Y|<|Z|$, y hay a lo m\'as
    una arista roja incidente en cada v\'ertice de $Y$, entonces existe alg\'un
    $x \in Z$ tal que no tiene aristas rojas incidentes. Sea $v \in Y$ y sea
    $e$, en caso de existir, la arista roja incidente en $v$. Tomamos $H'=
    (H-v)-x$ con $R'$ el conjunto de aristas rojas de $H'$. Observamos que $Y' =
    Y- v$ y $Z'= Z- x$ son las clases de color de $H'$. Por hip\'otesis de
    inducci\'on, existe un conjunto $M'$ de aristas azules incidentes en $H'$
    tal que cada v\'ertice de $Y'$ incide exactamente en una arista de $M'$.
    Adem\'as $R'\cup M'$ es ac\'\i{}clico.
    
    En $H$, definimos $M = M'\cup \{xv\}$. Al tomar $x$ sin aristas rojas
    tenemos que $M$ cumple tener una arista azul por v\'ertice en $Y$. Tambi\'en
    definimos $R= R'\cup \{ e \}$, si es que existe $e$.  Dado que  $M'\cup R'$
    es ac\'\i{}clica y las aristas $\{vx\}$ y $e$, en caso de existir, no
    est\'an en $M'\cup R'$, entonces tenemos que $M \cup R$ es ac\'\i{}clica.
\end{proof}

\begin{lema}%
\label{lem:trayectorias int. ajenas}
    Sea $G$ una gr\'afica $t$-conexa. Sean $A$ y $B$ v\'ertices de $F_{k}(G)$
    tales que $|A \triangle B| = 2$. Entonces hay $t$ $AB$-trayectorias
    internamente ajenas en $F_{k}(G)$. Adem\'as, si $t \geq k$, entonces hay
    $k(t- k + 1)$ $AB$-trayectorias internamente ajenas en $F_{k}(G)$.
\end{lema}

\begin{proof}
    Sea $G$ una gr\'afica $t$-conexa y sean $A$ y $B$ v\'ertices de $F_{k}(G)$
    tales que $|A \triangle B| = 2$. Primero buscamos probar el n\'umero de
    $AB$-trayectorias internamente ajenas en $F_{k}(G)$ es al menos $t$. 
    
    Sean $a \in A \setminus B$ y $b \in B \setminus A$ los v\'ertices en la
    diferencia sim\'etrica. Al ser $G$ una gr\'afica $t$-conexa, entonces por el
    Teorema General de Menger, existen $P_{1}, \dots, P_{t}$ $ab$-trayectorias
    internamente ajenas en $G$. Por lo tanto, por \cref{lem:relacion
    trayectorias int. ajenas de G y F(G)}, existen $A \xrightarrow[P_1]{}  B,
    \dots, A \xrightarrow[P_t]{}  B$ $AB$-trayectorias internamente ajenas en
    $F_{k}(G)$. 

    Ahora buscamos demostrar que, si $t \geq k$, entonces el n\'umero de
    $AB$-trayectorias internamente ajenas en $F_{k}(G)$ es al menos $k(t- k
    +1)$. Si $t=k$, entonces tenemos que $k(t - k + 1) = t(t-t+1) = t$ por lo
    que tenemos el caso anterior. Por lo tanto consideramos  $t \geq k + 1$. Sea
    $\mathcal{P}$ un conjunto m\'aximo de $ab$-trayectorias internamente ajenas
    en $G$. Por el Teorema General de Menger sabemos que $|\mathcal{P}| \ge t$.
    Elegimos un conjunto $\mathcal{P}$ para el que ninguna trayectoria tenga
    cuerdas. Definimos a las trayectorias $P_{1}, \dots, P_{l}$ como aquellas en
    $\mathcal{P}$ que no intersectan a $A \cap B$ y $Q_{1}, \dots, Q_{s}$ las
    trayectorias en $\mathcal{P}$ que intersectan a $A \cap B$. Entonces $l + s
    = |\mathcal{P}| \ge t$.

    Definimos a $C$ como el conjunto de v\'ertices en $A \cap B$ que intersectan
    alg\'un $Q_i$, con $i \in \{1, \dots, s\}$. Observamos que, al ser $Q_1,
    \dots, Q_s$ internamente ajenas, entonces cada v\'ertice de $C$ esta en
    exactamente una $Q_i$, con $i \in \{1, \dots, s\}$. Definimos a $D$ como el
    conjunto de v\'ertices en $A \cap B$ que no intersecta a $Q_i$ alguna, con
    $i \in \{1, \dots, s\}$. Observamos que $C$ y $D$ inducen una partici\'on
    de $A \cap B$. Entonces, tenemos que $|A\cap B| = |C| + |D| = k-1$. Adem\'as
    podemos ver que $s \leq |C| \leq k-1$ y como $ t - s \leq l$, entonces $l
    \geq t -|C| = t- (k-1-|D|)$.

    Podemos separar las $AB$-trayectorias que consideramos en $F_{k}(G)$ en tres
    tipos. Los primeros dos tipos son las trayectorias obtenidas
    d\cref{lem:relacion trayectorias int. ajenas de G y F(G)}, es decir, las
    trayectorias $A \xrightarrow[P_1]{}  B, \dots, A \xrightarrow[P_l]{}  B$ y
    $A \xrightarrow[Q_1]{}  B, \dots, A \xrightarrow[Q_s]{}  B$. Nombramos a
    estos tipos de $AB$-trayectorias trayectorias de tipo $P$ y de tipo $Q$,
    respectivamente. Notamos que, por como se defini\'o, toda $AB$-trayectoria
    de tipo $P$ no pasa por $A\cap B$, por lo que cada trayectoria de este tipo
    corresponde a la sucesi\'on de fichas obtenidas al mover la ficha de $a$ a
    trav\'es de $P_i$ hacia $b$, con $i \in \{1, \dots, l\}$.
    
    Las \'ultimas trayectorias a considerar son las del tipo $R$, que ahora
    construimos utilizando los vecinos de los v\'ertices en $A \cap B$. Primero
    consideramos $v \in C$, por lo que $v \in Q_i$ para un  $i \in \{1, \dots,
    s\}$. Definimos $Y_v = N_G(v) \setminus ((A \cap B) \cup V(Q_i))$. Dado que
    $G$ es una gr\'afica $t$-conexa, $d_G(v) \geq t$. A su vez, $|A \cap B| =k
    -1$ y $v \in (A \cap B) \setminus N_G(v)$. Por \'ultimo, como $Q_i$ no tiene
    cuerdas, $v$ s\'olo tiene dos vecinos en $Q_i$. Por lo tanto tenemos que
    $|Y_v| \geq t- (k-2)-2 = t-k$. 
    
    Ahora, sea $v \in D$ y definimos $Y_v = N_G(v) \setminus (A \cup B)$. Al ser
    $G$ una gr\'afica $t$-conexa, se cumple $d_G(v) \geq t$. Adem\'as tenemos
    que $|A \cup B| = k + 1$ y $v \in (A \cup B) \setminus N_G(v)$. Ahora,
    notemos que $(a, v, b)$ no es una trayectoria en $G$. \'Esto pues de lo
    contrario tendr\'i{}amos una trayectoria cuyo \'unico v\'ertice interno no
    est\'a en $P_i$ alguna ni $Q_j$ alguna, con $i \in \{1, \dots, l\}$ y $j \in
    \{1, \dots, s\}$, por lo que la trayectoria no estar\'i{}a en $\mathcal{P}$.
    Por lo que concluimos que $a \notin N_G(v)$ o $b \notin N_G(v)$. Por lo
    tanto tenemos que $|Y_v| \geq t- (k-1) = t-k + 1$. Elegimos a $Y_v '$,
    alg\'un subconjunto de $Y_v$ con cardinalidad $t-k$ si $v \in C$ y $t- k+ 1$
    si $v \in D$. Notamos que $Y_v ' \neq \varnothing$ porque tomamos $t \geq k
    + 1$, adem\'as tenemos que $a, b \notin Y_v '$ por definici\'on. 

    Sea $H_v$ la gr\'afica bipartita completa con clases de colores $Y_v '$ y
    $\{1,\dots, l\}$. Definimos la siguiente coloraci\'on, si $y \in P_i$, para
    alg\'un $i \in \{1, \dots, l\}$ y $y \in Y_v '$, entonces coloreamos la
    arista $iy$ de rojo. Coloreamos el resto de las aristas de la gr\'afica de
    azul. Ahora veamos si la coloraci\'on definida cumple las hip\'otesis
    d\cref{lem:azul-rojo graf. bipartita} Primero tenemos que por construcci\'on
    de los $P_i$, con $i \in \{1, \dots, l\}$, cada v\'ertice de $Y_v '$ est\'a
    en a lo m\'as un $P_i$, por lo que cada v\'ertice de $Y_v '$ incide en a lo
    m\'as una arista roja. Luego, notemos que $|\{1, \dots, l\}| = l  \geq t-k+
    1+ |D| > t-k + |D|$. Si $v \in D$, entonces $|D| \geq 1$ por lo que $l \geq
    t- k+1 = |Y_v '|$. Si $v \in C$, tenemos que $l \geq t-k = |Y_v '|$.
    Entonces podemos usar \cref{lem:azul-rojo graf. bipartita} en $H_v$ con $Z=
    \{1, \dots, l\}$ y $Y = Y_v '$. Por lo tanto hay un conjunto $M_v$ de
    aristas azules tal que cada v\'ertice en $Y$ incide en exactamente una
    arista de $M_v$ y la uni\'on de aristas rojas y aristas de $M_v$ es
    ac\'\i{}clica. Notamos que $|M_v|=|Y|$.

    Usando $M_v$ construimos las trayectorias de tipo $R$ de la siguiente
    manera. Para cada $jx \in M_v$ defininimos $R\langle v, x \rangle$ como la
    trayectoria en $F_k(G)$ que corresponde a mover la ficha en $v$ hacia $x$,
    luego recorrer la trayectoria $A\setminus \{v\} \xrightarrow[P_j]{}
    (A\setminus \{v\})'$ y por \'ultimo mover la ficha de $x$ hacia $v$. Notemos
    que las fichas en $(A\cap B)\setminus \{v\}$ estan estacionarias. Adem\'as,
    cada v\'ertice en $R\langle v,x \rangle$ se conforma por $((A\cap
    B)\setminus \{v\}) \cup \{x\}$ y alg\'un v\'ertice $y \in P_i$.

    Falta ver que cada trayectoria de tipo $R$ es internamente ajena al resto de
    las trayectorias. Primero veamos que las trayectorias de tipo $R$ son
    internamente ajenas dos a dos. Supongamos que existen $R \langle v, x
    \rangle$ y $R\langle v',x' \rangle$ tales que $(v,x) \neq (v',x')$ pero
    comparten un v\'ertice interno. Entonces tenemos que $ix \in M_v$ y $i'x'\in
    M_{v'}$, con $i, i' \in \{1, \dots, l\}$. Por construcci\'on de las
    trayectorias de tipo $R$ tenemos que $((A\cap B)\setminus \{v\}) \cup \{x,
    y\} =((A\cap B)\setminus \{v'\}) \cup \{x', y'\}$ para alg\'un $y \in P_i$ y
    $y' \in P_{i'}$. Al tener $x'\in Y_{v'} '$, y ya que $((A \cap B )\setminus
    \{v\}) \cap Y_{v'}'= \varnothing$, tenemos que $x' \in \{x,y\}$.
    An\'alogamente, como $(A \cap B) \cap P_i = \varnothing$, tenemos que $y'\in
    \{x, y\}$. Por lo tanto tenemos $\{x,y\}= \{x',y'\}$, lo cual implica que
    $(A\cap B)\setminus \{v\} = (A\cap B)\setminus \{v'\}$, entonces $v = v'$.
    \'Esto implica que $ix, i'x' \in M_v$, pero por construcci\'on cada
    v\'ertice de $Y_v '$ es incidente en s\'olo una arista de $M_v$, por lo que
    $x \neq x'$ y $i \neq i'$. Como tenemos que $\{x, y\}=\{x', y'\}$, entonces
    $x=y'$ y $y=x'$. Entonces tenemos que $x \in P_{i'}$ y $x'\in P_i$, lo cual
    implica que $xi'$ y $x'i$ son aristas rojas en $H_v$. Por lo tanto tenemos
    el ciclo $(x, i, x', i)$ con aristas de color azul, rojo, azul, rojo
    respectivamente, lo cual es una contradicci\'on. As\'i pues, las
    trayectorias de tipo $R$ son internamenta ajenas dos a dos.

    Ahora veamos que las trayectorias de tipo $R$ y las de tipo $P$ son
    internamente ajenas entre s\'i{}. Sean $R\langle v,x \rangle$ y $A
    \xrightarrow[P_i]{}  B$, para alg\'un $i \in \{1, \dots, l\}$, trayectorias
    en $F_k(G)$ de tipo $R$ y $P$, respectivamente. Por construcci\'on, $v$ no
    est\'a en v\'ertice interno alguno de $R \langle v,x \rangle$. Por otro
    lado, como $ v \in A\cap B$, entonces $v$ est\'a est\'atico en $A
    \xrightarrow[P_i]{}  B$, es decir, $v$ est\'a en cada v\'ertice de la
    trayectoria. Por lo tanto tenemos que las trayectorias $R\langle v,x
    \rangle$ y $A \xrightarrow[P_i]{}  B$, para alg\'un $i \in \{1, \dots, l\}$,
    son internamente ajenas.

    Por \'ultimo, veamos que las trayectorias de tipo $Q$ y las de tipo $R$ son
    internamente ajenas entre s\'i{}. Sean $R\langle v,x \rangle$ y $A
    \xrightarrow[Q_i]{}  B$, para alg\'un $i \in \{1, \dots, s\}$, trayectorias
    en $F_k(G)$ de tipo $R$ y $Q$, respectivamente. Sea $jx$ arista en $M_v$,
    con $j \in \{1, \dots, l\}$, entonces $x \notin P_j$. Si tomamos $v \notin
    Q_i$, entonces $v$ est\'a est\'atico en $A \xrightarrow[Q_i]{} B$, es decir,
    est\'a en cada v\'ertice de la trayectoria. Por otro lado, por
    construcci\'on tenemos que $v$ no est\'a en los v\'ertices internos de $R
    \langle v, x \rangle$. Por lo tanto las trayectorias son internamente
    ajenas. Ahora consideramos el caso en el que $v \in Q_i$, es decir $v \in
    C$. Por como definimos $R \langle v,x \rangle$, $x$ es est\'a en cada
    v\'ertice interno de la trayectoria. Por otro lado, como $x \in Y_v'$, y por
    construcci\'on, tenemos que $Y_v ' \cap ((A\cap B) \cup Q_i) = \varnothing$,
    entonces $x \notin ((A \cap B) \cup Q_i)$. Por definici\'on, cada v\'ertice
    de $A \xrightarrow[Q_i]{}  B$ est\'a contenido en $((A \cap B) \cup Q_i)$,
    entonces $x$ no esta en los v\'ertices internos de $A \xrightarrow[Q_i]{}
    B$. Por lo tanto tenemos que las trayectorias $R \langle v,x \rangle$ y $A
    \xrightarrow[Q_i]{}  B$ son internamente ajenas.

    Tenemos que hay $l$ y $s$ trayectorias de tipo $P$ y $Q$ respectivamente.
    Adem\'as, para cada $v \in C$ hay $t-k$ trayectorias de tipo $R$ y para cada
    $v \in D$ hay $t-k+1$ trayectorias de tipo $R$. Entonces en $F_k(G)$ hay $l+
    s+ |C|(t-k)+ |D|(t-k +1) = l + s + (|C| + |D|)(t-k) + |D| = l + s +
    (|k-1)(t-k) + |D|$ trayectorias internamente ajenas. Despejando tenemos que
    $l + s + (k-1)(t-k) + |D| \geq t+ (k-1)(t-k) = k (t -k +1)$. Por lo tanto el
    n\'umero de $AB$-trayectorias en $F_k(G)$ es al menos $k(t-k+1)$.
\end {proof}

% \begin{figure}[ht!]
%     \centering

%     \tikzset{every picture/.style={line width=0.75pt}} 

%     \begin{tikzpicture}[x=0.75pt,y=0.75pt,yscale=-1,xscale=1]
    
%     \begin{scope}[xshift=3cm,yshift=4.5cm,scale=1]
%     \foreach \i in {0,...,5}
%         \draw ({(360/6)*\i}:50) node(\i)[wvertex]{\pgfmathparse{int(\i+1)}
%         \pgfmathresult};
%     \foreach \i/\j in {0/1,0/2,0/3,0/5,1/2,1/4,1/5,2/3,2/4,2/5,3/4,3/5,4/5}
%         \draw [edge] (\i) to (\j);
%     \end{scope}


%     \draw    (481.08, 347.35) circle [x radius= 10.5, y radius= 10.5]   ; \draw
%     (481.08,341.85) node [anchor=north] [inner sep=0.75pt]  [font=\scriptsize]
%     [align=left] {123};
    
%     \draw    (418.03, 205.42) circle [x radius= 10.5, y radius= 10.5]   ; \draw
%     (418.03,199.92) node [anchor=north] [inner sep=0.75pt]  [font=\scriptsize]
%     [align=left] {124};
    
%     \draw    (306.57, 281.07) circle [x radius= 10.5, y radius= 10.5]   ; \draw
%     (306.57,275.57) node [anchor=north] [inner sep=0.75pt]  [font=\scriptsize]
%     [align=left] {234};
    
%     \draw    (401.14, 347.91) circle [x radius= 10.5, y radius= 10.5]   ; \draw
%     (401.14,342.41) node [anchor=north] [inner sep=0.75pt]  [font=\scriptsize]
%     [align=left] {125};
    
%     \draw    (299.31, 84.77) circle [x radius= 10.5, y radius= 10.5]   ; \draw
%     (299.31,79.27) node [anchor=north] [inner sep=0.75pt]  [font=\scriptsize]
%     [align=left] {135};
    
%     \draw    (492.42, 220.97) circle [x radius= 10.5, y radius= 10.5]   ; \draw
%     (492.42,215.47) node [anchor=north] [inner sep=0.75pt]  [font=\scriptsize]
%     [align=left] {235};
    
%     \draw    (577.93, 89.36) circle [x radius= 10.5, y radius= 10.5]   ; \draw
%     (577.93,83.86) node [anchor=north] [inner sep=0.75pt]  [font=\scriptsize]
%     [align=left] {136};
    
%     \draw    (538.83, 323) circle [x radius= 10.5, y radius= 10.5]   ; \draw
%     (538.83,317.5) node [anchor=north] [inner sep=0.75pt]  [font=\scriptsize]
%     [align=left] {126};
    
%     \draw    (278.83, 218.89) circle [x radius= 10.5, y radius= 10.5]   ; \draw
%     (278.83,213.39) node [anchor=north] [inner sep=0.75pt]  [font=\scriptsize]
%     [align=left] {134};
    
%     \draw    (275.59, 146.39) circle [x radius= 10.5, y radius= 10.5]   ; \draw
%     (275.59,140.89) node [anchor=north] [inner sep=0.75pt]  [font=\scriptsize]
%     [align=left] {145};
    
%     \draw    (407.18, 131.33) circle [x radius= 10.5, y radius= 10.5]   ; \draw
%     (407.18,125.83) node [anchor=north] [inner sep=0.75pt]  [font=\scriptsize]
%     [align=left] {146};
    
%     \draw    (538.83, 43.25) circle [x radius= 10.5, y radius= 10.5]   ; \draw
%     (538.83,37.75) node [anchor=north] [inner sep=0.75pt]  [font=\scriptsize]
%     [align=left] {156};
    
%     \draw    (604.47, 146.35) circle [x radius= 10.5, y radius= 10.5]   ; \draw
%     (604.47,140.85) node [anchor=north] [inner sep=0.75pt]  [font=\scriptsize]
%     [align=left] {236};
    
%     \draw    (345.16, 324.59) circle [x radius= 10.5, y radius= 10.5]   ; \draw
%     (345.16,319.09) node [anchor=north] [inner sep=0.75pt]  [font=\scriptsize]
%     [align=left] {245};
    
%     \draw    (577.42, 281.66) circle [x radius= 10.5, y radius= 10.5]   ; \draw
%     (577.42,276.16) node [anchor=north] [inner sep=0.75pt]  [font=\scriptsize]
%     [align=left] {246};
    
%     \draw    (605.66, 209.01) circle [x radius= 10.5, y radius= 10.5]   ; \draw
%     (605.66,203.51) node [anchor=north] [inner sep=0.75pt]  [font=\scriptsize]
%     [align=left] {256};
    
%     \draw    (343.39, 45.32) circle [x radius= 10.5, y radius= 10.5]   ; \draw
%     (343.39,39.82) node [anchor=north] [inner sep=0.75pt]  [font=\scriptsize]
%     [align=left] {345};
    
%     \draw    (403.16, 16.89) circle [x radius= 10.5, y radius= 10.5]   ; \draw
%     (403.16,11.39) node [anchor=north] [inner sep=0.75pt]  [font=\scriptsize]
%     [align=left] {346};
    
%     \draw    (509.83, 146.87) circle [x radius= 10.5, y radius= 10.5]   ; \draw
%     (509.83,141.37) node [anchor=north] [inner sep=0.75pt]  [font=\scriptsize]
%     [align=left] {356};
    
%     \draw    (481.83, 18.38) circle [x radius= 10.5, y radius= 10.5]   ; \draw
%     (481.83,12.88) node [anchor=north] [inner sep=0.75pt]  [font=\scriptsize]
%     [align=left] {456};

%     % Connection
%     \draw    (475.11,333.91) -- (424,218.86) ;
%     % Connection
%     \draw    (467.33,342.13) -- (320.32,286.29) ;
%     % Connection
%     \draw    (466.38,347.46) -- (415.85,347.81) ;
%     % Connection
%     \draw    (472.71,335.26) -- (307.68,96.86) ;
%     % Connection
%     \draw    (482.39,332.71) -- (491.11,235.61) ;
%     % Connection
%     \draw    (486.25,333.58) -- (572.76,103.13) ;
%     % Connection
%     \draw    (494.63,341.64) -- (525.27,328.72) ;
%     % Connection
%     \draw    (403.4,206.84) -- (293.46,217.48) ;
%     % Connection
%     \draw    (404.44,199.79) -- (289.18,152.02) ;
%     % Connection
%     \draw    (405.86,213.68) -- (318.74,272.81) ;
%     % Connection
%     \draw    (416.3,220.03) -- (402.87,333.3) ;
%     % Connection
%     \draw    (428.57,215.68) -- (528.29,312.74) ;
%     % Connection
%     \draw    (431.3,211.77) -- (564.15,275.31) ;
%     % Connection
%     \draw    (387.56,342.25) -- (358.73,330.25) ;
%     % Connection
%     \draw    (395.83,334.19) -- (304.62,98.49) ;
%     % Connection
%     \draw    (409.73,335.97) -- (483.84,232.91) ;
%     % Connection
%     \draw    (413.31,339.65) -- (593.49,217.27) ;
%     % Connection
%     \draw    (415.61,345.29) -- (524.35,325.62) ;
%     % Connection
%     \draw    (548.86,312.25) -- (567.38,292.41) ;
%     % Connection
%     \draw    (538.83,308.3) -- (538.83,57.95) ;
%     % Connection
%     \draw    (541.25,308.5) -- (575.5,103.87) ;
%     % Connection
%     \draw    (543.95,309.21) -- (599.35,160.14) ;
%     % Connection
%     \draw    (284.82,232.33) -- (300.57,267.64) ;
%     % Connection
%     \draw    (278.17,204.2) -- (276.24,161.08) ;
%     % Connection
%     \draw    (286.53,206.37) -- (395.45,29.42) ;
%     % Connection
%     \draw    (290.97,210.61) -- (395.04,139.62) ;
%     % Connection
%     \draw    (292.32,213.05) -- (564.43,95.2) ;
%     % Connection
%     \draw    (281.05,204.35) -- (297.09,99.31) ;
%     % Connection
%     \draw    (294.02,98.5) -- (280.87,132.66) ;
%     % Connection
%     \draw    (311.33,93.24) -- (480.41,212.49) ;
%     % Connection
%     \draw    (313.42,88.93) -- (495.72,142.71) ;
%     % Connection
%     \draw    (314.01,85.01) -- (563.23,89.12) ;
%     % Connection
%     \draw    (313.8,82.26) -- (524.34,45.76) ;
%     % Connection
%     \draw    (310.27,74.96) -- (332.44,55.12) ;
%     % Connection
%     \draw    (584.14,102.69) -- (598.26,133.02) ;
%     % Connection
%     \draw    (568.42,78.14) -- (548.34,54.46) ;
%     % Connection
%     \draw    (564.34,83.73) -- (416.75,22.53) ;
%     % Connection
%     \draw    (563.65,92.87) -- (421.47,127.82) ;
%     % Connection
%     \draw    (283.78,134.18) -- (335.2,57.53) ;
%     % Connection
%     \draw    (288.08,138.63) -- (469.34,26.13) ;
%     % Connection
%     \draw    (289.28,141.02) -- (525.13,48.61) ;
%     % Connection
%     \draw    (290.2,144.72) -- (392.57,133) ;
%     % Connection
%     \draw    (409.32,145.88) -- (415.9,190.87) ;
%     % Connection
%     \draw    (418.21,141.06) -- (566.39,271.93) ;
%     % Connection
%     \draw    (406.67,116.63) -- (403.68,31.59) ;
%     % Connection
%     \draw    (419.41,123.15) -- (526.6,51.42) ;
%     % Connection
%     \draw    (525.35,37.36) -- (495.31,24.26) ;
%     % Connection
%     \draw    (534.86,57.41) -- (513.79,132.71) ;
%     % Connection
%     \draw    (544.33,56.89) -- (600.16,195.37) ;
%     % Connection
%     \draw    (316.32,292.07) -- (335.4,313.59) ;
%     % Connection
%     \draw    (280.94,160.09) -- (339.81,310.89) ;
%     % Connection
%     \draw    (308.84,266.54) -- (341.12,59.85) ;
%     % Connection
%     \draw    (311.62,267.25) -- (398.11,30.71) ;
%     % Connection
%     \draw    (320.56,276.54) -- (478.43,225.49) ;
%     % Connection
%     \draw    (319.97,275.01) -- (591.07,152.41) ;
%     % Connection
%     \draw    (321.27,281.1) -- (562.71,281.63) ;
%     % Connection
%     \draw    (495.79,206.65) -- (506.47,161.19) ;
%     % Connection
%     \draw    (480.4,229.43) -- (357.18,316.13) ;
%     % Connection
%     \draw    (504.67,212.81) -- (592.23,154.51) ;
%     % Connection
%     \draw    (604.75,161.06) -- (605.38,194.31) ;
%     % Connection
%     \draw    (601.59,160.78) -- (580.3,267.24) ;
%     % Connection
%     \draw    (589.76,146.43) -- (524.53,146.79) ;
%     % Connection
%     \draw    (359.62,321.92) -- (562.95,284.33) ;
%     % Connection
%     \draw    (358.6,318.63) -- (592.22,214.98) ;
%     % Connection
%     \draw    (351.15,311.16) -- (475.84,31.81) ;
%     % Connection
%     \draw    (347.88,310.14) -- (400.43,31.35) ;
%     % Connection
%     \draw    (582.75,267.95) -- (600.33,222.72) ;
%     % Connection
%     \draw    (572.4,267.83) -- (486.85,32.2) ;
%     % Connection
%     \draw    (569.33,269.37) -- (411.24,29.18) ;
%     % Connection
%     \draw    (593.32,201.01) -- (522.17,154.87) ;
%     % Connection
%     \draw    (356.68,39) -- (389.88,23.21) ;
%     % Connection
%     \draw    (357.83,42.51) -- (467.39,21.18) ;
%     % Connection
%     \draw    (355.95,52.98) -- (497.27,139.21) ;
%     % Connection
%     \draw    (417.86,17.17) -- (467.13,18.1) ;
%     % Connection
%     \draw    (412.49,28.26) -- (500.5,135.5) ;
%     % Connection
%     \draw    (506.7,132.5) -- (484.96,32.75) ;
%     \end{tikzpicture}
% \caption{El diagrama $F_3(G)$.}
% \label{fig:lema1.1.4}
% \end{figure}


\begin{figure}[ht!]
\centering
   \begin{tikzpicture}
    
    \begin{scope}[xshift=0cm,yshift=0cm,scale=1]
       \foreach \i in {0,...,15} \draw ({(360/16)*\i}:4)
           node(\i)[wvertex]{};

        \draw ({(360/16)*1}:4.5) node (e1) {{\footnotesize $236$}};
        \draw ({(360/16)*2}:4.5) node (e1) {{\footnotesize $256$}};
        \draw ({(360/16)*3}:4.5) node (e1) {{\footnotesize $246$}};
        \draw ({(360/16)*4}:4.5) node (e1) {{\footnotesize $126$}};
        \draw ({(360/16)*5}:4.5) node (e1) {{\footnotesize $123$}};
        \draw ({(360/16)*6}:4.5) node (e1) {{\footnotesize $125$}};
        \draw ({(360/16)*7}:4.5) node (e1) {{\footnotesize $245$}};
        \draw ({(360/16)*8}:4.5) node (e1) {{\footnotesize $234$}};
        \draw ({(360/16)*9}:4.5) node (e1) {{\footnotesize $134$}};
        \draw ({(360/16)*10}:4.5) node (e1) {{\footnotesize $145$}};
        \draw ({(360/16)*11}:4.5) node (e1) {{\footnotesize $135$}};
        \draw ({(360/16)*12}:4.5) node (e1) {{\footnotesize $345$}};
        \draw ({(360/16)*13}:4.5) node (e1) {{\footnotesize $346$}};
        \draw ({(360/16)*14}:4.5) node (e1) {{\footnotesize $456$}};
        \draw ({(360/16)*15}:4.5) node (e1) {{\footnotesize $156$}};
        \draw ({(360/16)*16}:4.5) node (e1) {{\footnotesize $136$}};

        \foreach\i/\j in{0/1,1/2,2/3,3/4,4/5,5/6,6/7,7/8,8/9,9/10,10/11,
           11/12,12/13,13/14, 14/15, 15/0} 
           \draw [edge] (\i) to (\j);
        
        \foreach\i/\j in{0/2,0/3,1/5,1/6,1/14,2/6,2/7,3/5,3/14,3/15,4/7,
           4/15,5/10,6/9,6/12,7/11,8/10,8/12,9/11,9/13,9/14,10/14,10/15,11/14,
           12/15} 
           \draw [edge] (\i) to (\j);
       

        \draw (-0.8,0) node (17) [vertex, label=30:{\tiny $124$}] {};
        \draw (0.2,-0.4) node (18) [vertex, label=90:{\tiny $146$}] {};
        \draw (1.1,0.3) node (19) [vertex, label=180:{\tiny $356$}] {};
        \draw (0.4,1.2) node (20) [vertex, label=87:{\tiny $235$}] {};
                    

   \end{scope}

        \foreach \i/\j in{1/8,3/13,3/14,5/11,7/14,9/0,17/3,17/4,17/5,17/6,17/8,
        17/9,17/10,17/18,18/10,18/13,18/0,19/1,19/2,19/11,19/12,19/13,19/14,19/15,
        19/20,20/1,20/5,20/6,20/7,20/8,20/11}
          \draw [edge] (\i) to (\j);

\end{tikzpicture}
\end{figure}


\begin{teorema}%
\label{teo:F(G) t-conexa}
    Sea $G$ una gr\'afica $t$-conexa. Entonces $F_{k}(G)$ es $t$-conexa para
    todo $k>1$
\end{teorema}
        
\begin{proof}
Primero, observamos que $A$ y $B$ son v\'ertices adyacentes en $F_k(G)$ si y
s\'olo si $V(G) \setminus A$ y $V(G)\setminus B$ son adyacentes en $F_{n-k}(G)$.
Entonces tenemos que $F_k(G) \simeq F_{n-k}(G)$. Por lo tanto asumimos sin
p\'erdida de generalidad que $k \leq \frac{n}{2}$.

Sea $\mathcal{C}$ un m\'i{}nimo corte por v\'ertices de $F_k(G)$. Basta
demostrar que $|\mathcal{C}| \geq t$. Definimos $A$ y $B$ como los v\'ertices en
distintas componentes de $F_k(G)- \mathcal{C}$ tales que $|A \triangle B|$ es
m\'i{}nimo.

Si $|A \triangle B| = 2$, entonces por \cref{lem:trayectorias int. ajenas}
tenemos que hay $t$ $AB$-trayectorias internamente ajenas en $F_k(G)$. Por lo
que tenemos que $|\mathcal{C}| \geq t$.

Ahora consideramos $|A \triangle B| = 2r \geq 4$. Definimos $A \setminus B
=\{a_1, \dots, a_r\}$ y $B \setminus A =\{b_1, \dots, b_r\}$. Tambi\'en
definimos $A_{i,x} = A\setminus \{a_i\} \cup \{x\}$ y $B_{j,x} = B\setminus
\{b_j\} \cup \{x\}$ para cada $i \in \{1, \dots, r\}$ y $x \in V(G)\setminus
(A\cup B)$. Supongamos que para algunos $i, j$ y $x$ tenemos que $A_{i,x} \notin
\mathcal{C}$ y $B_{j,x} \notin \mathcal{C}$. Como $A \triangle A_{i,x} = \{a_i,
x\}$, entonces tenemos que $|A \triangle A_{i,x}|< |A \triangle B|$ por lo que
$A$ y $A_{i,x}$ est\'an en la misma componente de $F_k(G)- \mathcal{C}$.
An\'alogamente tenemos que  $B$ y $B_{j,x}$ est\'an en la misma componente de
$F_k(G)-\mathcal{C}$

Luego nos fijamos en que $A_{i,x} \triangle B_{j,x} = (A \triangle B) \setminus
\{a_i, b_j\}$. Por lo que $|A_{i,x} \triangle B_{j,x}| = 2(r-1)$. As\'i{}
tenemos que $A_{i,x}$ y $B_{j,x}$ est\'an en la misma componente de $F_k(G)-
\mathcal{C}$. Pero tenemos que $A$ y$A_{i,x}$ est\'an en la misma componente, de
$F_K(G) - \mathcal{C}$, de igual manera que $B$ y $B_{j,x}$. Por lo tanto $A$ y
$B$ est\'an en la misma componente de $F_k(G)- \mathcal{C}$. Esto es una
cotradicci\'on pues tomamos a $A$ y $B$ en distintas componentes, lo que implica
que $A_{i,x}$ o $B_{j,x}$ est\'a en $\mathcal{C}$, para todas las $i,j$ y $x$.

Por lo anterior tenemos que, para cada $x \in V(G)\setminus (A \cup B)$,
$\mathcal{C}$ tiene todos los $\{A_i,x | i \in \{1, \dots, r\}\}$ o todos los
$\{B_j,x | j \in \{1, \dots, r\}\}$. Dado que $|A\cup B|=k +r$, sabemos que
$\mathcal{C}$ tiene al menos $r(n-k-r)$ v\'ertices.

Ahora definimos $A_{i,j} = (A\setminus \{a_i\}) \cup \{b_j\}$ y $B_{i,j} =
(B\setminus \{b_i\}) \cup \{a_j\}$, para toda $ i, j \in \{1, \dots, r\}$.
Notamos que hay $2r^2$ de estos conjuntos. Ahora supongamos que $A_{i,j} \notin
\mathcal{C}$, para algunos $ i, j \in \{1, \dots, r\}$. Entonces tenemos que $A
\triangle A_{i,j} = \{a_i, b_j\}$. Entonces $A$ y $A_{i,j}$ est\'an en la misma
componente de $F_k(G)- \mathcal{C}$. Por otro lado, $|B \triangle A_{i,j}| = 2
(r-1)$, por lo que $B$ y $A_{i,j}$ est\'an en la misma componente de $F_k(G) -
\mathcal{C}$. Por lo tanto tenemos que $A$ y $B$ est\'an en la misma componente
de $F_k(G)-\mathcal{C}$, lo cu\'al nos lleva a la misma contradicci\'on que el
caso anterior. Por lo tanto tenemos que $A_{i,j} \in \mathcal{C}$ para toda $i,
j \in \{1, \dots, r\}$. De manera an\'aloga $B_{i,j} \in \mathcal{C}$. Por lo
tanto tenemos que $\mathcal{C}$ tiene $2r^2$ v\'ertices, adem\'as de los del
caso anterior.

As\'i, tenemos que $|\mathcal{C}|\geq r(n-k-r)+2r^2 = r(n-k) + r^2$, pero $r
\geq 2$ y $k \leq \frac{n}{2}$, por lo que tenemos que $|\mathcal{C}| \geq
r(n-k)+r^2 > r(n-k) \geq 2(n-k) \geq n >t$. Por lo tanto $|\mathcal{C}|>t$.
\end{proof} 

\begin{teorema}%
    \label{teo:F(G) k (t- k+ 1)-conexa}
        Sea $G$ una gr\'afica $t$-conexa con $t \ge k$ y $n \ge \frac{1}{2} kt$.
        Entonces $F_{k}(G)$ es $k (t- k+ 1)$-conexa.
    \end{teorema}

    \begin{proof}
        Sea $\mathcal{C}$ un corte por v\'ertices m\'i{}nimo de $F_k(G)$. Sean
        $A$ y $B$ v\'ertices en distintas componentes de $F_k(G)- \mathcal{C}$
        tales que $|A \triangle B|$ es m\'i{}nima.

        Si $|A \triangle B| = 2$, entonces por \cref{lem:trayectorias int. ajenas} tenemos que
        $F_k(G)$ tiene $k (t- k+ 1)$ $AB$-trayectorias. Por lo que tenemos que
        $|\mathcal{C}| \geq k (t- k+ 1)$.

        Ahora veamos el caso en el que $|A \triangle B| = 2r \ge 4$. Utilizando
        la desigualdad que se utiliza al final de la demostraci\'on
        d\cref{teo:F(G) t-conexa} tenemos que $|\mathcal{C}| \ge r(n-k-r)+2r^2$. Dado
        que $r \ge 2$ y $n \ge \frac{1}{2}kt$, entonces tenemos que
        $|\mathcal{C}| \ge r(n-k-r)+2r^2 \ge 2 (n- k -2) + 8 \ge tk - 2k+ 4$.
        Observamos que $k^2 -3k + 4 \ge 0$ para toda $k \ge 1$. Entonces tenemos
        que $-2k+4 \ge -k^2 + k$, esto implica que $kt -2k +4 \ge kt - k^2 + k =
        k (t - k +1)$. Entonces tenemos que $|\mathcal{C}| \ge tk -2k +4 \ge
        k(t-k+1)$. Por lo tanto $F_k(G)$ es $k(t-k+1)$-conexa.
    \end{proof}