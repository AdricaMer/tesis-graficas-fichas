\chapter{Conexidad}%
\label{cap:conexidad}

Este cap\'itulo se enfoca en la conexidad de las gr\'aficas de fichas. Se
empieza por estudiar el n\'umero de trayectorias internamente ajenas que se
pueden encontrar en una gr\'afica de fichas, bas\'andose en las
caracter\'isticas y par\'ametros de las gr\'aficas de las cu\'ales provienen.
Despu\'es pasa a estudiar la relaci\'on entre la conexidad de una gr\'afica y su
gr\'afica de fichas. Todas las proposiciones del cap\'itulo salen del art\'iculo
\textit{Token graphs} \cite{fabilaToken}. El art\'iculo proporciona
demostraciones de dichas proposiciones, en las cuales se basan las
demostraciones de este cap\'itulo.


\section{Trayectorias internamente ajenas}%
\label{sec:TrayIntAj}

El objetivo de este cap\'itulo es estudiar la relaci\'on de la conexidad entre
una gr\'afica y su gr\'afica de $k$-fichas. Para esto, empezamos enfoc\'andonos
en determinar la cantidad de trayectorias internamente ajenas entre un par de
v\'ertices en una gr\'afica de fichas. Veremos que esta cantidad est\'a
relacionada con la conexidad de la gr\'afica original. Adem\'as, para gr\'aficas
con conexidad suficientemente grande, podemos encontrar un n\'umero a\'un mayor
de trayectorias internamente ajenas. Para llegar a estos resultados, los
siguientes dos lemas nos son de utilidad.

\begin{lema}%
\label{lem:TrayIntAj-G-FG}
    Sea $A$ un $k$-conjunto en la gr\'afica $G$ y $a, b \in V(G)$, tales que $a
    \in A$ y $b \notin A$. Sea $A' = (A \setminus \{ a \}) \cup \{ b \}$. Si $P$
    y $Q$ son $ab$-trayectorias internamente ajenas en $G$, entonces $A
    \xrightarrow[P]{} A'$ y $A \xrightarrow[Q]{} A'$ son trayectorias
    internamente ajenas en $F_{k}(G)$.
\end{lema}

\begin{proof}
    Primero, suponemos que la cardinalidad del conjunto $V(P) \cap A$ es al
    menos $2$, con $V(P) \cap A = \{v_{1}, v_{2}, \dots, v_{p}\}$ y $v_{1} = a$.
    Notamos que, si $k > p$, entonces $k-p$ fichas no est\'an sobre $P$, por lo
    que est\'an est\'aticas en todos los v\'ertices de $A \xrightarrow[P]{} A'$.
    Ahora, consideramos $R$ un v\'ertice interno de $A \xrightarrow[P]{} A'$.
    Por la observaci\'on anterior, tenemos que  $|R \cap V(P)| = p$. Adem\'as,
    $R$ tiene una ficha en la trayectoria $(v_{p},b ]$, por construcci\'on de $A
    \xrightarrow[P]{} A'$ . Por ello, $R$ no contiene al conjunto $\{v_{2},
    \dots, v_{p}\}$. Por otro lado, como $\{v_{2}, \dots, v_{p}\}$ est\'a en $A
    \cap V(P)$, y $Q$ y $P$ son internamente ajenas, entonces las fichas de
    $\{v_{2}, \dots, v_{p}\}$ est\'an est\'aticos en cada v\'ertice de $A
    \xrightarrow[Q]{} A'$. Por lo tanto, tenemos que $A \xrightarrow[P]{} A'$ y
    $A \xrightarrow[Q]{}A'$ son trayectorias internamente ajenas. Al suponer que
    $|A \cap Q| \geq 2$, tenemos un caso an\'alogo al anterior.

    Suponemos que $|A \cap V(P)| = 1$ y $|A \cap V(Q)| = 1$, i.~e., $A \cap V(P)
    = \{a\} = A \cap V(Q)$. Sin p\'erdida de generalidad, suponemos que $P$ no
    es la arista $ab$, entonces tenemos que $P \setminus \{a,b\} \neq
    \varnothing$. Por lo tanto, tenemos que todo v\'ertice interno de $A
    \xrightarrow[P]{} A'$ tiene alg\'un v\'ertice de $P \setminus \{a, b\}$. Por
    otra parte, ya que $P$ y $Q$ son internamente ajenas y s\'olo comparten el
    v\'ertice $a$ con $A$, ning\'un v\'ertice interno de $A \xrightarrow[Q]{}
    A'$ tiene un v\'ertice de $P \setminus \{a, b\}$. Luego, cada v\'ertice
    interno de ambas trayectorias tiene intersecci\'on vac\'ia con $A$.
    Concluimos que $A \xrightarrow[P]{} A'$ y $A \xrightarrow[Q]{} A'$ son
    internamente ajenas.
\end{proof}

Enseguida se presenta el siguiente lema t\'ecnico, que es necesario para
demostrar el resultado principal de esta secci\'on. 

\begin{lema}%
\label{lem:rb-bipGraph}
    Sea $H$ una gr\'afica bipartita completa con clases de color $Y$ y $Z$,
    donde $|Y|<|Z|$. Si las aristas de $H$ est\'an coloreadas de azul y rojo, de
    manera que cada v\'ertice de $Y$ es incidente en a lo m\'as una arista roja,
    entonces $H$ tiene un conjunto $M$ de aristas azules, tal que cada v\'ertice
    en $Y$ incide en exactamente una arista de $M$. Adem\'as, la uni\'on de
    aristas rojas y aristas de $M$ es ac\'iclica.
\end{lema}

\begin{proof}
    Sea $H$ una gr\'afica bipartita completa con clases de color $Y$ y $Z$ como
    se especificaron, notamos que esta coloraci\'on no es propia. Demostramos el
    resultado por inducci\'on sobre $|Y|$. 

    Empezamos suponiendo que $Y=\{y\}$. Como $H$ es bipartita completa, existe
    una arista azul $e$ tal que $y$ es incidente en $e$, de ah\'i que $M = \{ e
    \}$. Por otro lado, a lo m\'as, existe una arista roja incidente en $y$,
    digamos $e'$. Por tanto, la uni\'on de aristas rojas y aristas en $M$ es
    $\{e, e'\}$, que no es un ciclo.

    Luego, suponemos que $|Y|>1$. Como tenemos que $|Y|<|Z|$ y hay, a lo m\'as,
    una arista roja incidente en cada v\'ertice de $Y$, entonces existe alg\'un
    $x \in Z$, tal que no tiene aristas rojas incidentes. Sea $v \in Y$ y sea
    $e$, en caso de existir, la arista roja incidente en $v$. Tomamos $H'=
    (H-v)-x$ con $R'$ el conjunto de aristas rojas de $H'$. Observamos que $Y' =
    Y- v$ y $Z'= Z- x$ son las clases de color de $H'$. Por hip\'otesis de
    inducci\'on, existe un conjunto $M' \subseteq E(H')$ de aristas azules, tal
    que cada v\'ertice de $Y'$ incide exactamente en una arista de $M'$.
    Adem\'as, $R'\cup M'$ es ac\'iclico.
    
    En $H$ definimos $M = M'\cup \{xv\}$. Al tomar $x$ sin aristas rojas,
    tenemos que $M$ cumple tener una arista azul por v\'ertice en $Y$.
    Tambi\'en, definimos $R= R'\cup \{ e \}$, si es que existe $e$.  Dado que
    $M'\cup R'$ es ac\'iclica y las aristas $\{vx\}$ y $e$, en caso de existir,
    no est\'an en $M'\cup R'$, entonces tenemos que $M \cup R$ es ac\'iclico.
\end{proof}

\begin{figure}[ht!]
    \centering
       \begin{tikzpicture}
    
        \begin{scope}[xshift=-5.5cm]
            \draw (0,2) node (1) [vertex,fill=baige ,label=90:{\footnotesize $v_1$}] {};
            \draw (0,1) node (2) [vertex,fill=baige ,label=270:{\footnotesize $v_2$}] {};
            \draw (0,-1) node (3) [vertex,fill=baige ,label=90:{\footnotesize $v_3$}] {};
            \draw (0,-2) node (4) [vertex,fill=baige ,label=270:{\footnotesize $v_4$}] {};
            \draw (1,0) node (5) [vertex,fill=crema ,label=0:{\footnotesize $v_5$}] {};
            \draw (-1,0) node (6) [vertex,fill=crema ,label=180:{\footnotesize $v_6$}] {};
            \draw (2,0) node (7) [vertex,fill=crema ,label=0:{\footnotesize $v_7$}] {};
            \draw (-2,0) node (8) [vertex,fill=crema ,label=180:{\footnotesize $v_8$}] {};
            \draw (3,0) node (9) [vertex,fill=crema ,label=0:{\footnotesize $v_9$}] {};
            \draw (-3,0) node (10) [vertex,fill=crema ,label=180:{\footnotesize $v_{10}$}] {};

            \foreach \i/\j in{1/5,1/6,1/7,1/10,2/5,2/6,2/8,2/10,
            3/5,3/6,3/8,3/9,3/10,4/5,4/6,4/7,4/9} 
                \draw [nedge,azulMetal!60] (\i) to (\j);
            
            \foreach \i/\j in{1/9,2/7,4/8} 
                \draw [wedge,fushia] (\i) to (\j);
            
            \foreach \i/\j in{1/8,2/9,3/7,4/10} 
                \draw [wedge,azulMetal] (\i) to (\j);
        \end{scope}
        
    \end{tikzpicture}

\caption{Una gr\'afica bipartita completa, resaltando la uni\'on de aristas
${\color{fushia} \bf rojas}$ y ${\color{azulMetal}\bf azules}$ mencionadas en
\cref{lem:rb-bipGraph}.}
\label{fig:ex-rb-bioGraph}       
\end{figure}

Para aclarar \cref{lem:rb-bipGraph}, \cref{fig:ex-rb-bioGraph} muestra una
gr\'afica bipartita completa con bipartici\'on $({\color{crema}
\boldsymbol{X}},{\color{baige} \boldsymbol {Y}})$, con ${\color{crema}
\boldsymbol {X}}=\{v_5, v_6,v_7, v_8, v_9, v_{10}\}$ y ${\color{baige}
\boldsymbol {Y}}=\{v_1, v_2, v_3, v_4\}$. Sea ${\color{azulMetal} \boldsymbol
{M=\{v_1v_8, v_2v_9,v_3v_7,v_4v_{10}\}}}$ el conjunto de aristas azules
d\cref{lem:rb-bipGraph}, y resaltadas en la figura. Notamos que, todo v\'ertice
en ${\color{baige} \boldsymbol {Y}}$ tiene exactamente una arista en
${\color{azulMetal}\boldsymbol {M}}$. Adem\'as, la uni\'on de las aristas de
${\color{azulMetal}\boldsymbol {M}}$ y las aristas rojas, no forma un ciclo.

Ahora pasamos a demostrar el lema principal de esta secci\'on.

\begin{lema}%
\label{lem:TrayIntAj}
    Sea $G$ una gr\'afica $t$-conexa y sean $A$ y $B$ v\'ertices de $F_{k}(G)$,
    tales que $|A \triangle B| = 2$, entonces hay $t$ $AB$-trayectorias
    internamente ajenas en $F_{k}(G)$. Adem\'as, si $t \geq k$, entonces hay
    $k(t- k + 1)$ $AB$-trayectorias internamente ajenas en $F_{k}(G)$.
\end{lema}

\begin{proof}
    Sea $G$ una gr\'afica $t$-conexa y sean $A$ y $B$ v\'ertices de $F_{k}(G)$,
    tales que $|A \triangle B| = 2$. Empezamos por probar que el n\'umero de
    $AB$-trayectorias internamente ajenas en $F_{k}(G)$ es, al menos, $t$. 
    
    Sean $a \in A \setminus B$ y $b \in B \setminus A$ los v\'ertices en la
    diferencia sim\'etrica. Al ser $G$ una gr\'afica $t$-conexa, entonces, por
    el Teorema General de Menger, existen $P_{1}, \dots, P_{t}$
    $ab$-trayectorias internamente ajenas en $G$. Por lo tanto, por
    \cref{lem:TrayIntAj-G-FG}, existen $A \xrightarrow[P_1]{}  B, \dots, A
    \xrightarrow[P_t]{}  B$ $AB$-trayectorias internamente ajenas en $F_{k}(G)$. 

    Ahora buscamos demostrar que, si $t \geq k$, entonces el n\'umero de
    $AB$-trayectorias internamente ajenas en $F_{k}(G)$ es, al menos, $k(t- k
    +1)$. Si $t=k$, entonces tenemos que $k(t - k + 1) = t(t-t+1) = t$, por lo
    que tenemos el caso anterior. Por lo tanto, consideramos  $t \geq k + 1$.
    Sea $\mathcal{P}$ un conjunto m\'aximo de $ab$-trayectorias internamente
    ajenas en $G$. Por el Teorema General de Menger, sabemos que $|\mathcal{P}|
    \ge t$. Elegimos un conjunto $\mathcal{P}$ para el que ninguna trayectoria
    tenga cuerdas. Definimos a las trayectorias $P_{1}, \dots, P_{l}$ como
    aquellas en $\mathcal{P}$ que no intersectan a $A \cap B$ y $Q_{1}, \dots,
    Q_{s}$ las trayectorias en $\mathcal{P}$ que intersectan a $A \cap B$. Por
    lo que $l + s = |\mathcal{P}| \ge t$.

    Definimos a $C$ como el conjunto de v\'ertices en $A \cap B$ que intersectan
    alg\'un $Q_i$, con $i \in \{1, \dots, s\}$. Observamos que, al ser $Q_1,
    \dots, Q_s$ internamente ajenas y sin cuerdas, entonces cada v\'ertice de
    $C$ est\'a en exactamente una $Q_i$, con $i \in \{1, \dots, s\}$. Definimos
    a $D$ como el conjunto de v\'ertices en $A \cap B$ que no intersecta a $Q_i$
    alguna, con $i \in \{1, \dots, s\}$. Observamos que $C$ y $D$ inducen una
    partici\'on de $A \cap B$. Por ello, tenemos que $|A\cap B| = |C| + |D| =
    k-1$. Adem\'as podemos ver que $s \leq |C| \leq k-1$ y, como $ t - s \leq
    l$, entonces $l \geq t -|C| = t- (k-1-|D|)$.

    Podemos separar en tres tipos a las $AB$-trayectorias que consideramos en
    $F_{k}(G)$. Los primeros dos tipos son las trayectorias obtenidas
    d\cref{lem:TrayIntAj-G-FG}, es decir, las trayectorias $A
    \xrightarrow[P_1]{}  B, \dots, A \xrightarrow[P_l]{}  B$ y $A
    \xrightarrow[Q_1]{}  B, \dots, A \xrightarrow[Q_s]{}  B$. Nombramos a estos
    tipos de $AB$-trayectorias trayectorias de tipo $P$ y de tipo $Q$,
    respectivamente. Notamos que, por como est\'an definidas, toda
    $AB$-trayectoria de tipo $P$ no pasa por $A\cap B$, por lo que cada
    trayectoria de este tipo corresponde a la sucesi\'on de fichas obtenidas al
    mover la ficha de $a$ a trav\'es de $P_i$ hacia $b$, con $i \in \{1, \dots,
    l\}$.
    
    El \'ultimo tipo de trayectorias a considerar son las de tipo $R$, que
    construimos a continuaci\'on utilizando los vecinos de los v\'ertices en $A
    \cap B$. Empezamos considerando $v \in C$, as\'i $v \in Q_i$ para un $i \in
    \{1, \dots, s\}$. Definimos $Y_v = N_G(v) \setminus ((A \cap B) \cup
    V(Q_i))$. Dado que $G$ es una gr\'afica $t$-conexa, $d_G(v) \geq t$. A su
    vez, $|A \cap B| =k -1$ y $v \in (A \cap B) \setminus N_G(v)$. Por \'ultimo,
    como $Q_i$ no tiene cuerdas, $v$ s\'olo tiene dos vecinos en $Q_i$. Por lo
    tanto, tenemos que $|Y_v| \geq t- (k-2)-2 = t-k$. 
    
    Ahora, sea $v \in D$ y definimos $Y_v = N_G(v) \setminus (A \cup B)$. Al ser
    $G$ una gr\'afica $t$-conexa, se cumple $d_G(v) \geq t$. Adem\'as, $|A \cup
    B| = k + 1$ y $v \in (A \cup B) \setminus N_G(v)$. Ahora, notemos que $(a,
    v, b)$ no es una trayectoria en $G$. Esto pues, de lo contrario,
    tendr\'iamos una trayectoria cuyo \'unico v\'ertice interno no est\'a en
    $P_i$ ni $Q_j$, con $i \in \{1, \dots, l\}$ y $j \in \{1, \dots, s\}$, por
    lo que la trayectoria no estar\'ia en $\mathcal{P}$. Por lo que concluimos
    que $a \notin N_G(v)$ o $b \notin N_G(v)$. Por lo tanto, tenemos que $|Y_v|
    \geq t- (k-1) = t-k + 1$. Elegimos a $Y_v '$, alg\'un subconjunto de $Y_v$
    con cardinalidad $t-k$ si $v \in C$ y $t- k+ 1$ si $v \in D$. Notamos que
    $Y_v ' \neq \varnothing$, porque tomamos $t \geq k + 1$. Adem\'as, tenemos
    que $a, b \notin Y_v '$, por definici\'on. 

    Sea $H_v$ la gr\'afica bipartita completa con clases de colores $Y_v '$ y
    $\{1,\dots, l\}$. Luego, definimos la siguiente coloraci\'on, si $y \in
    P_i$, para alg\'un $i \in \{1, \dots, l\}$ y $y \in Y_v '$, entonces
    coloreamos la arista $iy$ de rojo y coloreamos el resto de las aristas de la
    gr\'afica de azul. Ahora, pasamos a comprobar que la coloraci\'on definida
    cumple las hip\'otesis d\cref{lem:rb-bipGraph}. Primero, notamos que, por
    construcci\'on de los $P_i$, con $i \in \{1, \dots, l\}$, cada v\'ertice de
    $Y_v '$ est\'a en, a lo m\'as, un $P_i$, por lo que cada v\'ertice de $Y_v
    '$ incide en, a lo m\'as, una arista roja. A continuaci\'on, notemos que
    $|\{1, \dots, l\}| = l  \geq t-k+ 1+ |D| > t-k + |D|$. Si $v \in D$,
    entonces $|D| \geq 1$, por lo que $l \geq t- k+1 = |Y_v '|$. Si $v \in C$,
    tenemos que $l \geq t-k = |Y_v '|$. As\'i, podemos usar
    \cref{lem:rb-bipGraph} en $H_v$ con $Z= \{1, \dots, l\}$ y $Y = Y_v '$. Por
    lo tanto, hay un conjunto $M_v$ de aristas azules tal que cada v\'ertice en
    $Y$ incide en exactamente una arista de $M_v$ y la uni\'on de aristas rojas
    y aristas de $M_v$ es ac\'iclica. Notamos que $|M_v|=|Y|$.

    Usando $M_v$, construimos las trayectorias de tipo $R$ de la siguiente
    manera. Para cada $jx \in M_v$, defininimos $R\langle v, x \rangle$ como la
    trayectoria en $F_k(G)$ que corresponde a mover la ficha en $v$ hacia $x$,
    luego recorrer la trayectoria $A\setminus \{v\} \xrightarrow[P_j]{}
    (A\setminus \{v\})'$, donde $(A\setminus \{v\})' = (A\setminus \{v,a\})\cup
    \{b\}$, y por \'ultimo, mover la ficha de $x$ hacia $v$. Notemos que las
    fichas en $(A\cap B)\setminus \{v\}$ est\'an estacionarias. Adem\'as, cada
    v\'ertice en $R\langle v,x \rangle$ se conforma por $((A\cap B)\setminus
    \{v\}) \cup \{x\}$ y alg\'un v\'ertice $y \in P_i$.

    Falta ver que cada trayectoria de tipo $R$ es internamente ajena al resto de
    las trayectorias. Primero, veamos que las trayectorias de tipo $R$ son
    internamente ajenas dos a dos. Supongamos que existen $R \langle v, x
    \rangle$ y $R\langle v',x' \rangle$, tales que $(v,x) \neq (v',x')$ pero
    comparten un v\'ertice interno. Luego, tenemos que $ix \in M_v$ y $i'x'\in
    M_{v'}$, con $i, i' \in \{1, \dots, l\}$. Por construcci\'on de las
    trayectorias de tipo $R$, tenemos que $((A\cap B)\setminus \{v\}) \cup \{x,
    y\} =((A\cap B)\setminus \{v'\}) \cup \{x', y'\}$, para alg\'un $y \in P_i$
    y $y' \in P_{i'}$. Al tener $x'\in Y_{v'} '$, adem\'as de que $((A \cap B
    )\setminus \{v\}) \cap Y_{v'}'= \varnothing$, tenemos que $x' \in \{x,y\}$.
    An\'alogamente, como $(A \cap B) \cap P_i = \varnothing$, tenemos que $y'\in
    \{x, y\}$. Por lo tanto, tenemos $\{x,y\}= \{x',y'\}$. Esto implica que
    $(A\cap B)\setminus \{v\} = (A\cap B)\setminus \{v'\}$, as\'i $v = v'$.
    Se sigue que $ix, i'x' \in M_v$, pero, por construcci\'on, cada
    v\'ertice de $Y_v '$ es incidente en s\'olo una arista de $M_v$, por lo que
    $x \neq x'$ y $i \neq i'$. Como tenemos que $\{x, y\}=\{x', y'\}$, entonces
    $x=y'$ y $y=x'$. Por ello, tenemos que $x \in P_{i'}$ y $x'\in P_i$, lo
    cu\'al implica que $xi'$ y $x'i$ son aristas rojas en $H_v$. Por lo tanto,
    tenemos el ciclo $(x, i, x', i)$ con aristas de color azul, rojo, azul, rojo
    respectivamente, lo cu\'al es una contradicci\'on. As\'i pues, las
    trayectorias de tipo $R$ son internamente ajenas dos a dos.

    Pasamos a demostrar que los tipos de trayectorias formadas son internamente
    ajenas dos a dos. Observamos que, las trayectorias de tipo $Q$ y de tipo $P$
    son internamente ajenas por construcci\'on. As\'i pues, comenzamos con las
    trayectorias de tipo $R$ y las de tipo $P$. Sean $R\langle v,x \rangle$ y $A
    \xrightarrow[P_i]{}  B$, para alg\'un $i \in \{1, \dots, l\}$, trayectorias
    en $F_k(G)$ de tipo $R$ y $P$, respectivamente. Por construcci\'on, $v$ no
    est\'a en v\'ertice interno alguno de $R \langle v,x \rangle$. Por otro
    lado, como $ v \in A\cap B$, entonces $v$ est\'a est\'atico en $A
    \xrightarrow[P_i]{}  B$, es decir, $v$ est\'a en cada v\'ertice de la
    trayectoria. Por lo tanto, tenemos que las trayectorias $R\langle v,x
    \rangle$ y $A \xrightarrow[P_i]{}  B$, para alg\'un $i \in \{1, \dots, l\}$,
    son internamente ajenas.

    Ahora, nos enfocamos en las trayectorias de tipo $Q$ y las de tipo $R$. Sean
    $R\langle v,x \rangle$ y $A \xrightarrow[Q_i]{}  B$, para alg\'un $i \in
    \{1, \dots, s\}$, trayectorias en $F_k(G)$ de tipo $R$ y $Q$,
    respectivamente. Sea $jx$ arista en $M_v$, con $j \in \{1, \dots, l\}$,
    entonces $x \notin P_j$. Si tomamos $v \notin Q_i$, entonces $v$ est\'a
    est\'atico en $A \xrightarrow[Q_i]{} B$, es decir, est\'a en cada v\'ertice
    de la trayectoria. Por otro lado, por construcci\'on, tenemos que $v$ no
    est\'a en los v\'ertices internos de $R \langle v, x \rangle$. Por lo tanto,
    las trayectorias son internamente ajenas. Luego, consideramos el caso en el
    que $v \in Q_i$, es decir, $v \in C$. Por como definimos $R \langle v,x
    \rangle$, $x$ est\'a en cada v\'ertice interno de la trayectoria. Por otro
    lado, como $x \in Y_v'$, y por construcci\'on, tenemos que $Y_v ' \cap
    ((A\cap B) \cup Q_i) = \varnothing$, entonces $x \notin ((A \cap B) \cup
    Q_i)$. Por definici\'on, cada v\'ertice de $A \xrightarrow[Q_i]{}  B$ est\'a
    contenido en $((A \cap B) \cup Q_i)$, entonces $x$ no est\'a en los
    v\'ertices internos de $A \xrightarrow[Q_i]{} B$. Por lo tanto, tenemos que
    las trayectorias $R \langle v,x \rangle$ y $A \xrightarrow[Q_i]{}  B$ son
    internamente ajenas.

    Hay $l$ y $s$ trayectorias de tipo $P$ y $Q$, respectivamente. Adem\'as,
    para cada $v \in C$, hay $t-k$ trayectorias de tipo $R$ y, para cada $v \in
    D$, hay $t-k+1$ trayectorias de tipo $R$. Por lo que la cantidad de
    trayectorias internamente ajenas en $F_k(G)$ es
    \[
    l+ s+ |C|(t-k)+ |D|(t-k +1) = l + s + (|C| + |D|)(t-k) + |D|
    = l + s + (|k-1)(t-k) + |D|.
    \]
     Despejando, tenemos que $l + s + (k-1)(t-k) + |D| \geq t+ (k-1)(t-k) = k (t -k
    +1)$. Por lo tanto, el n\'umero de $AB$-trayectorias en $F_k(G)$ es, al
    menos, $k(t-k+1)$.

\end{proof}

Con esto queda demostrado el resultado principal de la secci\'on. Para
ejemplificar los tres tipos de trayectorias mencionados en \cref{lem:TrayIntAj},
\cref{fig:ex-tok-path}, mostrada a continuaci\'on, exhibe una gr\'afica $G$ del
lado izquierdo y su gr\'afica de $3$-fichas del lado derecho. Ahora, construimos
una trayectoria de cada tipo. Consideramos ${\color{baige}\boldsymbol {A=
\{1,2,3\}}}$ y ${\color{baige}\boldsymbol {B=\{2,3,5\}}}$ los v\'ertices en
$F_3(G)$ sobre los que resaltamos los distintos tipos de trayectorias
internamente ajenas obtenidas en el lema anterior. Tenemos que ${\color{baige}
\boldsymbol {A\triangle B=\{1,5\}}}$ y $A\cap B=\{2,3\}$, por lo que usamos
$15$-trayectorias para encontrar las trayectorias de tipo $P$ y las de tipo $Q$,
de manera que las de tipo $P$ no tienen a $2$ ni a $3$ como v\'ertice interno y
las de tipo $Q$ tienen, o bien a $2$ o bien a $3$ como v\'ertices internos.
Notamos que hay dos trayectorias de tipo $P$ en $G$, la trayectoria
$P_1=(1,4,5)$ y la trayectoria ${\color{naranja}\boldsymbol {P_2=(1,6,5)}}$. En
\cref{fig:ex-tok-path} se encuentra resaltada $P_2$ de ${\color{naranja} \bf
naranja}$. Tambi\'en podemos observar que hay dos trayectorias de tipo $Q$ en
$G$, la trayectoria ${\color{fushia}\boldsymbol {Q_1(1,3,5)}}$, resaltada en
\cref{fig:ex-tok-path} de ${\color{fushia}\bf rosa}$, y la trayectoria
$Q_2=(1,2,5)$. As\'i, tenemos que los conjuntos auxiliares d\cref{lem:TrayIntAj}
son $C=\{2,3\}$ y $D= \varnothing$. Utilizando $v=3$ y
${\color{menta}\boldsymbol {x=4}}$ del lema y ${\color{naranja}\boldsymbol
{P_2}}$, construimos la trayectoria $ {\color{verdeAzulado}\boldsymbol {R\langle
3, 4 \rangle =\{1,2,3\}\{1,2,4\} \{2,4,6\}\{2,4,5\}\{2,3,5\}}}$.

\begin{figure}[ht!]
\centering
   \begin{tikzpicture}

    \begin{scope}[xshift=-8.5cm,scale=0.9]
        
       
        \foreach \i in {0,4}
            \draw ({(360/6)*\i}:2) node(\i)[wvertex,fill=baige]{};
        \draw ({(360/6)}:2) node(1)[bvertex]{};
        \draw ({(360/6)*2}:2) node(2)[wvertex,fill=fushia]{}; 
        \draw ({(360/6)*3}:2) node(3)[wvertex,fill=menta]{};
        \draw ({(360/6)*5}:2) node(5)[wvertex,fill=naranja]{}; 
        \foreach \i in {0,...,5}
            \draw ({(360/6)*\i}:2.5) node(e0){\pgfmathparse{int(\i+1)}
           \pgfmathresult};
        \foreach \i/\j in {0/3,0/1,1/2,1/4,1/5,2/3,2/5,3/5,3/4}
            \draw [edge,grisOscuro!60] (\i) to (\j);
        \foreach \i/\j in {0/2,2/4}
            \draw [wedge,fushia] (\i) to (\j);
        \foreach \i/\j in {0/5,4/5}
            \draw [wedge,naranja] (\i) to (\j);
        
        \end{scope}
    
    \begin{scope}[xshift=-0.77cm,yshift=0cm,scale=1]

        \draw ({(360/16)*6}:4) node(6)[wvertex,fill=fushia]{}; 
        \foreach \i in {3,7} 
            \draw ({(360/16)*\i}:4) node(\i)[wvertex,fill=verdeAzulado]{}; 
        
        \draw ({(360/16)*5}:4) node(5)[wvertex,fill=baige]{};
        \draw ({(360/16)*1}:4) node(1)[wvertex,fill=naranja]{};     
        \foreach \i in {8,12,4,10,14,11,0,2,9,13,15} 
            \draw({(360/16)*\i}:4) node(\i)[wvertex]{};

        \draw ({(360/16)*1}:4.5) node (e1) {{\footnotesize $236$}};
        \draw ({(360/16)*2}:4.5) node (e1) {{\footnotesize $256$}};
        \draw ({(360/16)*3}:4.5) node (e1) {{\footnotesize $246$}};
        \draw ({(360/16)*4}:4.5) node (e1) {{\footnotesize $126$}};
        \draw ({(360/16)*5}:4.5) node (e1) {{\footnotesize $123$}};
        \draw ({(360/16)*6}:4.5) node (e1) {{\footnotesize $125$}};
        \draw ({(360/16)*7}:4.5) node (e1) {{\footnotesize $245$}};
        \draw ({(360/16)*8}:4.5) node (e1) {{\footnotesize $234$}};
        \draw ({(360/16)*9}:4.5) node (e1) {{\footnotesize $134$}};
        \draw ({(360/16)*10}:4.5) node (e1) {{\footnotesize $145$}};
        \draw ({(360/16)*11}:4.5) node (e1) {{\footnotesize $135$}};
        \draw ({(360/16)*12}:4.5) node (e1) {{\footnotesize $345$}};
        \draw ({(360/16)*13}:4.5) node (e1) {{\footnotesize $346$}};
        \draw ({(360/16)*14}:4.5) node (e1) {{\footnotesize $456$}};
        \draw ({(360/16)*15}:4.5) node (e1) {{\footnotesize $156$}};
        \draw ({(360/16)*16}:4.5) node (e1) {{\footnotesize $136$}};

        \draw (-1.1,0.2) node (17) [vertex,fill=verdeAzulado,label=10:{\tiny $124$}] {};
        \draw (0.1,-0.3) node (18) [vertex,label=70:{\tiny $146$}] {};
        \draw (1.2,0.7) node (19) [vertex,label=180:{\tiny $356$}] {};
        \draw (0.3,1.4) node (20) [vertex,fill=baige ,label=88:{\tiny $235$}] {};
        
        \foreach\i/\j in{0/1,1/2,2/3,3/4,4/5,6/7,7/8,8/9,9/10,10/11,
           11/12,12/13,13/14, 14/15,15/0} 
           \draw [edge,grisOscuro!60] (\i) to (\j);
        \foreach\i/\j in{0/4,0/11,0/13,1/3,1/4,2/6,2/7,3/8,
        4/6,4/15,6/15,7/10,7/12,8/12,8/13,9/11,9/13,10/12,10/14,10/15,
        11/15,12/14} 
           \draw [edge,grisOscuro!60] (\i) to (\j);
        \foreach \i/\j in{1/8,3/13,3/14,5/11,7/14,9/0,9/18,17/4,17/6,
        17/8,17/9,17/10,17/18,3/18,18/10,18/13,19/1,19/11,19/12,
        19/13,19/14,19/15,6/11,11/20,20/2,5/0,0/18,18/15,15/2,2/19,19/20,5/8,8/20}
          \draw [edge,grisOscuro!60] (\i) to (\j);
        \foreach\i/\j in{5/6,20/6} 
           \draw [wedge,fushia] (\i) to (\j);
        \foreach\i/\j in{1/5,20/1} 
           \draw [wedge,naranja] (\i) to (\j);
        \foreach\i/\j in{17/5,17/3,3/7, 20/7} 
          \draw [wedge,verdeAzulado] (\i) to (\j);
        
   \end{scope}

\end{tikzpicture}
\caption{Una gr\'afica $G$ (izquierda) y su gr\'afica de $3$-fichas (derecha),
resaltando las trayectorias internamente ajenas de ${\color{baige}\boldsymbol
{A}}$ y ${\color{baige}\boldsymbol {B}}$.}
\label{fig:ex-tok-path}
\end{figure}

\section{Conexidad}%
\label{sec:conexidad}

Ahora nos enfocamos en la conexidad de las gr\'aficas de fichas. Con ayuda de
los resultados encontrados en la secci\'on anterior, en esta secci\'on buscamos
estudiar la relaci\'on entre la conexidad de una gr\'afica y su gr\'afica de
$k$-fichas.  El siguiente teorema nos muestra la relaci\'on buscada.

\begin{teorema}%
\label{teo:FG-t-conexa}
    Si $G$ es una gr\'afica $t$-conexa, entonces $F_{k}(G)$ es $t$-conexa para
    todo $k>1$.
\end{teorema}
        
\begin{proof}
Primero, observamos que $A$ y $B$ son v\'ertices adyacentes en $F_k(G)$ si y
s\'olo si $V(G) \setminus A$ y $V(G)\setminus B$ son adyacentes en $F_{n-k}(G)$.
De esta manera, tenemos que $F_k(G) \cong F_{n-k}(G)$. Por lo tanto, asumimos,
sin p\'erdida de generalidad, que $k \leq \frac{n}{2}$.

Sea $\mathcal{C}$ un corte m\'inimo por v\'ertices de $F_k(G)$. Basta demostrar
que $|\mathcal{C}| \geq t$. Sean $A$ y $B$ v\'ertices en distintas componentes
de $F_k(G)- \mathcal{C}$, tales que $|A \triangle B|$ es m\'inimo. Si $|A
\triangle B| = 2$, entonces, por \cref{lem:TrayIntAj}, tenemos que hay $t$
$AB$-trayectorias internamente ajenas en $F_k(G)$. Por lo que tenemos que
$|\mathcal{C}| \geq t$.

Ahora, consideramos el caso en el que $|A \triangle B| = 2r \geq 4$. Definimos
$A \setminus B =\{a_1, \dots, a_r\}$ y $B \setminus A =\{b_1, \dots, b_r\}$.
Tambi\'en definimos $A_{i,x} = (A\setminus \{a_i\}) \cup \{x\}$ y 
\linebreak
$B_{j,x} = (B\setminus
\{b_j\}) \cup \{x\}$, para cada $i \in \{1, \dots, r\}$ y $x \in V(G)\setminus
(A\cup B)$. Supongamos que, para algunos $i, j$ y $x$, tenemos que $A_{i,x}
\notin \mathcal{C}$ y $B_{j,x} \notin \mathcal{C}$. Como $A \triangle A_{i,x} =
\{a_i, x\}$, entonces tenemos que $|A \triangle A_{i,x}|< |A \triangle B|$, por
lo que $A$ y $A_{i,x}$ est\'an en la misma componente de $F_k(G)- \mathcal{C}$.
An\'alogamente, tenemos que  $B$ y $B_{j,x}$ est\'an en la misma componente de
$F_k(G)-\mathcal{C}$.

Por otro lado, notemos que $A_{i,x} \triangle B_{j,x} = (A \triangle B)
\setminus \{a_i, b_j\}$. Por lo que $|A_{i,x} \triangle B_{j,x}| = 2(r-1)$.
As\'i, tenemos que $A_{i,x}$ y $B_{j,x}$ est\'an en la misma componente de
$F_k(G)- \mathcal{C}$. Pero, tenemos que $A$ y$A_{i,x}$ est\'an en la misma
componente de $F_K(G) - \mathcal{C}$; de igual manera que $B$ y $B_{j,x}$. Por
lo tanto, $A$ y $B$ est\'an en la misma componente de $F_k(G)- \mathcal{C}$.
Esto es una contradicci\'on, pues tomamos a $A$ y $B$ en distintas componentes,
lo que implica que $A_{i,x}$ o $B_{j,x}$ est\'a en $\mathcal{C}$, para todas las
$i,j$ y $x$.

Por lo anterior, tenemos que, para cada $x \in V(G)\setminus (A \cup B)$,
$\mathcal{C}$ tiene todos los $\{A_i,x \colon\  i \in \{1, \dots, r\}\}$ o todos
los $\{B_j,x \colon\ j \in \{1, \dots, r\}\}$. Dado que $|A\cup B|=k +r$,
sabemos que $\mathcal{C}$ tiene, al menos, $r(n-k-r)$ v\'ertices.

Definimos $A_{i,j} = (A\setminus \{a_i\}) \cup \{b_j\}$ y $B_{i,j} = (B\setminus
\{b_i\}) \cup \{a_j\}$, para toda 
\linebreak
$ i, j \in \{1, \dots, r\}$. Notamos que hay
$2r^2$ de estos conjuntos. Supongamos que $A_{i,j} \notin \mathcal{C}$, para
algunos $ i, j \in \{1, \dots, r\}$, entonces tenemos que $A \triangle A_{i,j} =
\{a_i, b_j\}$. Por tanto, $A$ y $A_{i,j}$ est\'an en la misma componente de
$F_k(G)- \mathcal{C}$. Por otro lado, $|B \triangle A_{i,j}| = 2 (r-1)$, por lo
que $B$ y $A_{i,j}$ est\'an en la misma componente de $F_k(G) - \mathcal{C}$.
Por ende, tenemos que $A$ y $B$ est\'an en la misma componente de
$F_k(G)-\mathcal{C}$, lo cu\'al nos lleva a la misma contradicci\'on que el caso
anterior. Por lo tanto, tenemos que $A_{i,j} \in \mathcal{C}$, para toda $i, j
\in \{1, \dots, r\}$. De manera an\'aloga, $B_{i,j} \in \mathcal{C}$. Por lo
tanto, tenemos que $\mathcal{C}$ tiene $2r^2$ v\'ertices, adem\'as de los del
caso anterior.

As\'i, tenemos que $|\mathcal{C}|\geq r(n-k-r)+2r^2 = r(n-k) + r^2$, pero $r
\geq 2$ y $k \leq \frac{n}{2}$, por lo que tenemos que $|\mathcal{C}| \geq
r(n-k)+r^2 > r(n-k) \geq 2(n-k) \geq n >t$. Por lo tanto, $|\mathcal{C}|>t$.
\end{proof} 

Utilizamos las gr\'aficas de \cref{fig:ex-tConect} para ejemplificar
\cref{teo:FG-t-conexa}. Nos fijamos que $G$, mostrada del lado izquierdo en la
figura, es una gr\'afica $4$-conexa y su gr\'afica de $2$-fichas, mostrada del
lado derecho, es $8$-conexa. En la figura tambi\'en se muestra el conjuntos de
corte de $G$, ${\color{naranja}\boldsymbol{ \{1,3,5,6\}}}$, y el conjunto de
corte de $F_k(G)$, es decir, ${\color{fushia}\boldsymbol
{\{\{1,2\},\{1,3\},\{1,5\},\{1,6\},\{2,4\},\{3,5\},\{3,6\},\{5,6\}\}}}$.
Adem\'as, utilizando \cref{fig:ex-tok-path}, podemos ver que $F_3(G)$ es
$9$-conexa. Por lo tanto, tenemos que $F_2(G)$ y $F_3(G)$ tambi\'en son
gr\'aficas $4$-conexas.

\begin{figure}[ht!]
    \centering
       \begin{tikzpicture}
    
        \begin{scope}[xshift=-8.5cm]
            \foreach \i in {1,3}
                \draw ({(360/6)*\i}:2) node(\i)[vertex,fill=azulMetal]{};

            \foreach \i in {0,2,4,5}
                \draw ({(360/6)*\i}:2) node(\i)[vertex,fill=naranja!75]{};

                
            \foreach \i in {0,...,5}
                \draw ({(360/6)*\i}:2.5) node(e0){\pgfmathparse{int(\i+1)}
               \pgfmathresult};

            \foreach \i/\j in {0/1,0/2,0/3,0/5,1/2,1/4,1/5,2/3,2/4,2/5,3/4,3/5,4/5}
                \draw [edge,grisOscuro!50] (\i) to (\j);
            
            \end{scope}
        
        \begin{scope}[xshift=-1cm,yshift=0cm,scale=1]
        
            \foreach \i in{0,3,7,8,10} \draw ({(360/12)*\i}:3.5)
            node(\i)[wvertex,fill=azulMetal]{};

            \foreach \i in{1,2,4,5,6,9,11} \draw ({(360/12)*\i}:3.5)
            node(\i)[wvertex,fill=fushia!60]{};
    
            \draw ({(360/12)*1}:4) node (e1) {{\footnotesize $13$}};
            \draw ({(360/12)*2}:4) node (e1) {{\footnotesize $12$}};
            \draw ({(360/12)*3}:4) node (e1) {{\footnotesize $26$}};
            \draw ({(360/12)*4}:4) node (e1) {{\footnotesize $16$}};
            \draw ({(360/12)*5}:4) node (e1) {{\footnotesize $15$}};
            \draw ({(360/12)*6}:4) node (e1) {{\footnotesize $56$}};
            \draw ({(360/12)*7}:4) node (e1) {{\footnotesize $46$}};
            \draw ({(360/12)*8}:4) node (e1) {{\footnotesize $45$}};
            \draw ({(360/12)*9}:4) node (e1) {{\footnotesize $35$}};
            \draw ({(360/12)*10}:4) node (e1) {{\footnotesize $34$}};
            \draw ({(360/12)*11}:4) node (e1) {{\footnotesize $24$}};
            \draw ({(360/12)*12}:4) node (e1) {{\footnotesize $23$}};
            \draw (0.3,1) node (13) [vertex,fill=azulMetal, label=10:{\scriptsize $25$}] {};
            \draw (0.9,-1) node (14) [vertex,fill=azulMetal, label=10:{\scriptsize $14$}] {};
            \draw (-0.8,-0.6) node (15) [vertex,fill=fushia!60, label=120:{\scriptsize $36$}] {};
          
            
            \foreach\i/\j in{0/3,0/13,3/13,7/10,7/14,8/10,7/8,10/14} 
            \draw [wedge,azulMetal] (\i) to (\j);

            \foreach\i/\j in{0/1,1/2,1/4,1/5,1/10,1/14,
            1/15,2/3,0/2,2/4,2/5,2/11,4/5,4/7,4/14,4/15,3/4,5/6,6/7,3/6,5/8,5/9,5/13,
            5/14,6/8,6/9,6/13,6/15,8/9,9/10,10/11,0/9,9/13,9/15,0/11,3/11,7/11,8/11,
            11/13,11/14,0/15,3/15,7/15,10/15} 
               \draw [edge,grisOscuro!50] (\i) to (\j);
           
            \end{scope}

    \end{tikzpicture}
    \caption{Cortes m\'inimos en $G$ (izquierda) y en $F_2(G)$ (derecha).}
    \label{fig:ex-tConect}
    \end{figure}

Por \'ultimo, \cref{teo:FG_k(t- k+ 1)-conexa} presenta una mejor cota para
gr\'aficas suficientemente conexas y suficientemente grandes.

\begin{teorema}%
    \label{teo:FG_k(t- k+ 1)-conexa}
        Sea $G$ una gr\'afica. Si $G$ es $t$-conexa, con $t \ge k$ y $n \ge
        \frac{1}{2} kt$, entonces $F_{k}(G)$ es $k (t- k+ 1)$-conexa.
    \end{teorema}

    \begin{proof}
        Sea $\mathcal{C}$ un corte por v\'ertices m\'inimo de $F_k(G)$. Sean $A$
        y $B$ v\'ertices en distintas componentes de $F_k(G)- \mathcal{C}$,
        tales que $|A \triangle B|$ es m\'inima.

        Si $|A \triangle B| = 2$, entonces, por \cref{lem:TrayIntAj}, tenemos
        que $F_k(G)$ tiene $k (t- k+ 1)$ $AB$-trayectorias. Por lo que tenemos
        que $|\mathcal{C}| \geq k (t- k+ 1)$.

        Ahora nos enfocamos en el caso en el que $|A \triangle B| = 2r \ge 4$.
        Utilizamos la desigualdad obtenida al final de la demostraci\'on
        d\cref{teo:FG-t-conexa}, es decir, 
        \[
            |\mathcal{C}| \ge r(n-k-r)+2r^2.
        \]
        Dado que $r \ge 2$ y $n \ge \frac{1}{2}kt$, entonces tenemos que 
        \[
            |\mathcal{C}| \ge r(n-k-r)+2r^2 \ge 2 (n- k -2) + 8 \ge tk - 2k+ 4.
        \]
        Observamos que $k^2 -3k + 4 \ge 0$, para toda $k \ge 1$. De ah\'i que
        $-2k+4 \ge -k^2 + k$, lo cu\'al implica que $kt -2k +4 \ge kt - k^2 + k
        = k (t - k +1)$. Luego, tenemos que $|\mathcal{C}| \ge tk -2k +4 \ge
        k(t-k+1)$. Por lo tanto, $F_k(G)$ es $k(t-k+1)$-conexa.
    \end{proof}