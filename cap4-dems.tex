\chapter{Hamiltonicidad}%
\label{cap:hamilt}

La hamiltonicidad de una gr\'afica es un \'area muy estudiada dentro de la
Teor\'ia de Gr\'aficas. En este cap\'itulo veremos un poco de la hamiltonicidad
de las gr\'aficas de fichas. Algo que es importante mencionar es que la
hamiltonicidad de una gr\'afica no implica hamiltonicidad en su gr\'afica de
fichas. De igual manera, una gr\'afica no hamiltoniana puede tener gr\'aficas de
fichas que si lo sean. Un ejemplo de ambos casos se puede observar en
\cref{fig:ex-H-NH} donde, del lado izquierdo, se muestran dos gr\'aficas y sus
gr\'aficas de $2$-fichas, del lado derecho.


\begin{figure}[ht!]
    \centering
       \begin{tikzpicture}
    
        \begin{scope}[xshift=0cm,yshift=0cm,scale=0.7]
            \foreach \i in {1,...,4}
                \draw ({(360/4)*\i}:1.5) node(\i)[vertex, label=(360/4)*\i:{${\i}$}]{};
            \foreach \i/\j in {1/2,1/4,2/3,3/4}
                \draw [edge,grisOscuro] (\i) to (\j);
        \end{scope}

        \begin{scope}[xshift=5cm]
            \draw (-1,1.1) node (1) [vertex,label=180:{$12$}] {};
            \draw (1,1.1) node (2) [vertex,label=0:{$23$}] {};
            \draw (-1,0) node (3) [vertex,label=180:{$13$}] {};
            \draw (1,0) node (4) [vertex,label=0:{$24$}] {};
            \draw (-1,-1.1) node (5) [vertex,label=180:{$34$}] {};
            \draw (1,-1.1) node (6) [vertex,label=0:{$14$}] {};

            \foreach \i/\j in{1/3,1/4,2/3,2/4,3/5,3/6,4/5,4/6} 
                \draw [edge,grisOscuro] (\i) to (\j);
        \end{scope}

        \begin{scope}[xshift=0cm,yshift=-3.5cm,scale=0.7]
            \foreach \i in {1,2,3}
                \draw ({(360/3)*\i}:1.7) node(\i)[vertex, label=(360/3)*\i:{${\i}$}]{};
            \draw (0,0) node (4) [vertex,label=180:{$4$}] {};
            \foreach \i/\j in {1/4,2/4,2/3,3/4}
                \draw [edge,grisOscuro] (\i) to (\j);
        \end{scope}

        \begin{scope}[xshift=5cm,yshift=-3.5cm,scale=0.7]
            \draw ({(360/6)}:1.7) node(1)[vertex, label=(360/6):{${12}$}]{};
            \draw ({(360/6)*2}:1.7) node(2)[vertex, label=(360/6)*2:{${24}$}]{};
            \draw ({(360/6)*3}:1.7) node(3)[vertex, label=(360/6)*3:{${14}$}]{};
            \draw ({(360/6)*4}:1.7) node(4)[vertex, label=(360/6)*4:{${34}$}]{};
            \draw ({(360/6)*5}:1.7) node(5)[vertex, label=(360/6)*5:{${13}$}]{};
            \draw ({(360/6)*6}:1.7) node(6)[vertex, label=(360/6)*6:{${23}$}]{};
            \foreach \i/\j in {1/2,1/6,2/3,2/6,3/4,3/5,4/5,5/6}
                \draw [edge,grisOscuro] (\i) to (\j);
        \end{scope}

    \end{tikzpicture}
    \caption{Las dos gr\'aficas superiores son una gr\'afica hamiltoniana 
    (izquierda) y su gr\'afica de $2$-fichas que no es hamiltoniana (derecha).
     Las dos gr\'aficas inferiores son una gr\'afica no hamiltoniana (izquierda) 
     y su gr\'afica de $2$-fichas que es hamiltoniana}
    \label{fig:ex-H-NH}
\end{figure}

El no tener una relaci\'on tan clara, con respecto a la hamiltonicidad, entre
una gr\'afica y sus gr\'aficas de fichas nos llevaa a buscar una familia de
gr\'aficas no hamiltionana cuyas gr\'aficas de $k$-fichas sean hamiltonianas.
Para lograrlo, primero nos enfocaremos en los abanicos, una familia de
gr\'aficas que despu\'es facilitar\'a encontrar la familia de gr\'aficas que
buscamos.

\section{Gr\'afica abanico \texorpdfstring{$\mathcal{F}_n$}{Fn}}%
\label{sec:SimpleFan}

En esta secci\'on nos enfocaremos en los abanicos, es decir $\mathcal{F}_n$.
Recordamos que $\mathcal{F}_n = P_{n-1} \oplus K_1$, donde los primeros $n-1$
v\'ertices son los v\'ertices pertenecientes a $P_{n-1}$ y el $n$-\'esimo
v\'ertice es el que le corresponde a $K_1$. A continuaci\'on veremos que las
gr\'aficas de $k$-fichas de abanicos son hamiltonianas, a partir de abanicos con
cierto n\'umero de v\'ertices. Para esto, primero es necesaria la siguiente
proposici\'on.

%TODO:modificacion de def de abanico para homogenizar concepto. (Agregar nombre
%de v\'ertices en generalizado.)
%TODO: correccion en la definicion de abanico, es union generalizada.

 %TODO: definir la grafica abanico como una trayectoria P_{n-1} y K_1 de tal 
%manera que los primeros n-1 elementos pertenezcan a la trayectoria


\begin{proposicion}
        \label{prop:iso-SFan}
            Sean $F_k(\mathcal{F}_n)$ una gr\'afica de $k$-fichas y $V_i=\{ Y \in
            V(F_k(\mathcal{F}_n) \colon\ v_i \in Y) \}$ un subconjunto de v\'ertices, con
            $1 \leq i \leq n-k$. Sea $S_i$ la subgr\'afica de $F_k(\mathcal{F}_n)$
            inducida  por $V_i$. Entoncess $S_i$ es isomorfa a
            $F_{k-1}(\mathcal{F}_{n-i})$.
        \end{proposicion}
    
        \begin{proof}
            Sea $V(\mathcal{F}_{n-i}) = \{v_{i+1}, v_{i+2}, \dots, v_n\}$, donde
            $V(P_{n-i-1}) = \{v_{i+1}, v_{i+2}, \dots, v_{n-1}\}$ y $V(K_1)=
            \{v_n\}$ por definici\'on. Basta demostrar que existe un isomorfismo
            entre las gr\'aficas $S_i$ y $F_{k-1}(\mathcal{F}_{n-1})$.
            Proponemos a la funci\'on $A \xrightarrow[]{} A \setminus \{v_i\}$
            como dicho isomorfismo. Nos fijamos que para $A \in
            V(S_i)$ tenemos que $A \setminus \{v_i\} \in \
            F_{k-1}(\mathcal{F}_{n-i})$  y para $B \in
            F_{k-1}(\mathcal{F}_{n-i})$ tenemos que $B \cup \{v_i\} \in S_i$,
            por lo que $|V(S_i)|=|V(F_{k-1}(\mathcal{F}_{n-i}))|$. Adem\'as,
            observamos que $A$ y $B $ son adyacentes en $V(S_i)$ si y s\'olo si
            $A \setminus \{v_i\}$ y $B \setminus \{v_i\}$ son adyacentes en
            $F_{k-1}(F_{n-i})$, esto pues $A\triangle B = \{a,b\}$, donde $a\neq
            v_i$ y $b \neq v_i$, por lo que $v_i$ no afecta la adyacencia en
            ninguna de las gr\'aficas. Esto quiere decir que la funci\'on
            respeta adyacencias. Por lo tanto $S_i \cong
            F_{k-1}(\mathcal{F}_{n-i})$, con $1 \leq i \leq n-k$.
        \end{proof}

        Ahora pasamos a demostrar el teorema.
    
        \begin{teorema}
        \label{teo:hamilt-SFan}
            Sean $n,k \in \mathbb{Z}$ tales que $n \geq 3$ y $1 \leq k \leq n-1$,
            entonces $F_k(\mathcal{F}_n)$ es hamiltoniana.
        \end{teorema}
    
        
        \begin{proof}
            Usando doble inducci\'on sobre $k$ y $n$, demostraremos que
            $F_k(\mathcal{F}_n)$ tiene un ciclo hamiltoniano donde los v\'ertices
            $\{v_{n-k}, v_{n-k+1}, \dots, v_{n-2}, v_n\}$ y $\{v_{n-k},
            v_{n-k+1}, \dots, v_{n-2}, v_{n-1}\}$ son adyacentes. Para esta
            demostraci\'on consideraremos que para todo v\'ertice $A = \{a_1,
            \dots, a_k\}$ de una gr\'afica de $k$-fichas se tiene $a_1 < \cdots
            < a_k$. Observamos que para $k =1$ tenemos que $F_1(\mathcal{F}_n) \cong \mathcal{F}_n$
            por lo que es hamiltoniana y, por definici\'on de gr\'afica abanico,
            se cumple que $\{v_{n-k}, v_{n-k+1}, \dots, v_{n-2}, v_n\}$ y
            $\{v_{n-k}, v_{n-k+1}, \dots, v_{n-2}, v_{n-1}\}$ son adyacentes.
            Entonces, asumiremos $k \geq 2$ para el resto de la demostraci\'on.
            Primero consideramos el caso en el que $k =2$. Si tomamos $n = 3$,
            tenemos la gr\'afica con conjunto de v\'ertices $V(\mathcal{F}_n)=\{v_1, v_2,
            v_3\}$. Entonces tenemos el siguiente ciclo hamiltoniano $C=
            (\{v_1,v_3\},\{v_1,v_2\},\{v_2,v_3\}, \{v_1,v_3\})$, donde $\{v_1,
            v_3\}$ y $\{v_1, v_2\}$ son adyacentes. 
            
            Ahora veamos elabanico caso de la gr\'afica de $2$-fichas de $\mathcal{F}_n$, con $n
            \geq 4$ y conjunto de v\'ertices $V(\mathcal{F}_n)=\{v_1,\dots, v_n\}$.
            Tomamos el ciclo
    
            $C = (\{v_1, v_{n-1}\},\{v_1, v_n\},\{v_1, v_{n-2}\}, \{v_1,
            v_{n-3}\}, \dots, \{v_1, v_2\},$
            
            $\{v_2, v_n\}, \{v_2, v_{n-1}\}, \{v_2, v_{n-2}\},
            \dots, \{v_2, v_3\},$
    
            $ \dots,$
    
            $\{v_{n-3},v_n\}, \{v_{n-3}, v_{n-1}\},\{v_{n-3},v_{n-2}\},
            \{v_{n-2}, v_n\},\{v_{n-2}, v_{n-1}\}, \{v_{n-1}, v_n\}, \{v_1,
            v_{n-1}\})$
    
            Observamos que $C$ es un ciclo hamiltoniano que cumple tener la
            adyacencia de  $\{v_{n-2}, v_n\}$ y $\{v_{n-2}, v_{n-1}\}$. 
    
            Ahora supongamos que $F_{k'}(\mathcal{F}_{n'})$ satisface las hip\'otesis de
            inducci\'on, con $k'< k$ y $n'<n$. La gr\'afica $F_{k-1}(\mathcal{F}_{n-i})$,
            con $1 \leq i \leq n-k$, satisface la hip\'otesis por lo que tiene
            un ciclo hamiltoniano, y, por \cref{prop:iso-SFan}, sabemos que
            $F_{k-1}(\mathcal{F}_{n-i}) \cong S_i$. Recordemos que  $S_i$ es la
            subgr\'afica de $F_k(\mathcal{F}_n)$ inducida por el conjunto $V_i$ definido
            en la Proposici\'on, en otras palabras, $S_i= F_k(\mathcal{F}_n, V_i)$. Ahora,
            llamamos $C_i$ al ciclo hamiltoniano de $F_{k-1}(\mathcal{F}_{n-i})$ y $X_i =
            \{v_i, v_{n-k+1}, v_{n-k+2}, \dots, v_{n-2}, v_n\}$, $Y_i= \{v_i,
            v_{n-k+1}, v_{n-k+2}, \dots, v_{n-2}, v_{n-1}\}$ a los v\'ertices
            adyacentes en $C_i$. Nombramos $P_i$ a la subtrayectoria
            hamiltoniana de $X_i$ a $Y_i$. Por \'ultimo, denotamos $Z$ al
            v\'ertice $\{v_{n-k+1}, v_{n-k+2}, \dots, v_n\}$. Observamos que,
            por la forma en la que ordenamos los v\'ertices de una gr\'afica de
            fichas, $Z$ es el \'unico v\'ertice que cumple tener elemento m\'inimo
            $v_{n-k+1}$. Por lo tanto  $V_{n-k+1} = \{Z\}$. 
    
            Ahora, nombramos $D_j =\{v_{n-k}, v_{n-k+1}, \dots, v_{n-1}, v_n\}
            \setminus \{v_j\}$, con $n-k+1 \leq j \leq n$.  De esta manera,
            podemos ver al conjunto de v\'ertices de $S_{n-k}$ como
            $V_{n-k}=\{D_n, D_{n-1}, \dots, D_{n-k+1}\}$. Observamos que
            $D_{j-1} \triangle D_j= \{v_{j-1}, v_j\}$, para $n-k+2 \leq j \leq
            n-2$. Entonces tenemos la trayectoria $Q= (D_n, D_{n-1}, \dots,
            D_{n-k+1})$ en $S_{n-k}$.     
            Luego, fij\'emonos en que, para $j= n-k$, tenemos que $Y_j = D_n$ y
            $X_j= D_{n-1}$. Adem\'as tenemos las siguientes diferencias
            sim\'etricas $X_{n-k}\triangle D_{n-2} =\{v_{n-2}, v_{n-1}\}$ ,
            $Y_{n-k}\triangle D_{n-2} =\{v_{n-2}, v_n\}$, $Z\triangle D_{n-k+1}
            =\{v_{n-k}, v_{n-k+1}\}$. Por lo tanto, en $F_k(\mathcal{F}_n)$, tenemos las
            siguientes adyacencias: $X_{n-k} \sim D_{n-2}$, $Y_{n-k} \sim
            D_{n-2}$ y $Z \sim D_{n-k+1}$. Adicionalmente, observamos que, para
            $1 \leq i \leq n- k- 1$, tenemos $X_{i} \triangle X_{i+1}= \{v_i,
            v_{i+1}\}$ y $Y_i \triangle Y_{i+1} = \{v_i, v_{i+1}\}$. Tambi\'en
            tenemos que $Y_1 \triangle Z = \{v_1, v_n\}$. Por lo tanto tenemos
            que $Y_1 \sim Z$, $X_i \sim X_{i+1}$ y $Y_i \sim Y_{i+1}$, para toda
            $1 \leq i \leq n-k-1$.
            
            Definimos el siguiente ciclo $C$ como uni\'on de trayectorias m\'as
            peque\~{n}as: 
    
            En el caso en el que $n-k$ sea par:
            
            $C =(Y_1 \xrightarrow[P_1^{-1}]{} X_1, X_2 \xrightarrow[P_2]{} Y_2,
            \dots, Y_{n-k-1} \xrightarrow[P_{n-k-1}^{-1}]{} X_{n-k-1}, X_{n-k},
            Y_{n-k}, D_{n-2} \xrightarrow[Q]{} D_{n-k+1}, Z, Y_1)$
    
            En el caso en el que $N-k$ sea impar:
    
            $C = (Y_1 \xrightarrow[P_1^{-1}]{} X_1, X_2 \xrightarrow[P_2]{} Y_2,
            \dots, X_{n-k-1} \xrightarrow[P_{n-k-1}]{} Y_{n-k-1}, Y_{n-k},
            X_{n-k}, D_{n-2} \xrightarrow[Q]{} D_{n-k+1}, Z, Y_1)$
    
            Afirmamos que en ambos casos $C$ es un ciclo hamiltoniano.
            Observamos que, al ser uni\'on de las trayectorias $P_i$, el ciclo
            tiene todos los v\'ertices de $F_k(\mathcal{F}_n)$ que empiezan con $v_i$, con
            $i \in \{1, \dots, n-k-1\}$. Adem\'as, al tener la trayectoria $Q$,
            el ciclo tambi\'en tiene los v\'ertices que faltan. Por lo tanto $C$
            es un ciclo hamiltoniano de $F_k(\mathcal{F}_n)$, donde los v\'ertices
            $X_{n-k}$ y $Y_{n-k}$ son adyacentes. 
    
        \end{proof}

\section{Gr\'afica abanico generalizada
\texorpdfstring{$\mathcal{F}_{m,n}$}{Fmn}}%
\label{sec:GeneralFan}

En esta secci\'on veremos la hamiltonicidad de las gr\'aficas de fichas para los
abanicos generalizados, es decir $\mathcal{F}_{m,n}=P_n \oplus \overline{K_m}$.
Para facilitar las demostraciones y el entendimiento de estas, durante esta
secci\'on denotaremos $w_i$ a los v\'ertices de $\overline{K_m}$, con $i \in
\{1, \dots, m\}$. 

Para demostrar el caso general, es de utlidad primero enfocarnos en las
gr\'aficas de $2$-fichas, teorema que es demostrado a continuaci\'on.

\begin{teorema}
\label{teo:2-TokGenerFan}
    La gr\'afica de 2-fichas de $\mathcal{F}_{m,n}$ es hamiltoniana si y s\'olo si $n \geq
    2$ y $1 \leq m \leq 2n$, o si $n=1$ y $m=3$.
\end{teorema}

% TODO: dem de prop sobre relacion entre componentes conexas y hamiltonicidad
%TODO: caso en el que n=1
\begin{proof}
    Primero veamos que para $n \geq 2$ y $1 \leq m \leq 2n$ la gr\'afica de
    $2$-fichas de $\mathcal{F}_{m,n}$ es hamiltoniana. Para esto construiremos un ciclo
    hamiltorniano espec\'ifico para cada caso. Consideramos $m=1$ para el primer
    caso. Si $n=2$, entonces $F_2(\mathcal{F}_{m,n}) \cong \mathcal{F}_{1,2}$ pues ambas son
    isomorfas a $K_3$. Ahora consideramos $n \geq 3$ y tomamos $1 \leq i \leq n$
    para construir una familia de trayectorias
    $T_i=(\{v_i,w_1\},\{v_i,v_{i+1}\},\{v_i,v_{i+2}\},\dots, \{v_i,v_n\})$. Observamos
    que $\{V(T_1),V(T_2),\dots, V(T_n)\}$ es una partici\'on de
    $V(F_2(\mathcal{F}_{m,n}))$. %especificar mas?
    Ahora analizamos el siguiente ciclo $C$.
    
    En el caso en el que $n$ es par, tomamos:

    $C=(\{v_1,v_n\} \xrightarrow[T_1]{}\{v_1,w_1\},\{v_2,w_1\}
    \xrightarrow[T_2]{}\{v_2,v_n\},\{v_3,v_n\} \xrightarrow[T_3]{}\{v_3,w_1\},
    \dots, \{v_{n-1},v_n\}
    \xrightarrow[T_{n-1}]{}\{v_{n-1},w_1\},\{v_n,w_1\},\{v_1,v_n\})$

    En el caso en el que $n$ es impar, tomamos:

    $C=(\{v_1,v_n\} \xrightarrow[T_1]{}\{v_1,w_1\},\{v_2,w_1\}
    \xrightarrow[T_2]{}\{v_2,v_n\},\{v_3,v_n\} \xrightarrow[T_3]{}\{v_3,w_1\},
    \dots, \{v_{n-1},w_1\}
    \xrightarrow[T_{n-1}]{}\{v_{n-1},v_n\},\{v_n,w_1\},\{v_1,v_n\})$

    Primero comprobemos que $C$ es un ciclo en $F_2(\mathcal{F}_{m,n})$. En $\mathcal{F}_{m,n}$
    tenemos que $v_i \sim v_{i+1}$. Adem\'as $|\{v_i,w_1\}\triangle
    \{v_{i+1},w_1\}|=2 =|\{v_i,v_n\}\triangle \{v_{i+1},v_n\}|$, por lo que
    $\{v_i,w_1\}\sim \{v_{i+1},w_1\}$ y $\{v_i,v_n\}\sim \{v_{i+1},v_n\}$ en
    $F_2(\mathcal{F}_{m,n})$. Por lo tanto tenemos que $C$ es un ciclo en ambos casos.
    Adicionalmente tenemos que los los v\'ertices de las trayectorias son una
    partici\'on de los v\'ertices de $F_2(\mathcal{F}_{m,n})$, por lo que $C$ es un ciclo
    hamiltoniano. Como nota adicional, notemos que los v\'ertices
    $\{v_{n-2},w_1\}$ y $\{v_{n-2},v_{n-1}\}$ son adyacentes en la trayectoria
    $T_{n-2}$. Esto nos servir\'a en el siguiente teorema.

    Para el segundo caso consideramos $m=2n$. Definimos la siguiente familia de
    trayectorias. A $P_1$ la definimos como la subtrayectoria hamiltoniana de
    $\{v_n,w_1\}$ a $\{v_n,w_1\}$, es decir $P_1= (\{v_n,w_1\}
    \xrightarrow[C]{}\{v_1,v_n\})$. 
    
    Para $1 < i \leq n$ definimos
    $P_i=(\{w_i,v_n\},\{w_i,w_1\},\{w_i,v_{n-1}\},\{w_i,w_{i+(n-1)}\},\{w_i,v_{n-2}\},$
    $\{w_i,w_{i+(n-2)}\},\dots,\{w_i,v_2\},\{w_i,w_{i+2}\},\{w_i,v_1\},\{w_i,w_{i+1}\})$

    Para $n+1\leq i \leq 2n$, consideramos los \'indices de los v\'ertices de
    $\overline{K_m}$ $\mod 2n$, usando $2n$ en vez de $0$, para definir
    $P_i=(\{w_i,v_n\},\{w_i,w_{n+1}\},\{w_i,v_{n-1}\},\{w_i,w_{i+(n-1)}\},\{w_i,v_{n-2}\},$
    $\{w_i,w_{i+(n-2)}\},\dots,
    \{w_i,v_2\},\{w_i,w_{i+2}\},\{w_i,v_1\},\{w_i,w_{i+1}\})$

    Dado que la trayectoria $P_i$ recorre el ciclo $C$, podemos afirmar que es
    una trayectoria en $F_2(\mathcal{F}_{m,n})$, con $2\leq i \leq 2n$. M\'as a\'un,
    podemos observar que los elementos de toda $P_i$ tienen configuraci\'on
    $\{w_i,v_j\},\{w_i,w_k\}$, con $i, k\in \{1, \dots, 2n\}$ y $j \in \{1,
    \dots, n\}$. Adicionalmente, a partir del tercer elemento de $P_i$, tenemos a
    los elementos con la forma $\{w_i,v_j\},\{w_i,w_{i+j}\}$, con $j \in \{1,
    \dots, n-1\}$.  Por otro lado sabemos que, para toda $i \in \{1, \dots, n\}$
    y $j \in \{1, \dots, m\}$, se cumple que $v_i \sim w_j$ en $\mathcal{F}_{m,n}$ y
    adem\'as $|\{w_i,v_j\} \triangle \{w_i, w_{i+j}\}|=2$, por lo que
    $\{w_i,v_j\} \sim \{w_i, w_{i+j}\}$ en $F_2(F_{m.n})$. Entonces $P_i$ s\'i
    cumple ser una trayectoria en la gr\'afica de $2$-fichas. Ahora construimos
    el ciclo deseado usando las trayectorias $P_i$. Notamos que para, $2 \leq i
    \leq 2n$, el primer elemento de $P_i$ es $\{w_i, v_n\}$ y el \'ultimo
    elemento es $\{w_i, w_{i+1}\}$, que son adyacentes por el atgumento
    anterior. Entonces tenemos que el \'ultimo elemento de $P_i$ y el primer
    elemento de $P_{i+1}$ son adyacentes. Adem\'as, usando el mismo argumento,
    tenemos que el \'ultimo elemento de $P_1$, es decir $\{v_1,v_n\}$, que es
    adyacente al primer elemento de $P_2$, que es $\{w_2,v_n\}$. De manera
    an\'aloga $\{w_{2n},w_1\}$, el \'ultimo elemento de $P_{2n}$, es adyacente a
    $\{v_n,w_1\}$, el primer elemento de $P_1$. Por lo tanto $C_1 = (P_1,P_2,
    \dots, P_{2n},\{v_n,w_1\})$ es un ciclo en $F_2(\mathcal{F}_{m,n})$. 

    Ahora determinamos si $C_1$ es un ciclo hamiltoniano de la gr\'afica de
    $2$-fichas. Al igual que en el caso anterior, veamos que las trayectorias de
    $P_i$ generan una partici\'on de $F_2(\mathcal{F}_{m,n})$. Primero tenemos que los
    v\'ertices de la forma $\{v_i,v_j\}$ est\'an en $P_1$, con $i,j \in \{1,
    \dots, n\}$ y $i \neq j$. Adem\'as, los v\'ertices $\{w_i,w_1\}$
    est\'an en $P_1$, con $i \in \{1, \dots, m\}$ y $1 \neq i$. Luego,
    fijemosnos que los v\'ertices $\{w_i,v_j\}$ est\'an en $P_i$, para $i\in
    \{1, \dots, m\}$ y $j \in \{1, \dots, n\}$. Por \'ultimo, veamos los
    v\'ertices $\{w_i,w_j\}$, tomando $i < j$. Si $1 < i \leq n$ y $i <
    j \leq i+n-1$, entonces tenemos que $\{w_i,w_j\} \in P_i$, mientras que, si
    tomamos $1 < i \leq n$ y $i+n-1 < j \leq 2n$, tenemos que $\{w_i,w_j\} \in
    P_j$. Ahora, si tomamos $n<i<j \leq 2n$, tenemos que $\{w_i,w_j\} \in P_i$.
    Por lo tanto $\{V(P_1),V(P_2), \dots, V(P_{2n})\}$ es una partici\'on de
    $V(F_2(\mathcal{F}_{m,n}))$. Por lo tanto $C_1$ es un ciclo hamiltoniano. La Figura
    \ref{fig:exe-tray-n-2n} muestra el movimiento de fichas en la gr\'afica
    $\mathcal{F}_{8,4}$ para generar la trayectoria $P_6$ en $F_2(\mathcal{F}_{8,4})$, donde la
    ficha en ${\color{coral}w_6}$ est\'a est\'atica.


    \begin{figure}[ht!]
        \centering
           \begin{tikzpicture}
        
            \begin{scope}[xshift=-5.5cm,yshift=0cm,scale=0.8]
                \draw (-1.8,0) node (v1) [vertex,label=240:{\small $v_1$}] {};
                \draw (-0.6,0) node (v2) [vertex,label=200:{\small $v_2$}] {};
                \draw (0.6,0) node (v3) [vertex,label=350:{\small $v_3$}] {}; 
                \draw (1.8,0) node (v4) [bvertex,fill=lila ,label=300:{\small $v_4$}] {}; 
                \draw (-2,2.3) node (w1) [vertex,label=90:{\small $w_1$}] {}; 
                \draw (-0.75,2.3) node (w2) [vertex,label=90:{\small $w_2$}] {};
                \draw (0.75,2.3) node (w3) [vertex, label=90:{\small $w_3$}] {};
                \draw (2,2.3) node (w4) [cvertex,label=90:{\small $\color{grisOscuro!75}w_4$}] {};
                \draw (-2,-2.3) node (w5) [cvertex,label=270:{\small $\color{grisOscuro!75}w_5$}] {}; 
                \draw (-0.75,-2.3) node (w6) [bcvertex,fill=coral,label=270:{\small $\color{grisOscuro!75}w_6$}] {};
                \draw (0.75,-2.3) node (w7) [cvertex,label=270:{\small $\color{grisOscuro!75}w_7$}] {};
                \draw (2,-2.3) node (w8) [cvertex,label=270 :{\small $\color{grisOscuro!75} w_8$}] {};
                
                \foreach \i/\j in{w1/v1,w1/v2,w1/v3,w1/v4,w2/v1,w2/v2,w2/v3,
                    w2/v4,w3/v1,w3/v2,w3/v3,w3/v4,v1/v2,v2/v3,v3/v4} 
                \draw [edge,grisOscuro] (\i) to (\j);

                \foreach \i/\j in{w4/v1,w4/v2,w4/v3,w4/v4,w5/v1,w5/v2,w5/v3,
                    w5/v4,w6/v1,w6/v2,w6/v3,w6/v4,w7/v1,w7/v2,w7/v3,w7/v4,w8/v1,
                    w8/v2,w8/v3,w8/v4} 
                \draw [edge,grisOscuro!50] (\i) to (\j);
            \end{scope}
        
            \begin{scope}[xshift=0cm,yshift=0cm,scale=0.8]
                \draw (-1.8,0) node (v1) [vertex,label=240:{\small $v_1$}] {};
                \draw (-0.6,0) node (v2) [vertex,label=200:{\small $v_2$}] {};
                \draw (0.6,0) node (v3) [vertex,label=350:{\small $v_3$}] {}; 
                \draw (1.8,0) node (v4) [vertex,label=300:{\small $v_4$}] {}; 
                \draw (-2,2.3) node (w1) [vertex,label=90:{\small $w_1$}] {}; 
                \draw (-0.75,2.3) node (w2) [vertex,label=90:{\small $w_2$}] {};
                \draw (0.75,2.3) node (w3) [vertex, label=90:{\small $w_3$}] {};
                \draw (2,2.3) node (w4) [cvertex,label=90:{\small $\color{grisOscuro!75}w_4$}] {};
                \draw (-2,-2.3) node (w5) [bcvertex,fill=lila ,label=270:{\small $\color{grisOscuro!75}w_5$}] {}; 
                \draw (-0.75,-2.3) node (w6) [bcvertex,fill=coral,label=270:{\small $\color{grisOscuro!75}w_6$}] {};
                \draw (0.75,-2.3) node (w7) [cvertex,label=270:{\small $\color{grisOscuro!75}w_7$}] {};
                \draw (2,-2.3) node (w8) [cvertex,label=270 :{\small $\color{grisOscuro!75} w_8$}] {};
                    
                \foreach \i/\j in{w1/v1,w1/v2,w1/v3,w1/v4,w2/v1,w2/v2,w2/v3,
                    w2/v4,w3/v1,w3/v2,w3/v3,w3/v4,v1/v2,v2/v3,v3/v4} 
                \draw [edge,grisOscuro] (\i) to (\j);
             
                \foreach \i/\j in{w4/v1,w4/v2,w4/v3,w4/v4,w5/v1,w5/v2,w5/v3,
                    w5/v4,w6/v1,w6/v2,w6/v3,w6/v4,w7/v1,w7/v2,w7/v3,w7/v4,w8/v1,
                    w8/v2,w8/v3,w8/v4} 
                \draw [edge,grisOscuro!50] (\i) to (\j);
            \end{scope}

            \begin{scope}[xshift=5cm,yshift=0cm,scale=0.8]
                \draw (-1.8,0) node (v1) [vertex,label=240:{\small $v_1$}] {};
                \draw (-0.6,0) node (v2) [vertex,label=200:{\small $v_2$}] {};
                \draw (0.6,0) node (v3) [bvertex,fill=lila ,label=350:{\small $v_3$}] {}; 
                \draw (1.8,0) node (v4) [vertex,label=300:{\small $v_4$}] {}; 
                \draw (-2,2.3) node (w1) [vertex,label=90:{\small $w_1$}] {}; 
                \draw (-0.75,2.3) node (w2) [vertex,label=90:{\small $w_2$}] {};
                \draw (0.75,2.3) node (w3) [vertex, label=90:{\small $w_3$}] {};
                \draw (2,2.3) node (w4) [cvertex,label=90:{\small $\color{grisOscuro!75}w_4$}] {};
                \draw (-2,-2.3) node (w5) [cvertex,label=270:{\small $\color{grisOscuro!75}w_5$}] {}; 
                \draw (-0.75,-2.3) node (w6) [bcvertex,fill=coral,label=270:{\small $\color{grisOscuro!75}w_6$}] {};
                \draw (0.75,-2.3) node (w7) [cvertex,label=270:{\small $\color{grisOscuro!75}w_7$}] {};
                \draw (2,-2.3) node (w8) [cvertex,label=270 :{\small $\color{grisOscuro!75} w_8$}] {};
                            
                \foreach \i/\j in{w1/v1,w1/v2,w1/v3,w1/v4,w2/v1,w2/v2,w2/v3,
                    w2/v4,w3/v1,w3/v2,w3/v3,w3/v4,v1/v2,v2/v3,v3/v4} 
                \draw [edge,grisOscuro] (\i) to (\j);
             
                \foreach \i/\j in{w4/v1,w4/v2,w4/v3,w4/v4,w5/v1,w5/v2,w5/v3,
                    w5/v4,w6/v1,w6/v2,w6/v3,w6/v4,w7/v1,w7/v2,w7/v3,w7/v4,w8/v1,
                    w8/v2,w8/v3,w8/v4} 
                \draw [edge,grisOscuro!50] (\i) to (\j);
            \end{scope}
                    
            \begin{scope}[xshift=-5cm,yshift=-5.5cm,scale=0.8]
                \draw (-1.8,0) node (v1) [vertex,label=240:{\small $v_1$}] {};
                \draw (-0.6,0) node (v2) [vertex,label=200:{\small $v_2$}] {};
                \draw (0.6,0) node (v3) [vertex,label=350:{\small $v_3$}] {}; 
                \draw (1.8,0) node (v4) [vertex,label=300:{\small $v_4$}] {}; 
                \draw (-2,2.3) node (w1) [bvertex,fill=lila ,label=90:{\small $w_1$}] {}; 
                \draw (-0.75,2.3) node (w2) [vertex,label=90:{\small $w_2$}] {};
                \draw (0.75,2.3) node (w3) [vertex, label=90:{\small $w_3$}] {};
                \draw (2,2.3) node (w4) [cvertex,label=90:{\small $\color{grisOscuro!75}w_4$}] {};
                \draw (-2,-2.3) node (w5) [cvertex,label=270:{\small $\color{grisOscuro!75}w_5$}] {}; 
                \draw (-0.75,-2.3) node (w6) [bcvertex,fill=coral,label=270:{\small $\color{grisOscuro!75}w_6$}] {};
                \draw (0.75,-2.3) node (w7) [cvertex,label=270:{\small $\color{grisOscuro!75}w_7$}] {};
                \draw (2,-2.3) node (w8) [cvertex,label=270 :{\small $\color{grisOscuro!75} w_8$}] {};
                    
                \foreach \i/\j in{w1/v1,w1/v2,w1/v3,w1/v4,w2/v1,w2/v2,w2/v3,
                    w2/v4,w3/v1,w3/v2,w3/v3,w3/v4,v1/v2,v2/v3,v3/v4} 
                \draw [edge,grisOscuro] (\i) to (\j);
             
                \foreach \i/\j in{w4/v1,w4/v2,w4/v3,w4/v4,w5/v1,w5/v2,w5/v3,
                    w5/v4,w6/v1,w6/v2,w6/v3,w6/v4,w7/v1,w7/v2,w7/v3,w7/v4,w8/v1,
                    w8/v2,w8/v3,w8/v4} 
                \draw [edge,grisOscuro!50] (\i) to (\j);
            \end{scope}

            \begin{scope}[xshift=0cm,yshift=-5.5cm,scale=0.8]
                \draw (-1.8,0) node (v1) [vertex,label=240:{\small $v_1$}] {};
                \draw (-0.6,0) node (v2) [bvertex,fill=lila ,label=200:{\small $v_2$}] {};
                \draw (0.6,0) node (v3) [vertex,label=350:{\small $v_3$}] {}; 
                \draw (1.8,0) node (v4) [vertex,label=300:{\small $v_4$}] {}; 
                \draw (-2,2.3) node (w1) [vertex,label=90:{\small $w_1$}] {}; 
                \draw (-0.75,2.3) node (w2) [vertex,label=90:{\small $w_2$}] {};
                \draw (0.75,2.3) node (w3) [vertex, label=90:{\small $w_3$}] {};
                \draw (2,2.3) node (w4) [cvertex,label=90:{\small $\color{grisOscuro!75}w_4$}] {};
                \draw (-2,-2.3) node (w5) [cvertex,label=270:{\small $\color{grisOscuro!75}w_5$}] {}; 
                \draw (-0.75,-2.3) node (w6) [bcvertex,fill=coral,label=270:{\small $\color{grisOscuro!75}w_6$}] {};
                \draw (0.75,-2.3) node (w7) [cvertex,label=270:{\small $\color{grisOscuro!75}w_7$}] {};
                \draw (2,-2.3) node (w8) [cvertex,label=270 :{\small $\color{grisOscuro!75} w_8$}] {};
                    
                \foreach \i/\j in{w1/v1,w1/v2,w1/v3,w1/v4,w2/v1,w2/v2,w2/v3,
                    w2/v4,w3/v1,w3/v2,w3/v3,w3/v4,v1/v2,v2/v3,v3/v4} 
                \draw [edge,grisOscuro] (\i) to (\j);
             
                \foreach \i/\j in{w4/v1,w4/v2,w4/v3,w4/v4,w5/v1,w5/v2,w5/v3,
                    w5/v4,w6/v1,w6/v2,w6/v3,w6/v4,w7/v1,w7/v2,w7/v3,w7/v4,w8/v1,
                    w8/v2,w8/v3,w8/v4} 
                \draw [edge,grisOscuro!50] (\i) to (\j);
            \end{scope}


            \begin{scope}[xshift=5cm,yshift=-5.5cm,scale=0.8]
                \draw (-1.8,0) node (v1) [vertex,label=240:{\small $v_1$}] {};
                \draw (-0.6,0) node (v2) [vertex,label=200:{\small $v_2$}] {};
                \draw (0.6,0) node (v3) [vertex,label=350:{\small $v_3$}] {}; 
                \draw (1.8,0) node (v4) [vertex,label=300:{\small $v_4$}] {}; 
                \draw (-2,2.3) node (w1) [vertex,label=90:{\small $w_1$}] {}; 
                \draw (-0.75,2.3) node (w2) [vertex,label=90:{\small $w_2$}] {};
                \draw (0.75,2.3) node (w3) [vertex, label=90:{\small $w_3$}] {};
                \draw (2,2.3) node (w4) [cvertex,label=90:{\small $\color{grisOscuro!75}w_4$}] {};
                \draw (-2,-2.3) node (w5) [cvertex,label=270:{\small $\color{grisOscuro!75}w_5$}] {}; 
                \draw (-0.75,-2.3) node (w6) [bcvertex,fill=coral,label=270:{\small $\color{grisOscuro!75}w_6$}] {};
                \draw (0.75,-2.3) node (w7) [cvertex,label=270:{\small $\color{grisOscuro!75}w_7$}] {};
                \draw (2,-2.3) node (w8) [bcvertex,fill=lila,label=270 :{\small $\color{grisOscuro!75} w_8$}] {};
                    
                \foreach \i/\j in{w1/v1,w1/v2,w1/v3,w1/v4,w2/v1,w2/v2,w2/v3,
                    w2/v4,w3/v1,w3/v2,w3/v3,w3/v4,v1/v2,v2/v3,v3/v4} 
                \draw [edge,grisOscuro] (\i) to (\j);
             
                \foreach \i/\j in{w4/v1,w4/v2,w4/v3,w4/v4,w5/v1,w5/v2,w5/v3,
                    w5/v4,w6/v1,w6/v2,w6/v3,w6/v4,w7/v1,w7/v2,w7/v3,w7/v4,w8/v1,
                    w8/v2,w8/v3,w8/v4} 
                \draw [edge,grisOscuro!50] (\i) to (\j);
            \end{scope}


            \begin{scope}[xshift=-2.5cm,yshift=-11cm,scale=0.8]
                \draw (-1.8,0) node (v1) [bvertex,fill=lila ,label=240:{\small $v_1$}] {};
                \draw (-0.6,0) node (v2) [vertex,label=200:{\small $v_2$}] {};
                \draw (0.6,0) node (v3) [vertex,label=350:{\small $v_3$}] {}; 
                \draw (1.8,0) node (v4) [vertex,label=300:{\small $v_4$}] {}; 
                \draw (-2,2.3) node (w1) [vertex,label=90:{\small $w_1$}] {}; 
                \draw (-0.75,2.3) node (w2) [vertex,label=90:{\small $w_2$}] {};
                \draw (0.75,2.3) node (w3) [vertex, label=90:{\small $w_3$}] {};
                \draw (2,2.3) node (w4) [cvertex,label=90:{\small $\color{grisOscuro!75}w_4$}] {};
                \draw (-2,-2.3) node (w5) [cvertex,label=270:{\small $\color{grisOscuro!75}w_5$}] {}; 
                \draw (-0.75,-2.3) node (w6) [bcvertex,fill=coral,label=270:{\small $\color{grisOscuro!75}w_6$}] {};
                \draw (0.75,-2.3) node (w7) [cvertex,label=270:{\small $\color{grisOscuro!75}w_7$}] {};
                \draw (2,-2.3) node (w8) [cvertex,label=270 :{\small $\color{grisOscuro!75} w_8$}] {};
                    
                \foreach \i/\j in{w1/v1,w1/v2,w1/v3,w1/v4,w2/v1,w2/v2,w2/v3,
                    w2/v4,w3/v1,w3/v2,w3/v3,w3/v4,v1/v2,v2/v3,v3/v4} 
                \draw [edge,grisOscuro] (\i) to (\j);
             
                \foreach \i/\j in{w4/v1,w4/v2,w4/v3,w4/v4,w5/v1,w5/v2,w5/v3,
                    w5/v4,w6/v1,w6/v2,w6/v3,w6/v4,w7/v1,w7/v2,w7/v3,w7/v4,w8/v1,
                    w8/v2,w8/v3,w8/v4} 
                \draw [edge,grisOscuro!50] (\i) to (\j);
            \end{scope}

            \begin{scope}[xshift=2.5cm,yshift=-11cm,scale=0.8]
                \draw (-1.8,0) node (v1) [vertex,label=240:{\small $v_1$}] {};
                \draw (-0.6,0) node (v2) [vertex,label=200:{\small $v_2$}] {};
                \draw (0.6,0) node (v3) [vertex,label=350:{\small $v_3$}] {}; 
                \draw (1.8,0) node (v4) [vertex,label=300:{\small $v_4$}] {}; 
                \draw (-2,2.3) node (w1) [vertex,label=90:{\small $w_1$}] {}; 
                \draw (-0.75,2.3) node (w2) [vertex,label=90:{\small $w_2$}] {};
                \draw (0.75,2.3) node (w3) [vertex, label=90:{\small $w_3$}] {};
                \draw (2,2.3) node (w4) [cvertex,label=90:{\small $\color{grisOscuro!75}w_4$}] {};
                \draw (-2,-2.3) node (w5) [cvertex,label=270:{\small $\color{grisOscuro!75}w_5$}] {}; 
                \draw (-0.75,-2.3) node (w6) [bcvertex,fill=coral,label=270:{\small $\color{grisOscuro!75}w_6$}] {};
                \draw (0.75,-2.3) node (w7) [bcvertex,fill=lila,label=270:{\small $\color{grisOscuro!75}w_7$}] {};
                \draw (2,-2.3) node (w8) [cvertex,label=270 :{\small $\color{grisOscuro!75} w_8$}] {};
                    
                \foreach \i/\j in{w1/v1,w1/v2,w1/v3,w1/v4,w2/v1,w2/v2,w2/v3,
                    w2/v4,w3/v1,w3/v2,w3/v3,w3/v4,v1/v2,v2/v3,v3/v4} 
                \draw [edge,grisOscuro] (\i) to (\j);
             
                \foreach \i/\j in{w4/v1,w4/v2,w4/v3,w4/v4,w5/v1,w5/v2,w5/v3,
                    w5/v4,w6/v1,w6/v2,w6/v3,w6/v4,w7/v1,w7/v2,w7/v3,w7/v4,w8/v1,
                    w8/v2,w8/v3,w8/v4} 
                \draw [edge,grisOscuro!50] (\i) to (\j);
            \end{scope}

        \end{tikzpicture}
        \caption{La gr\'afica $F_2(\mathcal{F}_{8,4})$, mostrando el movimiento de fichas
        a trav\'es de $P_6$ }
        \label{fig:exe-tray-n-2n}
    \end{figure}
    
    Por \'ultimo, veamos el caso en el que $1<m<2n$ y del mismo modo que en los
    casos anteriores, definiremos una familia de trayectorias en $F_2(\mathcal{F}_{m,n})$.
    Primero definimos $Q_1 =P_1$. Ahora nos fijamos en $1<i<m$ para definir
    $Q_i$. Queremos usar $P_i$ como referencia para generar las nuevas
    trayectorias, pero $P_i$ tiene m\'as v\'ertices de la forma $\{w_i,w_j\}$,
    en particular cuando $j< n+i-m$. Entonces defnimos $Q_i =
    (\{w_i,v_n\},\{w_i,w_1\},\{w_i,v_{n-1}\},\{w_i,v_{n-2}\}, \dots,
    ,\{w_i,v_{n-j}\},\{w_i,w_{n-j-1}\}\xrightarrow[P_i]{}\{w_i,w_{i+1}\})$, con
    $j = n+i-m-1$. Finalmente, definimos $Q_m$ como la trayectoria
    $(\{w_m,v_n\},\{w_m,w_1\},\{w_m,v_{n-1}\},\{w_m,v_{n-2}\}, \dots,
    ,\{w_m,v_{n-j}\},\{w_m,w_{n-j-1}\}\xrightarrow[P_m]{}\{w_m,w_{m+1}\})$ al
    intercambiar los v\'ertices $\{w_m,w_{m+1}\}$ y $\{w_m,w_1\}$. Observamos
    que las trayectorias $Q_i$ son trayectorias en $F_2(\mathcal{F}_{m,n})$ pues $v_i \sim
    v_{i+1}$ en $\mathcal{F}_{m,n}$. Adem\'as, esta familia de trayectorias forma una
    partici\'on para $F_2(\mathcal{F}_{m,n})$ pues son las trayectorias $P_i$ al quitarle
    los v\'ertices $\{w_i,w_j\}$, con $j< n+i-m$. Por \'ultimo, notamos que,
    salvo $Q_m$, las trayectorias $P_i$ y $Q_i$ tienen los mismos v\'ertices
    iniciales y finales, adicionalmente, el \'ultimo v\'ertice de $Q_i$, es
    decir $\{w_m,w_1\}$, es adyacente al primer v\'ertice de $P_1$, que es
    $\{v_n,w_1\}$. Por lo tanto el ciclo $C_2=(Q_1,Q_2, \dots, Q_m,
    \{v_n,w_1\})$ es el ciclo hamiltoniano buscado. La
    Figura\ref{fig:exe-tray-n-m} muestra el movimiento de las fichas en la
    gr\'afica $\mathcal{F}_{3,4}$ para generar la trayectoria $P_2$ en la gr\'afica de
    $2$-fichas, donde la ficha en ${\color{azulClaro}w_2}$ est\'a fija.


    \begin{figure}[ht!]
        \centering
           \begin{tikzpicture}
        
            \begin{scope}[xshift=-5cm,yshift=0cm,scale=0.8]
                \draw (-1.8,0) node (v1) [vertex,label=240:{\small $v_1$}] {};
                \draw (-0.6,0) node (v2) [vertex,label=200:{\small $v_2$}] {};
                \draw (0.6,0) node (v3) [vertex,label=350:{\small $v_3$}] {}; 
                \draw (1.8,0) node (v4) [bvertex,fill=amarillo ,label=300:{\small $v_4$}] {}; 
                \draw (-1.2,2.3) node (w1) [vertex,label=90:{\small $w_1$}] {}; 
                \draw (1.2,2.3) node (w2) [bvertex,fill=azulClaro ,label=90:{\small $w_2$}] {};
                \draw (0,-2.3) node (w3) [vertex,label=270:{\small $w_3$}] {};
                
                \foreach \i/\j in{w1/v1,w1/v2,w1/v3,w1/v4,w2/v1,w2/v2,w2/v3,
                    w2/v4,w3/v1,w3/v2,w3/v3,w3/v4,v1/v2,v2/v3,v3/v4} 
                \draw [edge,grisOscuro] (\i) to (\j);
            \end{scope}
        
            \begin{scope}[xshift=0cm,yshift=0cm,scale=0.8]
                \draw (-1.8,0) node (v1) [vertex,label=240:{\small $v_1$}] {};
                \draw (-0.6,0) node (v2) [vertex,label=200:{\small $v_2$}] {};
                \draw (0.6,0) node (v3) [vertex,label=350:{\small $v_3$}] {}; 
                \draw (1.8,0) node (v4) [vertex,label=300:{\small $v_4$}] {}; 
                \draw (-1.2,2.3) node (w1) [bvertex,fill=amarillo ,label=90:{\small $w_1$}] {}; 
                \draw (1.2,2.3) node (w2) [bvertex,fill=azulClaro ,label=90:{\small $w_2$}] {};
                \draw (0,-2.3) node (w3) [vertex,label=270:{\small $w_3$}] {};
                    
                \foreach \i/\j in{w1/v1,w1/v2,w1/v3,w1/v4,w2/v1,w2/v2,w2/v3,
                    w2/v4,w3/v1,w3/v2,w3/v3,w3/v4,v1/v2,v2/v3,v3/v4} 
                \draw [edge,grisOscuro] (\i) to (\j);
            \end{scope}

            \begin{scope}[xshift=5cm,yshift=0cm,scale=0.8]
                \draw (-1.8,0) node (v1) [vertex,label=240:{\small $v_1$}] {};
                \draw (-0.6,0) node (v2) [vertex,label=200:{\small $v_2$}] {};
                \draw (0.6,0) node (v3) [bvertex,fill=amarillo ,label=350:{\small $v_3$}] {}; 
                \draw (1.8,0) node (v4) [vertex,label=300:{\small $v_4$}] {}; 
                \draw (-1.2,2.3) node (w1) [vertex,label=90:{\small $w_1$}] {}; 
                \draw (1.2,2.3) node (w2) [bvertex,fill=azulClaro ,label=90:{\small $w_2$}] {};
                \draw (0,-2.3) node (w3) [vertex,label=270:{\small $w_3$}] {};
                            
                \foreach \i/\j in{w1/v1,w1/v2,w1/v3,w1/v4,w2/v1,w2/v2,w2/v3,
                    w2/v4,w3/v1,w3/v2,w3/v3,w3/v4,v1/v2,v2/v3,v3/v4} 
                \draw [edge,grisOscuro] (\i) to (\j);
            \end{scope}
                    
            \begin{scope}[xshift=-5cm,yshift=-5.5cm,scale=0.8]
                \draw (-1.8,0) node (v1) [vertex,label=240:{\small $v_1$}] {};
                \draw (-0.6,0) node (v2) [bvertex,fill=amarillo ,label=200:{\small $v_2$}] {};
                \draw (0.6,0) node (v3) [vertex,label=350:{\small $v_3$}] {}; 
                \draw (1.8,0) node (v4) [vertex,label=300:{\small $v_4$}] {}; 
                \draw (-1.2,2.3) node (w1) [vertex,label=90:{\small $w_1$}] {}; 
                \draw (1.2,2.3) node (w2) [bvertex,fill=azulClaro ,label=90:{\small $w_2$}] {};
                \draw (0,-2.3) node (w3) [vertex,label=270:{\small $w_3$}] {};
                    
                \foreach \i/\j in{w1/v1,w1/v2,w1/v3,w1/v4,w2/v1,w2/v2,w2/v3,
                    w2/v4,w3/v1,w3/v2,w3/v3,w3/v4,v1/v2,v2/v3,v3/v4} 
                \draw [edge,grisOscuro] (\i) to (\j);
            \end{scope}

            \begin{scope}[xshift=0cm,yshift=-5.5cm,scale=0.8]
                \draw (-1.8,0) node (v1) [bvertex,fill=amarillo ,label=240:{\small $v_1$}] {};
                \draw (-0.6,0) node (v2) [vertex,label=200:{\small $v_2$}] {};
                \draw (0.6,0) node (v3) [vertex,label=350:{\small $v_3$}] {}; 
                \draw (1.8,0) node (v4) [vertex,label=300:{\small $v_4$}] {}; 
                \draw (-1.2,2.3) node (w1) [vertex,label=90:{\small $w_1$}] {}; 
                \draw (1.2,2.3) node (w2) [bvertex,fill=azulClaro ,label=90:{\small $w_2$}] {};
                \draw (0,-2.3) node (w3) [vertex,label=270:{\small $w_3$}] {};
                    
                \foreach \i/\j in{w1/v1,w1/v2,w1/v3,w1/v4,w2/v1,w2/v2,w2/v3,
                    w2/v4,w3/v1,w3/v2,w3/v3,w3/v4,v1/v2,v2/v3,v3/v4} 
                \draw [edge,grisOscuro] (\i) to (\j);
            \end{scope}

            \begin{scope}[xshift=5cm,yshift=-5.5cm,scale=0.8]
                \draw (-1.8,0) node (v1) [vertex,label=240:{\small $v_1$}] {};
                \draw (-0.6,0) node (v2) [vertex,label=200:{\small $v_2$}] {};
                \draw (0.6,0) node (v3) [vertex,label=350:{\small $v_3$}] {}; 
                \draw (1.8,0) node (v4) [vertex,label=300:{\small $v_4$}] {}; 
                \draw (-1.2,2.3) node (w1) [vertex,label=90:{\small $w_1$}] {}; 
                \draw (1.2,2.3) node (w2) [bvertex,fill=azulClaro ,label=90:{\small $w_2$}] {};
                \draw (0,-2.3) node (w3) [bvertex,fill=amarillo ,label=270:{\small $w_3$}] {};
                    
                \foreach \i/\j in{w1/v1,w1/v2,w1/v3,w1/v4,w2/v1,w2/v2,w2/v3,
                    w2/v4,w3/v1,w3/v2,w3/v3,w3/v4,v1/v2,v2/v3,v3/v4} 
                \draw [edge,grisOscuro] (\i) to (\j);
            \end{scope}

        \end{tikzpicture}
    
        \caption{La gr\'afica $F_2(\mathcal{F}_{3,4})$, mostrando el movimiento de fichas
        a trav\'es de $P_2$ }
        \label{fig:exe-tray-n-m}     
    \end{figure}

    Ahora veamos que para cuando $m>2n$. Sea $S \subset V(F_2(\mathcal{F}_{m,n}))$, con
    $S=\{\{w_i,v_j\} \colon\ i \in \{1,\dots,m\},j\in\{1,\dots,n\}\}$. Primero,
    observamos que todo v\'ertice $\{w_i,w_j\}$ tiene a todos sus vecinos en $S$
    por lo que es un v\'ertice aislado en $F_2(\mathcal{F}_{m,n})-S$, donde $i,j \in
    \{1,\dots,m\}$ tal que $i \neq j$. Entonces tenemos $\binom{m}{2}$
    v\'ertices aislados en $F_2(\mathcal{F}_{m,n})-S$. Adem\'as, la subgr\'afica inducida
    por los v\'ertices de tipo $\{v_i,v_j\}$ forman una componente conexa en
    $F_2(\mathcal{F}_{m,n})-S$. Por lo tanto tenemos que $F_2(\mathcal{F}_{m,n})-S$ tiene al menos
    $\binom{m}{2} + 1$ componentes conexas. Despejando, se obtiene que
    $\binom{m}{2} + 1 = \frac{m(m-1)}{2} + 1 \geq \frac{m(2n)}{2} + 1 = mn + 1 >
    mn = |S|$. Por lo tanto, por la (Proposicion que falta) tenemos que
    $F_2(\mathcal{F}_{m,n})$ no es hamiltoniana.

\end{proof}

Ahora pasamos al caso general, es decir, enfocarnos en las gr\'aficas de
$k$-fichas, para $2 \leq k \leq n-1$.

\begin{teorema}
\label{teo:TokGenerFan}
    Sean $k, n, m$ enteros positivos tales que $2 \leq k \leq n$ y $1 \leq m
       \leq 2n$, entonces la gr\'afica $F_k(\mathcal{F}_{m,n})$ es hamiltoniana.
\end{teorema}

\begin{proof}
    Usando doble inducci\'on sobre $m$ y $k$, demostraremos que
    $F_k(\mathcal{F}_{m,n})$ tiene un ciclo hamiltoniano donde los v\'ertices $\{w_1,
    v_{n-k}, v_{n-k+1}, \dots, v_{n-2}\}$ y $\{v_{n-k}, v_{n-k+1}, \dots,
    v_{n-2}, v_{n-1}\}$ son adyacentes. Notamos que el caso en el que $m =1$
    equivale al Teorema \ref{teo:hamilt-SFan}, haciedo un cambio de
    notaci\'on de $w_1$ a $v_n$. Por lo tanto tenemos que la gr\'afica de
    $k$-fichas es hamiltoniana. Ahora veamos el caso en el que $m>1$,
    supongamos que la gr\'afica $F_{k'}(\mathcal{F}_{m-1,n}$) cumple la hip\'otesis de
    inducci\'on, con $2\leq k'\leq n$ y $1 \leq m-1 \leq 2n$. Para el subcaso de
    $k=2$ usaremos \cref{teo:2-TokGenerFan}. Basta ver que el ciclo
    obtenido en el teorema cumple tener la adyacencia deseada, es decir
    $\{w_1, v_{n-2}\}\sim\{v_{n-2}, v_{n-1}\}$. Recordando la observaci\'on
    de la demostraci\'on de dicho teorema, tenemos que $\{v_{n-2},w_1\}$ y
    $\{v_{n-2},v_{n-1}\}$ son adyacentes en la trayectoria $T_{n-2}$. Esta
    trayectoria es parte de la trayectoria $P_1$, usada para obtener un
    ciclo hamiltoniano en todos los casos del Teorema
    \ref{teo:2-TokGenerFan}. Por lo tanto, para $k=2$ se obtiene un ciclo
    hamiltoniano con la estructura deseada.

    Por \'ultimo, nos fijamos en el subcaso donde $k>2$. Sean $S_1$ y $S_2$ los
    conjuntos $S_1=\{A \in F_k(\mathcal{F}_{m,n}) \colon\ w_1 \in A\}$ y
    $S_2=\{A \in F_k(\mathcal{F}_{m,n}) \colon\ w_1 \notin A\}$. Estos conjuntos
    generan una partici\'on de $(F_k(\mathcal{F}_{m,n}))$, pues todo v\'ertice
    de la gr\'afica de $k$-fichas esta, por definici\'on, exactamente en alguno
    de estos conjuntos. Ahora, nombramos $H_1$ y $H_2$ a las subgr\'aficas de
    $F_k(\mathcal{F}_{m,n})$ inducidas por $S_1$ y $S_2$, respectivamente. Por
    un lado, $H_1$ es la gr\'afica de fichas donde todo v\'ertice contiene
    $w_1$, en otras palabras $H_1= F_k(\mathcal{F}_{m,n},\{w_1\})$. Sabemos que
    $F_k(\mathcal{F}_{m,n},\{w_1\}) \cong F_{k-1}(\mathcal{F}_{m,n}- w_1)$, por
    lo tanto $H_1 \cong F_{k-1}(\mathcal{F}_{m-1,n})$. Por otro lado, observamos
    que, al ser $H_2$ la gr\'afica de fichas donde ning\'un v\'ertice tiene a
    $w_1$, tenemos que $H_2 \cong F_k(\mathcal{F}_{m-1,n})$. Por lo tanto $H_1$
    y $H_2$ cumplen la hip\'otesis de induci\'on, es decir, ambos tienen un
    ciclo hamiltoniano que llamamos $C_1$ y $C_2$, respectivamente. Adem\'as,
    $C_1$ cumple ser un ciclo donde los v\'ertices $X_1 = \{w_1,
    w_2,v_{n-k},v_{n-k+1}, \dots, v_{n-3}\}$ y $Y_1 = \{w_1,v_{n-k},v_{n-k+1},
    \dots, v_{n-3},v_{n-2}\}$ son adyacentes. De manera an\'aloga, $C_2$ es un
    ciclo hamiltoniano donde los v\'ertices $X_2 = \{ w_2,v_{n-k},v_{n-k+1},
    \dots, v_{n-2}\}$ y $Y_2 = \{v_{n-k},v_{n-k+1}, \dots, v_{n-2},v_{n-1}\}$
    son adyacentes. Definimos $P_i$ como la subtrayectoria hamiltoniana de $X_i$
    a $Y_i$, con $i \in \{ 1,2 \}$. Observamos que $X_1 \triangle X_2 =
    \{w_1,v_{n-2}\}$ y $Y_1 \triangle Y_2 = \{w_1, v_{n-1}\}$, por lo que $X_1
    \sim X_2$ y $Y_1 \sim Y_2$. Ahora definimos el ciclo $C = (X_1
    \xrightarrow[P_1]{}Y_1,Y_2 \xrightarrow[P_2^{-1}]{}X_2,X_1)$. Afirmamos que
    $C$ es el ciclo buscado. Dado que $\{S_1,S_2\}$ es una partici\'on de
    $V(F_k(\mathcal{F}_{m,n}))$, entonces $C$ es un ciclo hamiltoniano. Adem\'as
    tenemos que $Y_1$ y $Y_2$ son adyacentes, por lo que $C$ cumple las
    condiciones deseadas. Por lo tanto $F_k(\mathcal{F}_{m,n})$ es una gr\'afica
    hamiltoniana.
\end{proof}

