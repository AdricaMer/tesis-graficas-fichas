\chapter{Hamiltonicidad}%
\label{cap:hamilt}

\section{Teoremas y demostraciones}%
%\label{sec:etiquetas}

 %TODO: definir la grafica abanico como una trayectoria P_{n-1} y K_1 de tal 
%manera que los primeros n-1 elementos pertenezcan a la trayectoria
%TODO: ver donde poner que cadaa v\'ertices de la grafica de fichas cumple 
%que sus elementos estan ordenados de forma creciente
\begin{proposicion}
        \label{prop:iso-SFan}
            Sean $F_k(F_n)$ una gr\'afica de $k$-fichas y $V_i=\{ Y \in
            V(F_k(F_n) \colon\ v_i \in Y) \}$ un subconjunto de v\'ertices, con
            $1 \leq i \leq n-k$. Sea $S_i$ la subgr\'afica de $F_k(F_n)$
            inducida  por $V_i$. Entoncess $S_i$ es isomorfa a
            $F_{k-1}(F_{n-i})$.
        \end{proposicion}
    
        \begin{proof}
            Sea $V(F_{n-i}) = \{v_{i+1}, v_{i+2}, \dots, v_n\}$, donde
            $V(P_{n-i-1}) = \{v_{i+1}, v_{i+2}, \dots, v_{n-1}\}$ y $V(K_1)=
            \{v_n\}$ por definici\'on. Basta demostrar que existe una funcion
            que sea un isomorfismo entre gr\'aficas entre $S_i$ y
            $F_{k-1}(F_{n-1})$. Proponemos a la funci\'on $A \xrightarrow[]{} A
            \setminus \{v_i\}$ y nos fijamos que cumple ser un isomorfismo de
            gr\'aficas, pues para $A \in V(S_i)$ tenemos que $A \setminus
            \{v_i\} \in \ F_{k-1}(F_{n-i})$  y para $B \in F_{k-1}(F_{n-i})$
            tenemos que $B \cup \{v_i\} \in S_i$. Por lo tanto $S_i \cong
            F_{k-1}(F_{n-i})$, con $1 \leq i \leq n-k$.
        \end{proof}
    
        \begin{teorema}
        \label{teo:hamilt-SFan}
            Sean $n,k \in \mathbb{Z}$ tales que $n \geq 3$ y $1 \leq k \leq n-1$,
            entonces $F_k(F_n)$ es hamiltoniana.
        \end{teorema}
    
        
        \begin{proof}
            Usando doble inducci\'on sobre $k$ y $n$, demostraremos que
            $F_k(F_n)$ tiene un ciclo hamiltoniano donde los v\'ertices
            $\{v_{n-k}, v_{n-k+1}, \dots, v_{n-2}, v_n\}$ y $\{v_{n-k},
            v_{n-k+1}, \dots, v_{n-2}, v_{n-1}\}$ son adyacentes. Para esta
            demostraci\'on consideraremos que para todo v\'ertice $A = \{a_1,
            \dots, a_k\}$ de una gr\'afica de $k$-fichas se tiene $a_1 < \cdots
            < a_k$. Observamos que para $k =1$ tenemos que $F_1(F_n) \cong F_n$
            por lo que es hamiltoniana y, por definici\'on de gr\'afica abanico,
            se cumple que $\{v_{n-k}, v_{n-k+1}, \dots, v_{n-2}, v_n\}$ y
            $\{v_{n-k}, v_{n-k+1}, \dots, v_{n-2}, v_{n-1}\}$ son adyacentes.
            Entonces, asumiremos $k \geq 2$ para el resto de la demostraci\'on.
            Primero consideramos el caso en el que $k =2$. Si tomamos $n = 3$,
            tenemos la gr\'afica con conjunto de v\'ertices $V(F_n)=\{v_1, v_2,
            v_3\}$. Entonces tenemos el siguiente ciclo hamiltoniano $C=
            (\{v_1,v_3\},\{v_1,v_2\},\{v_2,v_3\}, \{v_1,v_3\})$, donde $\{v_1,
            v_3\}$ y $\{v_1, v_2\}$ son adyacentes. 
            
            Ahora veamos el caso de la gr\'afica de $2$-fichas de $F_n$, con $n
            \geq 4$ y conjunto de v\'ertices $V(F_n)=\{v_1,\dots, v_n\}$.
            Tomamos el ciclo
    
            $C = (\{v_1, v_{n-1}\},\{v_1, v_n\},\{v_1, v_{n-2}\}, \{v_1,
            v_{n-3}\}, \dots, \{v_1, v_2\},$
            
            $\{v_2, v_n\}, \{v_2, v_{n-1}\}, \{v_2, v_{n-2}\},
            \dots, \{v_2, v_3\},$
    
            $ \dots,$
    
            $\{v_{n-3},v_n\}, \{v_{n-3}, v_{n-1}\},\{v_{n-3},v_{n-2}\},
            \{v_{n-2}, v_n\},\{v_{n-2}, v_{n-1}\}, \{v_{n-1}, v_n\}, \{v_1,
            v_{n-1}\})$
    
            Observamos que $C$ es un ciclo hamiltoniano que cumple tener la
            adyacencia de  $\{v_{n-2}, v_n\}$ y $\{v_{n-2}, v_{n-1}\}$. 
    
            Ahora supongamos que $F_{k'}(F_{n'})$ satisface las hipotesis de
            inducci\'on, con $k'< k$ y $n'<n'$. 
            %\'o $F_k(F_{n'})$ satisface las hipotesis de inducci\'on, con 
            %$n'<n$ y $k < n'$. 
            Sea $F_{k-1}(F_{n-i})$, con $1 \leq i \leq n-k$. Por
            \cref{prop:iso-SFan}, tenemos que $F_{k-1}(F_{n-i}) \cong S_i$,
            donde $S_i= F_k(F_n, V_i)$. Adem\'as, por hip\'otesis de
            inducci\'on, $F_{k-1}(F_{n-i})$ tiene un ciclo hamiltoniano, $C_i$,
            con los v\'ertices $X_i = \{v_i, v_{n-k+1}, v_{n-k+2}, \dots,
            v_{n-2}, v_n\}$ y $Y_i= \{v_i, v_{n-k+1}, v_{n-k+2}, \dots, v_{n-2},
            v_{n-1}\}$ adyacentes en $C_i$ . Nombramos $P_i$ a la subtrayectoria
            hamiltoniana de $X_i$ a $Y_i$ y denotamos $Z$ al v\'ertice
            $\{v_{n-k+1}, v_{n-k+2}, \dots, v_n\}$. Observamos que para $i=
            n-k+1$ tenemos que $V_i = \{Z\}$, donde $V_i$ es el conjunto
            mencionado en \cref{prop:iso-SFan}. Esto \'ultimo, pues los
            elementos estan colocados de forma creciente, de manera que $Z$ es
            el \'unico v\'ertice que cumple tener elemento m\'inimo $v_{n-k+1}$. 
    
            Ahora, sea $D_j =\{v_{n-k}, v_{n-k+1}, \dots, v_{n-1}, v_n\}
            \setminus \{v_j\}$, con $n-k+1 \leq j \leq n$.  De esta manera,
            podemos ver al conjunto de v\'ertices de $S_{n-k}$ como
            $V_{n-k}=\{D_n, D_{n-1}, \dots, D_{n-k+1}\}$. Observamos que
            $D_{j-1} \triangle D_j= \{v_{j-1}, v_j\}$, para $n-k+2 \leq j \leq
            n-2$. Entonces tenemos la trayectoria $Q= (D_n, D_{n-1}, \dots,
            D_{n-k+1})$ en $S_{n-k}$.     
            Luego, fij\'emonos en que, para $j= n-k$, tenemos que $Y_j = D_n$ y
            $X_j= D_{n-1}$. Adem\'as tenemos las siguientes diferencias
            sim\'etricas $X_{n-k}\triangle D_{n-2} =\{v_{n-2}, v_n\}$,
            $Y_{n-k}\triangle D_{n-2} =\{v_{n-2}, v_{n-1}\}$ , $Z\triangle
            D_{n-k+1} =\{v_{n-k}, v_{n-k+1}\}$. Por lo tanto, en $F_k(F_n)$,
            tenemos las siguientes adyacencias: $X_{n-k} \sim D_{n-2}$, $Y_{n-k}
            \sim D_{n-2}$ y $Z \sim D_{n-k+1}$. Adicionalmente, observamos que,
            para $1 \leq i \leq n- k- 1$, tenemos $X_{i} \triangle X_{i+1}=
            \{v_i, v_{i+1}\}$ y $Y_i \triangle Y_{i+1} = \{v_i, v_{i+1}\}$.
            Tambi\'en tenemos que $Y_1 \triangle Z = \{v_1, v_n\}$. Por lo tanto
            tenemos que $Y_1 \sim Z$, $X_i \sim X_{i+1}$ y $Y_i \sim Y_{i+1}$,
            para toda $1 \leq i \leq n-k-1$.
            
            Por lo tanto podemos definir el siguiente ciclo hamiltoniano $C$ en
            $F_k(F_n)$: 
    
            En el caso en el que $n-k$ sea par:
            
            $C =(X_1 \xrightarrow[P_1]{} Y_1, Y_2 \xrightarrow[P_2]{} X_2,
            \dots, X_{n-k-1} \xrightarrow[P_{n-k-1}]{} Y_{n-k-1}, Y_{n-k},
            X_{n-k}, D_{n-2} \xrightarrow[Q]{} D_{n-k+1}, Z, X_1)$
    
            En el caso en el que $N-k$ sea impar:
    
            $C = (X_1 \xrightarrow[P_1]{} Y_1, Y_2 \xrightarrow[P_2]{} X_2,
            \dots, Y_{n-k-1} \xrightarrow[P_{n-k-1}]{} X_{n-k-1}, X_{n-k},
            Y_{n-k}, D_{n-2} \xrightarrow[Q]{} D_{n-k+1}, Z, X_1)$
    
            En ambos casos $C$ es un ciclo hamiltoniano que, adem\'as, cumple
            que los v\'ertices $X_{n-k}$ y $Y_{n-k}$ son adyacentes. 
    
        \end{proof}