\chapter{Definiciones de Teor\'ia de Gr\'aficas}%
\label{cap:defs grafs}

\section{Definiciones b\'asicas}%
\label{sec:def-basicas}

El objeto a estudiar en el \'area de Teor\'ia de Gr\'aficas es, naturalmente,
una gr\'afica. Una \indice{gr\'afica} $G$ es una pareja ordenada de conjuntos
finitos $(V(G), E(G))$, donde $V(G)$ es no vac\'io y $E(G) \subseteq
\binom{V(G)}{2}$. Los elementos de $V(G)$ son llamados \indice{v\'ertices} y los
elementos de $E(G)$ son \indice{aristas}. Para una arista $e$ y v\'ertices $u,
v$ de $G$, decimos que $e$ es \indiceSub{aristas}{incidente} en $u$ y en $v$ si
$e= \{u, v\}$. A su vez, los v\'ertices $u$ y $v$ son
\indiceSub{v\'ertices}{extremos} de la arista $e$. De igual manera, podemos
decir que $u$ y $v$ son \textbf{v\'ertices}
\indiceSub{v\'ertices}{adyacentes}\index{adyacentes!v\'ertices} o
\indiceSub{v\'ertices}{vecinos} y lo denotamos $u \sim v$. Por otro lado,
decimos que son \textbf{aristas}
\indiceSub{aristas}{adyacentes}\index{adyacentes!aristas} cuando dos aristas
comparten un extremo. 

Al conjunto de vecinos de un v\'ertice $v$ se le llama
\indice{vecindad}, y se denota $N_G(v)$, mientras que al n\'umero de aristas
incidentes en $v$ se le llama \indice{grado} de $v$, $d(v)$. Un
\textbf{v\'ertice} \indiceSub{v\'ertices}{aislado}, es decir, un v\'ertice que
no tiene vecinos, tiene grado $0$.

\begin{figure}[ht!]
    \centering
       \begin{tikzpicture}
        
            \begin{scope}[xshift=6cm,scale=0.8]
                \draw (0.5,-1.5) node (1) [vertex, label=270:{\large $v_1$}] {};
                \draw (-1.9,-1.6) node (2) [vertex, label=270:{\large $v_2$}] {};
                \draw (1.2,1.5) node (3) [vertex, label=90:{\large $v_3$}] {};
                \draw (-2,0.5) node (4) [vertex, label=90:{\large $v_4$}] {};
                \draw (2.5,-0.6) node (5) [vertex, label=270:{\large $v_5$}] {};
                \draw (0,0) node (6) [vertex, label=90:{\large $v_6$}] {};
                \draw (3,1) node (7) [vertex, label=90:{\large $v_7$}] {};
                
                \foreach \i/\j in {1/3,1/4,1/6,2/6,3/4,3/5,5/6}
                \draw [edge,grisOscuro] (\i) to (\j);
            \end{scope}
    
        \begin{scope}[xshift=0cm,scale=0.6]
            \draw ({(360/4)*3}:2) node(1)[vertex, label=270:{\large $v_1$}]{};
            \draw (0,0) node (2) [vertex, label=90:{\large $v_2$}] {};
            \draw ({(360/4)*2}:2) node(3)[vertex, label=180:{\large $v_3$}]{};
            \draw (-2.5,-2) node (4) [vertex, label=270:{\large $v_4$}] {};
            \draw ({(360/4)*1}:2) node(5)[vertex, label=90:{\large $v_5$}]{};
            \draw ({(360/4)*0}:2) node(6)[vertex, label=0:{\large $v_6$}]{};
            \draw (-2.5,2) node (7) [vertex, label=90:{\large $v_7$}] {};

            \foreach \i/\j in {1/3,1/4,1/6,2/6,3/4,3/5,5/6}
                \draw [edge,grisOscuro] (\i) to (\j);  
            \end{scope}
                
    \end{tikzpicture}
    \caption{Diferentes diagramas de una gr\'afica $G$}
    \label{fig:diagGraf}
\end{figure}

Las gr\'aficas pueden ser representadas por \indice{diagramas} donde los
v\'ertices son c\'irculos y las arista son l\'ineas cuyos extremos son sus
v\'ertices incidentes. Cabe recalcar que una gr\'afica puede tener m\'ultiples
diagramas, como se muestra en \cref{fig:diagGraf}. Es f\'acil observar que en
ambos diagr\'amas de $G$ los v\'ertices tienen las mismas propiedades, por
ejemplo que $N_G(v_3)=\{v_1,v_4,v_5\}$ o que el grado de $v_2$ es $1$.   Esto se
debe a que la gr\'afica es un objeto matem\'atico, independiente del diagrama.


Al momento de estudiar objetos matem\'aticos, siempre es importante analizar
como se relacionan dichos objetos entre s\'i. Ya teniendo la noci\'on de una
gr\'afica, ahora nos enfocamos en c\'omo se ven las relaciones entre gr\'aficas.
Veamos que necesitar\'ian dos gr\'aficas para ser ``iguales''. Observamos que,
de igual manera que una gr\'afica puede tener varios diagramas, a un diagrama se
le pueden asociar distintas gr\'aficas, renombrando v\'ertices y aristas, para
obtener distintos conjuntos de v\'ertices y aristas. Un ejemplo de esto es
\cref{fig:isoGraf}. Al renombrar las aristas de la gr\'afica $G$, podemos
dibujar a $G$ como la gr\'afica $H$ y viseversa, usando como gu\'ia los colores
que se muestran en la figura. De manera intuitiva, las dos gr\'aficas que
comparten un mismo diagrama podr\'iamos decir que son ``iguales'' hasta donde
nos interesa, pues, al compartir un diagrama, comparten muchas propiedades como
el n\'umero de v\'ertices, las adyacencias o el grado de los v\'ertices. Estas
g\'aficas se dice que son isomorfas. Formalmente hablando, dos gr\'aficas $G$ y
$H$ son \indiceSub{gr\'afica}{isomorfas} si exite una biyecci\'on $\theta: V(G)
\rightarrow V(H)$ tal que para cualesquiera dos v\'ertices $u, v \in V(G)$, se
cumple que $uv \in E(G)$ si y s\'olo si $\theta(u)\theta(v) \in E(H)$. Es decir,
la biyecci\'on $\theta$ preserva adyacencias y no adyacencias. Cuando $G$ es
ismorfa a $H$ lo denotamos $G \cong H$ y la biyecci\'on entre las gr\'aficas es
llamada isomorfismo.

\begin{figure}[ht!]
    \centering
       \begin{tikzpicture}
    
            \begin{scope}[xshift=0cm,scale=0.8]
             \draw ({(360/5)*1}:2) node(1)[vertex,fill=morado, label=(360/5)*1:{\large
                    $v_{1}$}]{};
            \draw ({(360/5)*2}:2) node(2)[vertex,fill=amarillo, label=(360/5)*2:{\large
                    $v_{2}$}]{};
           \draw ({(360/5)*3}:2) node(3)[vertex,fill=azulCielo, label=(360/5)*3:{\large
                    $v_{3}$}]{};
            \draw ({(360/5)*4}:2) node(4)[vertex,fill=coral, label=(360/5)*4:{\large
                    $v_{4}$}]{};
            \draw ({(360/5)*5}:2) node(5)[vertex,fill=verde, label=(360/5)*5:{\large
                    $v_{5}$}]{};

            \foreach \i/\j in {5/1,1/2,2/3,3/4,4/5}
                \draw [edge,grisOscuro] (\i) to (\j);
            \end{scope}
    
        \begin{scope}[xshift=6cm,scale=0.8]
            \draw ({(360/5)*1}:2) node(1)[vertex,fill=morado, label=(360/5)*1:{\large
            $v_{1}$}]{};
    \draw ({(360/5)*2}:2) node(2)[vertex,fill=coral, label=(360/5)*2:{\large
            $v_{2}$}]{};
   \draw ({(360/5)*3}:2) node(3)[vertex,fill=amarillo, label=(360/5)*3:{\large
            $v_{3}$}]{};
    \draw ({(360/5)*4}:2) node(4)[vertex,fill=verde, label=(360/5)*4:{\large
            $v_{4}$}]{};
    \draw ({(360/5)*5}:2) node(5)[vertex,fill=azulCielo, label=(360/5)*5:{\large
            $v_{5}$}]{};

            \foreach \i/\j in {5/2,2/4,4/1,1/3,3/5}
                \draw [edge,grisOscuro] (\i) to (\j);
            \end{scope}
                
    \end{tikzpicture}
    \caption{Dos gr\'aficas isomorfas, $G$ y $H$.}
    \label{fig:isoGraf}
\end{figure}

Al momento de ver las relaciones entre dos gr\'aficas, tambi\'en es importante
analizar el caso en el que una gr\'afica este ``contenida'' en la otra. En el
caso de conjuntos nos referir\'iamos a subconjunto y en el caso de gr\'aficas
nos vamos a referir a subgr\'aficas. Sean $G$ y $H$ tales que $V(H) \subseteq
V(G)$ y $E(H) \subseteq E(G)$. Entonces decimos que $H$ es una
\indice{subgr\'afica} de $G$. Por otra parte, decimos que $G$ es una
\indice{supergr\'afica} de $H$. Esta relaci\'on la denotamos con la misma
notaci\'on que para subconjuntos, es decir $H \subseteq G$. Un concepto que se
deriva de las subgr\'aficas parte de fijarnos en la subgr\'afica que tiene la
mayor cantidad de aristas posibles. Formalmente hablando, nos referimos a $H$,
la subgr\'afica de $G$, cuyo conjunto de v\'ertices es $S \subseteq V(G)$ y cuyo
conjunto de aristas son aquellas aristas de $G$ que tienen ambos extremos en
$S$, es decir, $E(H) = \{uv \in E(G) \colon\ u,v \in S\}$. A esta subgr\'afica
la llamamos la \textbf{subgr\'afica} \indiceSub{subgr\'afica}{inducida} de $G$
por $S$ y la denotamos $G[S]$. Notemos que esta gr\'afica $G[S]$ es la
subgr\'afica de $G$ con conjunto de v\'ertices $S$ que m\'as se parece a $G$.
Otro ejemplo importante de subgr\'aficas es la subgr\'afica inducida obtenida al
quitarle uno o m\'as v\'ertices a una gr\'afica. En otras palabras, sea $S
\subset V(G)$, nombramos $H \subseteq G$ a la gr\'afica con $V(H)=V(G) \setminus
S$ y $E(H) = E(G)\setminus\{uv \in E(G) \colon\ u \in S \lor v \in S\}$. Tenemos
que $H$ es la subg\'afica obtenida a partir de $G$ al quitarle el conjunto $S$,
esta subgr\'afica se denota $G-S$. Un caso a se\~{n}alar es cuando $S$ es un
conjunto unitario, es decir, el caso en el que le quitamos un solo v\'ertice $v$
a $G$. En este caso escribiremos $G-v$ en vez de $G-\{v\}$. A continuaci\'on se
muestra un ejemplo de subgr\'aficas en \cref{fig:subgraf}, donde
${\color{morado}\bf H}$ es una subgr\'afica inducida con $V(H) =
\{v_1,v_2,v_3,v_4\}$ y ${\color{azulCielo}\bf H'}$ es una subgr\'afica con
$V(H')=\{v_5,v_6,v_7\}$ y $E(H')=\{v_5v_6,  v_6v_7\}$. Notamos que a
${\color{azulCielo}\bf H'}$ le falta la arista ${\color{grisOscuro!80}\bf
v_5v_7}$, resaltada en la figura, para ser una subgr\'afica inducida. 


%Como hacer que aparezcan letras con acento usando \color
\begin{figure}[ht!]
    \centering
       \begin{tikzpicture}
    
            \begin{scope}[xshift=0cm,scale=1]
                \foreach \i in {1,...,4} 
                    \draw ({(360/7)*\i}:2) node(\i)[vertex, fill=morado,
                    label=(360/7)*\i:{\large $v_{\i}$}]{};
                
                \foreach \i in {5,...,7} 
                    \draw ({(360/7)*\i}:2) node(\i)[vertex,fill=azulCielo,
                    label=(360/7)*\i:{\large $v_{\i}$}]{};

            \foreach \i/\j in
                {1/5,1/6,1/7,2/5,2/6,2/7,3/5,3/6,3/7,4/5,
            4/6,4/7}
                \draw [edge,grisOscuro!50] (\i) to (\j);
            
            \foreach \i/\j in
                {1/2,1/3,1/4,2/3,2/4,3/4}
                \draw [wedge,morado] (\i) to (\j);
            
            \foreach \i/\j in
                {5/6,6/7}
                \draw [wedge,azulCielo] (\i) to (\j);
            
            \draw [wedge,grisOscuro!80] (5) to (7);
            \end{scope}
                
    \end{tikzpicture}
    \caption{Una gr\'afica en la que se resalta una subgr\'afica inducida,
    ${\color{morado}\bf H}$, y una subgr\'afica, ${\color{azulCielo}\bf H'}$.}
    \label{fig:subgraf}
\end{figure}


\section{Clanes y conjuntos independientes}
\label{sec:clanes-CIndep}

    Algo que es relevante a trav\'es de varios temas de la Teor\'ia de
     Gr\'aficas es encontrar conjuntos de v\'ertices que sean adyacentes dos a
     dos o conjuntos de v\'ertices que, al contrario, no tengan vecinos dentro
     del conjunto. Teniendo una gr\'afica $G$ y un conjunto $S \subset V(G)$.
     Decimos que $S$ es un \indice{conjunto independiente} si cualesquiera dos
     v\'ertices de $S$ no son adyacentes. Esto quiere decir que todos los
     elementos de $S$ tienen grado $0$ en $G[S]$. La cardinalidad del conjunto
     independiente m\'as grande de una gr\'afica se llama \indice{n\'umero de
     independencia} y se denota $\alpha$. Por otro lado, decimos que $S$ es un
     \indice{clan} si cualesquiera dos v\'ertices en $S$ son adyacentes. En este
     caso, el grado de todo v\'ertice de $S$ es $|S-1|$ en $G[S]$. El
     \indice{n\'umero de clan} es la cardinaliad del clan con mayor n\'umero de
     elementos de una gr\'afica, se denota $\omega$. En \cref{fig:ClanInd} se
     muestra un ejemplo de un ${\color{coral}\bf clan}$ y un ${\color{verde}\bf
     conjunto}$ ${\color{verde}\bf independiente}$ de una gr\'afica. Es f\'acil
     observar que el ${\color{verde}\bf conjunto}$ ${\color{verde}\bf
     independiente}$ no es el conjunto independiente m\'as grande que podemos
     obtener, pues podr\'iamos agregar el v\'ertice $v_2$ y seguir\'ia siendo un
     conjunto independiente. De igual manera, podemos notar que existe un clan
     con cardinalidad mayor al ${\color{coral}\bf clan}$ mostrado. Este clan,
     llam\'emoslo $K$, est\'a mostrado con el per\'imetro de los v\'ertices de
     color ${\color{azulCielo}\bf azul}$. Es f\'acil ver que $|K| = \omega$.


\begin{figure}[ht!]
    \centering
       \begin{tikzpicture}
    
            \begin{scope}[xshift=0cm,scale=1]
                \foreach \i in {5,...,7} 
                    \draw ({(360/4)*\i}:1.3) node(\i)[avertex, fill=coral,
                    label=(360/4)*\i:{\large $v_{\i}$}]{};
                \draw ({(360/4)*8}:1.3)node(8)[avertex,label=(360/4)*8:{\large
                    $v_{8}$}]{};
                
                \foreach \i in {1,3} 
                    \draw ({(360/8)*(\i*2-1)}:3) node(\i)[bvertex,fill=verde,
                    label=(360/8)*(\i*2-1):{\large $v_{\i}$}]{};
                
                \foreach \i in {2,4}
                    \draw ({(360/8)*(\i*2-1)}:3)
                   node(\i)[bvertex,label=(360/8)*(\i*2-1):{\large $v_{\i}$}]{};


            \foreach \i/\j in
                {1/5,1/8,2/5,2/6,2/4,3/6,3/7,3/4,4/7,4/8}
                \draw [edge,grisOscuro!50] (\i) to (\j);
            
            \foreach \i/\j in
                {5/6,5/7,6/7}
                \draw [wedge,coral] (\i) to (\j);
            
            \foreach \i/\j in {8/5,8/6,8/7} 
                \draw [wedge,azulCielo!60] (\i) to (\j);
            \end{scope}
                
    \end{tikzpicture}
    \caption{Una gr\'afica, resaltando dos clanes y un conjunto independiente.}
    \label{fig:ClanInd}
\end{figure}


\section{Caminos y conexidad}
\label{sec:CamConex}

%camino/trayectoria/ciclo generador
%primera instancia
%comunmente

En Teor\'ia de Gr\'aficas resulta interesante preguntarse si, en una
gr\'afica, hay alg\'un ``v\'inculo'' entre dos v\'ertices. Si estos dos
v\'ertices son adyacentes, es claro que ese v\'inculo existe a trav\'es de
la arista que comparten. Un \indice{$uv$-camino} entre los v\'ertices $u$ y
$v$ de una gr\'afica $G$ es una sucesi\'on alternada de v\'ertices y aristas
de $G$ de la siguiente forma, $W=(v_0, e_1,v_1, \dots, e_{k-1},v_{k-1},
e_k,v_k)$ con $v_i \in V(G)$ y $v_{j-1}v_j = e_j \in E(G)$, para $i \in \{0,
\dots, k\}$ y $j \in \{ 1, \dots, k\}$. Este $uv$-camino es una forma de
vincular el v\'ertice $u$ con el v\'ertice $v$. En caso de que no sea
necesario especificar de qu\'e v\'ertice a qu\'e v\'ertice va la suceci\'on,
nos referiremos simplemente a un \indice{camino}. Al ser una sucesi\'on, a
un camino $W$ se le puede asociar una longitud, que ser\'a el n\'umero de
posiciones en la sucesi\'on $W$ que son ocupadas por aristas, se denota
$\ell(W)$ . 

Al poderle asignar una longitud a un camino, es natural utilizarla para ver
que tan ``cercano'' esta un v\'ertice de otro. Pero, observamos que, un
camino puede pasar m\'ultiples veces por el mismo v\'ertice o la misma
arista. En otras palabras, el concepto de camino no ayuda mucho para
encontrar que tan ``cerca'' esta un v\'ertice de otro, puesto que puede
existir otro camino que ocupe los mismos v\'ertices y aristas multiples
veces. Por lo tanto, introducimos el concepto de trayectoria. Una
\indice{trayectoria} es un camino que no repite v\'ertices, m\'as a\'un, una
\indice{$uv$-trayectoria} es una trayectoria que tiene v\'ertice inicial $u$
y v\'ertice final $v$. En una $uv$-trayectoria, los v\'ertices $u$ y $v$ son
llamados \indiceSub{v\'ertices}{extremos} y el resto de los v\'ertices de la
trayectoria son llamados \textbf{v\'ertices}
\indiceSub{v\'ertices}{internos}. Notemos que, al no repetir v\'ertices,
este camino tampoco repite aristas, pues, si lo hiciera, necesitar\'ia
volver a pasar por al menos uno de los extremos de dicha arista. De lo
anterior se sigue que todas las trayectorias que tengan los mismos
v\'ertices y aristas tambi\'en tienen la misma longitud. Esto es un mejor
acercamiento al concepto de ``cercan\'ia'' entre v\'ertices, pero se puede
tener el caso que haya m\'ultiples trayectorias de distintas longitudes
entre dos v\'ertices. Entonces, definimos la \indice{distancia} entre dos
v\'ertices $u$ y $v$ de una gr\'afica, denotada $d(u,v)$, como la longitud
de la $uv$-trayectoria m\'as corta. Sea $P$ una trayectoria en una
gr\'afica, decimos que $P$ no tiene \indice{cuerdas} si la subgr\'afica
inducida por $V(P)$ tiene grado m\'aximo 2. Observamos que cualquier
trayectoria de longitud m\'inima entre dos v\'ertices no tiene cuerdas.
Adicionalmente, podemos darle un ``tama\~{n}o'' a una gr\'afica $G$ al
definir su \indice{di\'ametro} como $\max_{v\in G}\{\max_{u\in
G}\{d(u,v)\}\}$. 
%Checar si esa notacion ya se uso

La siguiente proposici\'on nos muestra una relaci\'on entre caminos y
trayectorias.  

\begin{proposicion}
\label{prop:CamTray}
    En una gr\'afica $G$ con $u, v \in V(G)$, $u \ne v$, todo $uv$-camino
    contiene una $uv$-trayectoria.
\end{proposicion}

\begin{proof}
    Sea $W$ un $uv$-camino en una gr\'afica $G$, con $u,v \in V(G)$.
    Demostraremos, por inducci\'on sobre la longitud de $W$, que $W$
    contiene una $uv$-trayectoria. Primero, supongamos que $\ell(W)=1$, esto
    quiere decir que $W$ solo contiene una arista, por lo tanto $W$ es una
    trayectoria. Ahora, supongamos que todo $uv$-camino con longitud menor a
    $k$ contiene una $uv$-trayectoria. Tomamos $W$, tal que $\ell(W) = k$.
    Adem\'as, supongamos que $W$ no es una trayectoria, pues en el caso
    contrario, $W$ ser\'ia la trayectoria buscada. Tomamos $W=
    (w_0,e_1,w_1,e_2, \dots, e_n,w_n)$, con $w_0=u$, $w_n=v$, $e_i \in E(G)$
    y $w_j \in V(G)$, donde $i \in \{1, \dots, n\}$ y $j \in \{0, \dots,
    n\}$. Sea $m \in \{0, \dots, n\}$ el \'indice del primer v\'ertice de
    $W$ que se repite y sea $l \in \{m+1, \dots, n\}$ tal que $w_m = w_l$.
    Entonces, formamos el camino $W'= (w_0,e_1,w_1,\dots w_m, e_{l+1},
    w_{l+1} \dots, e_n,w_n)$. Observamos que $\ell(W')<\ell(W)$, por lo que,
    por hip\'otesis de inducci\'on, $W'$ contiene una $uv$-trayectoria. Por
    lo tanto $W$ tambi\'en contiene una $uv$-trayectoria.
\end{proof}

Por \cref{prop:CamTray}, sabemos que, cada que hablemos de caminos, siempre hay
una trayectoria contenida por lo que ahora nos enfocaremos en trayectorias. Como
se mencion\'o anteriormente, entre dos v\'ertices $u$ y $v$ pueden existir
m\'ultiples trayectorias. Decimos que $X$ y $Y$ son dos
\textbf{$uv$-trayectorias} \indiceSub{$uv$-trayectoria}{internamente ajenas} si
$V(X)\cap V(Y)=\{u,v\}$. Tambi\'en podemos definir un concepto similar
enfoc\'andonos en aristas, es decir, dos $uv$-trayectorias que no tengan aristas
en com\'un. A estas trayectorias las llamamoas \textbf{$uv$-trayectorias}
\indiceSub{$uv$-trayectoria}{ajenas por aristas}. Estos conceptos nos sevir\'ann
m\'as adelante cuando hablemos de conexidad. Adicionalmente, siempre podemos
recorrer una trayectoria en sentido contrario, es decir, empezando por el
\'ultimo v\'ertice y terminando por el primero. Al recorrer la trayectoria de
esta manera obtenemos otra trayectoria. Formalmente hablando, teniendo una
$uv$-trayectoria $X$, su \textbf{trayectoria} \indiceSub{trayectoria}{inversa},
a la cu\'al denotamos $X^{-1}$, es una $vu$-trayectoria que tiene los mismos
v\'ertices y aristas que $X$.

\begin{figure}[htb!]
    \centering
        \begin{tikzpicture}
    
            \begin{scope}[xshift=0cm,scale=1]
                \foreach \i in {1,2} \draw ({(360/6)*\i}:2.5)
                    node(\i)[vertex, label=(360/6)*\i:{\large $v_{\i}$}]{};
            
            \foreach \i in {3,5} \draw ({(360/6)*\i}:2.5)
                    node(\i)[vertex, fill =coral, label=(360/6)*\i:{\large $v_{\i}$}]{};
            
            \draw ({(360/6)*6}:2.5) node(6)[vertex, fill=morado, label=(360/6)*6:{\large $v_{6}$}]{};
            \draw ({(360/6)*4}:2.5) node(4)[vertex, fill=verde, label=(360/6)*4:{\large $v_{4}$}]{};
            \draw (1,0) node (7) [vertex,fill=morado, label=300:{\large $v_7$}] {};
            \draw (-1,0) node (8) [vertex,fill=morado, label=150:{\large $v_8$}] {};
                
            \foreach \i/\j in
                {1/6,2/3,2/8,4/8,5/7,5/8}
                \draw [edge,grisOscuro!50] (\i) to (\j);
            
                \foreach \i/\j in
                {3/8,7/8,2/7,1/2,1/7,6/7,5/6}
                \draw [wedge,morado] (\i) to (\j);
            
            \foreach \i/\j in
                {3/4,4/5}
                \draw [wedge,verde] (\i) to (\j);

            \end{scope}
                
    \end{tikzpicture}
    \caption{Una gr\'afica en la que se resaltan las ${\color{morado}\bf
    aristas}$ de un ${\color{coral}\bf v_3 v_5}$-camino con su
    ${\color{coral}\bf v_3 v_5}$-trayectoria contenida, resaltando sus
    ${\color{morado}\textbf{v\'ertices}}$.  Tambi\'en se resalta una
    ${\color{coral}\bf v_3 v_5}$-trayectoria sin cuerdas de color
    ${\color{verde}\bf verde}$.}
    \label{fig:ex-caminos}
\end{figure}
    
En \cref{fig:ex-caminos} podemos ver un ejemplo de \cref{prop:CamTray} donde se
resaltan las ${\color{morado}\bf aristas}$ de un $v_3 v_5$-camino y donde, para
mostrar la trayectoria contenida, se resaltan los
${\color{morado}\textbf{v\'ertices}}$ de una $v_3 v_5$-trayectoria.
Adicionalmente, en la misma figura se observan dos ${\color{coral}\bf v_3
v_5}$-trayectorias internamente ajenas. Ambas est\'an representadas con sus
v\'ertices internos coloreados del mismo color, una de ${\color{morado}\bf
morado}$ y la otra de ${\color{verde}\bf verde}$. Notamos que la
${\color{morado}\textbf{trayectoria morada}}$ tiene cuerdas, entre ellas la
arista $v_8v_5$. Por otro lado, la ${\color{verde}\textbf{trayectoria verde}}$
no tiene cuerdas y es la trayectoria m\'as corta entre $v_3$ y $v_5$, por lo que
la distancia entre $v_3$ y $v_5$ es $2$.

Cuando los extremos de un camino son iguales, se llama \textbf{camino}
\indiceSub{camino}{cerrado}. Al igual que en el caso de las trayectorias, vale
la pena resaltar el caso en el que un camino cerrado no repite v\'ertices (salvo
los extremos). Un \indice{ciclo} es un camino cerrado de longitud al menos $3$
que, adem\'as, no repite v\'ertices, salvo los extremos.  Muchas v\'eces nos
interesa saber si una gr\'afica contiene o no contiene ciclos.   Una gr\'afica
sin ciclos es llamada \indiceSub{gr\'afica}{ac\'iclica}.
   
Hasta ahora nos hemos enfocado en los ``v\'inculos'' entre dos v\'ertices de una
gr\'afica. Sin embargo, es importante mencionar que no siempre existe manera de
``vincular'' dos v\'ertices, como se puede ver \cref{fig:diagGraf}, donde no
existen trayectorias que conecten a $v_7$ con alg\'un otro v\'ertice.  Definimos
una gr\'afica $G$ como \indiceSub{gr\'afica}{conexa}\index{conexa!gr\'afica} si,
para cualquier par de v\'ertices $u,v \in V(G)$, existe una $uv$-trayectoria. Si
una gr\'afica no cumple esta propiedad, decimos que es
\indiceSub{gr\'afica}{inconexa}. Observamos que toda gr\'afica, incluso las
gr\'aficas inconexas, contienen ``partes'' que s\'i son conexas, por lo que en
todo caso, al estudiar la conexidad de las gr\'aficas, nos podemos enfocar en
dichas partes. En una gr\'afica $G$, las subgr\'aficas m\'aximas con la
propiedad de ser conexas son llamadas \indice{componentes
conexas}\index{conexa!componente}. Notamos que si una gr\'afica es conexa,
contiene s\'olo una componente conexa, ella misma. 

Algo que es de inter\'es en este tema es, que tan dif\'icil es ``desconectar''
una gr\'afica. Primero veamos un caso m\'as espec\'ifico, qu\'e tan dif\'icil es
``desconectar'' a dos v\'ertices espec\'ificos en una gr\'afica. Para esto puede
resultar natural pensar en las trayectorias internamente ajenas, pues la
cantidad de trayectorias internamente ajenas entre dos v\'ertices nos habla de
cu\'antas maneras diferentes hay de conectar un v\'ertice con otro. Definimos la
\textbf{conexidad} \indiceSub{conexidad}{local} entre dos v\'ertices distintos
$u$ y $v$ como el m\'aximo n\'umero de $uv$-trayectorias internamente ajenas.
Este n\'umero se denota por $p(u,v)$. Ahora, para el caso global de una
gr\'afica no trivial $G$, decimos que $G$ es \indiceSub{gr\'afica}{$t$-conexa}
si, para cualesquiera dos v\'ertices $u,v \in V(G)$, se tiene que $p(u,v) \geq
t$. Por convenci\'on, decimos que la gr\'afica trivial es $0$-conexa y no es
$k$-conexa para ning\'un $k > 0$. Adicionalmente, definimos la
\indice{conexidad} de una gr\'afica como el m\'aximo n\'umero $t$ para el cu\'al
la gr\'afica es $t$-conexa. Un ejemplo de conexidad es considerar las gr\'aficas
en \cref{fig:isoGraf} y \cref{fig:subgraf}. Notamos que ambas son conexas, sin
embargo, necesitar\'iamos ``borrar'' m\'as aristas para ``desvincular''
v\'ertices en la gr\'afica de \cref{fig:subgraf} que en las gr\'aficas de
\cref{fig:isoGraf}, a saber, las gr\'aficas de \cref*{fig:isoGraf} son
$2$-conexas y la gr\'afica de \cref{fig:subgraf} es $6$-conexa.

Otra manera de abordar el ``desconectar'' una gr\'afica es cuestionarse si
existe alg\'un v\'ertice o conjunto de v\'ertices que, al quitarlo,
``desconecte'' a la gr\'afica. A este conjunto lo llamamos corte por v\'ertices.
Formalmente, un \indice{corte por v\'ertices} $S$ de una gr\'afica $G$ es un
subconjunto de $V(G)$ tal que $G[V(G)-S]$ es inconexa. Cabe notar que no todas
las gr\'aficas tienen alg\'un corte por v\'ertices, a saber, las gr\'aficas
completas (introducidas en la siguiente secci\'on) no tienen corte por
v\'ertices y son las \'unicas con esta propiedad.

\begin{figure}[htb!]
    \centering
    \begin{tikzpicture}
    
        \begin{scope}[xshift=0cm,scale=1]
            \foreach \i in {2,3} \draw ({(360/6)*\i}:2.5)
                node(\i)[vertex, label=(360/6)*\i:{\large $v_{\i}$}]{};
            
            \foreach \i in {1,5} \draw ({(360/6)*\i}:2.5)
                node(\i)[wvertex, fill=verde, label=(360/6)*\i:{\large $v_{\i}$}]{};
            
            \draw ({(360/6)*6}:2.5) node(6)[vertex, label=(360/6)*6:{\large $v_{6}$}]{};
            \draw ({(360/6)*4}:2.5) node(4)[vertex, label=(360/6)*4:{\large $v_{4}$}]{};
            \draw (1,0) node (7) [vertex, label=300:{\large $v_7$}] {};
            \draw (-1,0) node (8) [vertex, label=150:{\large $v_8$}] {};
                
            \foreach \i/\j in
                {2/3,5/8,3/8,7/8,2/7,6/7,3/4}
                \draw [edge,grisOscuro!50] (\i) to (\j);
            
            \foreach \i/\j in
                {1/2,2/8,4/8,4/5}
                \draw [wedge,coral] (\i) to (\j);
            
            \foreach \i/\j in
                {1/7,5/7}
                \draw [wedge,azulCielo] (\i) to (\j);
            
            \foreach \i/\j in
                {1/6,5/6}
                \draw [wedge,amarillo] (\i) to (\j);

        \end{scope}

        \begin{scope}[xshift=8cm,scale=1]
            \foreach \i in {1,5} \draw ({(360/6)*\i}:2.5)
                node(\i)[wvertex, fill=verde, label=(360/6)*\i:{\large $v_{\i}$}]{};
            
            \draw ({(360/6)*2}:2.5) node(2)[coralvertex, fill=morado, label=(360/6)*2:{\large $v_{2}$}]{};
            \draw ({(360/6)*3}:2.5) node(3)[vertex, label=(360/6)*3:{\large $v_{3}$}]{};
            \draw ({(360/6)*4}:2.5) node(4)[vertex, label=(360/6)*4:{\large $v_{4}$}]{};
            \draw ({(360/6)*6}:2.5) node(6)[amarvertex, fill=morado, label=(360/6)*6:{\large $v_{6}$}]{};
            \draw (1,0) node (7) [avertex, fill=morado, label=300:{\large $v_7$}] {};
            \draw (-1,0) node (8) [vertex, label=150:{\large $v_8$}] {};
                
            \foreach \i/\j in
                {1/6,1/2,1/7,2/3,2/7,2/8,3/4,3/8,4/5,4/8,5/6,5/7,5/8,6/7,7/8}
                \draw [edge,grisOscuro!50] (\i) to (\j);
            
        \end{scope}
                
    \end{tikzpicture}
    \caption{Tres trayectorias internamente ajenas entre $v_1$ y $v_5$ (izquierda), y un conjunto de tres v\'ertices que separa a $v_1$ de $v_5$ (derecha).}
    \label{fig:ex-menger}
\end{figure}

En \cref{fig:ex-menger} se muestran las dos maneras de ``desconectar'' los
v\'ertices $v_1$ y $v_5$ de una gr\'afica conexa. Del lado izquierdo se muestran
tres $v_1v_5$-trayectorias internamente ajenas, cada una resaltada de un color
diferente. Del lado derecho se muestra el corte por v\'ertices
${\color{morado}\bf \{v_2, v_6, v_7\} }$, donde cada v\'ertice est\'a relleno de
morado. Observemos que cada v\'ertice del corte por v\'ertices se encuentra en
una $v_1v_5$-trayectoria internamente ajena con las dem\'as. Para facilitar esta
visualizaci\'on, cada v\'ertice del corte por v\'ertices tiene el per\'imetro
del mismo color que la $v_1v_5$-trayectoria internamente ajena a la que nos
referimos. La relaci\'on entre ambos conceptos fue demostrada por Menger y es un
teorema de mucha relevancia en Teor\'ia de gr\'aficas. En \cref*{teo:menger} se
demuestra una versi\'on global de este teorema.

% Otra forma de determinar como ``desvincular'' dos v\'ertices es preguntarse
% cuantas aristas se necesitar\'ian ``borrar'' para que no hubiera trayectorias
% entre dichos v\'ertices. Es f\'acil observar que esto est\'a relacionado con la
% cantidad de trayectorias ajenas por aristas
% %% TODO: agregar definición trayectorias ajenas por aristas
% que existan entre dichos
% v\'ertices.

\begin{teorema}
    \label{teo:menger}
Sea $G$ una gr\'afica, $G$ es $k$-conexa si y s\'olo si contiene $k$
trayectorias internamente ajenas entre cualesquiera dos v\'ertices de $G$.
\end{teorema}

\begin{proof}
    
\end{proof}

\section{Operaciones}
\label{sec:operaciones}

Al igual que en muchas \'areas de las matem\'aticas, al buscar maneras de
generar nuevas gr\'aficas, nos encontramos con las operaciones de gr\'aficas. A
continuaci\'on hablaremos de las operaciones de gr\'aficas que son pertinentes
para este trabajo. El primer grupo operaciones que se va a definir se basa en el
concepto de uni\'on de conjuntos. La \indice{uni\'on} de dos gr\'aficas $G$ y
$H$ la definimos como la gr\'afica cuyo conjunto de v\'ertices y de aristas son
la uni\'on de los conjuntos de v\'ertices  y de aristasde ambas gr\'aficas,
respectivamente. Se utiliza la misma notaci\'on que para la uni\'on conjuntos,
es decir $G \cup H$. El siguiente concepto es el de la uni\'on ajena de dos
gr\'aficas. La \textbf{uni\'on} \indiceSub{uni\'on}{ajena} de las gr\'aficas $G$
y $H$ es la uni\'on de $G$ y $H$, donde $V(G) \cap V(H) = \varnothing$. La
uni\'on ajena se denota $G + H$. Por \'ultimo, otra operaci\'on que surge del
concepto de uni\'on de conjuntos y que va a ser de utilidad en este trabajo es
la uni\'on completa. La \textbf{uni\'on} \indiceSub{uni\'on}{completa} de dos
gr\'aficas $G$ y $H$ es la gr\'afica con conjunto de v\'ertices $V(G) \cup V(H)$
y cuyo conjunto de aristas es $E(G) \cup E(H) \cup \{uv \colon u \in V(G), v \in
V(H) \}$. La uni\'on completa la denotamos $G \oplus H$. En \cref{fig:ex-unionCompleta} se muestra, del lado izquierdo, un
ejemplo de dos gr\'aficas, la primera con v\'ertices $v_i$ y la segunda con
v\'ertices $w_i$, ambas con $i \in \{1, 2 ,3\}$. Del lado izquierdo de la figura
se muestra la uni\'on completa de dichas gr\'aficas.

\begin{figure}[htb!]
    \centering
    \begin{tikzpicture}

        \begin{scope}[yshift=1.5cm,scale=1]
            \foreach \i in {1,...,3}
            \draw ({(360/3)*\i-150}:1)
                node(\i)[vertex, label=(360/3)*\i-150:{\large $v_{\i}$}]{};
            \end{scope}

        \begin{scope}[yshift=-1.5cm,scale=1]
            \foreach \i in {4,...,6}
            \draw ({(360/3)*\i+30}:1)
                node(\i)[vertex, label=(360/3)*\i+30:{\large $w_{{\pgfmathparse{int(\i-3)}\pgfmathresult}}$}]{};
        \end{scope}

        \foreach \i/\j in {1/2,2/3,3/1,4/5,5/6,6/4,1/4,1/5,1/6,2/4,2/5,2/6,3/4,3/5,3/6}
        \draw [edge,grisOscuro] (\i) to (\j);

        \begin{scope}[yshift=1.5cm,,xshift=-6cm,scale=1]
            \foreach \i in {1,...,3}
            \draw ({(360/3)*\i-150}:1)
                node(\i)[vertex, label=(360/3)*\i-150:{\large $v_{\i}$}]{};
            \end{scope}

        \begin{scope}[yshift=-1.5cm,xshift=-6cm,scale=1]
            \foreach \i in {4,...,6}
            \draw ({(360/3)*\i+30}:1)
                node(\i)[vertex, label=(360/3)*\i+30:{\large $w_{{\pgfmathparse{int(\i-3)}\pgfmathresult}}$}]{};
        \end{scope}


            \foreach \i/\j in {1/2,2/3,3/1,4/5,5/6,6/4}
                \draw [edge,grisOscuro] (\i) to (\j);
    \end{tikzpicture}
    \caption{Dos gr\'aficas (izquierda) y su uni\'on completa (derecha).}
\label{fig:ex-unionCompleta}
\end{figure}

Otra operaci\'on que se utilizar\'a en este trabajo es el producto cartesiano de
gr\'aficas. Dadas dos gr\'aficas $G$ y $H$, el \indice{producto cartesiano} de
$G$ y $H$, denotado $G \square H$, es la gr\'afica cuyo conjunto de v\'ertices
es el producto cartesiano de $V(G)$ y $V(H)$, es decir $V(G \square H) = V(G)
\times V(H)$. Adem\'as, $(g_1,g_2)(h_1,h_2)$ es arista de $G \square H$ si y
s\'olo si $g_1 = g_2$ y $h_1h_2 \in E(H)$, o $h_1 = h_2$ y $g_1g_2 \in E(G)$. A
continuaci\'on, en \cref{fig:ex-cartesiano}, se muestran dos gr\'aficas, la
primera del lado izquierdo y con conjunto de v\'ertices $\{v_1, v_2\}$ y la
segunda en la parte inferior y con conjunto de v\'ertices $\{w_1,w_2\}$. En
medio de estas gr\'aficas se muestra el producto cartesiano de ambas, resaltando
de donde vienen los v\'ertices de esta \'ultima gr\'afica.

\begin{figure}[ht!]
    \centering
        \begin{tikzpicture}
        
            \begin{scope}[xshift=0cm,scale=0.9]

                \draw ({(360/4)*2 - 45}:2) node(1)[avertex,fill=amarillo, label=(360/4)*2 -
                    45:{\large $k_{1}$}]{};
                \draw ({(360/4) - 45}:2) node(2)[avertex, fill=morado, label=(360/4) -
                    45:{\large $k_{2}$}]{};
                \draw ({(360/4)*3 - 45}:2) node(3)[coralvertex,,fill= amarillo, label=(360/4)*3 - 
                45:{\large $k_{3}$}]{};
                \draw ({(360/4)*4 - 45}:2) node(4)[coralvertex, fill= morado, label=(360/4)*4 - 
                45:{\large $k_{4}$}]{};
                \draw (-3,1.4) node (5) [avertex, label=180:{\large $v_1$}] {};
                \draw (-3,-1.4) node (6) [coralvertex, label=180:{\large $v_2$}] {}; 
                \draw (-1.4,-3) node (7) [bvertex, fill= amarillo, label=270:{\large $w_1$}] {};
                \draw (1.4,-3) node (8) [bvertex, fill= morado, label=270:{\large $w_2$}] {};

                \foreach \i/\j in {1/2,1/3,3/4,2/4,5/6,7/8}
                    \draw [edge,grisOscuro] (\i) to (\j);
                \end{scope}

                    
        \end{tikzpicture}
        \caption{A}
        \label{fig:ex-cartesiano}
    \end{figure}

Por \'ultimo, definimos el \indice{complemento} de una gr\'afica $G$ como la
gr\'afica que tiene el mismo conjunto de v\'ertices que $G$ y cuyo conjunto de
aristas es $\binom{V(G)}{2} - E(G)$, a esta gr\'afica la denotamos $\overline{G}$.


%notacion de complemento de grafica y el ejemplo de K complemento, que es una
%gr\'afica vacia, es decir su numero de independencia es igual al numero de vertices
\section{Algunas familias relevantes}
\label{sec:famGraf}
   
 Muchas veces, en Teor\'ia de Gr\'aficas, al momento de abordrar varios
problemas, es de utilidad separar las gr\'aficas en familias donde se compartan
caracteristcas. Hay varias familias relevantes en la Teor\'ia de Gr\'aficas, en
esta secci\'on hablaremos de las familias que son relevantes para este trabajo y
sus propiedades.

\begin{figure}[ht!]
    \centering
        \begin{tikzpicture}
        
            \begin{scope}[xshift=0cm,scale=0.8]

                \foreach \i in {1,2,3} \draw ({(360/3)*\i}:1.8)
                    node(\i)[vertex, label=(360/3)*\i:{\large $v_{\i}$}]{};
                \end{scope}
        
            \begin{scope}[xshift=7cm,scale=0.8]

                \foreach \i in {1,...,5} \draw ({(360/5)*\i}:2.5)
                    node(\i)[vertex, label=(360/5)*\i:{\large $v_{\i}$}]{};
                
                \foreach \i/\j in {1/2,1/3,1/4,1/5,2/3,2/4,2/5,3/4,3/5,4/5}
                    \draw [edge,grisOscuro] (\i) to (\j);
                \end{scope}
                    
        \end{tikzpicture}
        \caption{La gr\'afica vac\'ia de 3 v\'ertices (izquierda) y la
        gr\'afica completa de 5 v\'ertices (derecha).}
        \label{fig:ex-vacomp}
    \end{figure}

Decimos que una gr\'afica $G$ es \indiceSub{gr\'afica}{completa} si su conjunto
de aristas es igual a $\binom{V(G)}{2}$. A la gr\'afica completa de orden $n$ se
le denota $K_n$. Cuando una gr\'afica est\'a conformada \'unicamente por
v\'ertices aislados decimos que es una gr\'afica \indiceSub{gr\'afica}{vac\'ia}
y la denotamos $\overline{K_n}$, donde $n$ es el n\'umero de v\'ertices de la
gr\'afica. A $K_1$ tambi\'en se le conoce como la gr\'afica  \indice{trivial}.

%definimos particion?
Una familia de gr\'aficas que es muy estudiada en Teor\'ia de Gr\'aficas es la
familia de las \textbf{gr\'aficas }\indiceSub{gr\'afica}{$k$-partitas}. Una
gr\'afica es $k$-\textit{partita} si su conjunto de v\'ertices admite una
partici\'on en a lo m\'as $k$ conjuntos independientes. Un caso muy relevante, y
que nos ser\'a particularmente \'util en este trabajo, es el caso en el que
$k=2$. En este caso se dice que es una \textbf{gr\'afica}
\indiceSub{gr\'afica}{bipartita}\index{bipartita!gr\'afica}. Una gr\'afica
bipartita $G$ con partici\'on $(X,Y)$ se denota $G[X,Y]$, y $(X,Y)$ es llamada
\indice{bipartici\'on}. Adem\'as, existe un caso especial de las gr\'aficas
$k$-partitas, es cuando todas las aristas posibles est\'an presentes, por
cuestion de utilidad para el trabajo, definiremos formalmente este
comportamiento \'unicamente para las gr\'aficas bipartitas. Una gr\'afica es
\indiceSub{gr\'afica}{bipartita completa}\index{bipartita!completa} si es
bipartita y dada la partici\'on $X$ y $Y$ de sus v\'ertices, cada v\'ertice de
$X$ es adyacente a cada v\'ertice de $Y$. 

\begin{figure}[ht!]
    \centering
        \begin{tikzpicture}
        
            \begin{scope}[xshift=0cm,scale=0.8]

                \foreach \i in {1,3,5} \draw ({(360/6)*\i}:2)
                    node(\i)[vertex, fill=coral, label=(360/6)*\i:{\large $v_{\i}$}]{};
                
                \foreach \i in {2,4,6} \draw ({(360/6)*\i}:2)
                    node(\i)[vertex, fill=azulCielo, label=(360/6)*\i:{\large $v_{\i}$}]{};
    
                \foreach \i/\j in {1/2,2/3,3/4,4/5,5/6,6/1}
                    \draw [edge,grisOscuro!50] (\i) to (\j);
                \end{scope}
        
            \begin{scope}[xshift=6cm,scale=0.8]

                \foreach \i in {1,3} \draw ({(360/4)*\i}:2.3)
                    node(\i)[vertex, fill=morado, label=(360/4)*\i:{\large $v_{\i}$}]{};
                
                \foreach \i in {2,4} \draw ({(360/4)*\i}:2.3)
                    node(\i)[vertex, fill=verde, label=(360/4)*\i:{\large $v_{\i}$}]{};
               
                \draw (0,0) node (5) [vertex, fill=verde, label=300:{\large $v_5$}] {};

                \foreach \i/\j in {1/4,1/5,1/2,2/3,3/4,3/5}
                    \draw [edge,grisOscuro!50] (\i) to (\j);
                \end{scope}
                    
        \end{tikzpicture}
        \caption{El $6$-ciclo como ejemplo de una gr\'afica bipartita (izquierda), y una gr\'afica bipartita completa (derecha).}
        \label{fig:ex-bip}
    \end{figure}


Un teorema muy importante al hablar de gr\'aficas bipartitas es
\cref{teo:bip-CImpar} y sirve como caracterizaci\'on para esta familia de
gr\'aficas. Esto tambi\'en se puede observar en eljemplo de \cref{fig:ex-bip}.


\begin{teorema}
    \label{teo:bip-CImpar}
    Una gr\'afica $G$ es bipartita si y s\'olo si no contiene ciclos de longitud impar.
\end{teorema}


\begin{proof}
    
\end{proof}
    
Otra  familia de gr\'aficas que se estudiar\'a este trabajo son los abanicos.Un
\indice{abanico} $\mathcal{F}_n$ es la gr\'afica obtenida de la uni\'on de $K_1$
y $P_{n-1}$, donde los primeros $n-1$ v\'ertices de $\mathcal{F}_n$ son los
v\'ertices pertenecientes a $P_{n-1}$ y el $n$-\'esimo v\'ertice de la gr\'afica
es el que le corresponde a $K_1$. 
%TODO: checar si el cambio de notacion no afecta las demostraciones. "A los
%v\'ertices que vienen de $P_{n-1}$ los nombramos $v_i$, con $i \in\{1,...,
%n-1\}$ y al v\'ertice de $K_1$ lo nombramos $w_1$.
Al igual que en varios casos, existe una generalizaci\'on de esta
estructura de gr\'afica al sustituir $K_1$ por $\overline{K_n}$. Un
\textbf{abanico} \indiceSub{abanico}{generalizado}, denotiado
$\mathcal{F}_{m.n}$, se define como $\mathcal{F}_{m,n}=\overline{K_m} \oplus
P_n$. En \cref{fig:ex-abanico} se puede ver un ejemplo de un abanico del lado
izquierdo y un abanico generalizado del lado derecho.

\begin{figure}[ht!]
    \centering
        \begin{tikzpicture}

        \begin{scope}[xshift=0cm,scale=1]
            \foreach \i in {1,...,5} \draw ({(360/6)*\i}:1.8)
                    node(\i)[vertex, label=(360/6)*\i:{\large $v_{\i}$}]{};

            \draw (0,0) node (6) [vertex,label=0:{\small $v_6$}] {};

            \foreach \i/\j in{1/2,2/3,3/4,4/5,1/6,2/6,3/6,4/6,5/6} 
                \draw [edge,grisOscuro] (\i) to (\j);
        \end{scope}
            
        \begin{scope}[xshift=7cm,scale=1]
            \draw (-2.6,1) node (v1) [vertex,label=180:{\small $v_1$}] {};
            \draw (-1.3,1.6) node (v2) [vertex,label=135:{\small $v_2$}] {};
            \draw (0,2) node (v3) [vertex,label=90:{\small $v_3$}] {}; 
            \draw (1.3,1.6) node (v4) [bvertex,label=45:{\small $v_4$}] {}; 
            \draw (2.6,1) node (v5) [vertex,label=0:{\small $v_5$}] {}; 
            \draw (-1.2,-1.5) node (w1) [vertex,label=270:{\small $w_1$}] {}; 
            \draw (1.2,-1.5) node (w2) [bvertex, label=270:{\small $w_2$}] {};

            
            \foreach \i/\j in{w1/v1,w1/v2,w1/v3,w1/v4,w1/v5,w2/v1,w2/v2,w2/v3,
                w2/v4,w2/v5,v1/v2,v2/v3,v3/v4,v4/v5} 
            \draw [edge,grisOscuro] (\i) to (\j);
        \end{scope}
\end{tikzpicture}
\caption{A}
\label{fig:ex-abanico}
\end{figure}


La \'ultima familia de gr\'aficas que veremos en esta secci\'on es la familia de
gr\'aficas hamiltonianas. Para definir esta familia, es preciso definir algunos
conceptos necesarios. El primer concepto es el de trayectoria hamiltoniana. Una
\textbf{trayectoria}
\indiceSub{trayectoria}{hamiltoniana}\index{hamiltoniana!trayectoria} es una
trayectoria generadora, es decir, una trayectoria cuyo conjunto de v\'ertices es
el conjunto de v\'ertices de la gr\'afica. Asimismo, un \textbf{ciclo}
\indiceSub{ciclo}{hamiltoniano}\index{hamiltoniana!ciclo} es un ciclo generador.
Ahora, pasamos a definir una \textbf{gr\'afica}
\indiceSub{gr\'afica}{hamiltoniana}\index{hamiltoniana!gr\'afica}, que es una
gr\'afica que contiene un ciclo hamiltoniano. En \cref{fig:ex-hamilt} se muestra
un ejemplo de una gr\'afica que contiene una trayectoria hamiltoniana del lado
izquierdo y de una gr\'afica hamiltoniana del lado izquierdo, resaltando con
colores la trayectoria y el ciclo hamiltoniano respectivamente. Una definici\'on
que nos ser\'a de utilidad a lo largo de este trabajo es la de subtrayectoria
hamiltoniana. Definimos una \indice{subtrayectoria
hamiltoniana}\index{hamiltoniana!subtrayectoria} como la trayectoria
hamiltoniana resultante de quitarle una arista a un ciclo hamiltoniano. 

\begin{figure}[ht!]
    \centering
        \begin{tikzpicture}
        
            \begin{scope}[xshift=0cm,scale=0.8]

                \foreach \i in {1,...,3}
                 \draw ({(360/3)*\i-90}:0.7)
                    node(\i)[vertex, label=(360/3)*\i-90:{\large $v_{\i}$}]{};
                
                \draw (0,3) node (4) [vertex, label=90:{\large $v_4$}] {};
                \draw (3,-1.5) node (5) [vertex, label=0:{\large $v_5$}] {};
                \draw (-3,-1.5) node (6) [vertex, label=180:{\large $v_6$}] {};
                
                \foreach \i/\j in {4/5,2/3,3/6,1/4}
                    \draw [edge,grisOscuro!50] (\i) to (\j);
                
                \foreach \i/\j in {1/3,3/5,4/6,2/4,5/6}
                    \draw [wedge,verde!110] (\i) to (\j);
                \end{scope}
        
            \begin{scope}[xshift=7.5cm,scale=0.8]

                \foreach \i in {1,...,3}
                 \draw ({(360/3)*\i-90}:0.7)
                    node(\i)[vertex, label=(360/3)*\i-90:{\large $v_{\i}$}]{};
                
                \draw (0,3) node (4) [vertex, label=90:{\large $v_4$}] {};
                \draw (3,-1.5) node (5) [vertex, label=0:{\large $v_5$}] {};
                \draw (-3,-1.5) node (6) [vertex, label=180:{\large $v_6$}] {};
                
                \foreach \i/\j in {1/2,2/3,4/5,6/4,1/5,3/6}
                    \draw [edge,grisOscuro!50] (\i) to (\j);
                
                \foreach \i/\j in {2/4,2/6,5/6,3/5,1/3,1/4}
                    \draw [wedge,azulCielo] (\i) to (\j);
                \end{scope}
                    
    \end{tikzpicture}
    \caption{Una trayectoria hamiltoniana (izquierda) y un ciclo hamiltoniano
    (derecha).}
    \label{fig:ex-hamilt}
\end{figure}

\section{Coloraci\'on}
\label{sec:coloracion}

En Teor\'ia de Gr\'aficas, hablamos de colorear una gr\'afica como manera de
etiquetar a los elementos de una gr\'afica, ya sean a los v\'ertices o a las
aristas. Primero nos enfocaremos en la coloraci\'on por aristas. Formalmente,
una \textbf{$k$-coloraci\'on} \indiceSub{$k$-coloraci\'on}{por
aristas}\index{aristas!coloraci\'on} de una gr\'afica $G$, para un entero $k$,
es una funci\'on $c' \colon E(G)\to S$, con $S$ un conjunto de cardinalidad $k$.
La \indice{clase crom\'atica} de un color $i$ en una coloraci\'on por aristas
$c'$ es el conjunto de todas las aristas de la gr\'afica que estan coloreadas
del color $i$, es decir $c'^{-1}[i]$. Notemos que las clases crom\'aticas pueden
ser vac\'ias. Una \textbf{coloraci\'on} \indiceSub{$k$-coloraci\'on}{propia por
aristas}\index{aristas!$k$-coloraci\'on propia} es una coloraci\'on por aristas
donde se le asignan colores diferentes a todas las aristas adyacentes.

Por otro lado, una una \textbf{$k$-coloraci\'on}
\indiceSub{$k$-coloraci\'on}{por v\'ertices}\index{v\'ertices!coloraci\'on} de
una gr\'afica $G$, para un entero $k$, es una funci\'on $c \colon V(G)\to T$,
donde $T$ es un conjunto de cardinalidad $k$. En ambos casos, ya sea
coloraci\'on por v\'ertices o por aristas, a los elementos del codominio de la
funci\'on se les llama \indice{colores}. Una \textbf{coloraci\'on}
\indiceSub{$-$coloraci\'on}{propia por
v\'ertices}\index{v\'ertices!$k$-coloraci\'on propia} es una coloraci\'on por
v\'ertices donde se le asignan colores diferentes a todas las v\'ertices
adyacentes. Decimos que una gr\'afica $G$ es
\indiceSub{gr\'afica}{$k$-coloreable}, con $k \in \mathbb{Z}$, si $G$ admite una
$k$-coloraci\'on propia por v\'ertices. A continuaci\'on, en
\cref{fig:ex-color-prop}, se muestra una gr\'afica $G$ donde se resalta una
coloraci\'on propia por v\'ertices, del lado izquierdo, y una coloraci\'on
propia por aristas, del lado derecho. 

\begin{figure}[ht!]
    \centering
    \begin{tikzpicture}
    
        \begin{scope}[xshift=-8cm]
            \foreach \i in {2,5} 
                \draw ({(360/5)*\i}:2.5) node(\i)[vertex, fill=coral,
                label=(360/5)*\i:{\large
                $v_{\i}$}]{};

            \foreach \i in {1,3,4}
                \draw ({(360/5)*\i}:2.5) node(\i)[vertex, fill=azulCielo,
                label=(360/5)*\i:{\large $v_{\i}$}]{};
            
            \draw (0,0) node (6) [vertex, fill=amarillo, label=35:{\large $v_6$}] {};
           
            \foreach \i/\j in {1/6,2/6,3/6,4/6,5/6}
                \draw [edge,grisOscuro] (\i) to (\j);
            \end{scope}

        \begin{scope}[xshift=0cm]
            \foreach \i in {1,...,5}
                \draw ({(360/5)*\i}:2.5) node(\i)[vertex,
                label=(360/5)*\i:{\large $v_{\i}$}]{};
    
            \draw (0,0) node (6) [vertex, label=35:{\large $v_6$}] {};
               
            \draw[ edge] [wedge,coral] (1) to (6);
            \draw[ edge] [wedge,azulCielo] (2) to (6);
            \draw[ edge] [wedge,amarillo] (3) to (6);
            \draw[ edge] [wedge,morado] (4) to (6);
            \draw[ edge] [wedge,verde] (5) to (6);
        \end{scope}

    \end{tikzpicture}
    \caption{uns $3$-coloraci\'on propia por v\'ertices (izquierda) y una 
        $5$-coloraci\'on propia por aristas (derecha).}
        \label{fig:ex-color-prop}
\end{figure}

La \indice{clase crom\'atica} de un color $i$ en una coloraci\'on por v\'ertices
$c$ es el conjunto de todas las v\'ertices de la gr\'afica que estan coloreadas
del color $i$, es decir $c^{-1}[i]$. Notemos que las clases crom\'aticas pueden
ser vac\'ias, adem\'as, en una coloraci\'on propia, cada clase crom\'atica es un
conjunto independiente. El m\'inimo entero $k$ para el cu\'al una gr\'afica $G$
es $k$-coloreable es el \indice{n\'umero crom\'atico} de $G$, denotado por
$\chi(G)$. En este caso decimos que $G$ es
\indiceSub{gr\'afica}{$k$-crom\'atica}. Observamos que si $H$ es una
subgr\'afica de $G$, entonces $\chi(H) \leq \chi(G)$, m\'as a\'un, si si $H$ es
una gr\'afica completa de $n$ v\'ertices, entonces $n \leq \chi(G)$. Observemos
que, en \cref{fig:ex-color-prop}, se muestra una $3$-coloraci\'on propia por
v\'ertices, sin embargo el n\'umero crom\'atico de la gr\'afica es $2$.



\section{Gr\'aficas de Fichas}
\label{def:fichas}

Durante los a\~{n}os 90, se pueden encontrar nociones de gr\'aficas de
$2$-fichas, referidas como ``Dobule Vertex Graphs'', en (...) donde Alavi y
otros autores estudian la conectividad, planaridad, regularidad y hamiltonicidad
de algunas gr\'aficas de $2$-fichas. Por oto lado, en 1992, Zhu estudia las
gr\'aficas de $k$-fichas, para una $k \in \mathbb{N^{+}}$ m\'as general,
refiriendose a ellas como ``$k$-tuplex Vertex Graphs''. Posteriormente, en 2002
en (...), se estudian las gr\'aficas de $2$-fichas bajo otro nombre en (...)
donde Rudolph da un ejemplo de dos gr\'aficas coespectrales donde sus gr\'aficas
de $2$-fichas no lo son. M\'as adelante, en el 2007, se utilizan las gr\'aficas
de fichas en el articulo (...) (Audeanet), donde se refieren a ellas como
``potencias sim\'etricas'' de una gr\'afica. En este art\'iculo Audeanet
demuestra que las gr\'aficas de $2$-fichas de las gr\'aficas fuertemente
regulares con los mismos par\'ametros son coespectrales. M\'as adelante, de
manera independiente, Barghi y Ponomarenko (2009) y Alzaga (2010) prueban que
para alg\'un entero positivo dado existen una cantidad infinita de pares de
gr\'aficas no isomorfas con gr\'aficas de $k$-fichas coespectrales. En el 2012,
Fabila (...) reintroduce el concepto de gr\'aficas de fichas, asign\'andole este
nombre al pensarlo como configuraciones de $k$ fichas indistinguibles sobre los
v\'ertices de una gr\'afica $G$, a lo m\'as una ficha por v\'ertice. Una ficha
se puede ``mover'' de un v\'ertice de $G$ a otro siempre y cuando exista una
arista entre ellos y no haya una ficha en el segundo v\'ertice. Cada
configuraci\'on de $k$ fichas ser\'a un v\'ertice en nuestra nueva gr\'afica
donde dos v\'ertices ser\'an adyacentes siempre que se pueda llegar de una
configuraci\'on a otra al mover una ficha atrav\'es de una arista. A esta nueva
gr\'afica la nombramos la gr\'afica de $k$-fichas de $G$ y la denotamos
$F_k(G)$. M\'as adelante, tomando como base el articulo (Fabila) se han seguido
estudiando la conexidad, planaridad, hamiltonicidad, entre otras coasas, de las
gr\'aficas de fichas (...).

Formalmente hablando, dada una gr\'afica $G$ y $k$ un entero positivo, la
gr\'afica cuyo conjunto de v\'ertices es $\binom{V(G)}{k}$ y donde dos
v\'ertices $A$ y $B$ en son adyacentes si y s\'olo si $|A \triangle B| =
\{a,b\}$, con $a \in A$, $b \in B$ y $ab \in E(G)$, es la \textbf{gr\'afica de}
\indiceSub{gr\'afica}{$k$-fichas} \textbf{de $G$}\index{fichas!gr\'afica}.
Observamos que la gr\'afica de $1$-fichas es isomorfa a la gr\'afica original,
pues cada v\'ertice de la gr\'afica de fichas es el v\'ertice de la gr\'afica
original donde se encuentra la ficha. Por lo mismo tambi\'en tiene las mismas
adyacencias y no adyacencias que la gr\'afica original. Por lo tanto, en general
para los ejemplos usaremos gr\'aficas de $k$-fichas, con $k > 1$. En
\cref{fig:ex-tok-graph} se muestra un ejemplo de una gr\'afica $G$ y su
gr\'afica de $2$-fichas. Para ayudar al entendimiento de las gr\'aficas de
fichas, en la gr\'afica del lado derecho de \cref{fig:ex-tok-graph} en cada
v\'ertice de $F_2(G)$ se muestra una copia a escala de $G$, resaltando en
${\color{rosa}\bf rosa}$ los v\'ertices de $G$ donde se encuentran las fichas.

\begin{figure}[ht!]
    \centering
       \begin{tikzpicture}
    
        \begin{scope}[xshift=-9cm,scale=0.9]
            \foreach \i in {1,...,5}
                \draw ({(360/5)*\i}:2) node(\i)[vertex, label=(360/5)*\i:{${\i}$}]{};
            \foreach \i/\j in {1/2,1/3,1/4,1/5,2/3,3/4,4/5}
                \draw [edge,grisOscuro] (\i) to (\j);
        \end{scope}
        
        \begin{scope}[xshift=0cm,yshift=0cm,scale=2]
            \draw (0,1.8) node (1) [Bvertex, label=90:{\large $34$}] {};
            \draw (0,-1.4) node (2) [Bvertex, label=270:{\large $25$}] {};
            \draw (1,1) node (3) [Bvertex, label=45:{\large $35$}] {};
            \draw (-1,1) node (4) [Bvertex, label=135:{\large $24$}] {};
            \draw (2.1,0) node (5) [Bvertex, label=0:{\large $45$}] {}; 
            \draw (-2.1,0) node (6) [Bvertex, label=180:{\large $23$}] {};
            \draw (0.85,-0.6) node (7) [Bvertex, label=270:{\large $14$}] {};
            \draw (-0.85,-0.6) node (8) [Bvertex, label=270:{\large $13$}] {};
            \draw (1.6,-1.8) node (9) [Bvertex, label=290:{\large $15$}] {};
            \draw (-1.6,-1.8) node (10) [Bvertex, label=240:{\large $12$}] {};
           
            \foreach \i/\j in{1/4,1/3,1/7,1/8,2/3,2/4,2/9,2/10,3/5,3/8,3/9,4/6,
            4/7,4/10,5/7,5/9,6/8,6/10,7/8,7/9,8/10} 
            \draw[edge,grisOscuro] (\i) to (\j);
       \end{scope} 
       
       \begin{scope}[xshift=0cm,yshift=3.6cm,scale=0.3]
        \foreach \i in {1,2,5}
            \draw ({(360/5)*\i}:2) node(\i)[svertex]{};
        \foreach \i in {3,4}
            \draw ({(360/5)*\i}:2) node(\i)[svertex,fill=rosa]{};
        \foreach \i/\j in {1/2,1/3,1/4,1/5,2/3,3/4,4/5}
            \draw [edge,grisOscuro] (\i) to (\j);
        \end{scope}

        \begin{scope}[xshift=-2cm,yshift=2cm,scale=0.3]
            \foreach \i in {1,3,5}
                \draw ({(360/5)*\i}:2) node(\i)[svertex]{};
            \foreach \i in {2,4}
                \draw ({(360/5)*\i}:2) node(\i)[svertex,fill=rosa]{};
            \foreach \i/\j in {1/2,1/3,1/4,1/5,2/3,3/4,4/5}
                \draw [edge,grisOscuro] (\i) to (\j);
        \end{scope}

        \begin{scope}[xshift=2cm,yshift=2cm,scale=0.3]
            \foreach \i in {1,2,4}
                \draw ({(360/5)*\i}:2) node(\i)[svertex]{};
            \foreach \i in {3,5}
               \draw ({(360/5)*\i}:2) node(\i)[svertex,fill=rosa]{};
            \foreach \i/\j in {1/2,1/3,1/4,1/5,2/3,3/4,4/5}
                \draw [edge,grisOscuro] (\i) to (\j);
        \end{scope}

        \begin{scope}[xshift=-4.2cm,yshift=0cm,scale=0.3]
            \foreach \i in {1,4,5}
                \draw ({(360/5)*\i}:2) node(\i)[svertex]{};
            \foreach \i in {2,3}
                \draw ({(360/5)*\i}:2) node(\i)[svertex,fill=rosa]{};
            \foreach \i/\j in {1/2,1/3,1/4,1/5,2/3,3/4,4/5}
                \draw [edge,grisOscuro] (\i) to (\j);
        \end{scope}
        
        \begin{scope}[xshift=4.2cm,yshift=0cm,scale=0.3]
            \foreach \i in {1,2,3}
                \draw ({(360/5)*\i}:2) node(\i)[svertex]{};
            \foreach \i in {4,5}
                \draw ({(360/5)*\i}:2) node(\i)[svertex,fill=rosa]{};
            \foreach \i/\j in {1/2,1/3,1/4,1/5,2/3,3/4,4/5}
                \draw [edge,grisOscuro] (\i) to (\j);
        \end{scope}

        \begin{scope}[xshift=-1.7cm,yshift=-1.2cm,scale=0.3]
            \foreach \i in {2,4,5}
                \draw ({(360/5)*\i}:2) node(\i)[svertex]{};
            \foreach \i in {1,3}
                \draw ({(360/5)*\i}:2) node(\i)[svertex,fill=rosa]{};
            \foreach \i/\j in {1/2,1/3,1/4,1/5,2/3,3/4,4/5}
                \draw [edge,grisOscuro] (\i) to (\j);
        \end{scope}

        \begin{scope}[xshift=1.7cm,yshift=-1.2cm,scale=0.3]
            \foreach \i in {2,3,5}
                \draw ({(360/5)*\i}:2) node(\i)[svertex]{};
            \foreach \i in {1,4}
                \draw ({(360/5)*\i}:2) node(\i)[svertex,fill=rosa]{};
            \foreach \i/\j in {1/2,1/3,1/4,1/5,2/3,3/4,4/5}
                \draw [edge,grisOscuro] (\i) to (\j);
        \end{scope}

        \begin{scope}[xshift=0cm,yshift=-2.8cm,scale=0.3]
            \foreach \i in {1,3,4}
                \draw ({(360/5)*\i}:2) node(\i)[svertex]{};
            \foreach \i in {2,5}
                \draw ({(360/5)*\i}:2) node(\i)[svertex,fill=rosa]{};
            \foreach \i/\j in {1/2,1/3,1/4,1/5,2/3,3/4,4/5}
                \draw [edge,grisOscuro] (\i) to (\j);
        \end{scope}

        \begin{scope}[xshift=-3.2cm,yshift=-3.6cm,scale=0.3]
            \foreach \i in {3,4,5}
                \draw ({(360/5)*\i}:2) node(\i)[svertex]{};
            \foreach \i in {1,2}
                \draw ({(360/5)*\i}:2) node(\i)[svertex,fill=rosa]{};
            \foreach \i/\j in {1/2,1/3,1/4,1/5,2/3,3/4,4/5}
                \draw [edge,grisOscuro] (\i) to (\j);
        \end{scope}

        \begin{scope}[xshift=3.2cm,yshift=-3.6cm,scale=0.3]
            \foreach \i in {2,3,4}
                \draw ({(360/5)*\i}:2) node(\i)[svertex]{};
            \foreach \i in {1,5}
                \draw ({(360/5)*\i}:2) node(\i)[svertex,fill=rosa]{};
            \foreach \i/\j in {1/2,1/3,1/4,1/5,2/3,3/4,4/5}
                \draw [edge,grisOscuro] (\i) to (\j);
        \end{scope}
    
    \end{tikzpicture}
    \caption{El diagrama de una gr\'afica $G$ (izquierda) y su gr\'afica de
    $2$-fichas (derecha) donde se muestra como se ven las fichas en $G$ para
    cada v\'ertice.}
    \label{fig:ex-tok-graph}
\end{figure}



...Teniendo eso en mente, definimos el siguiente concepto. Sean $P$ una
$ab$-trayectoria en la gr\'afica $G$ y $A$ un $k$-conjunto en $G$, en otras
palabras $A \in V(F_k(G))$. Si al conjunto $A$ le pedimos que $a\in A$ y $b
\notin A$, entonces a la pareja $(A,P)$ le podemos asignar una trayectoria en la
gr\'afica de $k$-fichas, tal que el v\'ertice final est\'a dado por $A'=(A
\setminus \{a\}) \cup \{b\}$. Para definir esta nueva trayectoria tomamos $A\cap
P =\{v_1, \dots, v_q\}$, con $v_1 = a$, y ``movemos las fichas'' de la siguiente
manera. Primero movemos la ficha de $v_q$, v\'ertice de $G$, hacia el v\'ertice
$b$, v\'ertice de $G$, por $P$. Luego, para los v\'ertices $v_i$ en $G$, con $i
\in \{q-1, q-2, \dots 1\}$, movemos la ficha de $v_i$ a $v_{i+1}$. As\'i
susesivamente vamos moviendo fichas a trav\'es de $P$ por v\'ertices que estan
libres de fichas. Esta trayectoria en $F_k(G)$ la denotamos
\indiceSub{fichas}{$A \xrightarrow[P]{} A'$}. A continuaci\'on daremos un
ejemplo de esta trayectoria en una gr\'afica de $3$-fichas, apoyandonos de
\cref{fig:ex-tok-aux} y \cref{fig:ex-tok-path}. Para este ejemplo usaremos la
misma gr\'afica $G$ que en \cref{fig:ex-tok-graph}. Tanto en
\cref{fig:ex-tok-aux} como \cref{fig:ex-tok-path} la gr\'afica $G$ se muestra
del lado izquierdo y en \cref{fig:ex-tok-path} se muestra $F_3(G)$ del lado
derecho. Primero definamos $A=\{2,3,4\}$ en $F_3(G)$ y nuestra trayectoria en
$G$ como ${\color{vino}\bf P= (3,1,4,5)}$, resaltada en \cref{fig:ex-tok-aux}.
Por lo tanto tenemos que $A'=\{2,4,5\}$. Del lado derecho de la misma figura se
muestra el movimiento de las fichas que empiezan en $A \cap P =\{3,4\}$,
considerando de arriba a abajo como se van moviendo las fichas.

\begin{figure}[ht!]
    \centering
       \begin{tikzpicture}
    
        \begin{scope}[xshift=-8.5cm,scale=0.8]
            \foreach \i in {1,...,5}
                \draw ({(360/5)*\i}:2) node(\i)[vertex, label=(360/5)*\i:{${\i}$}]{};
            
            \foreach \i/\j in {1/2,1/5,2/3,3/4}
                \draw [edge,grisOscuro!70] (\i) to (\j);
            \foreach \i/\j in {1/3,1/4,4/5}
                \draw [wedge,vino] (\i) to (\j);
        \end{scope}

        \begin{scope}[yshift=54]
            \draw (-1,0) node (1) [vertex, label=90:{$1$}] {};
            \draw (-3,0) node (3) [vertex, fill=naranja, label=90:{$3$}] {};
            \draw (1,0) node (4) [vertex, fill=azulMetal, label=90:{$4$}] {};
            \draw (3,0) node (5) [vertex, label=90:{$5$}] {};

            \foreach \i/\j in {1/3,1/4,4/5}
                \draw [edge,vino] (\i) to (\j);
        \end{scope}

        \begin{scope}[yshift=18]
            \draw (-1,0) node (1) [vertex, label=90:{$1$}] {};
            \draw (-3,0) node (3) [vertex, fill=naranja, label=90:{$3$}] {};
            \draw (1,0) node (4) [vertex, label=90:{$4$}] {};
            \draw (3,0) node (5) [vertex, fill=azulMetal, label=90:{$5$}] {};

            \foreach \i/\j in {1/3,1/4,4/5}
                \draw [edge,vino] (\i) to (\j);
        \end{scope}

        \begin{scope}[yshift=-18]
            \draw (-1,0) node (1) [vertex, fill=naranja, label=90:{$1$}] {};
            \draw (-3,0) node (3) [vertex, label=90:{$3$}] {};
            \draw (1,0) node (4) [vertex, label=90:{$4$}] {};
            \draw (3,0) node (5) [vertex, fill= azulMetal, label=90:{$5$}] {};

            \foreach \i/\j in {1/3,1/4,4/5}
                \draw [edge,vino] (\i) to (\j);
        \end{scope}

        \begin{scope}[yshift=-54]
            \draw (-1,0) node (1) [vertex, label=90:{$1$}] {};
            \draw (-3,0) node (3) [vertex, label=90:{$3$}] {};
            \draw (1,0) node (4) [vertex, fill=naranja, label=90:{$4$}] {};
            \draw (3,0) node (5) [vertex, fill=azulMetal, label=90:{$5$}] {};

            \foreach \i/\j in {1/3,1/4,4/5}
                \draw [edge,vino] (\i) to (\j);
        \end{scope}
        
    \end{tikzpicture}
    \caption{El diagrama de una gr\'afica simple $G$ y $F_3(G)$}
    \label{fig:ex-tok-aux}
\end{figure}
    
M\'as adelante, en \cref{fig:ex-tok-path}, se muestra la trayectoria
${\color{vino}\bf \{2,3,4\}\xrightarrow[P]{}\{2,4,5\}}$ en $F_3(G)$. En cada
v\'ertice de esta trayecotira se resaltan los n\'umeros del color de la ficha
que se us\'o en \cref{fig:ex-tok-aux}, \'esto para facilitar el entendimiento de
la relaci\'on entre la trayectoria de $G$ y la de $F_3(G)$. Observamos que $2$
no est\'a en $A \cap P =\{3,4\}$ por lo que no se ve en \cref{fig:ex-tok-aux}.
Esto significa que la ficha en el v\'ertice $2$ se queda fija a trav\'es de la
trayectoria en la gr\'afica de fichas. Esto se puede ver en
\cref{fig:ex-tok-path} al fijarnos que el $2$ en todos los v\'ertices de la
trayectoria, resaltado en gris.

%A lo largo de este trabajo nos ser\'a de utilidad $A'$ por lo que .....


\begin{figure}[ht!]
    \centering
       \begin{tikzpicture}
    
        \begin{scope}[xshift=-8.5cm,scale=0.8]
            \foreach \i in {1,...,5}
                \draw ({(360/5)*\i}:2) node(\i)[vertex, label=(360/5)*\i:{\normalsize ${\i}$}]{};
            
            \foreach \i/\j in {1/2,1/5,2/3,3/4}
                \draw [edge,grisOscuro!75] (\i) to (\j);
            \foreach \i/\j in {1/3,1/4,4/5}
                \draw [wedge,vino] (\i) to (\j);
            \end{scope}
        
        %{\small $12{\bf 6}$}
        %${\color{azulCielo}\bf azul}$
        \begin{scope}[xshift=-2cm]
            \draw (0,1.5) node (1) [vertex, label=90:{\footnotesize $134$}] {};
            \draw (-2,2) node (2) [vertex, label=90:{${\bf {\color{grisOscuro} 2}{\color{naranja} 3}{\color{azulMetal} 4}}$}] {};
            \draw (2,2) node (3) [vertex, label=90:{\footnotesize $345$}] {};
            \draw (-3,1) node (4) [vertex, label=180:{\footnotesize $123$}] {};
            \draw (3,1) node (5) [vertex, label=0:{\footnotesize $145$}] {}; 
            \draw (-1.1,0.7) node (6) [vertex, label=87:{${\bf {\color{grisOscuro} 2}{\color{naranja} 3}{\color{azulMetal} 5}}$}] {};
            \draw (1.1,0.7) node (7) [vertex, label=93:{${\bf {\color{grisOscuro} 2}{\color{naranja} 4}{\color{azulMetal} 5}}$}] {};
            \draw (-1.1,-1) node (8) [vertex, label=240:{\footnotesize $124$}] {};
            \draw (1.1,-1) node (9) [vertex, label=330:{\footnotesize $135$}] {};
            \draw (0,-2.3) node (10) [vertex, label=270:{${\bf {\color{naranja} 1}{\color{grisOscuro} 2}{\color{azulMetal} 5}}$}] {};
           
            \foreach \i/\j in{1/2,1/3,1/8,1/9,2/4,2/8,3/5,3/7,3/9,4/6,4/8,5/7,
            5/9,6/7,6/9,7/8,8/10,9/10} 
                \draw [edge,grisOscuro!75] (\i)to (\j); 
            \foreach \i/\j in { 2/6,6/10,7/10} 
                \draw [wedge,vino] (\i) to (\j);
       \end{scope}       
    
    \end{tikzpicture}
    \caption{El diagrama de una gr\'afica simple $G$ y $F_3(G)$}
    \label{fig:ex-tok-path}
\end{figure}


Algo que tambi\'en resulta interesante en el ejemplo anterior es la ficha que se
queda fija en el v\'ertice $2$ de la gr\'afica $G$. Al estar fija, todos los
v\'ertices de la nueva trayectoria en $F_3(G)$ contienen a $2$. Generalizando un
poco m\'as, nos preguntamos que gr\'afica de $3$-fichas se obtendr\'ia si una
ficha se queda fija, digamos en $2$. Es f\'acil notar que esta gr\'afica es una
sugr\'afica de la gr\'afica de $3$-fichas de $G$, pues el conjunto de v\'ertices
de esta gr\'afica es el conjunto de v\'ertices de $F_3(G)$ que contengan a $2$.
(A continuaci\'on, este ejemplo se representa en \cref{fig:ex-tok-subgraph}.)
La idea de fijar una ficha se puede extender a $r\leq k$ fichas, aunque el caso
interesante ser\'a para $r<k$ pues fijamos todas las fichas obtenemos una
gr\'afica trivial. Dado un conjunto $X \subseteq V(G)$ con $|X|=r<k$, definimos
a $F_k(G,X)$ como la \textbf{subgr\'afica de $F_k(G)$
inducida}\index{subgr\'afica!inducida de fichas}\index{fichas!subgr\'afica
inducida} por los v\'ertices de $F_k(G)$ que contienen al subconjunto $X$. 

\begin{figure}[ht!]
    \centering
       \begin{tikzpicture}
    
        \begin{scope}[xshift=-8.5cm,scale=0.7]
            \foreach \i in {1,3,4,5}
                \draw ({(360/5)*\i}:2) node(\i)[vertex, label=(360/5)*\i:{\small ${\i}$}]{};
            \draw ({(360/5)*2}:2) node(2)[vertex, fill=rosa, label=(360/5)*2:{\small ${2}$}]{};
            
            \foreach \i/\j in {1/2,1/3,1/4,1/5,2/3,3/4,4/5}
                \draw [edge,grisOscuro] (\i) to (\j);
            \end{scope}
        
        %{\small $12{\bf 6}$}
        %${\color{azulCielo}\bf azul}$
        \begin{scope}[xshift=-2cm, scale=0.8]
            \draw (0,1.5) node (1) [vertex, label=90:{{ {\color{grisOscuro}\footnotesize $134$}}}] {};
            \draw (-2,2) node (2) [vertex, label=90:\small $234$] {};
            \draw (2,2) node (3) [vertex, label=90:{{ {\color{grisOscuro}\footnotesize $345$}}}] {};
            \draw (-3,1) node (4) [vertex, label=180:\small $123$] {};
            \draw (3,1) node (5) [vertex, label=0:{{ {\color{grisOscuro}\footnotesize $145$}}}] {}; 
            \draw (-1.1,0.7) node (6) [vertex, label=87:\small $235$] {};
            \draw (1.1,0.7) node (7) [vertex, label=93:\small $245$] {};
            \draw (-1.1,-1) node (8) [vertex, label=240:\small $124$] {};
            \draw (1.1,-1) node (9) [vertex, label=330:{{ {\color{grisOscuro}\footnotesize $135$}}}] {};
            \draw (0,-2.3) node (10) [vertex, label=270:\small $125$] {};
           
            \foreach \i/\j in{1/2,1/3,1/8,1/9,3/5,3/7,3/9,5/7,
            5/9,9/10} 
                \draw [edge,grisOscuro!75] (\i)to (\j); 
            \foreach \i/\j in {2/4,4/6,4/8,2/8,7/8,8/10,2/6,6/7,6/10,7/10} 
                \draw [wedge,rosa] (\i) to (\j);
       \end{scope}       

    \end{tikzpicture}
    \caption{El diagrama de una gr\'afica simple $G$ y $F_3(G)$}
    \label{fig:ex-tok-subgraph}
\end{figure}

Al momento de generar $F_3(G,\{2\})$ es f\'acil notar que las fichas se mueven
por $V(G) \setminus \{2\}$, excepto la ficha que est\'a fija. De igual manera,
las aristas por las que se mueven las fichas son las aristas de $G$ que no
tienen extremos en $2$. Entonces, podr\'iamos interpretar $F_3(G,\{2\})$ como la
gr\'afica de $(3-1)$-fichas de la gr\'afica $G-2$, es decir $F_2(G-2)$. Esta
gr\'afica est\'a exhibida en \cref{fig:ex-tok-subgraph-aux}. De manera general,
esto quiere decir que existe una relaci\'on entre la subgr\'afica de $F_k(G)$
inducida por $X$, donde $|r|=X \subset V(G)$, y la gr\'afica de $(k-r)$-fichas
de la gr\'afica $G-X$, es decir $F_{k-r}(G-X)$. Afirmamos que, para $k,r \in
\mathbb{N}$ y $k>r = |X|$, con $X \subseteq V(G)$, la gr\'afica $F_k(G,X)$ es
isomorfa a la gr\'afica $F_{k-r}(G-X)$, con el isomorfismo que a cada $A \in
F_k(G,X)$ le asocia $A \setminus X$ en $F_{k-r}(G-X)$.   Observamos que las
aristas en $G$ con alg\'un extremos en $X$ no se usan en $F_k(G,X)$ por lo que
el la funci\'on si preserva adyacencias y no adyacencias..... 

\begin{figure}[ht!]
    \centering
       \begin{tikzpicture}
    
       \begin{scope}[xshift=-8.5cm,scale=0.7]
        \foreach \i in {1,3,4,5}
            \draw ({(360/5)*\i}:2) node(\i)[vertex, label=(360/5)*\i:{\small ${\i}$}]{};
        \draw ({(360/5)*2}:2) node(2)[cvertex, label=(360/5)*2:{{ {\color{grisOscuro!75}\small $2$}}}]{};
        
        \foreach \i/\j in {1/3,1/4,1/5,3/4,4/5}
            \draw [edge,grisOscuro] (\i) to (\j);

        \foreach \i/\j in {1/2,2/3}
            \draw [edge,grisOscuro!50] (\i) to (\j);
        \end{scope}
    
    %{\small $12{\bf 6}$}
    %${\color{azulCielo}\bf azul}$
%     \begin{scope}[xshift=-1.3cm]
%         \draw (-2,2) node (2) [vertex, label=90:$234$] {};
%         \draw (-3,1) node (4) [vertex, label=180:{$123$}] {};
%         \draw (-1.1,0.7) node (6) [vertex, label=87:$235$] {};
%         \draw (1.1,0.7) node (7) [vertex, label=93:$245$] {};
%         \draw (-1.1,-1) node (8) [vertex, label=240:{$124$}] {};
%         \draw (0,-2.3) node (10) [vertex, label=270:$125$] {};
       
%        \foreach \i/\j in {2/4,4/6,4/8,2/8,7/8,8/10,2/6,6/7,6/10,7/10} 
%             \draw [edge,grisOscuro] (\i) to (\j);
%    \end{scope}       

    \begin{scope}[xshift=-2cm]
        \draw (0.5,0.87) node (1) [vertex,label=87:$235$] {};
        \draw ({(360/6)*2}:2) node(2)[vertex, label=(360/5)*2:{\small ${234}$}]{};
        \draw ({(360/6)*3}:2) node(3)[vertex, label=(360/5)*3:{\small ${123}$}]{};
        \draw (-0.5,-0.87) node (4) [vertex,label=240:\small $124$] {};
        \draw ({(360/6)*5}:2) node(5)[vertex, label=(360/5)*5:{\small ${125}$}]{};
        \draw ({(360/6)*6}:2) node(6)[vertex, label=(360/5)*6:{\small ${245}$}]{};
    %\draw ({(360/5)*2}:2) node(2)[vertex, fill=rosa, label=(360/5)*2:{\small ${2}$}]{};
    
    \foreach \i/\j in {1/2,1/3,1/5,1/6,2/3,2/4,3/4,4/5,4/6,5/6}
        \draw [edge,grisOscuro] (\i) to (\j);
    \end{scope}

    \end{tikzpicture}
    \caption{El diagrama de una gr\'afica simple $G$ y $F_3(G)$}
    \label{fig:ex-tok-subgraph-aux}
\end{figure}
