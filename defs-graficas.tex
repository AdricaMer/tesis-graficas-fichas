\chapter{Definiciones de Teor\'ia de Gr\'aficas}%
\label{cap:defs grafs}

\section{Definiciones b\'asicas}%
\label{sec:def-basicas}

El objeto a estudiar en el \'area de Teor\'ia de Gr\'aficas es, naturalmente,
una gr\'afica. Una \indice{gr\'afica} $G$ es una pareja ordenada de conjuntos
finitos $(V(G), E(G))$, donde $V(G)$ es no vac\'io y $E(G) \subseteq
\binom{V(G)}{2}$. A la cardinalidad de $V(G)$ se le llama
\indice{orden} de $G$ y sus elementos son llamados
\indice{v\'ertices}. A los elementos de $E(G)$ se les conoce como
\indice{aristas}. Para una arista $e$ y v\'ertices $u$ y $v$ de $G$, decimos que
$e$ es \indiceSub{aristas}{incidente} en $u$ y en $v$ si $e= \{u, v\}$. A su
vez, los v\'ertices $u$ y $v$ son \indiceSub{v\'ertices}{extremos} de la arista
$e$. De igual manera, podemos decir que $u$ y $v$ son \textbf{v\'ertices}
\indiceSub{v\'ertices}{adyacentes}\index{adyacentes!v\'ertices} o
\indiceSub{v\'ertices}{vecinos} y lo denotamos $u \sim v$. Por otro lado,
decimos que dos aristas son \textbf{aristas}
\indiceSub{aristas}{adyacentes}\index{adyacentes!aristas} cuando comparten un
extremo. 

Al conjunto de vecinos de un v\'ertice $v$ se le llama la \indice{vecindad} de
$v$, y se denota por $N_G(v)$, mientras que definimos al n\'umero de aristas
incidentes en $v$ como el \indice{grado} de $v$, se denota $d(v)$. Un
\textbf{v\'ertice} \indiceSub{v\'ertices}{aislado} es  un v\'ertice que no tiene
vecinos, es decir, tiene grado $0$.

\begin{figure}[ht!]
    \centering
       \begin{tikzpicture}
        
            \begin{scope}[xshift=6cm,scale=0.8]
                \draw (0.5,-1.5) node (1) [vertex, label=270:{\large $v_1$}] {};
                \draw (-1.9,-1.6) node (2) [vertex, label=270:{\large $v_2$}] {};
                \draw (1.2,1.5) node (3) [vertex, label=90:{\large $v_3$}] {};
                \draw (-2,0.5) node (4) [vertex, label=90:{\large $v_4$}] {};
                \draw (2.5,-0.6) node (5) [vertex, label=270:{\large $v_5$}] {};
                \draw (0,0) node (6) [vertex, label=90:{\large $v_6$}] {};
                \draw (3,1) node (7) [vertex, label=90:{\large $v_7$}] {};
                
                \foreach \i/\j in {1/3,1/4,1/6,2/6,3/4,3/5,5/6}
                \draw [edge,grisOscuro] (\i) to (\j);
            \end{scope}
    
        \begin{scope}[xshift=0cm,scale=0.6]
            \draw ({(360/4)*3}:2) node(1)[vertex, label=270:{\large $v_1$}]{};
            \draw (0,0) node (2) [vertex, label=90:{\large $v_2$}] {};
            \draw ({(360/4)*2}:2) node(3)[vertex, label=180:{\large $v_3$}]{};
            \draw (-2.5,-2) node (4) [vertex, label=270:{\large $v_4$}] {};
            \draw ({(360/4)*1}:2) node(5)[vertex, label=90:{\large $v_5$}]{};
            \draw ({(360/4)*0}:2) node(6)[vertex, label=0:{\large $v_6$}]{};
            \draw (-2.5,2) node (7) [vertex, label=90:{\large $v_7$}] {};

            \foreach \i/\j in {1/3,1/4,1/6,2/6,3/4,3/5,5/6}
                \draw [edge,grisOscuro] (\i) to (\j);  
            \end{scope}
                
    \end{tikzpicture}
    \caption{Dos diagramas diferentes de una gr\'afica $G$.}
    \label{fig:diagGraf}
\end{figure}

Las gr\'aficas pueden ser representadas por \indice{diagramas}, donde los
v\'ertices en ellas son c\'irculos y las aristas son l\'ineas cuyos extremos son
sus v\'ertices incidentes. Cabe recalcar que una gr\'afica puede tener
m\'ultiples diagramas, como se muestra en \cref{fig:diagGraf}. Es f\'acil
observar que, en ambos diagramas de $G$, los v\'ertices tienen las mismas
propiedades, por ejemplo, que $N_G(v_3)=\{v_1,v_4,v_5\}$ o que el grado de $v_2$
es $1$. Esto se debe a que la gr\'afica es un objeto matem\'atico, independiente
del diagrama.


Al momento de estudiar objetos matem\'aticos, siempre es importante analizar
c\'omo se relacionan dichos objetos entre s\'i. Ya teniendo la noci\'on de una
gr\'afica, nos enfocamos en c\'omo se ven las relaciones entre gr\'aficas.
Veamos qu\'e necesitan dos gr\'aficas para ser ``iguales''. Observamos que, de
igual manera que una gr\'afica puede tener varios diagramas, a un diagrama se le
pueden asociar distintas gr\'aficas, renombrando v\'ertices y aristas, para
obtener distintos conjuntos de v\'ertices y aristas. Un ejemplo de esto se
aprecia en \cref{fig:isoGraf}. Al renombrar los v\'ertices de la gr\'afica $G$,
podemos dibujar a $G$ como la gr\'afica $H$ y viceversa. En \cref{fig:isoGraf}
se muestra una forma de renombrar los v\'ertices al usar los colores que se
muestran en la figura como gu\'ia. De manera intuitiva, decimos que dos
gr\'aficas que comparten un mismo diagrama son ``iguales'' pues, al compartir un
diagrama, comparten muchas propiedades que nos interesan, como el n\'umero de
v\'ertices, las adyacencias o el grado de los v\'ertices. Estas g\'aficas se
dice que son isomorfas. Formalmente hablando, dos gr\'aficas $G$ y $H$ son
\indiceSub{gr\'afica}{isomorfas} si exite una biyecci\'on $\theta: V(G)
\rightarrow V(H)$ tal que para cualesquiera dos v\'ertices $u, v \in V(G)$, se
cumple que $uv \in E(G)$ si y s\'olo si $\theta(u)\theta(v) \in E(H)$. Es decir,
la biyecci\'on $\theta$ preserva adyacencias y no adyacencias. Cuando $G$ es
ismorfa a $H$ lo denotamos $G \cong H$ y la biyecci\'on entre las gr\'aficas es
llamada \indice{isomorfismo}.

\begin{figure}[ht!]
    \centering
       \begin{tikzpicture}
    
            \begin{scope}[xshift=0cm,scale=0.8]
             \draw ({(360/5)*1}:2) node(1)[vertex,fill=morado, label=(360/5)*1:{\large
                    $v_{1}$}]{};
            \draw ({(360/5)*2}:2) node(2)[vertex,fill=amarillo, label=(360/5)*2:{\large
                    $v_{2}$}]{};
           \draw ({(360/5)*3}:2) node(3)[vertex,fill=azulCielo, label=(360/5)*3:{\large
                    $v_{3}$}]{};
            \draw ({(360/5)*4}:2) node(4)[vertex,fill=coral, label=(360/5)*4:{\large
                    $v_{4}$}]{};
            \draw ({(360/5)*5}:2) node(5)[vertex,fill=verde, label=(360/5)*5:{\large
                    $v_{5}$}]{};

            \foreach \i/\j in {5/1,1/2,2/3,3/4,4/5}
                \draw [edge,grisOscuro] (\i) to (\j);
            \end{scope}
    
        \begin{scope}[xshift=6cm,scale=0.8]
            \draw ({(360/5)*1}:2) node(1)[vertex,fill=morado, label=(360/5)*1:{\large
            $v_{1}$}]{};
    \draw ({(360/5)*2}:2) node(2)[vertex,fill=coral, label=(360/5)*2:{\large
            $v_{2}$}]{};
   \draw ({(360/5)*3}:2) node(3)[vertex,fill=amarillo, label=(360/5)*3:{\large
            $v_{3}$}]{};
    \draw ({(360/5)*4}:2) node(4)[vertex,fill=verde, label=(360/5)*4:{\large
            $v_{4}$}]{};
    \draw ({(360/5)*5}:2) node(5)[vertex,fill=azulCielo, label=(360/5)*5:{\large
            $v_{5}$}]{};

            \foreach \i/\j in {5/2,2/4,4/1,1/3,3/5}
                \draw [edge,grisOscuro] (\i) to (\j);
            \end{scope}
                
    \end{tikzpicture}
    \caption{Dos gr\'aficas isomorfas, $G$ (izquierda) y $H$ (derecha).}
    \label{fig:isoGraf}
\end{figure}

Al momento de ver las relaciones entre dos objetos matem\'aticos, tambi\'en es
importante analizar el caso en el que uno est\'e ``contenido'' en el otro. As\'i
como en el caso de conjuntos nos referimos a subconjuntos, en el caso de
gr\'aficas, nos referimos a subgr\'aficas. Si $G$ y $H$ son gr\'aficas, tales
que $V(H) \subseteq V(G)$ y $E(H) \subseteq E(G)$, entonces decimos que $H$ es
una \indice{subgr\'afica} de $G$. En tal caso, tambi\'en decimos que $G$ es una
\indice{supergr\'afica} de $H$. Esta relaci\'on se denota con la misma
notaci\'on que para subconjuntos, es decir, $H \subseteq G$. Cuando una
subgr\'afica tiene el mismo conjunto de v\'ertices que la gr\'afica, es decir,
$H \subseteq G$ y $V(H)= V(G)$, entonces decimos que $H$ es una
\textbf{subgr\'afica} \indiceSub{subgr\'afica}{generadora}. Otro concepto que se
deriva de las subgr\'aficas parte de fijarnos en la subgr\'afica que tiene la
mayor cantidad de aristas posibles. Formalmente hablando, nos referimos a $H$,
la subgr\'afica de $G$, cuyo conjunto de v\'ertices es $S \subseteq V(G)$ y cuyo
conjunto de aristas son aquellas aristas de $G$ que tienen ambos extremos en
$S$, es decir, $E(H) = \{uv \in E(G) \colon\ u,v \in S\}$. A esta subgr\'afica
la llamamos la \textbf{subgr\'afica} \indiceSub{subgr\'afica}{inducida} de $G$
por $S$ y la denotamos $G[S]$. Notamos que $G[S]$ es la subgr\'afica de $G$, con
conjunto de v\'ertices $S$, que m\'as se parece a un pedazo de $G$. Un tercer
tipo importante de subgr\'aficas es el de las subgr\'aficas inducidas de $G$
obtenidas al quitarle uno o m\'as v\'ertices y las aristas que inciden en dichos
v\'ertices. En otras palabras, sea $S \subset V(G)$, nos referimos a la
subgr\'afica inducida $G[V(G)-S]$. Esta gr\'afica se denota $G-S$. Un caso a
se\~{n}alar es cuando $S$ es un conjunto unitario, en otras palabras, el caso en
el que le quitamos un solo v\'ertice $v$ a $G$. En este caso escribimos $G-v$,
en vez de $G-\{v\}$. A continuaci\'on se muestra un ejemplo de subgr\'aficas en
\cref{fig:subgraf}, donde ${\color{morado}\boldsymbol {H}}$ es una subgr\'afica
inducida con $V(H) = \{v_1,v_2,v_3,v_4\}$ y ${\color{azulCielo}{\boldsymbol{
H'}}}$ es una subgr\'afica con $V(H')=\{v_5,v_6,v_7\}$ y $E(H')=\{v_5v_6,
v_6v_7\}$. Notamos que a ${\color{azulCielo}\boldsymbol {H'}}$ le falta la
arista ${\color{grisOscuro!80}\boldsymbol {v_5v_7}}$, resaltada en la figura,
para ser una subgr\'afica inducida. 

\begin{figure}[ht!]
    \centering
       \begin{tikzpicture}
    
            \begin{scope}[xshift=0cm,scale=1]
                \foreach \i in {1,...,4} 
                    \draw ({(360/7)*\i}:2) node(\i)[vertex, fill=morado,
                    label=(360/7)*\i:{\large $v_{\i}$}]{};
                
                \foreach \i in {5,...,7} 
                    \draw ({(360/7)*\i}:2) node(\i)[vertex,fill=azulCielo,
                    label=(360/7)*\i:{\large $v_{\i}$}]{};

            \foreach \i/\j in
                {1/5,1/6,1/7,2/5,2/6,2/7,3/5,3/6,3/7,4/5,
            4/6,4/7}
                \draw [edge,grisOscuro!50] (\i) to (\j);
            
            \foreach \i/\j in
                {1/2,1/3,1/4,2/3,2/4,3/4}
                \draw [wedge,morado] (\i) to (\j);
            
            \foreach \i/\j in
                {5/6,6/7}
                \draw [wedge,azulCielo] (\i) to (\j);
            
            \draw [wedge,grisOscuro!80] (5) to (7);
            \end{scope}
                
    \end{tikzpicture}
    \caption{Una gr\'afica en la que se resalta una subgr\'afica inducida,
    ${\color{morado}\boldsymbol {H}}$, y una subgr\'afica, ${\color{azulCielo}\boldsymbol {H'}}$.}
    \label{fig:subgraf}
\end{figure}


\section{Clanes y conjuntos independientes}
\label{sec:clanes-CIndep}

    Algo que es relevante a trav\'es de varios temas de la Teor\'ia de
    Gr\'aficas es encontrar conjuntos de v\'ertices que sean adyacentes dos a
    dos o conjuntos de v\'ertices que, al contrario, no tengan vecinos dentro de
    dicho conjunto. Teniendo una gr\'afica $G$ y un conjunto $S \subseteq V(G)$,
    decimos que $S$ es un \indice{conjunto independiente} si cualesquiera dos
    v\'ertices de $S$ no son adyacentes. Esto quiere decir que todos los
    elementos de $S$ tienen grado $0$ en $G[S]$. La cardinalidad del conjunto
    independiente m\'as grande de una gr\'afica $G$ se llama \indice{n\'umero de
    independencia} y se denota $\alpha (G)$. Por otro lado, decimos que $S$ es
    un \indice{clan} si cualesquiera dos v\'ertices en $S$ son adyacentes. En
    este caso, el grado de todo v\'ertice de $S$ en $G[S]$ es $|S-1|$. De manera
    an\'aloga a los conjuntos independientes, a la cardinalidad del clan con
    mayor n\'umero de elementos de una gr\'afica $G$ se le llama
    \indice{n\'umero de clan} y se denota $\omega(G)$. En \cref{fig:ClanInd} se
    muestra un ejemplo de un ${\color{coral}\bf {clan}}$ y un ${\color{verde}\bf
    {conjunto}}$ ${\color{verde}\bf {independiente}}$ de una gr\'afica $G$,
    resaltados de color naranja y verde, respectivamente. Es f\'acil observar
    que el ${\color{verde}\bf {conjunto}}$ ${\color{verde}\bf {independiente}}$
    no es el conjunto independiente m\'as grande que podemos obtener, pues
    podr\'iamos agregar el v\'ertice $v_2$ y seguir\'ia siendo un conjunto
    independiente. De igual manera, podemos notar que existe un clan con
    cardinalidad mayor al ${\color{coral}\bf {clan}}$ mostrado. Este clan, que
    llamamos $K$, est\'a mostrado con el per\'imetro de los v\'ertices de color
    ${\color{azulCielo}\bf {azul}}$. Es f\'acil ver que $|K| = \omega (G)$.


\begin{figure}[ht!]
    \centering
       \begin{tikzpicture}
    
            \begin{scope}[xshift=0cm,scale=0.9]
                \foreach \i in {5,...,7} 
                    \draw ({(360/4)*\i}:1.3) node(\i)[avertex, fill=coral,
                    label=(360/4)*\i:{\large $v_{\i}$}]{};
                \draw ({(360/4)*8}:1.3)node(8)[avertex,label=(360/4)*8:{\large
                    $v_{8}$}]{};
                
                \foreach \i in {1,3} 
                    \draw ({(360/8)*(\i*2-1)}:3) node(\i)[bvertex,fill=verde,
                    label=(360/8)*(\i*2-1):{\large $v_{\i}$}]{};
                
                \foreach \i in {2,4}
                    \draw ({(360/8)*(\i*2-1)}:3)
                   node(\i)[bvertex,label=(360/8)*(\i*2-1):{\large $v_{\i}$}]{};


            \foreach \i/\j in
                {1/5,1/8,2/5,2/6,2/4,3/6,3/7,3/4,4/7,4/8}
                \draw [edge,grisOscuro!50] (\i) to (\j);
            
            \foreach \i/\j in
                {5/6,5/7,6/7}
                \draw [wedge,coral] (\i) to (\j);
            
            \foreach \i/\j in {8/5,8/6,8/7} 
                \draw [wedge,azulCielo!60] (\i) to (\j);
            \end{scope}
                
    \end{tikzpicture}
    \caption{Una gr\'afica, resaltando dos clanes y un conjunto independiente.}
    \label{fig:ClanInd}
\end{figure}


%%%%%%%%%%%%%%%%%%%%%%%%%%%%%%%%%%%%%%%%%%%%%%%%%%%%%%%%%%%%%%%%%%%%%%%%%%%%%%%%
%%%%%%%%%%%%%%%%%%%%%%%%%%%%%%%%%%%%%%%%%%%%%%%%%%%%%%%%%%%%%%%%%%%%%%%%%%%%%%%%
%%%%%%%%%%%%%%%%%%%%%%%%%%%%%%%%%%%%%%%%%%%%%%%%%%%%%%%%%%%%%%%%%%%%%%%%%%%%%%%%

\section{Operaciones}
\label{sec:operaciones}

Al igual que en muchas \'areas de las matem\'aticas, al buscar maneras de
generar nuevas gr\'aficas, nos encontramos con las operaciones de gr\'aficas. A
continuaci\'on nos enfocamos en las operaciones de gr\'aficas que son
pertinentes para este trabajo. El primer grupo de operaciones que se define se
basa en el concepto de uni\'on de conjuntos. La \indice{uni\'on} de dos
gr\'aficas $G$ y $H$ la definimos como la gr\'afica cuyo conjunto de v\'ertices
y de aristas son la uni\'on de los conjuntos de v\'ertices  y de aristas de
ambas gr\'aficas, respectivamente. Se utiliza la misma notaci\'on que para la
uni\'on de conjuntos, es decir, $G \cup H$. El segundo tipo de uni\'on de
gr\'aficas que definimos es la uni\'on ajena. La \textbf{uni\'on}
\indiceSub{uni\'on}{ajena} de las gr\'aficas $G$ y $H$, denotada $G + H$, es la
uni\'on de $G$ y $H$, donde $V(G) \cap V(H) = \varnothing$. Por \'ultimo, otra
operaci\'on que surge del concepto de uni\'on de conjuntos y que va a ser de
utilidad en este trabajo es la uni\'on completa. La \textbf{uni\'on}
\indiceSub{uni\'on}{completa} de dos gr\'aficas $G$ y $H$, tales que $V(G) \cap
V(H)=\varnothing$, es la gr\'afica con conjunto de v\'ertices $V(G) \cup V(H)$ y
cuyo conjunto de aristas es $E(G) \cup E(H) \cup \{uv \colon u \in V(G), v \in
V(H) \}$. La uni\'on completa la denotamos $G \oplus H$. En
\cref{fig:ex-unionCompleta} se muestra un ejemplo de este tipo de uni\'on. Del
lado izquierdo de la figura se muestran dos gr\'aficas, la primera, en la parte
superior, tiene v\'ertices $v_i$ y la segunda, en la parte inferior, tiene
v\'ertices $w_i$, ambas con $i \in \{1, 2 ,3\}$. Del lado derecho de la figura
se muestra la uni\'on completa de dichas gr\'aficas.

\begin{figure}[htb!]
    \centering
    \begin{tikzpicture}

        \begin{scope}[yshift=1cm,xshift=-2cm,rotate=45,scale=1]
            \foreach \i in {1,...,3}
            \draw ({(360/3)*\i-150}:1)
                node(\i)[vertex, label=(360/3)*\i-105:{\large $v_{\i}$}]{};
            \end{scope}

        \begin{scope}[yshift=-1cm,rotate=45,scale=1]
            \foreach \i in {4,...,6}
            \draw ({(360/3)*\i+30}:1)
                node(\i)[vertex, label=(360/3)*\i+75:{\large $w_{{\pgfmathparse{int(\i-3)}\pgfmathresult}}$}]{};
        \end{scope}

        \foreach \i/\j in {1/2,2/3,3/1,4/5,5/6,6/4,1/4,1/5,1/6,2/4,2/5,2/6,3/4,3/5,3/6}
        \draw [edge,grisOscuro] (\i) to (\j);

        \begin{scope}[yshift=1cm,xshift=-8cm,rotate=45,scale=1]
            \foreach \i in {1,...,3}
            \draw ({(360/3)*\i-150}:1)
                node(\i)[vertex, label=(360/3)*\i-150:{\large $v_{\i}$}]{};
            \end{scope}

        \begin{scope}[yshift=-1cm,xshift=-6cm,rotate=45,scale=1]
            \foreach \i in {4,...,6}
            \draw ({(360/3)*\i+30}:1)
                node(\i)[vertex, label=(360/3)*\i+30:{\large $w_{{\pgfmathparse{int(\i-3)}\pgfmathresult}}$}]{};
        \end{scope}


            \foreach \i/\j in {1/2,2/3,3/1,4/5,5/6,6/4}
                \draw [edge,grisOscuro] (\i) to (\j);
    \end{tikzpicture}
    \caption{Dos gr\'aficas (izquierda) y su uni\'on completa (derecha).}
\label{fig:ex-unionCompleta}
\end{figure}

La siguiente proposici\'on sobre la uni\'on completa de subgr\'aficas
generadoras nos ser\'a de utilidad m\'as adelante este trabajo, por lo que
pasamos a demostrarla.

\begin{proposicion}
\label{prop:HUnion-SSubG}
    Sean $H$, $H'$, $G$ y $G'$ cuatro gr\'aficas. Si $H$ y $H'$ son
    subgr\'aficas generadoras de $G$ y $G'$, respectivamente, entonces la
    uni\'on completa de $H$ y $H'$ es una subg\'afica generadora de la uni\'on
    completa de $G$ y $G'$.
\end{proposicion}

\begin{proof}
    Dadas dos gr\'aficas $G$ y $G'$ y dos gr\'aficas generadoras $H \subseteq G$
    y $H'\subseteq G'$, buscamos demostrar que $H \oplus H'$ es una subgr\'afica
    generadora de $G \oplus G'$. Observamos que $V(H)=V(G)$ y $V(H')=V(G)$, de
    ah\'i que $V(H \oplus H')=V(G \oplus G')$. Por consiguiente, basta ver que
    las aristas de $H \oplus H'$ son aristas de $G \oplus G'$. Sean $x$ y $y$
    v\'ertices adyacentes en $H \oplus H'$. Si $x,y \in V(H)$, de manera
    an\'aloga si $x,y \in V(H')$, entonces tenemos que $x \sim y$ en $G$,
    an\'alogamente $x \sim y$ en $G'$. Por ende, en ambos casos tenemos que $x$
    y $y$ son adyacentes en $G \oplus G'$. Ahora, sin p\'erdida de generalidad,
    de la definici\'on de  uni\'on completa tenemos que, si $x \in V(H)$ y $y
    \in V(H')$, entonces $x \sim y$ en $G \oplus G'$. De este modo tenemos que
    $H \oplus H'$ es una subgr\'afica generadora de $G \oplus G'$.
\end{proof}

Otra operaci\'on que se utilizar\'a m\'as adelante en este trabajo es el
producto cartesiano de gr\'aficas. Dadas dos gr\'aficas $G$ y $H$, el
\indice{producto cartesiano} de $G$ y $H$, denotado $G \square H$, es la
gr\'afica cuyo conjunto de v\'ertices es el producto cartesiano de $V(G)$ y
$V(H)$, es decir, $V(G \square H) = V(G) \times V(H)$. Adem\'as,
$(g_1,g_2)(h_1,h_2)$ es arista de $G \square H$ si y s\'olo si $g_1 = g_2$ y
$h_1h_2 \in E(H)$, o $h_1 = h_2$ y $g_1g_2 \in E(G)$. A continuaci\'on, en
\cref{fig:ex-cartesiano}, se muestran tres gr\'aficas, la primera del lado
izquierdo y con conjunto de v\'ertices $\{{\color{azulCielo}\boldsymbol{{v_1}}},
{\color{coral}\boldsymbol{{v_2}}}\}$ y la segunda en la parte inferior y con
conjunto de v\'ertices
$\{{\color{amarillo}\boldsymbol{{w_1}}},{\color{morado}\boldsymbol{{w_2}}}\}$.
Estas dos gr\'aficas son copias de la gr\'afica $K_2$ (introducida formalmente
en la siguiente secci\'on). En medio de estas gr\'aficas se encuentra la tercera
gr\'afica, el producto cartesiano de las gr\'aficas antes mencionadas,
resaltando de d\'onde vienen los v\'ertices de esta \'ultima gr\'afica.

\begin{figure}[ht!]
    \centering
        \begin{tikzpicture}
        
            \begin{scope}[xshift=0cm,scale=0.9]

                \draw ({(360/4)*2 - 45}:2) node(1)[avertex,fill=amarillo, label=(360/4)*2 -
                    45:{\large $k_{1}$}]{};
                \draw ({(360/4) - 45}:2) node(2)[avertex, fill=morado, label=(360/4) -
                    45:{\large $k_{2}$}]{};
                \draw ({(360/4)*3 - 45}:2) node(3)[coralvertex,fill= amarillo, label=(360/4)*3 - 
                45:{\large $k_{3}$}]{};
                \draw ({(360/4)*4 - 45}:2) node(4)[coralvertex, fill= morado, label=(360/4)*4 - 
                45:{\large $k_{4}$}]{};
                \draw (-3,1.4) node (5) [avertex, label=180:{\large $v_1$}] {};
                \draw (-3,-1.4) node (6) [coralvertex, label=180:{\large $v_2$}] {}; 
                \draw (-1.4,-3) node (7) [bvertex, fill= amarillo, label=270:{\large $w_1$}] {};
                \draw (1.4,-3) node (8) [bvertex, fill= morado, label=270:{\large $w_2$}] {};

                \foreach \i/\j in {1/2,1/3,3/4,2/4,5/6,7/8}
                    \draw [edge,grisOscuro] (\i) to (\j);
                \end{scope}

                    
        \end{tikzpicture}
        \caption{En el centro, la gr\'afica $K_2 \square K_2$. Una copia de
        $K_2$ a la izquierda y abajo, resaltando cada v\'ertice de distinto
        color.}
        \label{fig:ex-cartesiano}
    \end{figure}

Hasta ahora se han definido operaciones para dos o m\'as gr\'aficas, sin
embargo, tambi\'en existen operaciones sobre una misma gr\'afica. Tal es el caso
de la \indiceSub{aristas}{contraci\'on} \textbf{de aristas}. Dada una gr\'afica
$G$ y $e=uv$ una arista de $G$, denotamos $G/e$ a la gr\'afica obtenida de
contraer a la arista $e$. Donde $E(G/e)=E(G) \setminus \{e\}$ es el conjunto de
aristas de $G/e$ y su conjunto de v\'ertices es $V(G/e)=(V(G)\setminus \{u,v\})
\cup \{w\}$, donde $w$ es un v\'ertice tal que $N(w)=\{x \in V(G) \colon x \in
(N(u)\setminus \{v\}) \lor x \in (N(v) \setminus \{u\})\}$. En otras palabras,
$G/e$ es la gr\'afica obtenida de eliminar la arista $e$ y ``fusionar'' los
extremos de $e$ para crear un nuevo v\'ertice llamado $w$. La Figura
\ref{fig:ex-contract} muestra un peque\~{n}o ejemplo de la contracci\'on de
aristas.

\begin{figure}[htb!]
    \centering
    \begin{tikzpicture}

        \begin{scope}[xshift=0cm]
            \foreach \i in {1,2}
            \draw ({(360/3)*\i}:1.5)
                node(\i)[vertex, label=(360/3)*\i:{\large $v_{\i}$}]{};
                
            \draw (360:1.5) node(3)[vertex, fill=azulCielo, label=0:{\large $w$}]{};
                
            \foreach \i/\j in {1/2,1/3,2/3}
                \draw [edge,grisOscuro] (\i) to (\j);
            \end{scope}

        \begin{scope}[xshift=-6cm]
            \foreach \i in {1,2}
            \draw ({(360/4)*\i}:1.5)
                node(\i)[vertex, label=(360/4)*\i:{\large $v_{\i}$}]{};

            \draw ({(360/4)*3}:1.5) node(3)[vertex, fill=amarillo, label=(360/4)*3:{\large $v_{3}$}]{};
            \draw (360:1.5) node(4)[vertex, fill=coral, label=0:{\large $v_{4}$}]{};

            \foreach \i/\j in {1/2,1/3,1/4,2/3}
                \draw [edge,grisOscuro] (\i) to (\j);
            
            \foreach \i/\j in {3/4}
                \draw[wedge,azulCielo] (\i) to (\j);
            \end{scope}

    \end{tikzpicture}
    \caption{La gr\'afica $G$ con la arista $e=v_3v_4$ resaltada de
    ${\color{azulCielo}\bf azul}$ (izquierda) y la gr\'afica
    $G/{\color{azulCielo}\boldsymbol{e}}$ (derecha).}
\label{fig:ex-contract}
\end{figure}

Por \'ultimo, definimos el \indice{complemento} de una gr\'afica
$G$ como la gr\'afica que tiene el mismo conjunto de v\'ertices que $G$ y cuyo
conjunto de aristas es $\binom{V(G)}{2} - E(G)$. A esta gr\'afica la denotamos
$\overline{G}$.



%%%%%%%%%%%%%%%%%%%%%%%%%%%%%%%%%%%%%%%%%%%%%%%%%%%%%%%%%%%%%%%%%%%%%%%%%%%%%%%%
%%%%%%%%%%%%%%%%%%%%%%%%%%%%%%%%%%%%%%%%%%%%%%%%%%%%%%%%%%%%%%%%%%%%%%%%%%%%%%%%
%%%%%%%%%%%%%%%%%%%%%%%%%%%%%%%%%%%%%%%%%%%%%%%%%%%%%%%%%%%%%%%%%%%%%%%%%%%%%%%%

\section{Caminos y conexidad}
\label{sec:CamConex}

En Teor\'ia de Gr\'aficas resulta interesante preguntarse si, en una gr\'afica,
hay alg\'un ``v\'inculo'' entre dos v\'ertices. Si estos dos v\'ertices son
adyacentes, es claro que ese v\'inculo existe a trav\'es de la arista que
comparten. Un $uv$\textbf{-camino}\index{u @$uv$-camino} entre los v\'ertices
$u$ y $v$ de una gr\'afica $G$ es una sucesi\'on alternada de v\'ertices y
aristas de $G$ de la siguiente forma, $W=(v_0, e_1,v_1, \dots, e_{k-1},v_{k-1},
e_k,v_k)$ con $v_i \in V(G)$ y donde $e_j = v_{j-1}v_j$ para $i \in \{0, \dots,
k\}$ y $j \in \{ 1, \dots, k\}$. Este $uv$-camino es una forma de vincular el
v\'ertice $u$ con el v\'ertice $v$. En caso de que no sea necesario especificar
de qu\'e v\'ertice a qu\'e v\'ertice va la sucesi\'on, nos referimos simplemente
a un \indice{camino}. Es importante notar que un camino no siempre tiene
aristas. Un camino es \indiceSub{camino}{trivial} si tiene \'unicamente un
v\'ertice y no tiene aristas. Este es el \'unico caso en el que un camino no
tiene aristas. Al ser una sucesi\'on, a un camino $W$ se le puede asociar una
\indiceSub{camino}{longitud}, definido como el n\'umero de posiciones en $W$ que
son ocupadas por aristas. Denotamos la longitud de $W$ como $\ell(W)$. 

Al poderle asignar una longitud a un camino, es natural utilizar esta longitud
para ver que tan ``cercano'' est\'a un v\'ertice de otro. Pero, observamos que,
un camino puede pasar m\'ultiples veces por el mismo v\'ertice o la misma
arista. En otras palabras, el concepto de camino no ayuda mucho para encontrar
la forma m\'as corta para llegar de un v\'ertice a otro. Por lo tanto,
introducimos el concepto de trayectoria. Una \indice{trayectoria} es un camino
que no repite v\'ertices. Al igual que con el concepto de camino, una
$uv$\textbf{-trayectoria} \index{u @$uv$-trayectoria} es una trayectoria que
tiene v\'ertice inicial $u$ y v\'ertice final $v$. En una $uv$-trayectoria, los
v\'ertices $u$ y $v$ son llamados \indiceSub{v\'ertices}{extremos} y el resto de
los v\'ertices de la trayectoria son llamados \textbf{v\'ertices}
\indiceSub{v\'ertices}{internos}. A la $uv$-trayectoria $P$, la podemos escribir
como $u\xrightarrow[P]{} v$. Notamos que, al no repetir v\'ertices, este camino
tampoco repite aristas, pues, si lo hiciera, necesitar\'ia volver a pasar por al
menos uno de los extremos de dicha arista. De lo anterior se sigue que todas las
trayectorias que tengan los mismos v\'ertices y aristas tambi\'en tienen la
misma longitud. El usar trayectorias es una mejor aproximaci\'on al concepto de
``cercan\'ia'' entre v\'ertices, sin embargo, se puede tener el caso que haya
m\'ultiples trayectorias de distintas longitudes entre dos v\'ertices. Definimos
la \indice{distancia} entre dos v\'ertices $u$ y $v$ de una gr\'afica, denotada
$d(u,v)$, como la longitud de la $uv$-trayectoria m\'as corta. En caso de que no
existan $uv$-trayectorias en la gr\'afica, decimos que la distancia entre $u$ y
$v$ es infinito. Sea $P$ una trayectoria en una gr\'afica, decimos que $P$ no
tiene \indice{cuerdas} si la subgr\'afica inducida por $V(P)$ tiene grado
m\'aximo 2. Observamos que cualquier trayectoria de longitud m\'inima entre dos
v\'ertices no tiene cuerdas. Adicionalmente, podemos darle una ``dimensi\'on'' a
una gr\'afica $G$ al definir su \indice{di\'ametro} como $\max_{v\in
G}\{\max_{u\in G}\{d(u,v)\}\}$ si $G$ es conexa e infinito si es inconexa. 

La siguiente proposici\'on nos muestra una relaci\'on entre caminos y
trayectorias.

\begin{proposicion}
\label{prop:CamTray}
    En una gr\'afica $G$ con $u, v \in V(G)$, $u \ne v$, todo $uv$-camino
    contiene una $uv$-trayectoria.
\end{proposicion}

\begin{proof}
    Sea $W$ un $uv$-camino en una gr\'afica $G$, con $u,v \in V(G)$.
    Demostramos, por inducci\'on sobre la longitud de $W$, que $W$ contiene una
    $uv$-trayectoria. Primero, tomamos un camino $W$ tal que $\ell(W)=1$. Esto
    quiere decir que $W$ s\'olo contiene una arista, por lo tanto, $W$ es una
    trayectoria. Ahora, supongamos que todo $uv$-camino con longitud menor a $k$
    contiene una $uv$-trayectoria. Tomamos un camino $W$ con longitud $k$, tal
    que no es una trayectoria, ya que de serlo, $W$ ser\'ia la trayectoria
    buscada. Tomamos $W= (w_0,e_1,w_1,e_2, \dots, e_n,w_n)$, con $w_0=u$,
    $w_n=v$, $e_i \in E(G)$ y $w_j \in V(G)$, donde $i \in \{1, \dots, n\}$ y $j
    \in \{0, \dots, n\}$. Sea $m \in \{0, \dots, n\}$ el \'indice del primer
    v\'ertice de $W$ que se repite y sea $l \in \{m+1, \dots, n\}$ tal que $w_m
    = w_l$. Formamos el camino $W'= (w_0,e_1,w_1,\dots, w_m, e_{l+1}, w_{l+1},
    \dots, e_n,w_n)$. Observamos que $\ell(W')<\ell(W)$, por lo que $W'$
    contiene una $uv$-trayectoria, por hip\'otesis de inducci\'on. Por lo tanto,
    $W$ tambi\'en contiene una $uv$-trayectoria.
\end{proof}

Por \cref{prop:CamTray}, sabemos que, cada que hablamos de caminos, siempre hay
una trayectoria contenida. Por esta raz\'on, de ahora en adelante nos enfocamos
en trayectorias. Como se mencion\'o anteriormente, entre dos v\'ertices $u$ y
$v$ pueden existir m\'ultiples trayectorias. Decimos que $X$ y $Y$ son dos
\textbf{$uv$-trayectorias} \indiceSub{u @$uv$-trayectoria}{internamente ajenas}
si $V(X)\cap V(Y)=\{u,v\}$, i.~e. los v\'ertices en com\'un entre $X$ y $Y$ son
\'unicamente sus extremos. El m\'aximo n\'umero de $uv$-trayectorias
internamente ajenas se denota por $p(u,v)$. Tambi\'en podemos definir un
concepto similar enfoc\'andonos en aristas. Llamamos \textbf{$uv$-trayectorias}
\indiceSub{u @$uv$-trayectoria}{ajenas por aristas} a dos o m\'as
$uv$-trayectorias que no tengan aristas en com\'un. Estos conceptos nos
sevir\'an m\'as adelante cuando hablemos de conexidad. Adicionalmente, siempre
podemos recorrer una trayectoria en sentido contrario, es decir, empezando por
el \'ultimo v\'ertice y terminando por el primero. Al recorrer la trayectoria de
esta manera obtenemos una nueva trayectoria. Formalmente hablando, teniendo una
$uv$-trayectoria $X$, su \textbf{trayectoria} \indiceSub{trayectoria}{inversa},
a la cual denotamos $X^{-1}$, es una $vu$-trayectoria que tiene los mismos
v\'ertices y aristas que $X$.

Algo interesante de observar es que, a pesar de que las trayectorias las
definimos con v\'ertices como extremos, tambi\'en se pueden definir con
conjuntos de la siguiente manera. Sean $X,Y \subseteq V(G)$, definimos la
trayectoria $P=(v_1,\dots, v_r)$ como una $XY$-trayectoria si sucede que $V(P)
\cap X= \{v_1\}$ y $V(P) \cap Y= \{v_r\}$.

\begin{figure}[htb!]
    \centering
        \begin{tikzpicture}
    
            \begin{scope}[xshift=0cm,scale=0.8]
                \foreach \i in {1,2} \draw ({(360/6)*\i}:2.5)
                    node(\i)[vertex, label=(360/6)*\i:{\large $v_{\i}$}]{};
            
            \foreach \i in {3,5} \draw ({(360/6)*\i}:2.5)
                    node(\i)[vertex, fill =coral, label=(360/6)*\i:{\large $v_{\i}$}]{};
            
            \draw ({(360/6)*6}:2.5) node(6)[vertex, fill=morado, label=(360/6)*6:{\large $v_{6}$}]{};
            \draw ({(360/6)*4}:2.5) node(4)[vertex, fill=verde, label=(360/6)*4:{\large $v_{4}$}]{};
            \draw (1,0) node (7) [vertex,fill=morado, label=300:{\large $v_7$}] {};
            \draw (-1,0) node (8) [vertex,fill=morado, label=150:{\large $v_8$}] {};
                
            \foreach \i/\j in
                {1/6,2/3,2/8,4/8,5/7,5/8}
                \draw [edge,grisOscuro!50] (\i) to (\j);
            
                \foreach \i/\j in
                {3/8,7/8,2/7,1/2,1/7,6/7,5/6}
                \draw [wedge,morado] (\i) to (\j);
            
            \foreach \i/\j in
                {3/4,4/5}
                \draw [wedge,verde] (\i) to (\j);

            \end{scope}
                
    \end{tikzpicture}
    \caption{Una gr\'afica, donde se resaltan las ${\color{morado}\bf
    {aristas}}$ de un ${\color{coral}\boldsymbol {v_3 v_5}}$-camino y resaltando
    los ${\color{morado}\textbf{v\'ertices}}$  de su ${\color{coral}\boldsymbol
    {v_3 v_5}}$-trayectoria contenida.  Tambi\'en se resalta, en
    ${\color{verde}\bf {verde}}$, una ${\color{coral}\boldsymbol {v_3
    v_5}}$-trayectoria sin cuerdas.}
    \label{fig:ex-caminos}
\end{figure}
    
En \cref{fig:ex-caminos} podemos ver un ejemplo de \cref{prop:CamTray}, donde se
resaltan las aristas de ${\color{morado}\bf {morado}}$ de un $v_3 v_5$-camino y
donde, para mostrar una trayectoria contenida, se resaltan los v\'ertices de
${\color{morado}\textbf{morado}}$ de dicha $v_3 v_5$-trayectoria.
Adem\'as, en la misma figura se observan dos ${\color{coral}\boldsymbol
{v_3 v_5}}$-trayectorias internamente ajenas. Ambas est\'an representadas con
sus v\'ertices internos coloreados del mismo color, una de ${\color{morado}\bf
{morado}}$ y la otra de ${\color{verde}\bf {verde}}$. Notamos que la
${\color{morado}\bf{trayectoria}}$ ${\color{morado}\bf {morada}}$ tiene cuerdas,
entre ellas, la arista $v_8v_5$. Por otro lado, la
${\color{verde}\bf{trayectoria}}$ ${\color{verde}\bf {verde}}$ no tiene cuerdas
y tiene la longitud m\'as corta entre $v_3$ y $v_5$, de donde obtenemos que la
distancia entre $v_3$ y $v_5$ es $2$.

Cuando un camino tiene extremos iguales, se le llama \textbf{camino}
\indiceSub{camino}{cerrado}. Al igual que en el caso de las trayectorias, vale
la pena resaltar el caso en el que un camino cerrado no repite v\'ertices (salvo
los extremos). Un \indice{ciclo} es un camino cerrado de longitud al menos $3$
que, adicionalmente, no repite v\'ertices, salvo los extremos.  Muchas veces nos
interesa saber si una gr\'afica contiene o no contiene ciclos. Una gr\'afica sin
ciclos es llamada \textbf{gr\'afica} \indiceSub{gr\'afica}{ac\'iclica}. Algo que
tambi\'en nos ser\'a de utilidad m\'as adelante, es el concepto de ciclo
generador. De manera similar a una subgr\'afica generadora, un \textbf{ciclo}
\indiceSub{ciclo}{generador} es un ciclo cuyo conjunto de v\'ertices es el mismo
que el de la gr\'afica.
   
Hasta ahora nos hemos enfocado en los ``v\'inculos'' entre dos v\'ertices de una
gr\'afica. Sin embargo, es importante mencionar que no siempre existe una manera
de ``vincular'' dos v\'ertices, como se puede ver en \cref{fig:diagGraf}, donde
no existen trayectorias que conecten a $v_7$ con alg\'un otro v\'ertice.
Definimos una gr\'afica $G$ como
\indiceSub{gr\'afica}{conexa}\index{conexa!gr\'afica} si, para cualquier par de
v\'ertices $u,v \in V(G)$, existe una $uv$-trayectoria. Si una gr\'afica no
cumple esta propiedad, decimos que es \indiceSub{gr\'afica}{inconexa}.
Observamos que toda gr\'afica, incluso las gr\'aficas inconexas, contienen
``partes'' que s\'i son conexas, dado que un v\'ertice aislado es trivialmente
conexo. Por lo anterior, al estudiar la conexidad de las gr\'aficas, en todo
caso, nos podemos enfocar en dichas partes. En una gr\'afica $G$, las
subgr\'aficas m\'aximas con la propiedad de ser conexas son llamadas
\indice{componentes conexas}\index{conexa!componente}. Notamos que si una
gr\'afica es conexa, contiene s\'olo una componente conexa, ella misma.

Una pregunta interesante en este tema es, qu\'e tan dif\'icil es ``desconectar''
una gr\'afica. Para esto puede resultar natural cuestionarse si existe alg\'un
v\'ertice o conjunto de v\'ertices que, al quitarlo, ``desconecte'' a la
gr\'afica. A este conjunto lo llamamos corte por v\'ertices. Formalmente, un
\indice{corte por v\'ertices}\index{v\'ertices!corte} $S$ de una gr\'afica $G$
es un subconjunto de $V(G)$ tal que $G[V(G)-S]$ es inconexa, es decir, el
conjunto $S$ \indice{separa} a la gr\'afica en dos o m\'as componentes conexas.
Decimos que $S$ es un $k$\textbf{-corte por v\'ertices}\index{corte por
v\'ertices!k @ $k$-corte}\index{v\'ertices!k @$k$-corte} si $|S|=k$. Un corte
por v\'ertices $S$ entre dos v\'ertices no adyacentes $u$ y $v$ es un corte por
v\'ertices que deja a $u$ y a $v$ en diferentes componentes conexas en $G-S$. A
este corte por v\'ertices se le llama $uv$\textbf{-corte por
v\'ertices}\index{corte por v\'ertices!u @$uv$-corte}\index{v\'ertices!u
@$uv$-corte}. La m\'inima cardinalidad de un $uv$-corte por v\'ertices se denota
$c(u,v)$ y no est\'a definida si $u$ y $v$ son adyacentes o $x=y$. Tambi\'en
pueden existir cortes por v\'ertices entre dos conjuntos de v\'ertices. Sean
$X,Y \subseteq V(G)$, $S$ es un $XY$\textbf{-corte por v\'ertices}\index{corte
por v\'ertices!X @$XY$-corte}\index{v\'ertices!X @$XY$-corte} si $G-S$ es una
gr\'afica inconexa donde $X$ y $Y$ se encuentran en distintas componentes
conexas. Vale la pena notar que $X$ y $Y$ pueden compartir v\'ertices. En este
caso, para separar a los conjuntos, $S$ debe contener a los elementos de la
intersecci\'on, por lo que el $XY$-corte por v\'ertices $S$ deja a $X\setminus
S$ y $Y \setminus S$ en componentes conexas distintas en $G-S$. Cabe notar que
no todas las gr\'aficas tienen alg\'un corte por v\'ertices, a saber, las
gr\'aficas en las que cualesquiera dos v\'ertices son adyacentes no tienen
cortes por v\'ertices, y son las \'unicas con esta propiedad. Ahora, usando el
concepto de corte por v\'ertices, definimos la conexidad de una gr\'afica. El
\indice{n\'umero de conexidad} de una gr\'afica $G$, denotado por $\kappa(G)$,
se define como la m\'inima cardinalidad de un conjunto de corte de $G$, es
decir, $\kappa(G)=\min \{|S| \colon\ S \textnormal{ es corte por v\'ertices}\}$.
Como ya se observ\'o, las gr\'aficas completas no tienen cortes por v\'ertices,
no obstante, por convenci\'on decimos que una gr\'afica completa de $n$
v\'ertices tiene n\'umero de conexidad $n-1$. Por otra parte, el vac\'io es un
corte por v\'ertices en cualquier gr\'afica inconexa, por lo que una gr\'afica
inconexa tiene n\'umero de conexidad $0$. Adicionalmente, decimos que una
gr\'afica $G$ es $t$\textbf{-conexa} \index{gr\'afica!t @$t$-conexa} si
$|V(G)|>t$ y $G-X$ es conexa para todo conjunto de v\'ertices $X$ tal que
$|X|<t$. Para ejemplificar el concepto de conexidad nos fijamos en las
gr\'aficas en \cref{fig:isoGraf} y \cref{fig:subgraf}. Notamos que ambas son
conexas, sin embargo, necesitamos m\'as v\'ertices para ``desconectar'' la
gr\'afica de \cref{fig:subgraf} que las gr\'aficas de \cref{fig:isoGraf}. A
saber, las gr\'aficas de \cref{fig:isoGraf} son $2$-conexas y la gr\'afica de
\cref{fig:subgraf} es $6$-conexa.

Otra manera de abordar el tema de ``desconectar'' una gr\'afica es  pensar en
las trayectorias internamente ajenas, pues la cantidad de trayectorias
internamente ajenas entre dos v\'ertices nos habla de cu\'antas maneras
diferentes hay de conectar un v\'ertice con otro. De hecho, existe una
relaci\'on entre la conexidad de una gr\'afica y la cantidad de trayectorias
internamente ajenas que hay entre cualesquiera dos de sus v\'ertices. Para
ayudar a visualizar dicha relaci\'on, en \cref{fig:ex-menger} se muestran dos
copias de una gr\'afica, donde cada copia busca la manera de separar los
v\'ertices $\color{verde} \bf v_1$ y $\color{verde} \bf v_5$ utilizando
conceptos diferentes. Del lado izquierdo se muestran tres $v_1v_5$-trayectorias
internamente ajenas, cada una resaltada de un color diferente. Del lado derecho
se muestra el corte por v\'ertices ${\color{morado}\boldsymbol {\{v_2, v_6,
v_7\} }}$, donde cada v\'ertice est\'a relleno de morado. Observemos que cada
v\'ertice del corte por v\'ertices se encuentra en una $v_1v_5$-trayectoria que
es internamente ajena de las dem\'as. Para facilitar esta visualizaci\'on, cada
v\'ertice del corte por v\'ertices tiene el per\'imetro del mismo color que la
$v_1v_5$-trayectoria internamente ajena a la que nos referimos. La relaci\'on
entre la cantidad de trayectorias internamente ajenas entre cualesquiera dos
v\'ertices de una gr\'afica y la conexidad de \'esta fue demostrada por Menger y
es un teorema de mucha relevancia en Teor\'ia de Gr\'aficas. A continuaci\'on
demostramos dicho teorema, conocido como el Teorema de Menger, y una versi\'on
global de \'este. 

\begin{figure}[htb!]
    \centering
    \begin{tikzpicture}
    
        \begin{scope}[xshift=0cm,scale=0.8]
            \foreach \i in {2,3} \draw ({(360/6)*\i}:2.5)
                node(\i)[vertex, label=(360/6)*\i:{\large $v_{\i}$}]{};
            
            \foreach \i in {1,5} \draw ({(360/6)*\i}:2.5)
                node(\i)[wvertex, fill=verde, label=(360/6)*\i:{\large $v_{\i}$}]{};
            
            \draw ({(360/6)*6}:2.5) node(6)[vertex, label=(360/6)*6:{\large $v_{6}$}]{};
            \draw ({(360/6)*4}:2.5) node(4)[vertex, label=(360/6)*4:{\large $v_{4}$}]{};
            \draw (1,0) node (7) [vertex, label=300:{\large $v_7$}] {};
            \draw (-1,0) node (8) [vertex, label=150:{\large $v_8$}] {};
                
            \foreach \i/\j in
                {2/3,5/8,3/8,7/8,2/7,6/7,3/4}
                \draw [edge,grisOscuro!50] (\i) to (\j);
            
            \foreach \i/\j in
                {1/2,2/8,4/8,4/5}
                \draw [wedge,coral] (\i) to (\j);
            
            \foreach \i/\j in
                {1/7,5/7}
                \draw [wedge,azulCielo] (\i) to (\j);
            
            \foreach \i/\j in
                {1/6,5/6}
                \draw [wedge,amarillo] (\i) to (\j);

        \end{scope}

        \begin{scope}[xshift=8cm,scale=0.8]
            \foreach \i in {1,5} \draw ({(360/6)*\i}:2.5)
                node(\i)[wvertex, fill=verde, label=(360/6)*\i:{\large $v_{\i}$}]{};
            
            \draw ({(360/6)*2}:2.5) node(2)[coralvertex, fill=morado, label=(360/6)*2:{\large $v_{2}$}]{};
            \draw ({(360/6)*3}:2.5) node(3)[vertex, label=(360/6)*3:{\large $v_{3}$}]{};
            \draw ({(360/6)*4}:2.5) node(4)[vertex, label=(360/6)*4:{\large $v_{4}$}]{};
            \draw ({(360/6)*6}:2.5) node(6)[amarvertex, fill=morado, label=(360/6)*6:{\large $v_{6}$}]{};
            \draw (1,0) node (7) [avertex, fill=morado, label=300:{\large $v_7$}] {};
            \draw (-1,0) node (8) [vertex, label=150:{\large $v_8$}] {};
                
            \foreach \i/\j in
                {1/6,1/2,1/7,2/3,2/7,2/8,3/4,3/8,4/5,4/8,5/6,5/7,5/8,6/7,7/8}
                \draw [edge,grisOscuro!50] (\i) to (\j);
            
        \end{scope}
                
    \end{tikzpicture}
    \caption{Tres trayectorias internamente ajenas entre $v_1$ y $v_5$ 
    (izquierda), y un conjunto de tres v\'ertices que separa a $v_1$ de $v_5$
     (derecha).}
    \label{fig:ex-menger}
\end{figure}

% Otra forma de determinar como ``desvincular'' dos v\'ertices es preguntarse
% cuantas aristas se necesitar\'ian ``borrar'' para que no hubiera trayectorias
% entre dichos v\'ertices. Es f\'acil observar que esto est\'a relacionado con
% la cantidad de trayectorias ajenas por aristas %% TODO: agregar definición
% trayectorias ajenas por aristas que existan entre dichos v\'ertices.

\begin{teorema}
    \label{teo:menger}
    Sea $G$ una gr\'afica y sean $X, Y \subseteq V(G)$, entonces el m\'aximo
    n\'umero de $XY$-trayectorias internamente ajenas en $G$ es igual al
    m\'inimo n\'umero de v\'ertices que separan a $X$ de $Y$.
\end{teorema}

\begin{proof}
    Sea $G$ una gr\'afica y sean $X,Y \subseteq V(G)$. Primero, consideramos el
    caso en el que un conjunto de v\'ertices est\'a contenido en el otro, sin
    p\'erdida de generalidad, consideramos $X \subseteq Y$. Por un lado, cada
    v\'ertice de $X$ es una $XY$-trayectoria tivial, por lo que el n\'umero de
    $XY$-trayectorias internamente ajenas es $|X|$. Por otro lado, cualquier
    conjunto de v\'ertices que separe a $X$ de $Y$ debe contener a los
    v\'ertices de $X$, por lo tanto, el m\'inimo n\'umero de v\'ertices que
    separan a $X$ de $Y$ es $|X|$.

    En segunda instancia, consideramos el caso en el que $X$ y $Y$ sean
    completamente adyacentes, i.~e., todo elemento de $X$ tiene, al menos, un
    vecino en $Y$ y todo elemento de $Y$ tiene, al menos, un vecino en $X$. Sin
    p\'erdida de generalidad, consideramos $|X| \leq |Y|$. Al ser completamente
    adyacentes, el m\'inimo n\'umero de $XY$-trayectorias internamente ajenas es
    $|X|$. Por otro lado, al tener que todos los elementos de $X$ tienen vecinos
    en $Y$, el m\'inimo n\'umero de v\'ertices que separan a $X$ de $Y$
    tambi\'en es $|X|$.

    Ahora, sean $X, Y \subseteq V(G)$ tales que no cumplan los casos anteriores.
    Denotamos por $k$ al m\'inimo n\'umero de v\'ertices que separan a $X$ de
    $Y$ en $G$. Observamos que, dada una familia de $XY$-trayectorias
    internamente ajenas, cada trayectoria contiene al menos un v\'ertice del
    conjunto que separa a $X$ de $Y$. Por eso, no pueden existir m\'as de $k$
    trayectorias internamente ajenas entre $X$ y $Y$ pues, de ser as\'i,
    ning\'un $k$-conjunto separar\'ia a $X$ de $Y$. Por lo tanto, basta
    demostrar que existen $k$ trayectorias internamente ajenas entre $X$ y $Y$.
    Se demuestra por inducci\'on sobre el n\'umero de aristas de $G$. Primero
    veamos el caso en el que $G$ no tiene aristas, entonces la \'unica manera de
    separar a $X$ de $Y$ es quitar los v\'ertices que tienen en com\'un, en
    otras palabras $|X \cap Y|=k$. Por otro lado, las $XY$-trayectorias
    internamente ajenas de $G$ son las trayectorias triviales tomando los
    v\'ertices de $X \cap Y$. Por ende, tenemos el n\'umero de $XY$-trayectorias
    internamente ajenas es $|X\cap Y|=k$.

    Ahora suponemos que $E(G) \neq \varnothing$ y que el teorema es v\'alido
    para gr\'aficas $G'$, tales que $|E(G')| < |E(G)|$. Sea $e = uv$ una arista
    de $G$. Primero consideramos $G/e$, la gr\'afica obtenida de ``contraer'' a
    la arista $e$, es decir, sustituir $u$ y $v$ por un \'unico v\'ertice cuya
    vecindad es igual a las uniones de las vecindades $u$ y $v$ (salvo $u$ y $v$
    mismos), definido en la siguiente secci\'on. Al v\'ertice obtenido de la
    contracci\'on de $e$ lo nombramos $x_e$. Puede suseder que alg\'un extremo
    de $e$ estaba en $X$ o $Y$, por lo que es necesario redefinir estos
    conjuntos. Si $u\in X$ o $v\in X$, entonces definimos $X'=(X \setminus
    \{u,v\}) \cup \{x_e\}$; en otro caso definimos $X'=X$. An\'alogamente,
    definimos $Y'= (Y \setminus \{u,v\}) \cup \{x_e\}$ si $u \in Y$ o $v \in Y$
    y definimos $Y'=Y$ en otro caso. Sea $A \subseteq V(G \setminus e)$ un
    $X'Y'$-corte por v\'ertices, entonces, por hip\'otesis de inducci\'on, el
    n\'umero de $X'Y'$-trayectorias internamente ajenas en $G/e$ es $|A|$. De lo
    que se sigue que, en $G$, el n\'umero de $XY$-trayectorias internamente
    ajenas es al menos $|A|$. Por eso, si $|A| \geq k$, entonces tenemos que el
    n\'umero de $XY$-trayectorias internamente ajenas en $G$ es $k$. Ahora, si
    $|A| \leq k-1$, entonces podemos afirmar que $x_e \in A$, pues de lo
    contrario tendr\'iamos que $A$ es un $XY$-corte por v\'ertices en $G$.
    Definimos $A'$ como el conjunto obtenido de $A$ al quitarle el v\'ertice
    $x_e$ y a\~{n}adirle los v\'ertices $u$ y $v$, es decir, $A'=(A \setminus
    \{x_e\})\cup \{u,v\}$. Observamos que el conjunto $A'$ separa  a $X$ de $Y$.
    La cardinalidad de $A'$ es $|A|+1 \leq (k-1)+1= k$, por otro lado, al
    separar a $X$ y a $Y$ tenemos que $|A'| \geq k$. As\'i pues, $|A'|=k$.

    Ahora, consideramos la gr\'afica $G-e$. Afirmamos que cualquier conjunto de
    v\'ertices que separa a $X$ de $A'$ en $G-e$ tambi\'en separa a $X$ de $Y$
    en $G$. Para demostrar esto, tomamos $B$ un conjunto que separa a $X$ de
    $A'$ en $G-e$. Observamos que toda $XY$-trayectoria en $G$ tiene alg\'un
    v\'ertice de $A'$. Luego, dada una familia de $XY$-trayectorias, es posible
    tomar una $XA'$-trayectoria por cada $XY$-trayectoria. Como $x,y \in A'$,
    entonces tenemos que dichas $XA'$-trayectorias tambi\'en son trayectorias de
    $G-e$, pues cada una contiene exactamente un v\'ertice de $A'$. De esta
    manera, cada $XA'$-trayectoria tiene un v\'ertice en $B$. Por este motivo,
    cada $XY$-trayectoria contiene un v\'ertice de $B$, de donde se sigue que
    $B$ es un conjunto que separa a $X$ de $Y$ en $G$. Observamos que
    $E(G-e)<E(G)$, por lo que se cumple la hip\'otesis de inducci\'on, i.~e., la
    familia de estas $XA'$-trayectorias tiene la misma cardinalidad de $B$.
    Notamos que, por construcci\'on, las $XA'$-trayectorias tambi\'en son
    trayectorias de $G$. Adem\'as tenemos que, al ser $B$ un conjunto que separa
    a $X$ de $Y$, $|B| \leq k$. Por lo tanto, tenemos que el n\'umero de
    $XA'$-trayectorias internamente ajenas en $G$ es $k$. De manera an\'aloga
    tenemos que hay $k$ trayectorias internamente ajenas de $Y$ a $A'$ en $G$.
    Como $A'$ separa a $X$ y a $Y$, entonces los v\'ertices que comparten las
    $XA'$-trayectorias y las $YA'$-trayectorias son v\'ertices de $A'$. Sean
    $a_1, \dots, a_k$ los v\'ertices de $A'$. Denotamos por $P_i$ a la
    $XA'$-trayectoria que contiene a $a_i$ y $Q_i$ a la $YA'$-trayectoria que
    contiene a $a_i$, con $i \in \{1. \dots, k\}$. A continuaci\'on, construimos
    una familia de $XY$-trayectorias internamente ajenas en $G$ de la siguiente
    manera, $X \xrightarrow[P_i]{} a_i \xrightarrow[Q_i]{} Y$, para toda $i \in
    \{1, \dots, k\}$. As\'i, concluimos que $G$ tiene $k$ $XY$-trayectorias
    internamente ajenas. 
\end{proof}

 Tambi\'en existe una versi\'on del Teorema de Menger para el caso cuando se
 desea separar v\'ertices, en lugar de conjuntos de v\'ertices. Pasamos a
 demostrar dicha versi\'on en \cref{coro:menger}, la cual es de utilidad al
 demostrar la versi\'on global del Teorema de Menger.

 \begin{corolario}
    \label{coro:menger}
    Sea $G$ una gr\'afica y sean $u, v \in V(G)$. Si $u$ y $v$ no son adyacentes
    en $G$, entonces el m\'aximo n\'umero de $uv$-trayectorias internamente
    ajenas es igual a la m\'inima cardinalidad de un $uv$-corte por v\'ertices.
\end{corolario}

\begin{proof}
    Sean $u$ y $v$ v\'ertices no adyacentes de una gr\'afica $G$ y sean $X$ y
    $Y$ las vecindades de $u$ y $v$, respectivamente. Por \cref{teo:menger}
    sabemos que el m\'aximo n\'umero de $XY$-trayectorias internamente ajenas en
    $G$ es igual al m\'inimo n\'umero de v\'ertices que separan a $X$ de $Y$.
    Adem\'as, ni $u$ ni $v$ son v\'ertices de dichas trayectorias y, por ende,
    tampoco son v\'ertices de los conjuntos separan a $X$ y $Y$. Sea  $A$ un
    conjunto que separa a $X$ y a $Y$ con cardinalidad $r$. Si $|X| \geq r$ y
    $|Y| \geq r$, entonces tenemos que $A$ tambi\'en separa a $u$ de $v$. Sean
    $P_1, \dots, P_r$ una familia de $XY$-trayectorias internamente ajenas,
    entonces $uX\xrightarrow[P_1]{}Yv$,\dots, $uX\xrightarrow[P_r]{}Yv$ forma
    una familia de $uv$-trayectorias internamente ajenas. 

    Nos enfocamos en el caso en el que alguna de las vecindades tiene
    cardinalidad menor que $r$.  Sin p\'erdida de generalidad, consideramos
    $|X|<r$. Sea $X = \{x_1, \dots, x_s\}$, con $s<r$, entonces tenemos que $X$
    es un $uv$-corte por v\'ertices. Por otro lado, sea $\mathcal{P}$ una
    familia de $XY$-trayectorias internamente ajenas. Tomamos $\mathcal{P'}
    \subseteq \mathcal{P}$ la familia con mayor n\'umero de trayectorias que no
    comparten extremo en $X$. Observamos que $|\mathcal{P'}| = |X|$. Nombramos
    $P_i$ a la $XY$-trayectoria de $\mathcal{P'}$ que contiene al v\'ertice $x_i
    \in X$, con $i \in \{1, \dots, s\}$. Por \'ultimo, consideramos siguiente
    familia de $uv$-trayectorias internamente ajenas, $uX\xrightarrow[P_1]{}Yv,
    \dots, uX\xrightarrow[P_s]{}Yv$. Por lo tanto, tenemos que el m\'aximo
    n\'umero de $uv$-trayectorias internamente ajenas es $|X|=s$.
\end{proof}


A continuaci\'on  se enuncia y demuestra la versi\'on global del Teorema de
Menger, es decir, la conexidad de una gr\'afica y la cantidad de trayectorias
internamente ajenas entre cualesquiera dos v\'ertices de dicha gr\'afica.

\begin{teorema}
    \label{teo:mengerGen}
     Sea $G$ una gr\'afica, $G$ es $k$-conexa si y s\'olo si contiene
     $k$ trayectorias internamente ajenas entre cualesquiera dos v\'ertices de
     $G$.
 \end{teorema}

 \begin{proof}
     Primero, sea $G$ una gr\'afica $k$-conexa. Suponemos que existen $u,v \in
     V(G)$, de manera que el m\'aximo n\'umero de $uv$-trayectorias internamente
     ajenas sea $t$, con $t<k$. Por \cref{coro:menger} tenemos que $u \sim v$.
     Ahora, consideramos la subgr\'afica de $G$ obtenida al borrar la arista
     $uv$, a la que llamamos $G'$. De este modo, el m\'aximo n\'umero de
     $uv$-trayectorias internamente ajenas en $G'$ es $t-1$. Por ende,
     utilizando \cref{coro:menger}, tenemos que un $uv$-corte por v\'ertices $X$
     tiene cardinalidad, a lo m\'as, $t-1$. Ahora, como $|V(G)| > k > t$,
     entonces existe un v\'ertice $w \in V(G)$ tal que $w \notin X\cup\{u,v\}$.
     Observamos que $X$ separa a $w$ de $u$ o $v$ en $G'$. Suponemos, sin
     p\'erdida de generalidad, que $X$ es un $uw$-corte por v\'ertices en $G'$.
     De este modo, $X\cup\{v\}$ es un conjunto de cardinalidad a lo m\'as $t$,
     que separa a los v\'ertices $u$ y $w$ en $G$. Esto es una contradicci\'on,
     pues $G$ es $k$-conexa y $t < k$. Por lo tanto, $G$ contiene $k$
     trayectorias internamente ajenas entre cualesquiera dos v\'ertices.
 
     Suponemos que $G$ contiene $k$ trayectorias internamente ajenas
     entre cualesquiera dos v\'ertices, por lo que tenemos que $|V(G)|>k$.
     Adem\'as, dados dos v\'ertices $u$ y $v$, cualquier $uv$-corte por
     v\'ertices tiene cardinalidad al menos $k$, a saber un v\'ertice por cada
     $uv$-trayectoria internamente ajena. De ah\'i que $G$ es $k$-conexa.
         
 \end{proof}


%%%%%%%%%%%%%%%%%%%%%%%%%%%%%%%%%%%%%%%%%%%%%%%%%%%%%%%%%%%%%%%%%%%%%%%%%%%%%%%%
%%%%%%%%%%%%%%%%%%%%%%%%%%%%%%%%%%%%%%%%%%%%%%%%%%%%%%%%%%%%%%%%%%%%%%%%%%%%%%%%
%%%%%%%%%%%%%%%%%%%%%%%%%%%%%%%%%%%%%%%%%%%%%%%%%%%%%%%%%%%%%%%%%%%%%%%%%%%%%%%%

\section{Algunas familias relevantes}
\label{sec:famGraf}
   
Muchas veces, en Teor\'ia de Gr\'aficas, al momento de abordar varios problemas,
es de utilidad separar las gr\'aficas en familias donde se compartan
caracter\'isticas. Hay varias familias relevantes en la Teor\'ia de Gr\'aficas;
en esta secci\'on nos enfocamos en las familias y sus propiedades que son
relevantes para este trabajo.

Decimos que una gr\'afica $G$ es \indiceSub{gr\'afica}{completa} si su conjunto
de aristas es igual a $\binom{V(G)}{2}$. A la gr\'afica completa de orden $n$ se
le denota $K_n$. Nos fijamos que $K_1$ es la gr\'afica que s\'olo tiene un
v\'ertice y su conjunto de aristas es vac\'io. A esta gr\'afica tambi\'en se le
conoce como la gr\'afica \indice{trivial}. Cuando una gr\'afica est\'a
conformada \'unicamente por v\'ertices aislados decimos que es una gr\'afica
\indiceSub{gr\'afica}{vac\'ia}. Algo que es importante de mencionar es que,
tanto las gr\'aficas completas, como las gr\'aficas vac\'ias, son \'unicas salvo
isomorfismos, por lo que les podemos dar un nombre propio. Vale la pena notar
que el complemento de la gr\'afica completa $K_n$ es gr\'afica vac\'ia de $n$
elementos. \Cref{fig:ex-va-comp}, muestra un ejemplo de una gr\'afica vac\'ia y
una completa.

\begin{figure}[ht!]
    \centering
        \begin{tikzpicture}
        
            \begin{scope}[xshift=0cm,scale=0.5]

                \foreach \i in {1,2,3} \draw ({(360/3)*\i}:1.8)
                    node(\i)[vertex, label=(360/3)*\i:{\large $v_{\i}$}]{};
                \end{scope}
        
            \begin{scope}[xshift=7cm,scale=0.6]

                \foreach \i in {1,...,5} \draw ({(360/5)*\i}:2.5)
                    node(\i)[vertex, label=(360/5)*\i:{\large $v_{\i}$}]{};
                
                \foreach \i/\j in {1/2,1/3,1/4,1/5,2/3,2/4,2/5,3/4,3/5,4/5}
                    \draw [edge,grisOscuro] (\i) to (\j);
                \end{scope}
                    
        \end{tikzpicture}
        \caption{La gr\'afica vac\'ia de 3 v\'ertices (izquierda) y $K_5$
        (derecha).}
        \label{fig:ex-va-comp}
    \end{figure} 

Una familia de gr\'aficas muy estudiada en Teor\'ia de Gr\'aficas es la familia
de las gr\'aficas $k$-partitas. Antes de definir esta familia, es importante
se\~{n}alar que una \indiceSub{v\'ertices}{partici\'on}\index{partici\'on} de
$V(G)$ en $V_1, V_2, \dots, V_k$ partes debe satisfacer $V_i \cap V_j =
\varnothing$, para $i \ne j$ y $V_1 \cup V_2 \cup \cdots \cup V_k = V(G)$, sin
embargo, no pedimos que cada parte sea no vac\'ia. Estamos listos para definir
la familia de gr\'aficas. Una gr\'afica es
$k$\textbf{-partita}\index{gr\'afica!k @$k$-partita} si su conjunto de
v\'ertices admite una partici\'on en, a lo m\'as, $k$ conjuntos independientes.
Un caso muy relevante, el cu\'al se utiliza en este trabajo, es cuando $k=2$. En
este caso se dice que es una \textbf{gr\'afica}
\indiceSub{gr\'afica}{bipartita}\index{bipartita!gr\'afica}. Una gr\'afica
bipartita $G$ con partici\'on $(X,Y)$ se denota $G[X,Y]$, y $(X,Y)$ es una
\indice{bipartici\'on} de $G$. Una propiedad de suma utilidad se demuestra en
\cref{prop:bip-compConect}. 
\begin{proposicion}
\label{prop:bip-compConect}
    Una gr\'afica $G$ es bipartita si y s\'olo si cada componente conexa de $G$
    es bipartita.
\end{proposicion}

\begin{proof}
    En el caso en el que el que $G$ es una gr\'afica conexa, tenemos que s\'olo
    tiene una componente conexa, ella misma, por lo que la proposici\'on se
    cumple de manera trivial. Por ello, nos enfocamos en el caso en el que $G$
    es inconexa, i.~e., $G$ tiene al menos dos componentes conexas. Sean $G_1,
    \dots, G_n$ las componentes conexas de $G$, con $n \geq 2$. Empezamos
    suponiendo que $G$ es una gr\'afica bipartita con bipartici\'on $(X,Y)$.
    Sean $X_i, Y_i\subseteq V(G_i)$, tales que $X_i \subseteq X$ y $Y_i
    \subseteq Y$, para $i \in \{1, \dots, n\}$. Afirmamos que $(X_i,Y_i)$ es una
    partici\'on, para cada $G_i$. Tenemos que $X_i \cap Y_i = \varnothing$, pues
    $X \cap Y = \varnothing$ por definici\'on de partici\'on. Adem\'as, de
    manera an\'aloga, tenemos que $X_i \cup Y_i = V(G_i)$.  Luego, sean $a, b
    \in X_i$, como $X_i \subseteq X$, entonces tenemos que $a,b \in X$. Por
    ende, $a$ y $b$ no son adyacentes. An\'alogamente, si $a,b \in Y_i$, tenemos
    que $a$ y $b$ no son adyacentes. Por lo tanto, $(X_i, Y_i)$ es una
    partici\'on de $G_i$, en otras palabras, $G_i$ es una gr\'afica bipartita.

    Nos enfocamos en la segunda implicaci\'on, es decir, si $G_i$ es una
    gr\'afica bipartita, para $i \in \{1, \dots, n\}$, entonces $G$ tambi\'en lo
    es. Proponemos la bipartici\'on $(X,Y)$ para $G$ donde $X =
    \bigcup\limits_{i} X_i$ y $Y =\bigcup\limits_{i}  Y_i$. Nos fijamos en que
    $X \cup Y = \bigcup \limits_{i}  V(G_i)=V(G)$. De manera an\'aloga $X \cap Y
    = \varnothing$. Por \'ultimo, dados $u,v \in X$ tenemos que $u \in X_i$ y $v
    \in X_j$, para algunos $i,j \in \{1, \dots,n\}$. Si $i \neq j$, entonces los
    v\'ertices est\'an en distintas componentes conexas, por lo que no son
    adyacentes. Si $j=i$, entonces tenemos que $u,v \in X_i$, por lo que tampoco
    son adyacentes. An\'alogamente si tomamos a $u,v \in Y$. Con esto tenemos
    que $(X,Y)$ es una bipartici\'on de $G$; en otras palabras, $G$ es
    bipartita.
\end{proof}

La Proposici\'on\ref{prop:bip-compConect} nos permite estudiar a las gr\'aficas
bipartitas de manera m\'as general, pues afirma que no dependemos de si una
gr\'afica bipartita es conexa o no.


Existe un caso especial de las gr\'aficas $k$-partitas, es cuando todas las
aristas posibles est\'an presentes. Por cuesti\'on de utilidad para el trabajo,
definimos este comportamiento de manera formal \'unicamente para las gr\'aficas
bipartitas. Una gr\'afica es \indiceSub{gr\'afica}{bipartita
completa}\index{bipartita!completa} si es bipartita y, dada la partici\'on $X$ y
$Y$ de sus v\'ertices, cada v\'ertice de $X$ es adyacente a cada v\'ertice de
$Y$. En \cref{fig:ex-Bip} se muestra un ejemplo de una gr\'afica bipartita y una
gr\'afica bipartita completa, en ambas resaltando de diferentes colores cada
conjunto de la partici\'on. 

\begin{figure}[ht!]
    \centering
        \begin{tikzpicture}
        
            \begin{scope}[xshift=0cm,scale=0.8]

                \foreach \i in {1,3,5} \draw ({(360/6)*\i}:2)
                    node(\i)[vertex, fill=coral, label=(360/6)*\i:{\large $v_{\i}$}]{};
                
                \foreach \i in {2,4,6} \draw ({(360/6)*\i}:2)
                    node(\i)[vertex, fill=azulCielo, label=(360/6)*\i:{\large $v_{\i}$}]{};
    
                \foreach \i/\j in {1/2,2/3,3/4,4/5,5/6,6/1}
                    \draw [edge,grisOscuro!50] (\i) to (\j);
                \end{scope}
        
            \begin{scope}[xshift=6cm,scale=0.8]

                \foreach \i in {1,3} \draw ({(360/4)*\i}:2.3)
                    node(\i)[vertex, fill=morado, label=(360/4)*\i:{\large $v_{\i}$}]{};
                
                \foreach \i in {2,4} \draw ({(360/4)*\i}:2.3)
                    node(\i)[vertex, fill=verde, label=(360/4)*\i:{\large $v_{\i}$}]{};
               
                \draw (0,0) node (5) [vertex, fill=verde, label=300:{\large $v_5$}] {};

                \foreach \i/\j in {1/4,1/5,1/2,2/3,3/4,3/5}
                    \draw [edge,grisOscuro!50] (\i) to (\j);
                \end{scope}
                    
        \end{tikzpicture}
        \caption{El $6$-ciclo como ejemplo de una gr\'afica bipartita
        (izquierda), y una gr\'afica bipartita completa (derecha).}
        \label{fig:ex-Bip}
    \end{figure}


Un teorema muy importante al hablar de gr\'aficas bipartitas es
\cref{teo:bip-CImpar}, que sirve como caracterizaci\'on para esta familia de
gr\'aficas. Esto tambi\'en se puede observar en el ejemplo de \cref{fig:ex-Bip}.

\begin{teorema}
    \label{teo:bip-CImpar}
    Una gr\'afica $G$ es bipartita si y s\'olo si no contiene ciclos de longitud
    impar.
\end{teorema}

\begin{proof}
    Por \cref{prop:bip-compConect}, podemos suponer que $G$ es conexa. Empezamos
    considerando $G$ una gr\'afica bipartita con partici\'on $(X,Y)$. Buscamos
    probar que $G$ no tiene ciclos impares. Para esto, tomamos
    $C=(v_1,v_2,\dots, v_k,v_1)$ un ciclo en $G$. Sin p\'erdida de generalidad,
    consideramos $v_1 \in X$, entonces tenemos que $v_2 \in Y$, pues $G$ es
    bipartita. Inductivamente, tenemos que $v_i \in X$ y $v_j \in Y$ si y s\'olo
    si $i$ es impar y $j$ es par. Adem\'as, tenemos que $v_k \in Y$ pues $v_k
    v_1 \in E(G)$ y $v_1 \in X$. Por lo tanto, $k$ es par, i.~e., $C$ tiene
    longitud par. En consecuencia $G$ no contiene ciclos impares.

    Ahora pasamos a la segunda implicaci\'on. Para esto, consideramos $G$ una
    gr\'afica que no contiene ciclos impares. Dado un v\'ertice $v \in V(G)$,
    sea  $G_1$ la componente conexade $G$, tal que $v \in V(G_1)$. Construimos
    la siguiente partici\'on de $V(G_1)$. Sea $X$ al conjunto de los v\'ertices
    de $G_1$ que est\'en a distancia impar de $v$, en otras palabras, $X = \{u
    \in V(G_1) \colon d(u,v) \textnormal{ es impar}\}$. Tomamos al otro conjunto
    de la partici\'on como $Y = V(G_1)\setminus X$. Por construcci\'on, tenemos
    que $X \cap Y = \varnothing$ y $X \cup Y = V(G_1)$. S\'olo falta verificar
    que no existen aristas entre cualesquiera dos v\'ertices de $X$ o entre
    cualesquiera dos v\'ertices de $Y$. Ambos casos tienen demostraciones
    an\'alogas, por lo que s\'olo lo demostramos para $X$. Supongamos que
    existen $u,w \in X$, tales que $e=uw \in E(G_1)$. Sean $P$ una
    $vu$-trayectoria y $Q$ una $vw$-trayectoria, ambas de longitud m\'inima.
    Ahora, sea $z$ el \'ultimo v\'ertice que comparten estas trayectorias.
    Observemos que las subtrayectorias $v \xrightarrow[P]{}z$ y $v
    \xrightarrow[Q]{}z$ tienen la misma longitud. Esto pues, si alguna tuviera
    menor longitud, sin p\'erdida de generalidad, tomamos a $v
    \xrightarrow[P]{}z$ con menor longitud, entonces tendr\'iamos que la
    trayectoria $v \xrightarrow[P]{}z\xrightarrow[Q]{}w$ tiene longitud menor a
    $d(v,w)$, lo cu\'al es una contradicci\'on. Como $P$ y $Q$ tienen longitud
    par y $v \xrightarrow[P]{}z$ y $v \xrightarrow[Q]{}z$, tienen la misma
    longitud, entonces tenemos que $z \xrightarrow[P]{}u$ y $z
    \xrightarrow[Q]{}w$ tiene ambas longitud par o tienen ambas longitud impar.
    Ahora consideramos la trayectoria $ R= u \xrightarrow[P^{-1}]{}z
    \xrightarrow[Q]{}w$, por el argumento anterior sabemos que esta trayectoria
    tiene longitud par. Por \'utimo, observamos que existe el siguiente ciclo,
    $C=u\xrightarrow[R]{}w\xrightarrow[e]{}u$. Tenemos que $C$ es un ciclo de
    longitud impar, lo cu\'al es una contradicci\'on. Por lo tanto, no existen
    aristas entre v\'ertices de $X$. De manera an\'aloga podemos ver que $Y$
    tampoco tiene v\'ertices adyacentes, As\'i $G_1$ es una gr\'afica bipartita
    con bipartici\'on $(X,Y)$. Por lo tanto, todas las componentes conexas de
    $G$ son bipartitas, en el caso de que $G$ sea inconexa. Por ende, usando
    \cref{prop:bip-compConect}, tenemos que $G$ es bipartita.
\end{proof}
    
Otra  familia de gr\'aficas que se estudia m\'as adelante en este trabajo son
los abanicos. Un \indice{abanico} $\mathcal{F}_n$ es la gr\'afica obtenida de la
uni\'on completa de $K_1$ y $P_{n-1}$, donde los primeros $n-1$ v\'ertices de
$\mathcal{F}_n$ son los v\'ertices pertenecientes a $P_{n-1}$ y el $n$-\'esimo
v\'ertice de la gr\'afica es el que le corresponde a $K_1$. Al igual que en
varios casos, existe una generalizaci\'on de esta estructura al sustituir $K_1$
por $\overline{K_n}$. Un \textbf{abanico} \indiceSub{abanico}{generalizado},
denotado $\mathcal{F}_{m,n}$, se define como $\mathcal{F}_{m,n}=\overline{K_m}
\oplus P_n$. Por simplicidad, llamamos $v_i \in V(\mathcal{F}_{m,n})$ y $w_j \in
V(\mathcal{F}_{m,n})$ a los v\'ertices que provienen de $P_n$ y
$\overline{K_m}$, respectivamente, donde $i \in \{1, \dots, n\}$ y $j \in \{1,
\dots, m\}$. En \cref{fig:ex-abanico} se puede ver un ejemplo de un abanico del
lado izquierdo y un abanico generalizado del lado derecho.

\begin{figure}[ht!]
    \centering
        \begin{tikzpicture}

        \begin{scope}[xshift=0cm,scale=1]
            \foreach \i in {1,...,5} \draw ({(360/6)*\i}:1.8)
                    node(\i)[vertex, label=(360/6)*\i:{\large $v_{\i}$}]{};

            \draw (0,0) node (6) [vertex,label=0:{\small $v_6$}] {};

            \foreach \i/\j in{1/2,2/3,3/4,4/5,1/6,2/6,3/6,4/6,5/6} 
                \draw [edge,grisOscuro] (\i) to (\j);
        \end{scope}
            
        \begin{scope}[xshift=7cm,scale=1]
            \draw (-2.6,1) node (v1) [vertex,label=180:{\small $v_1$}] {};
            \draw (-1.3,1.6) node (v2) [vertex,label=135:{\small $v_2$}] {};
            \draw (0,2) node (v3) [vertex,label=90:{\small $v_3$}] {}; 
            \draw (1.3,1.6) node (v4) [bvertex,label=45:{\small $v_4$}] {}; 
            \draw (2.6,1) node (v5) [vertex,label=0:{\small $v_5$}] {}; 
            \draw (-1.2,-1.5) node (w1) [vertex,label=270:{\small $w_1$}] {}; 
            \draw (1.2,-1.5) node (w2) [bvertex, label=270:{\small $w_2$}] {};

            
            \foreach \i/\j in{w1/v1,w1/v2,w1/v3,w1/v4,w1/v5,w2/v1,w2/v2,w2/v3,
                w2/v4,w2/v5,v1/v2,v2/v3,v3/v4,v4/v5} 
            \draw [edge,grisOscuro] (\i) to (\j);
        \end{scope}
\end{tikzpicture}
\caption{Las gr\'aficas $\mathcal{F}_6$ (izquierda) y $\mathcal{F}_{2,5}$
(derecha).}
\label{fig:ex-abanico}
\end{figure}

La \'ultima familia de gr\'aficas que definimos en esta secci\'on es la familia
de gr\'aficas hamiltonianas. Para definir esta familia, es preciso definir
algunos conceptos necesarios. El primer concepto es el de trayectoria
hamiltoniana. Una \textbf{trayectoria}
\indiceSub{trayectoria}{hamiltoniana}\index{hamiltoniana!trayectoria} es una
trayectoria generadora, i.~e., una trayectoria cuyo conjunto de v\'ertices es el
conjunto de v\'ertices de la gr\'afica. Asimismo, un \textbf{ciclo}
\indiceSub{ciclo}{hamiltoniano}\index{hamiltoniana!ciclo} es un ciclo generador.
Por \'ultimo, una \indice{subtrayectoria
hamiltoniana}\index{hamiltoniana!subtrayectoria} es la trayectoria hamiltoniana
resultante de quitarle una arista a un ciclo hamiltoniano. Definimos una
\textbf{gr\'afica}
\indiceSub{gr\'afica}{hamiltoniana}\index{hamiltoniana!gr\'afica} como una
gr\'afica que contiene un ciclo hamiltoniano. En \cref{fig:ex-hamilt} se muestra
un ejemplo de una gr\'afica que contiene una trayectoria hamiltoniana del lado
izquierdo y de una gr\'afica hamiltoniana del lado izquierdo, respectivamente.

\begin{figure}[ht!]
    \centering
        \begin{tikzpicture}
        
            \begin{scope}[xshift=0cm,scale=0.7]

                \foreach \i in {1,...,3}
                 \draw ({(360/3)*\i-90}:0.7)
                    node(\i)[vertex, label=(360/3)*\i-90:{$v_{\i}$}]{};
                
                \draw (0,3) node (4) [vertex, label=90:{\large $v_4$}] {};
                \draw (3.1,-1.5) node (5) [vertex, label=0:{\large $v_5$}] {};
                \draw (-3.1,-1.5) node (6) [vertex, label=180:{\large $v_6$}] {};
                
                \foreach \i/\j in {4/5,2/3,3/6,1/4}
                    \draw [edge,grisOscuro!50] (\i) to (\j);
                
                \foreach \i/\j in {1/3,3/5,4/6,2/4,5/6}
                    \draw [wedge,verde!110] (\i) to (\j);
                \end{scope}
        
            \begin{scope}[xshift=7.5cm,scale=0.7]

                \foreach \i in {1,...,3}
                 \draw ({(360/3)*\i-90}:0.7)
                    node(\i)[vertex, label=(360/3)*\i-90:{$v_{\i}$}]{};
                
                \draw (0,3) node (4) [vertex, label=90:{\large $v_4$}] {};
                \draw (3.1,-1.5) node (5) [vertex, label=0:{\large $v_5$}] {};
                \draw (-3.1,-1.5) node (6) [vertex, label=180:{\large $v_6$}] {};
                
                \foreach \i/\j in {1/2,2/3,4/5,6/4,1/5,3/6}
                    \draw [edge,grisOscuro!50] (\i) to (\j);
                
                \foreach \i/\j in {2/4,2/6,5/6,3/5,1/3,1/4}
                    \draw [wedge,azulCielo] (\i) to (\j);
                \end{scope}
                    
    \end{tikzpicture}
    \caption{Una trayectoria hamiltoniana (izquierda) y un ciclo hamiltoniano
    (derecha).}
    \label{fig:ex-hamilt}
\end{figure}

\newpage

Una proposici\'on que nos servir\'a m\'as adelante es \cref{prop:hamilt},
demostrada a continuaci\'on.

\begin{proposicion}
\label{prop:hamilt}
   Sea $G$ una gr\'afica y sea $S \subseteq V(G)$ un subconjunto no vac\'io. Si
    $G$ es una gr\'afica hamiltoniana, entonces, $G-S$ tiene, a lo m\'as, $|S|$
    componentes conexas.
\end{proposicion}


\begin{proof}
    Sea $G$ una gr\'afica con ciclo hamiltoniano $C= (v_1, v_2, \dots, v_n,
    v_1)$ y sea $S \subseteq V(G)$ un subconjunto no vac\'io. Si $G-S$ es
    conexa, entonces se cumple la proposici\'on, pues $|S| \geq 1$ por
    definici\'on.
    
    Suponemos ahora que $G-S$ es inconexa. Sea $k$ el n\'umero de componentes
    conexas de $G-S$ y sean $G_1, G_2, \cdots, G_ k$ dichas componentes conexas,
    con $k < n$. Nombramos $u_i$ al \'ultimo v\'ertice de $C$ que est\'a en
    $G_i$, para $i \in \{1, \dots, k\}$. Luego, sea $w_i$ el v\'ertice que sigue
    de $u_i$ en $C$. Observamos que $w_i \notin V(G_i)$ por como se defini\'o.
    Adem\'as, $w_i \notin V(G_j)$, con $j \in \{1, \dots, k\}$ y $j \neq i$, por
    definici\'on de componentes conexas. Por lo tanto, tenemos que $w_i$ es
    elemento de $S$. Adem\'as, notamos que, al ser $C$ un ciclo, tenemos que
    $w_i \neq w_j$, para diferentes $i,j \in \{1, \dots, k\}$. Por lo tanto,
    tenemos que $\{w_1, \dots, w_k\} \subseteq S$, lo que implica que $k \leq
    |S|$. 
\end{proof} 

\section{Coloraci\'on}
\label{sec:coloracion}

En Teor\'ia de Gr\'aficas, cuando hablamos de colorear a una gr\'afica nos
referimos a una manera de etiquetar a todos sus v\'ertices o a todas sus
aristas. Primero, veamos la coloraci\'on por aristas. Formalmente, una
\textbf{$k$-coloraci\'on} \indiceSub{$k$-coloraci\'on}{por
aristas}\index{aristas!coloraci\'on} de una gr\'afica $G$, para un entero $k$,
es una funci\'on $c' \colon E(G)\to S$, con $S$ un conjunto de cardinalidad $k$.
La \indice{clase crom\'atica} de un color $i$ en una coloraci\'on por aristas
$c'$ es el conjunto de todas las aristas de la gr\'afica que est\'an coloreadas
del color $i$, es decir, $c'^{-1}[i]$. Notamos que las clases crom\'aticas
pueden ser vac\'ias. Una \textbf{coloraci\'on}
\indiceSub{k@$k$-coloraci\'on}{propia por aristas}\index{aristas!k
@$k$-coloraci\'on propia} es una coloraci\'on por aristas, donde se le asignan
colores diferentes a aristas adyacentes.

Por otro lado, una una \textbf{$k$-coloraci\'on}
\indiceSub{k@$k$-coloraci\'on}{por v\'ertices}\index{v\'ertices!coloraci\'on} de
una gr\'afica $G$, para un entero $k$, es una funci\'on $c \colon V(G)\to T$,
donde $T$ es un conjunto de cardinalidad $k$. En ambos casos, ya sea
coloraci\'on por v\'ertices o por aristas, a los elementos del codominio de la
funci\'on se les llama \indice{colores}. Una \textbf{coloraci\'on}
\indiceSub{k@$k$-coloraci\'on}{propia por
v\'ertices}\index{v\'ertices!k $k$-coloraci\'on propia} es una coloraci\'on por
v\'ertices, donde se le asignan colores diferentes a v\'ertices
adyacentes. Decimos que una gr\'afica $G$ es $k$\textbf{-coloreable}
\index{gr\'afica!k @$k$-coloreable}, con $k \in \mathbb{Z}$, si $G$ admite
una $k$-coloraci\'on propia por v\'ertices. A continuaci\'on, en
\cref{fig:ex-color-prop}, se muestra una gr\'afica $G$, donde se resalta una
coloraci\'on propia por v\'ertices del lado izquierdo y una coloraci\'on propia
por aristas del lado derecho. 

La \indice{clase crom\'atica} de un color $i$ en una coloraci\'on por v\'ertices
$c$ es el conjunto de todos los v\'ertices de la gr\'afica que est\'an
coloreados del color $i$, es decir, $c^{-1}[i]$. Notamos que las clases
crom\'aticas pueden ser vac\'ias. Adem\'as, en una coloraci\'on propia, cada
clase crom\'atica es un conjunto independiente. El m\'inimo entero $k$ para el
cu\'al una gr\'afica $G$ es $k$-coloreable es el \indice{n\'umero crom\'atico}
de $G$, denotado por $\chi(G)$. En este caso, decimos que $G$ es
$k$\textbf{-crom\'atica} \index{gr\'afica!k @$k$-crom\'atica}. En
\cref{fig:ex-color-prop}, se muestra una $3$-coloraci\'on propia por v\'ertices,
sin embargo, el n\'umero crom\'atico de la gr\'afica es $2$. 


\begin{figure}[ht!]
    \centering
    \begin{tikzpicture}
    
        \begin{scope}[xshift=-8cm,scale=0.8]
            \foreach \i in {2,5} 
                \draw ({(360/5)*\i}:2.5) node(\i)[vertex, fill=coral,
                label=(360/5)*\i:{\large
                $v_{\i}$}]{};

            \foreach \i in {1,3,4}
                \draw ({(360/5)*\i}:2.5) node(\i)[vertex, fill=azulCielo,
                label=(360/5)*\i:{\large $v_{\i}$}]{};
            
            \draw (0,0) node (6) [vertex, fill=amarillo, label=35:{\large $v_6$}] {};
           
            \foreach \i/\j in {1/6,2/6,3/6,4/6,5/6}
                \draw [edge,grisOscuro] (\i) to (\j);
            \end{scope}

        \begin{scope}[xshift=0cm,scale=0.8]
            \foreach \i in {1,...,5}
                \draw ({(360/5)*\i}:2.5) node(\i)[vertex,
                label=(360/5)*\i:{\large $v_{\i}$}]{};
    
            \draw (0,0) node (6) [vertex, label=35:{\large $v_6$}] {};
               
            \draw[ edge] [wedge,coral] (1) to (6);
            \draw[ edge] [wedge,azulCielo] (2) to (6);
            \draw[ edge] [wedge,amarillo] (3) to (6);
            \draw[ edge] [wedge,morado] (4) to (6);
            \draw[ edge] [wedge,verde] (5) to (6);
        \end{scope}

    \end{tikzpicture}
    \caption{Una $3$-coloraci\'on propia por v\'ertices (izquierda) y una
        $5$-coloraci\'on propia por aristas (derecha).}
        \label{fig:ex-color-prop}
\end{figure}

Es f\'acil observar que una gr\'afica es $1$-crom\'atica si y s\'olo si es
vac\'ia, pues, de tener alguna arista ser\'ia necesario utilizar un segundo
color. Tambi\'en podemos determinar c\'omo se ven las gr\'aficas
$2$-crom\'aticas. Para esto, notamos que, al tener dos clases de colores de una
coloraci\'on propia, garantizamos que la gr\'afica es una gr\'afica bipartita.
Adem\'as, todas las gr\'aficas bipartitas cumplen ser $2$-crom\'aticas,
asign\'andole un color a cada conjunto de la partici\'on.

Algo que es interesante de notar es que, si $H$ es una subgr\'afica de $G$,
entonces $\chi(H) \leq \chi(G)$; m\'as a\'un, si $H$ es una gr\'afica completa
de $n$ v\'ertices, entonces tenmos $n \leq \chi(G)$. Sin embargo, la suma del
n\'umero crom\'atico de dos subgr\'aficas puede ser mayor que el n\'umero
crom\'atico de la gr\'afica. A continuaci\'on se demuestra este resultado.

\begin{proposicion}
    Sea $G$ una gr\'afica.   Si $S \subseteq V(G)$ y $S \neq \varnothing$,
    entonces se cumple la siguiente desigualdad
    \[
        \chi(G) \leq \chi(G[S])+\chi(G-S).
    \] 
\end{proposicion}

\begin{proof}
    Sean $G$ una gr\'afica y $S \subseteq V(G)$, con $S \neq \varnothing$.  Sea
    $c_1$ una coloraci\'on propia de $G[S]$ con $ k= \chi(G[S])$ colores,
    tomando el conjunto de colores $\{1, \dots, k\}$. An\'alogamente, sea $c_2$
    una coloraci\'on propia de $G-S$ con $l= \chi(G-S)$ colores, tomando el
    siguiente conjunto con colores $\{k+1, k+2, \dots, k+l\}$. Observamos que
    $c_1$ y $c_2$ no comparten colores, por lo que podemos construir una
    coloraci\'on propia de $G$ con los colores $\{1, \dots, k+l\}$,
    asign\'andole los colores de $c_1$ y $c_2$ a los v\'ertices de $G$. De esta
    manera tenemos que $G$ es $(k+l)$-coloreable, donde $k = \chi(G[S])$ y $l =
    '\chi(G-S)$. Por lo tanto, tenemos que $\chi(G) \leq \chi(G[S])+\chi(G-S)$.
\end{proof}

Por \'ultimo, vale la pena mencionar la relaci\'on entre el n\'umero crom\'atico
de una gr\'afica y el n\'umero de clan de \'esta, i.~e., $\omega$. Como los
clanes cumplen que todos sus v\'ertices son adyacentes entre s\'i, entonces la
clase crom\'atica de una gr\'afica $G$ tiene que ser al menos el n\'umero de
clan de $G$, de esta manera tenemos que $\omega(G) \leq \chi(G)$. 

Con \'esto se concluyen los conceptos de Teor\'a de Gr\'aficas que se necesitan
en este trabajo, por lo que pasamos a estudiar las gr\'aficas de fichas, el
enfoque principal de este trabajo.