\chapter{Definiciones de Teor\'ia de Gr\'aficas}%
\label{cap:defs grafs}

\section{Definiciones b\'asicas}%
\label{sec:def-basicas}

El objeto a estudiar en el \'area de Teor\'ia de Gr\'aficas es, naturalmente,
una gr\'afica. Una \indice{gr\'afica} $G$ es una pareja ordenada de conjuntos
finitos $(V(G), E(G))$, donde $V(G)$ es no vac\'io y $E(G) \subseteq
\binom{V(G)}{2}$. Los elementos de $V(G)$ son llamados \indice{v\'ertices} y los
elementos de $E(G)$ son \indice{aristas}. Para una arista $e$ y v\'ertices $u,
v$ de $G$, decimos que $e$ es \indiceSub{aristas}{incidente} en $u$ y en $v$ si
$e= \{u, v\}$. A su vez, los v\'ertices $u$ y $v$ son extremos de la arista $e$.
De igual manera, podemos decir que $u$ y $v$ son v\'ertices \indice{adyacentes}
o vecinos y lo denotamos $u \sim v$. Esta relaci\'on entre v\'ertices la
llamamos adyacencia. Al conjunto de vecinos de un v\'ertice $v$ se le llama
\indice{vecindad}, y se denota $N_G(v)$, mientras que al n\'umero de aristas
incidentes en $v$ se le llama \indice{grado} de $v$, $d(v)$. Un v\'ertice
\indiceSub{v\'ertices}{aislado}, es decir, un v\'ertice que no tiene vecinos, tiene grado $0$.

\begin{figure}[ht!]
    \centering
       \begin{tikzpicture}
        
            \begin{scope}[xshift=6cm,scale=0.8]
                \draw (0.5,-1.5) node (1) [vertex, label=270:{\large $v_1$}] {};
                \draw (-1.9,-1.6) node (2) [vertex, label=270:{\large $v_2$}] {};
                \draw (1.2,1.5) node (3) [vertex, label=90:{\large $v_3$}] {};
                \draw (-2,0.5) node (4) [vertex, label=90:{\large $v_4$}] {};
                \draw (2.5,-0.6) node (5) [vertex, label=270:{\large $v_5$}] {};
                \draw (0,0) node (6) [vertex, label=90:{\large $v_6$}] {};
                \draw (3,1) node (7) [vertex, label=90:{\large $v_7$}] {};
                
                \foreach \i/\j in {1/3,1/4,1/6,2/6,3/4,3/5,5/6}
                \draw [edge] (\i) to (\j);
            \end{scope}
    
        \begin{scope}[xshift=0cm,scale=0.6]
            \draw ({(360/4)*3}:2) node(1)[vertex, label=270:{\large $v_1$}]{};
            \draw (0,0) node (2) [vertex, label=90:{\large $v_2$}] {};
            \draw ({(360/4)*2}:2) node(3)[vertex, label=180:{\large $v_3$}]{};
            \draw (-2.5,-2) node (4) [vertex, label=270:{\large $v_4$}] {};
            \draw ({(360/4)*1}:2) node(5)[vertex, label=90:{\large $v_5$}]{};
            \draw ({(360/4)*0}:2) node(6)[vertex, label=0:{\large $v_6$}]{};
            \draw (-2.5,2) node (7) [vertex, label=90:{\large $v_7$}] {};

            \foreach \i/\j in {1/3,1/4,1/6,2/6,3/4,3/5,5/6}
                \draw [edge] (\i) to (\j);  
            \end{scope}
                
    \end{tikzpicture}
    \caption{Diferentes diagramas de una gr\'afica $G$}
    \label{fig:diagGraf}
\end{figure}

Las gr\'aficas pueden ser representadas por \indice{diagramas} donde los
v\'ertices son c\'irculos y las arista son l\'ineas cuyos extremos son sus
v\'ertices incidentes. Cabe recalcar que una gr\'afica puede tener m\'ultiples
diagramas, como se muestra en \cref{fig:diagGraf}. Es f\'acil observar que en
ambos diagr\'amas de $G$ los v\'ertices tienen las mismas propiedades, por
ejemplo que $N_G(v_3)=\{v_1,v_4,v_5\}$ o que el grado de $v_2$ es $1$.   Esto se
debe a que la gr\'afica es un objeto matem\'atico, independiente del diagrama.


Al momento de estudiar objetos matem\'aticos, siempre es importante analizar
como se relacionan dichos objetos entre s\'i. Ya teniendo la noci\'on de una
gr\'afica, ahora nos enfocamos en c\'omo se ven las relaciones entre gr\'aficas.
Veamos que necesitar\'ian dos gr\'aficas para ser ``iguales''. Observamos que,
de igual manera que una gr\'afica puede tener varios diagramas, a un diagrama se
le pueden asociar distintas gr\'aficas, renombrando v\'ertices y aristas, para
obtener distintos conjuntos de v\'ertices y aristas. Un ejemplo de esto es
\cref{fig:isoGraf}. Al renombrar las aristas de la gr\'afica $G$, podemos
dibujar a $G$ como la gr\'afica $H$ y viseversa, usando como gu\'ia los colores
que se muestran en la figura. De manera intuitiva, las dos gr\'aficas que
comparten un mismo diagrama podr\'iamos decir que son ``iguales'' hasta donde
nos interesa, pues, al compartir un diagrama, comparten muchas propiedades como
el n\'umero de v\'ertices, las adyacencias o el grado de los v\'ertices. Estas
g\'aficas se dice que son isomorfas. Formalmente hablando, dos gr\'aficas $G$ y
$H$ son \indiceSub{gr\'afica}{isomorfas} si exite una biyecci\'on $\theta: V(G)
\rightarrow V(H)$ tal que para cualesquiera dos v\'ertices $u, v \in V(G)$, se
cumple que $uv \in E(G)$ si y s\'olo si $\theta(u)\theta(v) \in E(H)$. Es decir,
la biyecci\'on $\theta$ preserva adyacencias y no adyacencias. Cuando $G$ es
ismorfa a $H$ lo denotamos $G \cong H$ y la biyecci\'on entre las gr\'aficas es
llamada isomorfismo.

\begin{figure}[ht!]
    \centering
       \begin{tikzpicture}
    
            \begin{scope}[xshift=0cm,scale=0.8]
             \draw ({(360/5)*1}:2) node(1)[vertex,fill=morado, label=(360/5)*1:{\large
                    $v_{1}$}]{};
            \draw ({(360/5)*2}:2) node(2)[vertex,fill=amarillo, label=(360/5)*2:{\large
                    $v_{2}$}]{};
           \draw ({(360/5)*3}:2) node(3)[vertex,fill=azulCielo, label=(360/5)*3:{\large
                    $v_{3}$}]{};
            \draw ({(360/5)*4}:2) node(4)[vertex,fill=coral, label=(360/5)*4:{\large
                    $v_{4}$}]{};
            \draw ({(360/5)*5}:2) node(5)[vertex,fill=verde, label=(360/5)*5:{\large
                    $v_{5}$}]{};

            \foreach \i/\j in {5/1,1/2,2/3,3/4,4/5}
                \draw [edge] (\i) to (\j);
            \end{scope}
    
        \begin{scope}[xshift=6cm,scale=0.8]
            \draw ({(360/5)*1}:2) node(1)[vertex,fill=morado, label=(360/5)*1:{\large
            $v_{1}$}]{};
    \draw ({(360/5)*2}:2) node(2)[vertex,fill=coral, label=(360/5)*2:{\large
            $v_{2}$}]{};
   \draw ({(360/5)*3}:2) node(3)[vertex,fill=amarillo, label=(360/5)*3:{\large
            $v_{3}$}]{};
    \draw ({(360/5)*4}:2) node(4)[vertex,fill=verde, label=(360/5)*4:{\large
            $v_{4}$}]{};
    \draw ({(360/5)*5}:2) node(5)[vertex,fill=azulCielo, label=(360/5)*5:{\large
            $v_{5}$}]{};

            \foreach \i/\j in {5/2,2/4,4/1,1/3,3/5}
                \draw [edge] (\i) to (\j);
            \end{scope}
                
    \end{tikzpicture}
    \caption{Dos gr\'aficas isomorfas, $G$ y $H$.}
    \label{fig:isoGraf}
\end{figure}

Al momento de ver las relaciones entre dos gr\'aficas, tambi\'en es importante
analizar el caso en el que una gr\'afica este ``contenida'' en la otra. En el
caso de conjuntos nos referir\'iamos a subconjunto y en el caso de gr\'aficas
nos vamos a referir a subgr\'aficas. Sean $G$ y $H$ tales que $V(H) \subseteq
V(G)$ y $E(H) \subseteq E(G)$. Entonces decimos que $H$ es una
\indice{subrgr\'afica} de $G$. Por otra parte, decimos que $G$ es una
\indice{supergr\'afica} de $H$. Esta relaci\'on la denotamos con la misma
notaci\'on que para subconjuntos, es decir $H \subseteq G$. Un concepto que se
deriva de las subgr\'aficas parte de fijarnos en la subgr\'afica que tiene la
mayor cantidad de aristas posibles. Formalmente hablando, nos referimos a $H$,
la subgr\'afica de $G$, cuyo conjunto de v\'ertices es $S \subseteq V(G)$ y cuyo
conjunto de aristas son aquellas aristas de $G$ que tienen ambos extremos en
$S$, es decir, $E(H) = \{uv \in E(G) \colon\ u,v \in S\}$. A esta subgr\'afica
la llamamos la \textbf{subgr\'afica} \indiceSub{subgr\'afica}{inducida} de $G$
por $S$ y la denotamos $G[S]$. Notemos que esta gr\'afica $G[S]$ es la
subgr\'afica de $G$ con conjunto de v\'ertices $S$ que m\'as se parece a $G$.
Otro ejemplo importante de subgr\'aficas es la subgr\'afica inducida obtenida al
quitarle uno o m\'as v\'ertices a una gr\'afica. En otras palabras, sea $S
\subset V(G)$, nombramos $H \subseteq G$ a la gr\'afica con $V(H)=V(G) \setminus
S$ y $E(H) = E(G)\setminus\{uv \in E(G) \colon\ u \in S \lor v \in S\}$. Tenemos
que $H$ es la subg\'afica obtenida a partir de $G$ al quitarle el conjunto $S$,
esta subgr\'afica se denota $G-S$. Un caso a se\~{n}alar es cuando $S$ es un
conjunto unitario, es decir, el caso en el que le quitamos un solo v\'ertice $v$
a $G$. En este caso escribiremos $G-v$ en vez de $G-\{v\}$. A continuaci\'on se
muestra un ejemplo de subgr\'aficas en \cref{fig:subgraf}, donde
${\color{morado}\bf H}$ es una subgr\'afica inducida con $V(H) =
\{v_1,v_2,v_3,v_4\}$ y ${\color{azulCielo}\bf H'}$ es una subgr\'afica con
$V(H')=\{v_5,v_6,v_7\}$ y $E(H')=\{v_5v_6,  v_6v_7\}$. Notamos que a
${\color{azulCielo}\bf H'}$ le falta la arista ${\color{grisOscuro!80}\bf
v_5v_7}$, resaltada en la figura, para ser una subgr\'afica inducida. 


%Como hacer que aparezcan letras con acento usando \color
\begin{figure}[ht!]
    \centering
       \begin{tikzpicture}
    
            \begin{scope}[xshift=0cm,scale=1]
                \foreach \i in {1,...,4} 
                    \draw ({(360/7)*\i}:2) node(\i)[vertex, fill=morado,
                    label=(360/7)*\i:{\large $v_{\i}$}]{};
                
                \foreach \i in {5,...,7} 
                    \draw ({(360/7)*\i}:2) node(\i)[vertex,fill=azulCielo,
                    label=(360/7)*\i:{\large $v_{\i}$}]{};

            \foreach \i/\j in
                {1/5,1/6,1/7,2/5,2/6,2/7,3/5,3/6,3/7,4/5,
            4/6,4/7}
                \draw [edge,grisOscuro!50] (\i) to (\j);
            
            \foreach \i/\j in
                {1/2,1/3,1/4,2/3,2/4,3/4}
                \draw [wedge,morado] (\i) to (\j);
            
            \foreach \i/\j in
                {5/6,6/7}
                \draw [wedge,azulCielo] (\i) to (\j);
            
            \draw [wedge,grisOscuro!80] (5) to (7);
            \end{scope}
                
    \end{tikzpicture}
    \caption{Una gr\'afica en la que se resalta una subgr\'afica inducida,
    ${\color{morado}\bf H}$, y una subgr\'afica, ${\color{azulCielo}\bf H'}$.}
    \label{fig:subgraf}
\end{figure}


\section{Clanes y conjuntos independientes}
\label{sec:clanes-CIndep}

    Algo que es relevante a trav\'es de varios temas de la Teor\'ia de
     Gr\'aficas es encontrar conjuntos de v\'ertices que sean adyacentes dos a
     dos o conjuntos de v\'ertices que, al contrario, no tengan vecinos dentro
     del conjunto. Teniendo una gr\'afica $G$ y un conjunto $S \subset V(G)$.
     Decimos que $S$ es un \indice{conjunto independiente} si cualesquiera dos
     v\'ertices de $S$ no son adyacentes. Esto quiere decir que todos los
     elementos de $S$ tienen grado $0$ en $G[S]$. La cardinalidad del conjunto
     independiente m\'as grande de una gr\'afica se llama \indice{n\'umero de
     independencia} y se denota $\alpha$. Por otro lado, decimos que $S$ es un
     \indice{clan} si cualesquiera dos v\'ertices en $S$ son adyacentes. En este
     caso, el grado de todo v\'ertice de $S$ es $|S-1|$ en $G[S]$. El
     \indice{n\'umero de clan} es la cardinaliad del clan con mayor n\'umero de
     elementos de una gr\'afica, se denota $\omega$. En \cref{fig:ClanInd} se
     muestra un ejemplo de un ${\color{coral}\bf clan}$ y un ${\color{verde}\bf
     conjunto}$ ${\color{verde}\bf independiente}$ de una gr\'afica. Es f\'acil
     observar que el ${\color{verde}\bf conjunto}$ ${\color{verde}\bf
     independiente}$ no es el conjunto independiente m\'as grande que podemos
     obtener, pues podr\'iamos agregar el v\'ertice $v_2$ y seguir\'ia siendo un
     conjunto independiente. De igual manera, podemos notar que existe un clan
     con cardinalidad mayor al ${\color{coral}\bf clan}$ mostrado. Este clan,
     llam\'emoslo $K$, est\'a mostrado con el per\'imetro de los v\'ertices de
     color ${\color{azulCielo}\bf azul}$. Es f\'acil ver que $|K| = \omega$.


\begin{figure}[ht!]
    \centering
       \begin{tikzpicture}
    
            \begin{scope}[xshift=0cm,scale=1]
                \foreach \i in {5,...,7} 
                    \draw ({(360/4)*\i}:1.3) node(\i)[avertex, fill=coral,
                    label=(360/4)*\i:{\large $v_{\i}$}]{};
                \draw ({(360/4)*8}:1.3)node(8)[avertex,label=(360/4)*8:{\large
                    $v_{8}$}]{};
                
                \foreach \i in {1,3} 
                    \draw ({(360/8)*(\i*2-1)}:3) node(\i)[bvertex,fill=verde,
                    label=(360/8)*(\i*2-1):{\large $v_{\i}$}]{};
                
                \foreach \i in {2,4}
                    \draw ({(360/8)*(\i*2-1)}:3)
                   node(\i)[bvertex,label=(360/8)*(\i*2-1):{\large $v_{\i}$}]{};


            \foreach \i/\j in
                {1/5,1/8,2/5,2/6,2/4,3/6,3/7,3/4,4/7,4/8}
                \draw [edge,grisOscuro!50] (\i) to (\j);
            
            \foreach \i/\j in
                {5/6,5/7,6/7}
                \draw [wedge,coral] (\i) to (\j);
            
            \foreach \i/\j in {8/5,8/6,8/7} 
                \draw [wedge,azulCielo!60] (\i) to (\j);
            \end{scope}
                
    \end{tikzpicture}
    \caption{Una gr\'afica, resaltando dos clanes y un conjunto independiente.}
    \label{fig:ClanInd}
\end{figure}


\section{Caminos y conexidad}
\label{sec:CamConex}

%camino/trayectoria/ciclo generador
%primera instancia
%comunmente

En Teor\'ia de Gr\'aficas resulta interesante preguntarse si, en una
gr\'afica, hay alg\'un ``v\'inculo'' entre dos v\'ertices. Si estos dos
v\'ertices son adyacentes, es claro que ese v\'inculo existe a trav\'es de
la arista que comparten. Un \indice{$uv$-camino} entre los v\'ertices $u$ y
$v$ de una gr\'afica $G$ es una sucesi\'on alternada de v\'ertices y aristas
de $G$ de la siguiente forma, $W=(v_0, e_1,v_1, \dots, e_{k-1},v_{k-1},
e_k,v_k)$ con $v_i \in V(G)$ y $v_{j-1}v_j = e_j \in E(G)$, para $i \in \{0,
\dots, k\}$ y $j \in \{ 1, \dots, k\}$. Este $uv$-camino es una forma de
vincular el v\'ertice $u$ con el v\'ertice $v$. En caso de que no sea
necesario especificar de qu\'e v\'ertice a qu\'e v\'ertice va la suceci\'on,
nos referiremos simplemente a un \indice{camino}. Al ser una sucesi\'on, a
un camino $W$ se le puede asociar una longitud, que ser\'a el n\'umero de
posiciones en la sucesi\'on $W$ que son ocupadas por aristas, se denota
$\ell(W)$ . 

Al poderle asignar una longitud a un camino, es natural utilizarla para ver
que tan ``cercano'' esta un v\'ertice de otro. Pero, observamos que, un
camino puede pasar m\'ultiples veces por el mismo v\'ertice o la misma
arista. En otras palabras, el concepto de camino no ayuda mucho para
encontrar que tan ``cerca'' esta un v\'ertice de otro, puesto que puede
existir otro camino que ocupe los mismos v\'ertices y aristas multiples
veces. Por lo tanto, introducimos el concepto de trayectoria. Una
\indice{trayectoria} es un camino que no repite v\'ertices, m\'as a\'un, una
\indice{$uv$-trayectoria} es una trayectoria que tiene v\'ertice inicial $u$
y v\'ertice final $v$. En una $uv$-trayectoria, los v\'ertices $u$ y $v$ son
llamados \indiceSub{v\'ertices}{extremos} y el resto de los v\'ertices de la
trayectoria son llamados \textbf{v\'ertices}
\indiceSub{v\'ertices}{internos}. Notemos que, al no repetir v\'ertices,
este camino tampoco repite aristas, pues, si lo hiciera, necesitar\'ia
volver a pasar por al menos uno de los extremos de dicha arista. De lo
anterior se sigue que todas las trayectorias que tengan los mismos
v\'ertices y aristas tambi\'en tienen la misma longitud. Esto es un mejor
acercamiento al concepto de ``cercan\'ia'' entre v\'ertices, pero se puede
tener el caso que haya m\'ultiples trayectorias de distintas longitudes
entre dos v\'ertices. Entonces, definimos la \indice{distancia} entre dos
v\'ertices $u$ y $v$ de una gr\'afica, denotada $d(u,v)$, como la longitud
de la $uv$-trayectoria m\'as corta. Sea $P$ una trayectoria en una
gr\'afica, decimos que $P$ no tiene \indice{cuerdas} si la subgr\'afica
inducida por $V(P)$ tiene grado m\'aximo 2. Observamos que cualquier
trayectoria de longitud m\'inima entre dos v\'ertices no tiene cuerdas.
Adicionalmente, podemos darle un ``tama\~{n}o'' a una gr\'afica $G$ al
definir su \indice{di\'ametro} como $\max_{v\in G}\{\max_{u\in
G}\{d(u,v)\}\}$. 
%Checar si esa notacion ya se uso

La siguiente proposici\'on nos muestra una relaci\'on entre caminos y
trayectorias.  

\begin{proposicion}
\label{prop:CamTray}
    En una gr\'afica $G$ con $u, v \in V(G)$, $u \ne v$, todo $uv$-camino
    contiene una $uv$-trayectoria.
\end{proposicion}

\begin{proof}
    Sea $W$ un $uv$-camino en una gr\'afica $G$, con $u,v \in V(G)$.
    Demostraremos, por inducci\'on sobre la longitud de $W$, que $W$
    contiene una $uv$-trayectoria. Primero, supongamos que $\ell(W)=1$, esto
    quiere decir que $W$ solo contiene una arista, por lo tanto $W$ es una
    trayectoria. Ahora, supongamos que todo $uv$-camino con longitud menor a
    $k$ contiene una $uv$-trayectoria. Tomamos $W$, tal que $\ell(W) = k$.
    Adem\'as, supongamos que $W$ no es una trayectoria, pues en el caso
    contrario, $W$ ser\'ia la trayectoria buscada. Tomamos $W=
    (w_0,e_1,w_1,e_2, \dots, e_n,w_n)$, con $w_0=u$, $w_n=v$, $e_i \in E(G)$
    y $w_j \in V(G)$, donde $i \in \{1, \dots, n\}$ y $j \in \{0, \dots,
    n\}$. Sea $m \in \{0, \dots, n\}$ el \'indice del primer v\'ertice de
    $W$ que se repite y sea $l \in \{m+1, \dots, n\}$ tal que $w_m = w_l$.
    Entonces, formamos el camino $W'= (w_0,e_1,w_1,\dots w_m, e_{l+1},
    w_{l+1} \dots, e_n,w_n)$. Observamos que $\ell(W')<\ell(W)$, por lo que,
    por hip\'otesis de inducci\'on, $W'$ contiene una $uv$-trayectoria. Por
    lo tanto $W$ tambi\'en contiene una $uv$-trayectoria.
\end{proof}

Por \cref{prop:CamTray}, cada que hablemos de caminos, siempre hay una
trayectoria contenida... Ahora nos enfocaremos en trayectorias. Como se
mencion\'o anteriormente, entre dos v\'ertices $u$ y $v$ pueden existir
m\'ultiples trayectorias. Decimos que $X$ y $Y$ son dos $uv$-trayectorias
\indiceSub{$uv$-trayectoria}{internamente ajenas} si $V(X)\cap
V(Y)=\{u,v\}$. Este concepto nos sevir\'a m\'as adelante en la secci\'on de
conexidad. Adicionalemtne, podemos ver que al recorrer una trayectoria de
manera inversa obtenemos otra trayectoria. Formalmente hablando, teniendo
una $uv$-trayectoria $X$, su trayectoria \indiceSub{trayectoria}{inversa}, a
la cu\'al denotamos $X^{-1}$, es una $vu$-trayectoria que tiene los mismos
v\'ertices y aristas que $X$.

\begin{figure}[htb!]
    \centering
        \begin{tikzpicture}
    
            \begin{scope}[xshift=0cm,scale=1]
                \foreach \i in {1,2} \draw ({(360/6)*\i}:2.5)
                    node(\i)[vertex, label=(360/6)*\i:{\large $v_{\i}$}]{};
            
            \foreach \i in {3,5} \draw ({(360/6)*\i}:2.5)
                    node(\i)[vertex, fill =coral, label=(360/6)*\i:{\large $v_{\i}$}]{};
            
            \draw ({(360/6)*6}:2.5) node(6)[vertex, fill=morado, label=(360/6)*6:{\large $v_{6}$}]{};
            \draw ({(360/6)*4}:2.5) node(4)[vertex, fill=verde, label=(360/6)*4:{\large $v_{4}$}]{};
            \draw (1,0) node (7) [vertex,fill=morado, label=300:{\large $v_7$}] {};
            \draw (-1,0) node (8) [vertex,fill=morado, label=150:{\large $v_8$}] {};
                
            \foreach \i/\j in
                {1/6,2/3,2/8,4/8,5/7,5/8}
                \draw [edge,grisOscuro!50] (\i) to (\j);
            
                \foreach \i/\j in
                {3/8,7/8,2/7,1/2,1/7,6/7,5/6}
                \draw [wedge,morado] (\i) to (\j);
            
            \foreach \i/\j in
                {3/4,4/5}
                \draw [wedge,verde] (\i) to (\j);

            \end{scope}
                
    \end{tikzpicture}
    \caption{Una gr\'afica en la que se resaltan las ${\color{morado}\bf
    aristas}$ de un ${\color{coral}\bf v_3 v_5}$-camino con su
    ${\color{coral}\bf v_3 v_5}$-trayectoria contenida, resaltando sus
    ${\color{morado}\textbf{v\'ertices}}$.  Tambi\'en se resalta una
    ${\color{coral}\bf v_3 v_5}$-trayectoria sin cuerdas de color
    ${\color{verde}\bf verde}$.}
    \label{fig:ex-caminos}
\end{figure}
    
En \cref{fig:ex-caminos} podemos ver un ejemplo de \ref{prop:CamTray} donde se
resaltan las ${\color{morado}\bf aristas}$ de un $v_3 v_5$-camino y donde, para
mostrar la trayectoria contenida, se resaltan los
${\color{morado}\textbf{v\'ertices}}$ de una $v_3 v_5$-trayectoria.
Adicionalmente, en la misma figura se observan dos ${\color{coral}\bf v_3
v_5}$-trayectorias internamente ajenas. Ambas est\'an representadas con sus
v\'ertices internos coloreados del mismo color, una de ${\color{morado}\bf
morado}$ y la otra de ${\color{verde}\bf verde}$. Notamos que la
${\color{morado}\textbf{trayectoria morada}}$ tiene cuerdas, entre ellas la
arista $v_8v_5$. Por otro lado, la ${\color{verde}\textbf{trayectoria verde}}$
no tiene cuerdas y es la trayectoria m\'as corta entre $v_3$ y $v_5$, por lo que
la distancia entre $v_3$ y $v_5$ es $2$.

Cuando los extremos de un camino son iguales, se llama \textbf{camino}
\indiceSub{camino}{cerrado}. Al igual que en el caso de las trayectorias, vale
la pena resaltar el caso en el que un camino cerrado no repite v\'ertices (salvo
los extremos). Un \indice{ciclo} es un camino cerrado de longitud al menos $3$
que, adem\'as, no repite v\'ertices, salvo los extremos.  Muchas v\'eces nos
interesa saber si una gr\'afica contiene o no contiene ciclos.   Una gr\'afica
sin ciclos es llamada \indiceSub{gr\'afica}{ac\'iclica}.
   
Hasta ahora nos hemos enfocado en los ``v\'inculos'' entre dos v\'ertices de una
gr\'afica. Sin embargo, es importante mencionar que no siempre existe manera de
``vincular'' dos v\'ertices, como se puede ver \cref{fig:diagGraf}, donde no
existen trayectorias que conecten a $v_7$ con alg\'un otro v\'ertice.  Definimos
una gr\'afica $G$ como \indiceSub{gr\'afica}{conexa}\index{conexa!gr\'afica} si,
para cualquier par de v\'ertices $u,v \in V(G)$, existe una $uv$-trayectoria. Si
una gr\'afica no cumple esta propiedad, decimos que es
\indiceSub{gr\'afica}{inconexa}. Observamos que toda gr\'afica, incluso las
gr\'aficas inconexas, contienen ``partes'' que s\'i son conexas, por lo que en
todo caso, al estudiar la conexidad de las gr\'aficas, nos podemos enfocar en
dichas partes. En una gr\'afica $G$, las subgr\'aficas m\'aximas con la
propiedad de ser conexas son llamadas \indice{componentes
conexas}\index{conexa!componente}. Notamos que si una gr\'afica es conexa,
contiene s\'olo una componente conexa, ella misma. 

Algo que tambi\'en puede ser de inter\'es en este tema es, que tan dif\'icil es
``desvincular'' dos v\'ertices. Una manera de verlo es cuestionarse si existe
alg\'un v\'ertice o conjunto de v\'ertices que, al quitarlo, ``desvincule'' a
los v\'ertices. A este conjunto lo llamamos corte por v\'ertices. Formalmente,
un \indice{corte por v\'ertices} $S$ de una gr\'afica $G$ es un subconjunto de
$V(G)$ tal que $G[V(G)-S]$ es inconexa. Cabe notar que no todas las gr\'aficas
tienen corte por v\'ertices, a saber, las gr\'aficas completas (introducidas en
la siguiente secci\'on) no tienen corte por v\'ertices y son las \'unicas con
esta propiedad.
%% TODO: conectar estos dos párrafos
La \indice{conexidad local} entre dos v\'ertices distintos $u$ y $v$ se define
como el m\'aximo n\'umero de $uv$-trayectorias internamente ajenas. Este
n\'umero se denota por $p(u,v)$. Decimos que una gr\'afica no trivial es
\indiceSub{gr\'afica}{$t$-conexa} si, para cualesquiera dos v\'ertices $u,v \in
V(G)$, se tiene que $p(u,v) \geq t$. Por convenci\'on, decimos que la gr\'afica
trivial es $0$-conexa y no es $k$-conexa par ning\'un $k > 0$. Adicionalmente,
definimos la \indice{conexidad} de una gr\'afica como el m\'aximo n\'umero $t$
para el cu\'al la gr\'afica es $t$-conexa. Un ejemplo de conexidad es considerar
las gr\'aficas en \cref{fig:isoGraf} y \cref{fig:subgraf}. Notamos que ambas son
conexas, sin embargo, necesitar\'iamos ``borrar'' m\'as aristas para
``desvincular'' v\'ertices en la gr\'afica de \cref{fig:subgraf} que en las
gr\'aficas de \cref{fig:isoGraf}, a saber, las gr\'aficas de \cref*{fig:isoGraf}
son $2$-conexas y la gr\'afica de \cref{fig:subgraf} es $6$-conexa.

Otra forma de determinar como ``desvincular'' dos v\'ertices es preguntarse
cuantas aristas se necesitar\'ian ``borrar'' para que no hubiera trayectorias
entre dichos v\'ertices. Es f\'acil observar que esto est\'a relacionado con la
cantidad de trayectorias ajenas por aristas
%% TODO: agregar definición trayectorias ajenas por aristas
que existan entre dichos
v\'ertices.

\section{Operaciones}
\label{sec:operaciones}

\begin{definicion} Definiciones que faltan (Operaciones):
\begin{enumerate}
    \item Dadas las gr\'aficas $G$ y $H$, el producto carteciano de ambas
    gr\'aficas, denotado $G \square H$, es la gr\'afica cuyo conjunto de
    v\'ertices es el producto carteciano de $V(G)$ y $V(H)$, es decir $V(G
    \square H) = V(G) \times V(H)$. Adem\'as, $(g_1,g_2)(h_1,h_2)$ es arista
    de $G \square H$ si y s\'olo si $g_1 = g_2$ y $h_1h_2 \in E(H)$, o $h_1
    = h_2$ y $g_1g_2 \in E(G)$.
    \item Dadas dos gr\'aficas $G$ y $H$, la uni\'on de ambas gr\'aficas,
    denotada $G \bigcup H$ es la gr\'afica cuyo conjunto de v\'ertices es la
    uni\'on de los conjuntos de v\'ertices de ambas gr\'aficas y su conjunto
    de aristas es $E(G) \bigcup E(H)$. 
    \item La uni\'on ajena de gr\'aficas $G$ y $H$ es la uni\'on de $G$ y
    $H$, donde $v(G) \bigcap V(H) = \varnothing$. A esta funci\'on la
    denotamos $G + H$. 
\end{enumerate}
\end{definicion}


%notacion de complemento de grafica y el ejemplo de K complemento, que es una
%gr\'afica vacia, es decir su numero de independencia es igual al numero de vertices
\section{Algunas familias relevantes}
\label{sec:famGraf}
    
 es de ayuda
separarlas en familias... Muchas veces, en Teor\'ia de Gr\'aficas, es de utilidad
el enfocarnos en familias de gr\'aficas .... En
Teor\'ia de Gr\'aficas, muchas veces se estudian las gr\'aficas que comparten
cierta propiedad, por eso son importantes las familias de gr\'aficas     
Hay varias familias relevantes en la Teor\'ia de Gr\'aficas, en esta
    secci\'on hablaremos de las familias que son relevantes para este trabajo y
    sus propiedades.

    Decimos que una gr\'afica $G$ es \indiceSub{gr\'afica}{completa} si su
    conjunto de aristas es igual a $\binom{V(G)}{2}$. A la gr\'afica completa de
    orden $n$ se le denota $K_n$. Cuando una gr\'afica est\'a conformada
    \'unicamente por v\'ertices aislados la denotamos $\overline{K_n}$, donde
    $n$ es el n\'umero de v\'ertices de la gr\'afica. A $K_1$ tambi\'en se le
    conoce como la gr\'afica  \indice{trivial}.

    %definimos particion?
    Una familia de gr\'aficas que es muy estudiada en Teor\'ia de Gr\'aficas es
    la familia de las gr\'aficas \indiceSub{gr\'afica}{$k$-partitas}. Una
    gr\'afica es $k$-\textit{partita} si su conjunto de v\'ertices admite una
    partici\'on en a lo m\'as $k$ conjuntos independientes. Un caso muy
    relevante, y que nos ser\'a particularmente \'util en este trabajo, es el
    caso en el que $k=2$. En este caso se dice que es una gr\'afica
    \indiceSub{gr\'afica}{bipartita}. Una gr\'afica bipartita $G$ con
    partici\'on $(X,Y)$ se denota $G[X,Y]$, y $(X,Y)$ es llamada
    \indice{bipartici\'on}.
    %% TODO: conectar la siguiente idea
    ... una gr\'afica $k$-partita puede ser completa y, por cuestion de utilidad
    para el trabajo, lo definiremos formalmente para una gr\'afica bipartita.
    Una gr\'afica es \indiceSub{gr\'afica}{bipartita completa} si es bipartita y
    dada la partici\'on $X$ y $Y$ de sus v\'ertices, cada v\'ertice de $X$ es
    adyacente a cada v\'ertice de $Y$. 

    Otra  familia de gr\'aficas que se estudiar\'a este trabajo es la familia de
    las gr\'aficas hamiltonianas. Para definir esta familia, es preciso definir
    algunos conceptos necesarios. El primer concepto es el de trayectoria
    hamiltoniniana. Una trayectoria \indiceSub{trayectoria}{hamiltoniana} es una
    trayectoria generadora, es decir, una trayectoria cuyo conjunto de
    v\'ertices es el conjunto de v\'ertices de la gr\'afica. Asimismo, un ciclo
    \indiceSub{ciclo}{hamiltoniano} es un ciclo generador. Ahora, pasamos a
    definir una gr\'afica \indiceSub{gr\'afica}{hamiltoniana}, que es una
    gr\'afica que contiene un ciclo hamiltoniano. En la Figura... se muestra un
    ejemplo .... Una definici\'on que nos ser\'a de utilidad a lo largo de este
    trabajo es la de subtrayectoria hamiltoniana. Definimos una
    \indice{subtrayectoria hamiltoniana} como la trayectoria hamiltoniana
    resultante de quitarle una arista a un ciclo hamiltoniano. 

    La \'ultima familia de gr\'aficas que veremos en esta secci\'on son los
    abanicos. Un \indice{abanico} $\mathcal{F}_n$ es la gr\'afica obtenida de la
    uni\'on de $K_1$ y $P_{n-1}$, donde los primeros $n-1$ v\'ertices de
    $\mathcal{F}_n$ son los v\'ertices pertenecientes a $P_{n-1}$ y el
    $n$-\'esimo v\'ertice de la gr\'afica es que le corresponde a $K_1$. Al
    igual que en varios casos, existe una generalizaci\'on de esta estructura de
    gr\'afica al sustituir $K_1$ por $\overline{K_n}$. Un abanico
    \indiceSub{abanico}{generalizado}, denotiado $\mathcal{F}_{m.n}$, se define
    como $\mathcal{F}_{m,n}=\overline{K_m}+P_n$. En la Figura.... se puede ver
    un ejemplo de un abanico y un abanico generalizado.....



\section{Conceptos de coloraci\'on}
\label{sec:coloracion}

\begin{definicion} Conceptos de coloraci\'on:
    \label{def:coloracion}
    \begin{enumerate}
        \item Sea $G$ una gr\'afica, una $k$-\textit{coloraci\'on}, o
        simplemente una $\textit{coloraci\'on}$, es una funcion $c \colon
        E(G)\to S$, con $S$ un conjunto de cardinalidad $k$.
        \item Sea $G$ una gr\'afica y $S$ un conjunto de cardinalidad $k$, con
        $k$ entero positivo. La funci\'on $c: V \to S$ es una
        $k$-\textit{coloraci\'on por v\'ertices}, o simplemente una
        $k$-\textit{coloraci\'on}, de $G$. Los elementos de $S$ son llamados
        $\textit{colores}$.
        \item Una coloraci\'on es una \textit{coloraci\'on propia} si todos los
        v\'ertices adyacentes ocupan colores diferentes.
        \item Al subconjunto de v\'ertices que comparten un color en una
        coloraci\'on se le llama una clase de color.
        \item Sea $G$ una gr\'afica y $c$ una coloraci\'on con $S$ colores. Para
        cada color $i \in S$, se define la clase crom\'atica de $i$ de la
        coloraci\'on $c$, denotado $V_i$, como el conjunto de todos los
        v\'ertices de $G$ que ocupan el color $i$, en otras palabras $V_i =
        c^{-1}[i]$.
        \item  Sea $G$ una gr\'afica, el m\'inimo entero $k$ para el cu\'al $G$
        es $k$-coloreable es el n\'umero crom\'atico de $G$, denotado por
        $\chi(G)$. En este caso decimos que $G$ es $k$-crom\'atica.
    \end{enumerate}
\end{definicion}d

\begin{definicion} Gr\'afica de fichas:
    \label{def:fichas}


    %F_k(G,L)
    \begin{enumerate}
        \item Sean $G$ una gr\'afica y $k \geq 1$ un entero. Definimos la
        gr\'afica de $k$-fichas de $G$, denotada por $F_k(G)$, como la
        gr\'afica con conjunto de v\'ertices $\binom{V(G)}{k}$ y donde dos
        v\'ertices $A$ y $B$ en $F_k(G)$ son adyacentes si y s\'olo si $|A
        \triangle B| ={a,b}$, con $a \in A$, $b \in B$ y $ab \in E(G)$.
        \item Sea $A$ un $k$-comjunto en la gr\'afica $G$ tal que $a \in A$
        y $b\notin A$. Definimos $A'= (A \setminus \{a\}) \cup \{b\}$.
        \item Sea $P$ una trayectoria en la gr\'afica $G$. Tomamos $A$ un
        $k$-conjunto en $G$ de tal manera que $a\in A$ y $b \notin A$. Si
        $A\cap P =\{v_1, \dots, v_q\}$, con $v_1 = a$, definimos $A
        \xrightarrow[P]{} A'$ como la trayectoria en $F_k(G)$ entre $A$ y
        $A'$ que sigue la siguiente secuencia. Primero movemos la ficha de
        $v_q$, v\'ertice de $G$, hacia el v\'ertice $b$, v\'ertice de $G$,
        por $P$. Luego, para los v\'ertices $v_i$ en $G$,con $i \in \{q-1,
        q-2, \dots 1\}$, movemos la ficha de $v_i$ a $v_{i+1}$. 
    \end{enumerate}
\end{definicion}

\begin{figure}[ht!]
    \centering
       \begin{tikzpicture}
    
        \begin{scope}[xshift=-8.5cm]
            \foreach \i in {0,...,4}
                \draw ({(360/5)*\i}:2) node(\i)[vertex]{\pgfmathparse{int(\i+1)}
                \pgfmathresult};
            \foreach \i/\j in {0/1,0/2,0/3,0/4,1/2,2/3,3/4}
                \draw [edge] (\i) to (\j);
            \end{scope}
        
        \begin{scope}[xshift=0cm,yshift=0cm,scale=1]
            \draw (0,1.5) node (1) [vertex, label=90:{\scriptsize $134$}] {};
            \draw (-2,2) node (2) [vertex, label=90:{\scriptsize $234$}] {};
            \draw (2,2) node (3) [vertex, label=90:{\scriptsize $345$}] {};
            \draw (-3,1) node (4) [vertex, label=180:{\scriptsize $123$}] {};
            \draw (3,1) node (5) [vertex, label=0:{\scriptsize $145$}] {}; 
            \draw (-1,0.7) node (6) [vertex, label=87:{\scriptsize $235$}] {};
            \draw (1,0.7) node (7) [vertex, label=93:{\scriptsize $245$}] {};
            \draw (-1.1,-1) node (8) [vertex, label=240:{\scriptsize $124$}] {};
            \draw (1.1,-1) node (9) [vertex, label=330:{\scriptsize $135$}] {};
            \draw (0,-2.3) node (10) [vertex, label=270:{\scriptsize $125$}] {};
           
            \foreach \i/\j in{1/2, 1/3, 1/8, 1/9, 2/4, 2/6, 2/8, 3/5, 3/7, 3/9,
            4/6, 4/8, 5/7, 5/9, 6/7, 6/9, 6/10, 7/8, 7/10, 8/10, 9/10} 
            \draw [edge] (\i) to (\j);
       \end{scope}
    
            
    
    \end{tikzpicture}
    \caption{El diagrama de una gr\'afica simple $G$ y $F_3(G)$}
    \label{fig:ex-tok-path}
\end{figure}