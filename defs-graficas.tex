\chapter{Definiciones de Teor\'ia de Gr\'aficas}%
\label{cap:defs grafs}

\section{Definiciones b\'asicas}%
\label{sec:def-basicas}

Una gr\'afica $G$ es una pareja ordenada de conjuntos finitos $(V(G), E(G))$,
donde $V(G)$ es no vac\'io y $E(G) \subseteq \binom{V(G)}{2}$. Los elementos de
$V(G)$ son llamados v\'ertices y los elementos de $E(G)$ son aristas. Para un $e
\in E(G)$ y $u,v \in V(G)$, decimos que $e$ es incidente en $u$ y en $v$ si $e=
\{u, v\}$. A su vez, los v\'ertices $u$ y $v$ son extremos de la arista $e$. De
igual manera podemos decir que $u$ y $v$ son v\'ertices adyacentes o vecinos. Al
conjunto de vecinos de un v\'ertice $v$ se le llama vecindad, se denota
$N_G(v)$, mientras que n\'umero de aristas incidentes en $v$ se le llama grado
de $v$. Un v\'ertice aisaldo, es decir, un v\'ertice que no tiene vecinos, tiene
grado $0$.

Las gr\'aficas pueden ser representadas por diagramas en los que los v\'ertices
son c\'irculos y las arista son l\'inea cuyos extremos los v\'ertices
incidentes. Cabe recalcar que una gr\'afica puede tener  m\'ultiples diagramas,
como se muestra en \cref{fig:diagGraf}. Es f\'acil observar que en ambos
diagr\'amas de $G$ los v\'ertices tienen las mismas propiedades, por ejemplo
que $N_G(v_3)=\{v_1,v_4,v_5\}$ o que el grado de $v_2$ es $1$.

De igual manera que una gr\'afica puede tener varios diagramas, a un diagr\'ama
se le pueden asociar distintas gr\'aficas \cref{fig:isoGraf}. A pesar de que
estas gr\'aficas son distintas, comparten muchas propiedades como el n\'umero de
v\'ertices, las adyacencias o el grado de los v\'ertices. Estas g\'aficas se
dice que son isomorfas. Formalmente hablando, dos gr\'aficas $G$ y $H$ son
isomorfas si exite una biyecci\'on $\theta: V(G) \rightarrow V(H)$ tal que para
cualesquiera dos v\'ertices $u, v \in V(G)$, se cumple que $uv \in E(G)$ si y
s\'olo si $\theta(u)\theta(v) \in E(H)$. Es decir, la biyecci\'on $\theta$
preserva adyacencias y no adyacencias. Cuando $G$ es ismorfa a $H$ lo denotamos
$G \cong H$ y la biyecci\'on entre las gr\'aficas es llamada isomorfismo.

\begin{definicion}
    \label{def:propiedades}
    \begin{enumerate}
        \item Sean $G$ y $H$ tales que $V(H) \subseteq V(G)$ y $E(H) \subseteq
        E(G)$. Entonces decimos que $H$ es una subrgr\'afica de $G$ y $G$ es una
        supergr\'afica de $H$. Esta relaci\'on la denotamos por $H \subseteq G$.
        \item Sea $G$ una gr\'afica y $V \subseteq V(G)$ con $V \neq
        \varnothing$. Llamamos subgr\'afica inducida de $G$ por $V$, denotada
        $G[V]$ a la subgr\'afica de $G$ cuyo conjunto de v\'ertices es $V$ y
        cuyo conjunto de aristas son aquellas aristas de $G$ que tienen ambos
        extremos en $V$.
    \end{enumerate}
\end{definicion}

\begin{figure}[ht]
    \centering
       \begin{tikzpicture}
    
            \begin{scope}[xshift=6cm,scale=0.8]
                \draw (0.5,-1.5) node (1) [vertex, label=270:{\large $v_1$}] {};
                \draw (-1.9,-1.6) node (2) [vertex, label=270:{\large $v_2$}] {};
                \draw (1.2,1.5) node (3) [vertex, label=90:{\large $v_3$}] {};
                \draw (-2,0.5) node (4) [vertex, label=90:{\large $v_4$}] {};
                \draw (2.5,-0.6) node (5) [vertex, label=270:{\large $v_5$}] {};
                \draw (0,0) node (6) [vertex, label=90:{\large $v_6$}] {};
                
                \foreach \i/\j in {1/3,1/4,1/6,2/6,3/4,3/5,5/6}
                \draw [edge] (\i) to (\j);
            \end{scope}
    
        \begin{scope}[xshift=0cm,scale=0.6]
            \draw ({(360/4)*3}:2) node(1)[vertex, label=270:{\large $v_1$}]{};
            \draw (0,0) node (2) [vertex, label=90:{\large $v_2$}] {};
            \draw ({(360/4)*2}:2) node(3)[vertex, label=180:{\large $v_3$}]{};
            \draw (-2.5,-2) node (4) [vertex, label=270:{\large $v_4$}] {};
            \draw ({(360/4)*1}:2) node(5)[vertex, label=90:{\large $v_5$}]{};
            \draw ({(360/4)*0}:2) node(6)[vertex, label=0:{\large $v_6$}]{};

            \foreach \i/\j in {1/3,1/4,1/6,2/6,3/4,3/5,5/6}
                \draw [edge] (\i) to (\j);  
            \end{scope}
                
    \end{tikzpicture}
    \caption{Diferentes diagramas de una gr\'afica $G$}
    \label{fig:diagGraf}
\end{figure}

\begin{figure}[ht]
    \centering
       \begin{tikzpicture}
    
            \begin{scope}[xshift=0cm,scale=0.8]
                \foreach \i in {1,...,5} 
                    \draw ({(360/5)*\i}:2) node(\i)[vertex,
                    label=(360/5)*\i:{\large $v_{\i}$}]{};

            \foreach \i/\j in {5/1,1/2,2/3,3/4,4/5}
                \draw [edge] (\i) to (\j);
            \end{scope}
    
        \begin{scope}[xshift=6cm,scale=0.8]
                \foreach \i in {1,...,5} 
                    \draw ({(360/5)*\i}:2) node(\i)[vertex,
                    label=(360/5)*\i:{\large $v_{\i}$}]{};

            \foreach \i/\j in {5/2,2/4,4/1,1/3,3/5}
                \draw [edge] (\i) to (\j);
            \end{scope}
                
    \end{tikzpicture}
    \caption{Dos gr\'aficas isomorfas}
    \label{fig:isoGraf}
\end{figure}


\section{Clanes y conjuntos independientes}
\label{sec:clanes-CIndep}

\begin{definicion} Clanes y conjuntos independientes:
    \label{def:clanes y conjunto independiente}
    \begin{enumerate}
        \item Sean $G$ una gr\'afica y $S \subset V(G)$. Decimos que $S$ es
        independiente si cualesquiera dos v\'ertices de $S$ no son adyacentes.
        \item Sean $G$ una gr\'afica y $S \subset V(G)$. Decimos que $S$ es un
        clan si cualesquiera dos v\'ertices en $S$ son adyacentes.
        \item El n\'umero de clan de una gr\'afica $G$, denotado por $\omega_G$,
        es la cardinalidad del clan con mayor n\'umero de elementos en $G$.
        \item Agregar n\'umero de independencia $\alpha$.
    \end{enumerate}
\end{definicion}

\section{Algunas familias relevantes}
\label{sec:famGraf}
\begin{definicion} Tipos de familias de gr\'aficas:
    \label{def:familias}
    \begin{enumerate}
        \item Sea $G$ una gr\'afica, decimos que $G$ es completa si su
        conjunto de aristas es igual a $\binom{V(G)}{2}$.
        \item Una gr\'afica $G$ es una gr\'afica bipartita si su conjunto de
        v\'ertices admite una partici\'on en $X$ y $Y$ de tal manera que $X$ y
        $Y$ son conjuntos independientes. De igual manera, definimos una
        gr\'afica $k$-\textit{partita} como la gr\'afica cuyo conjunto de
        v\'ertices admite una partici\'on en a lo m\'as $k$ conjuntos
        independientes.
        \item Una gr\'afica es bipartita completa si es bipartita y dada la
        partici\'on $X$ y $Y$ de sus v\'ertices, cada v\'ertice de $X$ es
        adyacente a cada v\'ertice de $Y$.   
    \end{enumerate}
\end{definicion}

\section{Tipos de recorridos}
\label{sec:recorridos}


\begin{definicion} Tipos de recorridos de una gr\'afica:
    \label{def:tipos de recorridos}
    \begin{enumerate}
        \item Sea $G$ una gr\'afica, un camino $W$ en $G$ es una sucesi\'on
        alternante de v\'ertices y aristas de $G$ de la siguiente forma $W=(v_0,
        e_1,v_1, \dots, e_{k-1},v_{k-1}, e_k,v_k)$ con $v_i \in V(G)$ y
        $v_{j-1}v_j = e_j \in E(G)$, para $i \in \{0, \dots, k\}$ y $j \in \{ 1,
        \dots, k\}$.
        \item Una trayectoria es un camino que no repite v\'ertices. Llamamos
        $uv$-\textit{trayectoria} a una trayectoria con v\'ertice inicial $u$ y
        \'ultimo v\'ertice $v$.
        \item Decimos que un camino es cerrado si su v\'ertice inicial y su
        v\'ertice final son el mismo.
        \item Un ciclo es un camino cerrado de longitud al menos $3$ que no
        repite v\'ertices, salvo el primer y \'ultimo v\'ertice.
        \item Sea $P$ una trayectoria en una gr\'afica, decimos que $P$ no tiene
        cuerdas si la subgr\'afica inducida por $V(P)$ tiene grado m\'aximo 2.
    \end{enumerate}
\end{definicion}

\section{Conceptos de conexidad}
\label{sec:conexidad}

\begin{definicion} Conceptos de conexidad:
    \label{def:conexidad}
    \begin{enumerate}    
        \item Una gr\'afica $G$ es conexa si para cualquier partici\'on de
        $V(G)$ en dos conjuntos $X$ y $Y$, existe al menos una arista con un
        extremo en $X$ y el otro extremo en $Y$. Si una gr\'afica no es conexa,
        decimos que es inconexa.
        \item Sea $G$ una gr\'afica, nombramos componentes conexas a las
        subr\'aficas de $G$ que son m\'aximas con la propiedad de ser conexas.
        \item Sea $G$ una gr\'afica, un corte por v\'ertices $V$ es un
        subconjunto de $V(G)$ tal que $G[V(G)-V]$ es inconexa.
    \end{enumerate}
\end{definicion}

\section{Conceptos de coloraci\'on}
\label{sec:coloracion}

\begin{definicion} Conceptos de coloraci\'on:
    \label{def:coloracion}
    \begin{enumerate}
        \item Sea $G$ una gr\'afica, una $k$-\textit{coloraci\'on}, o
        simplemente una $\textit{coloraci\'on}$, es una funcion $c \colon
        E(G)\to S$, con $S$ un conjunto de cardinalidad $k$.
        \item Sea $G$ una gr\'afica y $S$ un conjunto de cardinalidad $k$, con
        $k$ entero positivo. La funci\'on $c: V \to S$ es una
        $k$-\textit{coloraci\'on por v\'ertices}, o simplemente una
        $k$-\textit{coloraci\'on}, de $G$. Los elementos de $S$ son llamados
        $\textit{colores}$.
        \item Una coloraci\'on es una \textit{coloraci\'on propia} si todos los
        v\'ertices adyacentes ocupan colores diferentes.
        \item Sea $G$ una gr\'afica y $c$ una coloraci\'on con $S$ colores. Para
        cada color $i \in S$, se define la clase crom\'atica de $i$ de la
        coloraci\'on $c$, denotado $V_i$, como el conjunto de todos los
        v\'ertices de $G$ a que ocupan el color $i$, en otras palabras $V_i =
        c^{-1}[i]$.
        \item  Sea $G$ una gr\'afica, el m\'inimo entero $k$ para el cu\'al $G$
        es $k$-coloreable es el n\'umero crom\'atico de $G$, denotado por
        $\chi(G)$.
    \end{enumerate}
\end{definicion}d

\begin{definicion} Gr\'afica de fichas:
    \label{def:fichas}
    \begin{enumerate}
        \item Sean $G$ una gr\'afica y $k \geq 1$ un entero. Definimos la
        gr\'afica de $k$-fichas de $G$, denotada por $F_k(G)$ como la
        gr\'afica con conjunto de v\'ertices $\binom{V(G)}{k}$ y donde dos
        v\'ertices $A$ y $B$ en $F_k(G)$ son adyacentes si y s\'olo si $|A
        \triangle B| ={a,b}$, con $a \in A$, $b \in B$ y $ab \in E(G)$.
        \item Sea $A$ un $k$-comjunto en la gr\'afica $G$ tal que $a \in A$
        y $b\notin A$. Definimos $A'= (A \setminus \{a\}) \cup \{b\}$.
        \item Sea $P$ una trayectoria en la gr\'afica $G$. Tomamos $A$ un
        $k$- conjunto en $G$ de tal manera que $a\in A$ y $b \notin A$. Si
        $A\cap P =\{v_1, \dots, v_q\}$, con $v_1 = a$, definimos $A
        \xrightarrow[P]{} A'$ como la trayectoria en $F_k(G)$ entre $A$ y
        $A'$ que sigue la siguiente secuencia. Primero movemos la ficha de
        $v_q$, v\'ertice de $G$, hacia el v\'ertice $b$, v\'ertice de $G$,
        por $P$. Luego, para los v\'ertices $v_i$ en $G$,con $i \in \{q-1,
        q-2, \dots 1\}$, movemos la ficha de $v_i$ a $v_{i+1}$. 
    \end{enumerate}
\end{definicion}

\begin{figure}[ht!]
    \centering
       \begin{tikzpicture}
    
        \begin{scope}[xshift=-8.5cm]
            \foreach \i in {0,...,4}
                \draw ({(360/5)*\i}:2) node(\i)[vertex]{\pgfmathparse{int(\i+1)}
                \pgfmathresult};
            \foreach \i/\j in {0/1,0/2,0/3,0/4,1/2,2/3,3/4}
                \draw [edge] (\i) to (\j);
            \end{scope}
        
        \begin{scope}[xshift=0cm,yshift=0cm,scale=1]
            \draw (0,1.5) node (1) [vertex, label=90:{\scriptsize $134$}] {};
            \draw (-2,2) node (2) [vertex, label=90:{\scriptsize $234$}] {};
            \draw (2,2) node (3) [vertex, label=90:{\scriptsize $345$}] {};
            \draw (-3,1) node (4) [vertex, label=180:{\scriptsize $123$}] {};
            \draw (3,1) node (5) [vertex, label=0:{\scriptsize $145$}] {}; 
            \draw (-1,0.7) node (6) [vertex, label=87:{\scriptsize $235$}] {};
            \draw (1,0.7) node (7) [vertex, label=93:{\scriptsize $245$}] {};
            \draw (-1.1,-1) node (8) [vertex, label=240:{\scriptsize $124$}] {};
            \draw (1.1,-1) node (9) [vertex, label=330:{\scriptsize $135$}] {};
            \draw (0,-2.3) node (10) [vertex, label=270:{\scriptsize $125$}] {};
           
            \foreach \i/\j in{1/2, 1/3, 1/8, 1/9, 2/4, 2/6, 2/8, 3/5, 3/7, 3/9,
            4/6, 4/8, 5/7, 5/9, 6/7, 6/9, 6/10, 7/8, 7/10, 8/10, 9/10} 
            \draw [edge] (\i) to (\j);
       \end{scope}
    
            
    
    \end{tikzpicture}
    \caption{El diagrama de una gr\'afica simple $G$ y $F_3(G)$}
    \label{fig:ex-tok-path}
\end{figure}