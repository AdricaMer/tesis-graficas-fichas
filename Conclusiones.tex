\chapter{Conclusiones}
\label{sec:conclusiones}


A trav\'es del estudio de las gr\'aficas de fichas realizado en este trabajo se
llega a varias concluciones. Esto es de esperarse, debido a que se estudian
varias propiedades de estas gr\'aficas. En primera instancia, al estudiar la
conexidad de las gr\'aficas, se encuentra que las gr\'aficas de fichas heredan
la conexidad de las gr\'aficas originales. M\'as a\'un, para gr\'aficas
$t$-conexas con $t \geq k$, donde $k$ es el n\'umero de fichas, y $t$ es
suficientemente grande, es decir, $|V(G)| \geq \frac{1}{2}kt$, se demuestra una
mejor cota para la conexidad de sus gr\'aficas de fichas. Dentro del art\'iculo
\textit{Token graphs} \cite{fabilaToken}, en el que se basa el cap\'itulo de
conexidad, se presenta la siguiente conjetura: Sea $G$ una gr\'afica. Si $G$ es
$t$-conexa, con $t \geq k$, entonces $F_k(G)$ es $k(t-k+1)$-conexa. En otras
palabras, no se necesita que la gr\'afica tenga un tama\~{n}o espec\'ifico.


Los siguientes dos temas que se trabajan tambi\'en salen del art\'iculo
\textit{Token graphs} \cite{fabilaToken}. Estos temas son el de clanes y
n\'umero crom\'atico. Se da una caracterizaci\'on de la estructura y tama\~no de
los clanes en las gr\'aficas de fichas. Algo que es interesante de observar es
que los clanes obtenidos en dichas gr\'aficas, tienen la misma cardinalidad que
el clan de la gr\'afica origial, de donde se construyen. El resultado principal
de la secci\'on de clanes es encontrar una f\'ormula para el n\'umero de clan de
las gr\'aficas de fichas.   En la siguiente secci\'on del mismo cap\'itulo,  se
trabaja el n\'umero crom\'atico de la gr\'afica de fichas a partir del n\'umero
crom\'atico de la gr\'afica original. Se logran obtener cotas usando el n\'umero
crom\'atico de la gr\'afica original que no dependan del n\'umero de fichas. 

Por \'ultimo, se estudia la hamiltonicidad de las gr\'aficas de fichas. El tema
de hamiltonicidad que se desarrolla en este trabajo sale de los art\'iculos
\textit{Hamiltonicity of token graphs of fan graphs} \cite{riveraHamilt} y
\textit{Hamiltonicity of Token Graphs of Some Join Graphs} \cite{adameHamilt}.
Ambos art\'iculos son m\'as recientes que el art\'iculo trabajado en las
secciones anteriores, por lo que el enfoque y acercamiento al tema son
diferentes. Es f\'acil observar que la propiedad de ser o no hamiltoniana no se
preserva en las gr\'aficas de fichas, al inicio del \Cref{cap:fichass} se
proporciona un ejemplo de ambos casos. No hay relaci\'on definida entre ambas en
este aspecto. Tanto los art\'iculos como el cap\'itulo de este trabajo se
enfocan en obtener una familia de gr\'aficas no hamiltonianas cuyas gr\'aficas
de fichas sean hamiltonianas. 

Para finalizar, vale la pena recalcar que este trabajo es meramente una
introducci\'on al \'area de gr\'aficas de fichas. Hay varios temas a desarrollar
que surgen de los resultados de este trabajo, como es la conjetura antes
mencionada o la busqueda de encontrar una familia hamiltoniana que preserve
hamiltonicidad en sus gr\'aficas de fichas. Las gr\'aficas de fichas es un
\'area relativamente nueva que se sigue investigando por lo que actualmente
existen m\'as resultados que no se abordaron en este trabajo. Sin embargo, se
espera que este trabajo haya servido como introducci\'on a las gr\'aficas de
fichas, as\'i como de despertar la curiosidad del lector en el tema.
