\chapter{Cap\'i{}tulo 2}%
\label{cap:ejemplos}

\section{Teoremas y demostraciones de clanes}%
\label{sec:etiquetas}


\begin{lema}%
        \label{lem:primero}
        Sean $A$, $B$ y $C$ v\'ertices en $F_k(G)$ tales que son adyacentes
        dos a dos. Entonces pasa s\'olo una de las siguientes
        opciones/relaciones $B \cap C \subset A$ o $A \subset B \cup C$
        \end{lema}

    
    \begin{proof}
        Primero, supongamos que existen $A$, $B$, $C$ v\'ertices en $F_k(G)$
        tales que son adyacentes dos a dos y cumplen $B \cap C \not\subset A$ y
        $A \not\subset B \cup C$. Entonces existen los v\'ertices  $a \in A
        \setminus (B \cup C)$ y $a \in (B \cap C)\setminus A$. Notamos que $b
        \in B \setminus A$ y $b \in C \setminus A$. De manera an\'aloga tenemos
        que $a \in A \setminus B$ y $a \in A \setminus C$. Por otro lado sabmeos
        que $A$ y $B$ son adyacentes en $F_k(G)$, al igual que $A$ y $C$, por lo
        tanto tenemos que $A \triangle B = \{a,b\}$ y $A \triangle C = \{a,
        b\}$. Entonces tenemos que $(B \cup C \cup \{a\})\setminus \{b\}
        \subseteq A$. Pero $B$ y $C$ son adyacentes en $F_k(G)$ por lo que $|B
        \cup C| = k+1$, entonces tenemos que $|A| \geq k+1$, lo c\'ual es una
        contradicci\'on pues $A \in V (F_k(G))$. 

        Ahora supongamos que existen $A$, $B$ y $C$ v\'ertices en $F_k(G)$ que
        son adyacentes dos a dos y cumplen $A \subset B \cup C$ y $B \cap C
        \subset A$. Como $B$ y $C$ son adyacentes, entonces tenemos que $B
        \triangle C = \{b,c\}$ con $b \in B$ y $c \in C$. Adem\'as tenemos que
        $|B \cap C| = k-1$ y $|B \cup C| = k +1$. Notamos que $B \cup C = (B\cap
        C) \cup \{b,c\}$ Por otro lado, sabemos que $|A|=k$ y da que $B \cap C
        \subset A$, tenemos que $A = (B \cap C) \cup \{a\}$. Adicionalmente, $A
        \subset B \cup C$, por lo que $a \in \{b, c\}$. Entonces tenemos que $A
        = (B \cap C) \cup \{b\}$ o $A = (B \cap C) \cup \{c\}$, es decir $A = B$
        o $A=C$, lo cu\'al es una contradicci\'on. Por lo tanto tenemos que $B
        \cap C \subset A$ o $A \subset B \cup C$.
    \end{proof}

    \begin{teorema}
        \label{teo:primero}
        Sea $X$ un conjunto de v\'ertices de $F_k(G)$. Entonces $X$ es un clan
        de $F_k(G)$ si y s\'olo si hay un clan $K$ de $G$ y un conjunto $S
        \subseteq V(G)$ tales que $K \cap S = \varnothing$ y pasa uno de los
        siguientes casos:
        \begin{enumerate}
            \item $X = \{S \cup \{v\}| v \in K\}$ y $|S| = k-1$
            \item $X = \{(S\cup K) \setminus \{v\}| v \in K \}$ y $|S| + |K| =
            k+1$
        \end{enumerate}
    \end{teorema}

    \begin{proof}
        
    \end{proof}